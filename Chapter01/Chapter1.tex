\section{Introduction}\label{s1}

\subsection{Major Sources}
The sources which I  consulted most frequently whilst developing this 
course  are:
\begin{description}
\item [\em Analytical Mechanics:] G.R.~Fowles, Third Edition (Holt, Rinehart, \& Winston,
New York NY, 1977).
\item [\em Physics:] R.~Resnick, D.~Halliday, and K.S.~Krane, Fourth Edition, Vol.~1
(John Wiley \& Sons, New York NY, 1992).
\item [\em Encyclop\ae dia Brittanica:] Fifteenth Edition (Encyclop\ae dia Brittanica, Chicago IL,
1994).
\item [\em Physics for Scientists and Engineers:] R.A.~Serway, and
R.J.~Beichner, Fifth Edition, Vol.~1 (Saunders College Publishing, Orlando FL, 2000).
\end{description}

\subsection{What is Classical Mechanics?}
{\em Classical mechanics} is the study of the {\em motion} of bodies (including the
special case in which bodies remain at rest) in accordance with the general
principles first enunciated by Sir Isaac Newton in his {\em Philosophiae
Naturalis Principia Mathematica} (1687), commonly known as the {\em Principia}. 
Classical mechanics was the first branch of Physics to be discovered, and is
the foundation upon which all other branches of Physics are built. 
Moreover, classical mechanics 
has many important applications in other areas of science,
such as Astronomy ({\em e.g.}, celestial mechanics), Chemistry
({\em e.g.}, the dynamics of molecular collisions), Geology ({\em e.g.}, the propagation
of seismic  waves, generated by earthquakes, through the Earth's crust), and Engineering
({\em e.g.}, the equilibrium and stability of structures). Classical
mechanics is also of great  significance outside the realm of science. After all, the sequence of
events leading  to the discovery of classical mechanics---starting with the ground-breaking
work of Copernicus, continuing with the researches of Galileo, Kepler, and Descartes,
and culminating in the monumental achievements of Newton---involved the complete overthrow
of the Aristotelian picture of the Universe, which had previously prevailed for more than a
millennium, and its replacement by a recognizably modern picture in which humankind
no longer played a privileged role.

In our investigation of classical mechanics we shall study many different types
of motion, including:
\begin{description}
\item {\bf\em Translational motion}---motion by which a body shifts from one point in space to
another ({\em e.g.}, the motion of a bullet fired from a gun).
\item {\bf\em Rotational motion}---motion by which an extended body changes orientation, with respect
to other bodies in space, without changing position ({\em e.g.}, the motion of a
spinning top).
\item {\bf\em Oscillatory motion}---motion which continually repeats in time with a fixed period
({\em e.g.}, the motion of a pendulum in a grandfather clock).
\item {\bf\em Circular motion}---motion by which a body executes a circular orbit about another
fixed body [{\em e.g.}, the (approximate) motion of the Earth about the Sun].
\end{description}
Of course, these different types of motion can be combined: for instance, the motion
of a properly bowled bowling ball  consists of a combination of translational and rotational
motion, whereas wave propagation is a combination of translational and oscillatory motion.
Furthermore, the above mentioned types of motion are not entirely
distinct: {\em e.g.}, circular motion contains elements of both rotational and oscillatory motion.
We shall also study {\em statics}: {\em i.e.}, the subdivision of mechanics
 which is concerned with the forces that
act on bodies {\em at rest} and  in equilibrium. Statics is obviously of great importance in civil
engineering: for instance, the principles of statics were used to design the building
in which this lecture is taking place, so as to ensure that it does not collapse.

\subsection{MKS Units}
The first principle of any exact science is {\em measurement}. In mechanics
there are three fundamental quantities which are subject to measurement: 
\begin{enumerate}
\item Intervals in space: {\em i.e.}, lengths.
\item Quantities of inertia, or mass, possessed by various bodies.
\item Intervals in time.
\end{enumerate}
Any other type of measurement in mechanics can be reduced to some combination of
measurements of these three quantities.

Each of the three fundamental quantities---{\em length}, {\em mass}, and {\em time}---is
measured with respect to some convenient standard. The system of units
currently used by all scientists, and most engineers, is called the {\em mks system}---after
the first initials of the names of the units of length, mass, and time, respectively, in this system:
{\em i.e.}, the {\em meter}, the {\em kilogram}, and the {\em second}.

The mks unit of length is the {\em meter} (symbol m), which was formerly the distance between
two scratches on a platinum-iridium alloy bar kept at the International Bureau of Metric
Standard in S\`{e}vres, France, but is now defined as the distance occupied by
$1,650,763.73$ wavelengths of light of the orange-red spectral line of the
isotope Krypton 86 in vacuum.

The mks unit of mass is the {\em kilogram} (symbol kg), which is defined as the mass
of a  platinum-iridium alloy cylinder  kept at the International Bureau of Metric
Standard in S\`{e}vres, France.

The mks unit of time is the {\em second} (symbol s), which was formerly defined in terms of the
Earth's rotation, but is now defined as the time for $9,192,631,770$ oscillations
associated
with the transition between the two hyperfine levels of the ground state of the
isotope Cesium 133.

In addition to the three fundamental quantities, classical mechanics 
also deals with {\em derived quantities},
such as velocity, acceleration, momentum, angular momentum, {\em etc.} Each of
these derived quantities can be reduced to some particular combination of 
length, mass, and time. The mks units of these derived quantities
are, therefore, the corresponding combinations of the  mks units of length, mass, and time. 
For instance, a velocity can be reduced to a length divided by a time. Hence,
the mks units of velocity are meters per second:
\begin{equation}
[v] = \frac{[L]}{[T]} = {\rm m\,s^{-1}}.
\end{equation}
Here, $v$ stands for a velocity, $L$ for a length, and $T$ for a time, whereas the
operator $[\cdots]$ represents the units, or {\em dimensions}, of the quantity
contained within the brackets. Momentum can be reduced to a mass
times a velocity. Hence, the mks units of momentum are kilogram-meters per second:
\begin{equation}
[p] = [M][v] = \frac{[M][L]}{[T]} = {\rm kg\,m\,s^{-1}}.
\end{equation}
Here, $p$ stands for a momentum, and $M$ for a mass. In this manner, the mks units
of all derived quantities appearing  in classical dynamics can easily be obtained.

\subsection{Standard Prefixes}
mks units are specifically designed to conveniently describe those motions which occur in everyday life.
Unfortunately, mks units tend to become rather unwieldy when dealing with motions on very
small scales ({\em e.g.}, the motions of molecules) or very large scales ({\em e.g.}, the
motion of stars in the Galaxy). In order to help cope with this problem, a set
of standard prefixes has been devised, which allow the mks units of length, mass, and time
to be modified so as to deal more
easily with very small and very large quantities: 
these prefixes are specified in Tab.~\ref{t1}. Thus, a {\em kilometer} (km) represents $10^3$\,m, 
a {\em nanometer}  (nm) represents
$10^{-9}$\,m, and a {\em femtosecond} (fs) represents $10^{-15}$\,s. The standard prefixes
can also be used to modify the units of derived quantities. 

\begin{table}
\centering
\begin{tabular}{lll|lll}\hline\hline
{\em Factor} & {\em Prefix} & {\em Symbol} & {\em Factor} & {\em Prefix} & {\em Symbol} \\
\hline
$10^{18}$ & exa- & E & $10^{-1}$ & deci- & d\\
$10^{15}$ & peta- & P & $10^{-2}$ & centi- & c\\
$10^{12}$ & tera- & T & $10^{-3}$ & milli- & m\\
$10^{9}$ & giga- & G & $10^{-6}$ & micro- & $\mu$\\
$10^{6}$ & mega- & M & $10^{-9}$ & nano- & n\\
$10^{3}$ & kilo- & k & $10^{-12}$ & pico- & p\\
$10^{2}$ & hecto- & h & $10^{-15}$ & femto- & f\\
$10^{1}$ & deka- & da & $10^{-18}$ & atto- & a\\
\hline\hline
\end{tabular}
\caption{\em Standard prefixes}\label{t1}
\end{table}

 \subsection{Other Units}
The mks system is not the only system of units in existence. Unfortunately, the obsolete
cgs (centimeter-gram-second) system and the even more obsolete fps (foot-pound-second) system are
still in use today, although their continued employment is now {\em strongly discouraged} in science and
engineering (except in the US!). Conversion between different systems of units is, in principle, perfectly
straightforward, but, in practice, a frequent source of error. Witness, for example,
the recent loss of the Mars Climate Orbiter because the engineers who designed its rocket
engine used fps units whereas the NASA mission controllers employed mks units. Table~\ref{t2}
specifies the various conversion factors between mks, cgs, and fps units.
Note that, rather confusingly (unless you are an engineer in the US!),
 a pound is a unit of force, rather than mass.
Additional non-standard units of length include the inch ($1\,{\rm ft} = 12\,{\rm in}$), the
yard ($1\,{\rm ya} = 3\,{\rm ft}$), and the mile ($1\,{\rm mi} = 5,280\,{\rm ft}$).
Additional non-standard units of mass include the ton (in the US, $1\,{\rm ton} = 2,000\,{\rm lb}$;
in the UK,  $1\,{\rm ton} = 2,240 \,{\rm lb}$), and the metric ton ($1\,{\rm tonne} = 1,000\,
{\rm kg}$). Finally, additional non-standard units of time include the minute ($1\,{\rm min}
= 60\,{\rm s}$), the hour ($1\,{\rm hr}
= 3,600\,{\rm s}$), the day ($1\,{\rm da} = 86,400\,{\rm s}$), and the year
($1\,{\rm yr} = 365.26\,{\rm da} = 31,558,464\,{\rm s}$). 

\begin{table}
\centering
\begin{tabular}{rcl}\hline\hline
1\,cm & $=$ & $10^{-2}$\,m\\
1\,g & $=$ & $10^{-3}$\,kg\\
1\,ft & $=$ & $0.3048$\,m\\
1\,lb & $=$ & $4.448$\,N\\
1\,slug & $=$ & $14.59$\,kg\\
\hline\hline
\end{tabular}
\caption{\em Conversion factors}\label{t2}
\end{table}

\subsection{Precision and Significant Figures}
In this course, you are expected to perform calculations to a relative accuracy
of 1\%: {\em i.e.}, to {\em three significant figures}. Since rounding errors tend
to accumulate during lengthy calculations, the easiest way in which to
achieve this accuracy is to perform all intermediate calculations to
{\em four significant figures}, and then to round the final result down to three significant figures.
If one of the quantities in your calculation turns out to the the small difference between
two much larger numbers, then you may need to keep {\em  more} than four significant figures.
Incidentally, you are strongly urged to use {\em scientific notation} in {\em all} of your
calculations: the use of non-scientific notation is generally a major source of error
in this course.
If your calculators are capable of operating in a mode in which {\em all} numbers 
(not just very small or very large numbers)  are
displayed in scientific form  then you are 
advised to
 perform your  calculations in this mode. 

\subsection{Dimensional Analysis}
As we have already mentioned, length, mass, and time are three {\em fundamentally different} quantities
which are measured in three completely independent units. It, therefore, makes no sense for a
prospective law of physics to express an equality between (say) a length and a mass. In other
words, the example law
\begin{equation}\label{e2}
m = l,
\end{equation}
where $m$ is a mass and $l$ is a length, cannot possibly be correct. One easy way of
seeing that Eq.~(\ref{e2}) is invalid (as a law of physics), is to note that this equation
is dependent on the adopted system of units: {\em i.e.}, if $m=l$ in mks units, then
$m\neq l$ in fps units, because the conversion factors which must be
applied to the left- and right-hand
sides differ. Physicists hold very strongly to the assumption that the laws of physics
possess {\em objective reality}: in other words,  the laws of physics are the same for
all  observers. One immediate consequence of  this assumption is that a law
of physics must take the same form in all possible systems of units that a prospective
observer might choose to employ. The only way in which this can be the case is if
all laws of physics are {\em dimensionally consistent}: {\em i.e.}, the quantities
on the left- and right-hand sides of the equality sign in any given law of
physics must have the same dimensions ({\em i.e.},
the same combinations of length, mass, and time). A dimensionally consistent equation
naturally takes the same form in all possible systems of units, since the same conversion
factors are applied to both sides of the equation when transforming from one system  to another.

As an example, let us consider what is probably the most famous equation in physics:
\begin{equation}\label{e3}
E = m\,c^2.
\end{equation}
Here, $E$ is the energy of a body, $m$ is its mass, and $c$ is the velocity of light in
vacuum. The dimensions of energy are $[M][L^2]/[T^2]$, and the dimensions of
velocity are $[L]/[T]$. Hence, the dimensions of the left-hand side are
$[M][L^2]/[T^2]$, whereas the dimensions of the right-hand side are 
$[M]\,([L]/[T])^2= [M][L^2]/[T^2]$. It follows that Eq.~(\ref{e3}) is indeed dimensionally
consistent. Thus, $E=m\,c^2$ holds good in mks units, in cgs units, in  fps units, and in 
any other sensible
set of units. Had Einstein proposed $E=m\,c$, or $E=m\,c^3$, then his error would
 have been immediately apparent to other physicists, since these prospective laws
are not dimensionally consistent. In fact, $E=m\,c^2$ represents the {\em only}\/ simple, dimensionally
consistent way of combining an energy, a mass, and the velocity of light in a law of
physics.

The last comment leads naturally to the subject of {\em dimensional analysis}: {\em i.e.,}
the use of the idea of dimensional consistency to {\em guess} the forms of simple laws of physics.
It should be noted that dimensional analysis is of fairly limited applicability, and
is a poor substitute for analysis employing the actual laws of physics; nevertheless,
it is occasionally useful. Suppose that a special effects studio wants to film a scene
in which the Leaning Tower of Pisa topples to the ground. In order to achieve this, the studio
might make a scale model of the tower, which is (say) 1\,m tall, and then film the model falling over.
The only problem is that the resulting footage would look completely unrealistic, because the model
tower would fall over too quickly. The studio could easily fix this problem by slowing the
film down. The question is by what factor should the film be slowed down in order to make it
look realistic?

\begin{figure}
\epsfysize=3in
\centerline{\epsffile{Chapter01/fig001.eps}}
\caption{\em The Leaning Tower of Pisa}\label{f1}   
\end{figure}

Although, at this stage, we do not know how to apply the laws of physics to the
problem of a tower falling over, we can, at least,  make some educated guesses as to what factors
the time $t_f$ required for this process to occur depends on. In fact, it
seems reasonable to suppose that $t_f$ depends principally on the mass of the tower, $m$, the
height of the tower, $h$, and the acceleration due to gravity, $g$. See Fig.~\ref{f1}. In other words,
\begin{equation}\label{e4}
t_f = C\,m^x\,h^y\,g^z,
\end{equation}
where $C$ is a dimensionless constant, and $x$, $y$, and $z$ are unknown exponents. The
exponents $x$, $y$, and $z$
can be determined by the requirement that the above equation be dimensionally
consistent. Incidentally, the dimensions of an acceleration are $[L]/[T^2]$. Hence,
equating the dimensions of both sides of Eq.~(\ref{e4}), we obtain
\begin{equation}
[T] = [M]^x\,[L]^y\,\left(\frac{[L]}{[T^2]}\right)^z.
\end{equation}
We can now compare the exponents of $[L]$, $[M]$, and $[T]$ on either side of the
above expression: these exponents must all match in order for Eq.~(\ref{e4}) to be dimensionally
consistent. Thus,
\begin{eqnarray}
0&=& y + z,\\
0 &=& x,\\
1 &=& -2\,z.
\end{eqnarray}
It immediately follows that $x=0$, $y=1/2$, and $z=-1/2$. Hence,
\begin{equation}
t_f = C\,\,\sqrt{\frac{h}{g}}.
\end{equation}
Now, the actual tower of Pisa is approximately 100\,m tall. It follows that since
$t_f\propto \sqrt{h}$ ($g$ is the same for both the real  and the model tower) then the
1\,m high model tower falls over a factor of $\sqrt{100/1}=10$ times faster than the real
tower. Thus, the film must be slowed down by a factor 10 in order to make it look realistic. 

\subsection*{\em Worked Example 1.1: Conversion of Units}
{\em Question:} Farmer Jones has recently brought a 40 acre field and wishes
to replace the fence surrounding it. Given that the field is square, what
 length of fencing (in meters) should Farmer Jones purchase? Incidentally,
1 acre equals  43,560 square feet.\\
~\\
{\em Answer:} If 1~acre equals 43,560 ${\rm ft}^2$ and 1~ft equals $0.3048\,{\rm m}$
(see Tab.~\ref{t2}) then
$$
1~{\rm acre} = 43560 \times (0.3048)^2 = 4.047\times 10^3\,{\rm m}^2.
$$
Thus, the area of the field in mks units is
$$
A = 40\times 4.047\times 10^3 = 1.619\times 10^5\,{\rm m}^2.
$$
Now, a square field with sides of length $l$ has an area $A=l^2$ and a circumference $D=4l$.
Hence, $D=4\sqrt{A}$. It follows that the length of the fence is
$$
D = 4\times \sqrt{1.619\times 10^5} = 1.609\times 10^3 \,{\rm m}.
$$

\subsection*{\em Worked Example 1.2: Tire Pressure}
{\em Question:} The recommended tire pressure in a Honda Civic is 28 psi (pounds per
square inch). What is this pressure in atmospheres (1 atmosphere is
$10^5\,{\rm N}\,{\rm m}^{-2}$)?\\
~\\
{\em Answer:} First, 28 pounds per
square inch is the same as $28\times (12)^2 = 4032$ pounds per square
foot (the standard fps unit of pressure).
Now, 1 pound equals $4.448$ Newtons (the standard SI unit of force), and
1 foot equals $0.3048$\,m (see Tab.~\ref{t2}). Hence,
$$
P = 4032\times (4.448) / (0.3048)^2 = 1.93\times 10^5\,{\rm N}{\rm m}^{-2}.
$$
It follows that 28 psi is equivalent to $1.93$ atmospheres.


\subsection*{\em Worked Example 1.3: Dimensional Analysis}
{\em Question:} The speed of sound $v$  in a gas might plausibly depend on
the pressure $p$, the density $\rho$, and the volume $V$ of the gas. Use
dimensional analysis to determine the exponents $x$, $y$, and $z$ in the
formula
$$
v = C\,p^x\,\rho^y\,V^z,
$$
where $C$ is a dimensionless constant. Incidentally, the mks units
of pressure are kilograms per meter per second squared. \\
~\\
{\em Answer:} Equating the dimensions of both sides of the above equation, we
obtain
$$
\frac{[L]}{[T]} = \left(\frac{[M]}{[T^2][L]}\right)^x\left(
\frac{[M]}{[L^3]}\right)^y [L^3]^z.
$$
A comparison of the exponents of $[L]$, $[M]$, and $[T]$ on either side of the
above expression yields
\begin{eqnarray}
1 &=& -x -3 y+ 3z,\nonumber\\
0 &=& x + y,\nonumber\\
-1&=& -2 x.\nonumber
\end{eqnarray}
The third equation immediately gives $x=1/2$; the second equation then yields $y=-1/2$; finally,
the first equation gives $z=0$. Hence,
$$
v = C \sqrt{\frac{p}{\rho}}.
$$
