\section{Motion in One Dimension}
\subsection{Introduction}
The purpose of this section is to introduce the concepts of {\em displacement},
{\em velocity}, and {\em acceleration}. For the sake of simplicity, we shall
restrict our attention to 1-dimensional motion.

\subsection{Displacement}
Consider a body moving in 1 dimension: {\em e.g.}, a train traveling down a straight
railroad track, or a truck driving down an interstate in Kansas. Suppose that we have
a team of observers who continually report the location of this body to us as time
progresses. To be more exact, our observers report the distance $x$ of the body
  from
some arbitrarily chosen reference point located on the track on which it is constrained to move. This
point is known as the {\em origin} of our coordinate system. A {\em positive} $x$ value implies
that the body is located $x$ meters to the {\em right} of the origin, whereas a {\em negative}
$x$ value implies that the body is located $|x|$ meters to the {\em left} of the origin.
Here, $x$ is termed the {\em displacement} of the body from the origin. See Fig.~\ref{f2}.
Of course, if the body is extended then our observers will have to report the displacement
$x$ of some conveniently chosen reference point on the body ({\em e.g.}, its centre
of mass) from the origin.

Our information regarding the body's motion  consists of a set of data points,
each specifying the displacement $x$ of the body at some time $t$. It is
usually illuminating to graph these points. Figure~\ref{f3} shows an example of such a graph. 
As is often the case, it is possible to fit the data points appearing in this
graph using a relatively
simple analytic curve. Indeed, the curve associated with Fig.~\ref{f3} is
\begin{equation}\label{e21}
x = 1 + t + \frac{t^2}{2} - \frac{t^4}{4}.
\end{equation}

\begin{figure}[b]
\epsfysize=1in
\centerline{\epsffile{Chapter02/fig002.eps}}
\caption{\em Motion in 1 dimension}\label{f2}   
\end{figure}

\begin{figure}[t]
\epsfysize=3in
\centerline{\epsffile{Chapter02/fig003.eps}}
\caption{\em Graph of displacement versus time}\label{f3}   
\end{figure}

\subsection{Velocity}
Both Fig.~\ref{f3} and formula (\ref{e21}) effectively specify the location
of the body whose motion we are studying as time progresses.
Let us now consider how we can use this information to determine the body's
instantaneous {\em velocity} as a function of time. The conventional
definition of velocity is as follows:
\begin{quote}
{\sf Velocity is the rate of change of displacement with time.}
\end{quote}
This definition implies that
\begin{equation}\label{e22}
v = \frac{{\mit\Delta}x}{{\mit\Delta} t},
\end{equation}
where $v$ is the body's velocity at time $t$, and ${\mit\Delta} x$ is the change in displacement
of the body between times $t$ and $t+{\mit\Delta} t$. 

How should we choose the
time interval ${\mit\Delta} t$ appearing in Eq.~(\ref{e22})?
Obviously,  in the simple case in which the
body is moving with {\em constant} velocity,  we can make ${\mit\Delta} t$ as
large or small as we like, and it will not affect the value of $v$. Suppose, however,
that $v$ is constantly changing in time, as is generally the case. 
In this situation, ${\mit\Delta} t$ must be kept sufficiently small that the body's velocity
does not change appreciably between times $t$ and $t+{\mit\Delta} t$. If
${\mit\Delta} t$ is made too large then formula (\ref{e22}) becomes invalid. 

Suppose that we require a {\em general}  expression for instantaneous velocity which
is valid irrespective of how rapidly or slowly the body's velocity changes in time. 
We can achieve this goal by taking the limit of Eq.~(\ref{e22}) as ${\mit\Delta} t$ approaches
zero. This ensures that no matter how rapidly $v$ varies with time, the velocity
of the body is always approximately constant in the interval $t$ to $t+{\mit\Delta} t$. 
Thus,
\begin{equation}\label{e23}
v = \lim_{{\mit\Delta}t\rightarrow 0}\frac{{\mit\Delta}x}{{\mit\Delta} t}=
\frac{dx}{dt},
\end{equation}
where $dx/dt$ represents the {\em derivative} of $x$ with respect to $t$.  The above definition 
is particularly useful if we can represent $x(t)$ as an
analytic function, because it allows us to  immediately evaluate the instantaneous velocity
$v(t)$ via the rules of calculus. Thus, if $x(t)$ is given by formula (\ref{e21})
then
\begin{equation}
v = \frac{dx}{dt} = 1 + t - t^3.
\end{equation}
Figure~\ref{f4} shows the graph of $v$ versus $t$ obtained from  the above expression.
Note that when $v$ is positive  the body is moving to the right ({\em i.e.}, $x$ is
increasing in time). Likewise, when $v$ is negative  the body is moving to the
left ({\em i.e.}, $x$ is
decreasing  in time). Finally, when $v=0$ the body is instantaneously at rest.

\begin{figure}
\epsfysize=3in
\centerline{\epsffile{Chapter02/fig004.eps}}
\caption{\em Graph of instantaneous velocity versus time associated with the motion
specified in Fig.~\ref{f3}}\label{f4}   
\end{figure}

The terms velocity and speed are often confused with one another.
A velocity can be either  positive or negative, depending on the 
direction of motion. The conventional definition of {\em speed} is that
it is the magnitude of velocity ({\em i.e.}, it is $v$ with the sign stripped
off). It follows that a body can never possess a negative speed.

\subsection{Acceleration}
The conventional
definition of acceleration is as follows:
\begin{quote}
{\sf Acceleration is the rate of change of velocity with time.}
\end{quote}
This definition implies that
\begin{equation}\label{e25}
a = \frac{{\mit\Delta}v}{{\mit\Delta} t},
\end{equation}
where $a$ is the body's acceleration at time $t$, and ${\mit\Delta} v$ is the change in velocity
of the body between times $t$ and $t+{\mit\Delta} t$. 

How   should we choose the
time interval ${\mit\Delta} t$ appearing in Eq.~(\ref{e25})? Again, in the
simple case in which  the
body is moving with {\em constant} acceleration,  we can make ${\mit\Delta} t$ as
large or small as we like, and it will not affect the value of $a$. Suppose, however,
that $a$ is constantly changing in time, as is generally the case. 
In this situation, ${\mit\Delta} t$ must be kept sufficiently small that the body's acceleration
does not change appreciably between times $t$ and $t+{\mit\Delta} t$. 

A general  expression for instantaneous acceleration,  which
is valid irrespective of how rapidly or slowly the body's acceleration changes in time, 
can be obtained by taking the limit of Eq.~(\ref{e25}) as ${\mit\Delta} t$ approaches
zero:
\begin{equation}\label{e26}
a = \lim_{{\mit\Delta}t\rightarrow 0}\frac{{\mit\Delta}v}{{\mit\Delta} t}=
\frac{dv}{dt}=\frac{d^2 x}{dt^2}.
\end{equation}
  The above definition is particularly useful if we can represent $x(t)$ as an
analytic function, because it allows us to immediately evaluate the instantaneous acceleration
$a(t)$  via the rules of calculus. Thus, if $x(t)$ is given by formula (\ref{e21})
then
\begin{equation}
a = \frac{d^2 x}{dt^2} = 1 - 3 t^2.
\end{equation}
Figure~\ref{f5} shows the graph of $a$ versus time obtained from the above expression.
Note that when $a$ is positive  the body is {\em accelerating} to the right  ({\em i.e.}, $v$ is
increasing in time). Likewise, when $a$ is negative  the body is {\em decelerating}
  ({\em i.e.}, $v$ is
decreasing  in time). 

\begin{figure}[t]
\epsfysize=3in
\centerline{\epsffile{Chapter02/fig005.eps}}
\caption{\em Graph of instantaneous acceleration versus time associated with the  motion
specified in Fig.~\ref{f3}}\label{f5}   
\end{figure}

Fortunately, it is generally not necessary to evaluate the rate of change of acceleration with time,
since this quantity does not appear in Newton's laws of motion.

\subsection{Motion with Constant Velocity}
The simplest type of motion (excluding the trivial case in which the body under
investigation remains at rest) consists of motion with {\em constant velocity}.
This type of motion occurs in everyday life whenever an object slides over a horizontal,
low friction surface: {\em e.g.}, a puck sliding across a hockey rink.
 
Fig.~\ref{f6} shows the graph of displacement versus time for a body
moving with constant velocity. It can be seen that the graph consists of
a {\em straight-line}. This line can be represented algebraically as
\begin{equation}
x = x_0 + v\,t.
\end{equation}
Here, $x_0$ is the displacement at time $t=0$: this quantity can be  determined from the
graph as the {\em intercept} of the
straight-line with the $x$-axis. Likewise, $v=dx/dt$ is the constant velocity of
the body: this quantity can be determined from the graph as the {\em gradient} of the straight-line
({\em i.e.}, the ratio ${\mit\Delta}x/{\mit\Delta} t$, as shown). Note that
$a=d^2 x/dt^2=0$, as expected.

\begin{figure}
\epsfysize=3in
\centerline{\epsffile{Chapter02/fig006.eps}}
\caption{\em Graph of displacement versus time for a body moving with constant velocity}\label{f6}   
\end{figure}

Fig.~\ref{f7} shows a  displacement versus time graph for a slightly  more
complicated case of motion with constant velocity. The body in question moves
to the right (since $x$ is clearly increasing with $t$) with a constant velocity (since
the graph is a straight-line) between times $A$ and $B$. The body then moves to
the right (since $x$ is still increasing in time) with a somewhat larger constant velocity
(since the graph is again a straight line, but possesses a larger gradient than before) between times
$B$ and $C$. The body remains at rest (since the graph is horizontal) between times $C$ and $D$.
Finally, the body moves to the left (since $x$ is decreasing with $t$) with a constant
velocity (since
the graph is a straight-line) between times $D$ and $E$.

\begin{figure}
\epsfysize=3in
\centerline{\epsffile{Chapter02/fig007.eps}}
\caption{\em Graph of displacement versus time}\label{f7}   
\end{figure}

\subsection{Motion with Constant Acceleration}\label{caccn}
Motion with constant acceleration occurs in everyday life whenever an
object is dropped: the object moves downward with the constant
acceleration $9.81\,{\rm m\,s^{-2}}$, under the influence of gravity.

Fig.~\ref{f8} shows the graphs of displacement versus time and
velocity versus time for a body
moving with constant acceleration. It can be seen that the displacement-time graph consists of
a {\em curved-line} whose gradient (slope) is increasing in time. 
This line can be represented algebraically as
\begin{equation}\label{e2a}
x = x_0 + v_0\,t + \frac{1}{2}\,a\,t^2.
\end{equation}
Here, $x_0$ is the displacement at time $t=0$: this quantity can be  determined from the
graph as the {\em intercept} of the
curved-line with the $x$-axis. Likewise, $v_0$ is the body's instantaneous
velocity at time $t=0$. 

\begin{figure}
\epsfysize=5.in
\centerline{\epsffile{Chapter02/fig008.eps}}
\caption{\em Graphs of displacement versus time and velocity versus time
for a body moving with constant acceleration}\label{f8}   
\end{figure}

The velocity-time graph consists of a {\em straight-line} which can be represented
algebraically as
\begin{equation}\label{e2b}
v = \frac{dx}{dt}= v_0 + a\,t.
\end{equation}
The quantity $v_0$ is determined from the graph as the  {\em intercept} of the
straight-line with the $x$-axis. The quantity $a$ is the constant acceleration: this 
 can be determined  graphically as the {\em gradient} of the straight-line
({\em i.e.}, the ratio ${\mit\Delta}v/{\mit\Delta} t$, as shown). Note that
$dv/dt=a$, as expected.

Equations (\ref{e2a}) and (\ref{e2b}) can be rearranged to give the
following set of three useful formulae which characterize motion with
constant acceleration:
\begin{eqnarray}
s &=& v_0\,t + \frac{1}{2} \,a\,t^2,\label{e211}\\
v  &=& v_0 + a\,t,\\
v^2 &=& v_0^{\,2} + 2\,a\,s.\label{e214}
\end{eqnarray}
Here, $s=x-x_0$ is the net distance traveled after $t$ seconds.

Fig.~\ref{f9} shows a  displacement versus time graph for a slightly  more
complicated case of accelerated motion. The body in question accelerates
to the right [since the gradient (slope) of the graph is increasing in time] 
 between times $A$ and $B$. The body then moves to
the right (since $x$ is  increasing in time) with a  constant velocity
(since the graph is a straight line) between times
$B$ and $C$. 
Finally, the body decelerates [since the gradient (slope) of the graph is decreasing in time]
 between times $C$ and $D$.

\begin{figure}
\epsfysize=3in
\centerline{\epsffile{Chapter02/fig009.eps}}
\caption{\em Graph of displacement versus time}\label{f9}   
\end{figure}

\subsection{Free-Fall Under Gravity}
Galileo Galilei was the first scientist to appreciate that, {\em neglecting the
effect of air resistance},  all bodies in free-fall close to the Earth's
surface accelerate vertically downwards with the same acceleration: namely,
$g=9.81\,{\rm m\,s^{-2}}$.\footnote{Actually, the acceleration due to gravity varies
slightly over the Earth's surface because of the combined effects of the Earth's rotation and
the Earth's slightly flattened shape. The acceleration at the poles is about
$9.834\,{\rm m\,s^{-2}}$, whereas the acceleration at the equator is
only $9.780\,{\rm m\,s^{-2}}$. The {\em average} acceleration is $9.81\,{\rm m\,s^{-2}}$.}
The neglect of air resistance is a fairly good approximation for large objects
which travel relatively slowly ({\em e.g.}, a shot-putt, or a basketball), but
becomes a poor approximation for small objects which travel relatively rapidly
({\em e.g.}, a golf-ball, or a bullet fired from a pistol).

Equations~(\ref{e211})--(\ref{e214}) can easily be modified to deal with the
special case of an object free-falling under gravity:
\begin{eqnarray}
s &=& v_0\,t -\frac{1}{2} \,g\,t^2,\label{e215}\\
v  &=& v_0 - g\,t,\label{e217}\\
v^2 &=& v_0^{\,2} - 2\,g\,s.\label{e218}
\end{eqnarray}
Here, $g=9.81\,{\rm m\,s^{-2}}$ is the downward acceleration due to gravity, $s$ is the
distance the object has moved vertically between times  $t=0$ and $t$ (if $s>0$ then the object has
risen $s$ meters, else if $s<0$ then the object has
fallen $|s|$ meters), and  $v_0$ is the object's instantaneous velocity at $t=0$. Finally,
$v$ is the object's instantaneous velocity at time $t$. 

Let us illustrate the use of Eqs.~(\ref{e215})--(\ref{e218}). Suppose that
a ball is released from rest and allowed to fall under the influence of gravity.
How long does it take the ball to fall $h$ meters? Well, according to Eq.~(\ref{e215})
[with $v_0=0$ (since the ball is released from rest), and $s=-h$ (since we wish the
ball to {\em fall} $h$ meters)], $h= g\,t^2/2$, so the time of fall is
\begin{equation}\label{e2.19}
t = \sqrt{\frac{2\,h}{g}}.
\end{equation}

Suppose that a ball is thrown vertically upwards from ground level with velocity $u$. 
To what height does the ball rise, 
how
long does it remain in the air, and with what velocity does it strike the
ground? The ball attains its maximum height when it is momentarily at rest
({\em i.e.}, when $v=0$). According to Eq.~(\ref{e217}) (with $v_0=u$), 
this occurs at time $t=u/g$. It follows from Eq.~(\ref{e215}) (with
$v_0=u$,  and $t=u/g$) that the maximum height of the ball
is given by
\begin{equation}
h = \frac{u^2}{2\,g}.
\end{equation} 
When the ball strikes the ground it has traveled zero net meters vertically, so $s=0$.
It follows from   Eqs.~(\ref{e217}) and (\ref{e218}) (with $v_0=u$ and $t>0$) that $v=-u$. 
In other words, the ball hits the ground with an equal and opposite velocity to that
with
which it was thrown into the air. Since the ascent and decent phases of the ball's
trajectory are clearly symmetric, the ball's time of flight is
simply twice the time required for the ball to attain its maximum height:
{\em i.e.},
\begin{equation}
t = \frac{2\,u}{g}.
\end{equation}

\subsection*{\em Worked Example 2.1: Velocity-Time Graph}
\begin{figure*}[h]
\epsfysize=3in
\centerline{\epsffile{Chapter02/fig009a.eps}}
\end{figure*}
{\em Question:} Consider the motion of the object whose velocity-time
graph is given in the diagram.
\begin{enumerate}
\item What is the acceleration of the object between times $t=0$ and $t=2$?
\item What is the acceleration of the object between times $t=10$ and $t=12$?
\item What is the net displacement of the object between times $t=0$ and
$t=16$?
\end{enumerate}
~\\
{\em Answer:}
\begin{enumerate}
\item The $v$-$t$ graph is a straight-line between $t=0$ and $t=2$, indicating 
constant acceleration during this time period. Hence,
$$
a = \frac{{\mit\Delta}v}{{\mit\Delta} t} = \frac{v(t=2)-v(t=0)}{2-0} = 
\frac{8-0}{2} = 4\,{\rm m\,s^{-2}}.
$$
\item The $v$-$t$ graph is a straight-line between $t=10$ and $t=12$, indicating 
constant acceleration during this time period. Hence,
$$
a = \frac{{\mit\Delta}v}{{\mit\Delta} t} = \frac{v(t=12)-v(t=10)}{12-10} = 
\frac{4-8}{2} = -2\,{\rm m\,s^{-2}}.
$$
The negative sign indicates that the object is decelerating.
\item Now, $v=dx/dt$, so
$$
x(16)-x(0) = \int_0^{16} v(t)\,dt.
$$
In other words, the net displacement between times $t=0$ and $t=16$ equals
the area under the $v$-$t$ curve, evaluated  between these two times. Recalling that the
area of a triangle is half its width times its height, the number of grid-squares
under the $v$-$t$ curve is 25. The area of each grid-square is $2\times 2=4\,{\rm m}$.
Hence,
$$
x(16)-x(0) = 4\times 25 = 100\,{\rm m}.
$$
\end{enumerate}

\subsection*{\em Worked Example 2.2: Speed Trap}
{\em Question:} In a speed trap, two pressure-activated strips are placed $120\,{\rm m}$
apart on a highway on which the speed limit is $85\,{\rm km/h}$. A driver going $110\,{\rm km/h}$
notices a police car just as he/she  activates the first strip, and immediately slows down. What
deceleration is needed so that the car's average speed is within the speed limit
when the car crosses the second strip?\\
~\\
{\em Answer:} Let $v_1= 110\,{\rm km/h}$ be the speed of the car at the first strip.
 Let ${\mit\Delta} x = 120\,{\rm m}$ be the
distance between the two strips, and let ${\mit\Delta} t$ be the time taken by the
car to travel from one strip to the other. The average velocity of the car is
$$
\bar{v} = \frac{{\mit\Delta}x}{{\mit\Delta} t}.
$$
We need this velocity to be $85\,{\rm km/h}$. Hence, we require
$$
{\mit\Delta}t = \frac{{\mit\Delta}x}{\bar{v}} = \frac{120}{85\times (1000/3600)} = 5.082\,{\rm s}.
$$ 
Here, we have changed units from ${\rm km/h}$ to ${\rm m/s}$. Now,
assuming that the acceleration $a$ of the car is uniform, we have
$$
{\mit\Delta} x = v_1\,{\mit\Delta} t + \frac{1}{2}\,a\,({\mit\Delta t})^2,
$$
which can be rearranged to give
$$
a = \frac{2\,({\mit\Delta}x - v_1\,{\mit\Delta} t)}{({\mit\Delta}t)^2} =
\frac{ 2\,(120 - 110\times (1000/3600)\times 5.082)}{(5.082)^2} = -2.73\,
{\rm m\,s^{-2}}.
$$
Hence, the required deceleration is $2.73\,
{\rm m\,s^{-2}}$.

\subsection*{\em Worked Example 2.3: The Brooklyn Bridge}
{\em Question:} In 1886, Steve Brodie achieved notoriety by allegedly jumping off
the recently completed Brooklyn bridge, for a bet,  and surviving. Given
that the bridge rises 135\,ft over the East River, how long would Mr.~Brodie have
been in the air, and with what speed would he have struck the water? Give
all answers in mks units. You may neglect air resistance.\\
~\\
{\em Answer:} Mr.~Brodie's net vertical
displacement was $h = -135\times 0.3048 = -41.15\, {\rm m}$. 
Assuming that his initial velocity was zero, 
$$
h = -\frac{1}{2}\,g\,t^2,
$$
where $t$ was his time of flight. Hence,
$$
t = \sqrt{\frac{-2\,h}{g}} = \sqrt{\frac{2\times 41.15}{9.81}} = 2.896\,{\rm s}.
$$
His final velocity was
$$
v =- g\,t = -9.81\times 2.896 = -28.41\,{\rm m\,s}^{-1}.
$$
Thus, the speed with which he plunged into the East River was
$28.41\,{\rm m\,s}^{-1}$, or $63.6\,{\rm mi/h}$\,! Clearly,
Mr.~Brodie's story should be taken with a pinch of salt. 
