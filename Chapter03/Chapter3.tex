\section{Motion in Three Dimensions}

\subsection{Introduction}
The purpose of this section is to generalize the previously introduced concepts of {\em displacement},
{\em velocity}, and {\em acceleration} in order to deal with motion in 3
dimensions.

\subsection{Cartesian Coordinates}
Our first task, when dealing with 3-dimensional motion, is to set up a suitable
coordinate system. The most straight-forward type of coordinate system is called a {\em Cartesian
system}, after Ren\'{e} 
Descartes. A Cartesian coordinate system consists of three mutually perpendicular axes,
the $x$-, $y$-, and $z$-axes (say). By convention, the orientation of these
axes is such that when the index finger, the middle finger, and the thumb of the
right-hand are configured so as to be mutually perpendicular,  the index finger, the
middle finger, and the thumb can be aligned along the $x$-, $y$-, and $z$-axes, respectively.
Such a coordinate system is termed {\em right-handed}. See Fig.~\ref{f10}. The
point of intersection of the three coordinate axes is termed the {\em origin}
of the coordinate system.

\begin{figure}[b]
\epsfysize=1.4in
\centerline{\epsffile{Chapter03/fig010.eps}}
\caption{\em A right-handed Cartesian coordinate system}\label{f10}   
\end{figure}

\subsection{Vector Displacement}
Consider the motion of a body moving in 3 dimensions. The 
body's instantaneous
position is most conveniently specified by giving its displacement from the origin of our
coordinate system. Note, however, that in 3 dimensions such a displacement possesses both
 {\em magnitude} and  {\em direction}. In other words, we not only have to
 specify how far the body is situated from the origin, we also
have to specify in which direction it lies. A quantity which possesses both
magnitude and direction is termed a {\em vector}. By contrast, a quantity
which possesses only magnitude is termed a {\em scalar}. Mass and time are scalar
quantities. However, in general, displacement is a vector. 

The vector displacement ${\bf r}$ of some point $R$ from the origin $O$
can be visualized as an arrow running from point $O$ to point $R$. See Fig.~\ref{f11}.
Note that in typeset documents vector quantities are conventionally written in a {\em bold-faced} font
({\em e.g.}, ${\bf r}$) to
distinguish them from scalar quantities. In free-hand notation, vectors are usually
{\em under-lined} ({\em e.g.}, $\underline{r}$). 

\begin{figure}
\epsfysize=2in
\centerline{\epsffile{Chapter03/fig011.eps}}
\caption{\em A vector displacement}\label{f11}   
\end{figure}

The vector displacement ${\bf r}$ can also be specified in terms of its {\em coordinates}:
\begin{equation}
{\bf r} = (x, y, z).
\end{equation}
The above expression is interpreted as follows: in order to get from point $O$ to
point $R$, first move $x$ meters along the $x$-axis (perpendicular to both the $y$- and $z$-axes),
then move $y$ meters along the $y$-axis (perpendicular to both the $x$- and $z$-axes),
finally move $z$ meters along the $z$-axis (perpendicular to both the $x$- and $y$-axes).
Note that a {\em positive}  $x$ value is interpreted as an instruction to move $x$ meters
along the $x$-axis in the direction of {\em increasing} $x$, whereas
 a {\em negative} $x$ value is interpreted as an 
instruction to move $|x|$ meters
 along the $x$-axis in the opposite direction, and so on.

\subsection{Vector Addition}
Suppose that the vector displacement 
 ${\bf r}$ of some point $R$ from the origin $O$ is specified as follows:
\begin{equation}
{\bf r} = {\bf r}_1 + {\bf r}_2.
\end{equation}
Figure~\ref{f12} illustrates how this expression is interpreted diagrammatically: in order to
get from point $O$ to point $R$, we first move from point $O$ to point $S$ along vector
${\bf r}_1$, and we then move from point $S$ to point $R$ along vector ${\bf r}_2$. The
net result is the same as if we had moved from point $O$ directly to point $R$ along
vector ${\bf r}$. Vector ${\bf r}$ is termed the {\em resultant} of adding vectors
${\bf r}_1$ and ${\bf r}_2$. 

\begin{figure}
\epsfysize=2in
\centerline{\epsffile{Chapter03/fig012.eps}}
\caption{\em Vector addition}\label{f12}   
\end{figure}

Note that we have two ways of specifying the vector displacement of point $S$ from
the origin: we can either write ${\bf r}_1$ or ${\bf r} - {\bf r}_2$. The
expression ${\bf r} - {\bf r}_2$ is interpreted as follows: starting at the origin,
move along vector ${\bf r}$ in the direction of the arrow, then move along
vector ${\bf r}_2$ in the {\em opposite} direction to the arrow. In other words,
a {\em minus} sign in front of a vector indicates that we should move along that vector in
the {\em opposite} direction to its arrow. 

Suppose that the components of vectors ${\bf r}_1$ and ${\bf r}_2$ are
$(x_1, y_1, z_1)$ and $(x_2, y_2, z_2)$, respectively. As is easily demonstrated,
the components $(x,y,z)$ of the
resultant vector ${\bf r}= {\bf r}_1 + {\bf r}_2$ are 
\begin{eqnarray}
x &=& x_1+x_2,\\
y&=& y_1 + y_2,\\
z &=& z_1 + z_2.
\end{eqnarray}
In other words, the components of the sum of two vectors are simply the algebraic
sums of the components of the individual vectors. 

\subsection{Vector Magnitude}
If ${\bf r}=(x,y,z)$ represents  the vector displacement of point $R$ from the origin,
what is the distance between these two points? In other words, what is the length,
or {\em magnitude}, $r=|{\bf r}|$,
of vector ${\bf r}$. It follows from a 3-dimensional generalization  of
Pythagoras' theorem that
\begin{equation}\label{e36}
r = \sqrt{x^2+ y^2 + z^2}.
\end{equation}

Note that if ${\bf r} = {\bf r}_1 + {\bf r}_2$ then
\begin{equation}\label{teq}
|{\bf r}| \leq |{\bf r}_1| + 
|{\bf r}_2|.
\end{equation}
In other words, the magnitudes of  vectors cannot, in general, be added
algebraically. The only exception to this rule (represented by the equality sign in the
above expression) occurs when  the vectors in question all point in the same direction. 
According to inequality (\ref{teq}), if we move 1\,m to the North (say) and next move
1\,m to the West (say) then, although we have moved a total distance of 2\,m, our net distance
from the starting point is less than 2\,m---of course, this is just common sense.

\subsection{Scalar Multiplication}
Suppose that ${\bf s} = \lambda\,{\bf r}$. This expression is interpreted as
follows: vector ${\bf s}$ points in the {\em same} direction as vector ${\bf r}$, but
the length of the former vector is $\lambda$ times that of the latter. Note
that if $\lambda$ is negative then vector ${\bf s}$ points in the {\em opposite} direction to
 vector ${\bf r}$, and
the length of the former vector is $|\lambda|$ times that of the latter. In terms
of components:
\begin{equation}
{\bf s} = \lambda\,(x,y,z) = (\lambda\,x, \lambda\,y, \lambda\,z).
\end{equation}
In other words, when we multiply a vector by a scalar then  the components of the resultant
vector are obtained 
by multiplying {\em all} the components of the original vector by the scalar.

\subsection{Diagonals of a Parallelogram}
The use of vectors is very well illustrated by the following rather famous proof
that the diagonals of a parallelogram mutually bisect one another.

\begin{figure}
\epsfysize=1.5in
\centerline{\epsffile{Chapter03/fig013.eps}}
\caption{\em A parallelogram}\label{f13}   
\end{figure}

Suppose that the quadrilateral  ABCD in Fig.~\ref{f13} is a parallelogram. It follows that
the opposite sides of ABCD can be represented by the
{\em same} vectors, ${\bf a}$ and ${\bf b}$: this merely indicates that these sides are of
equal length and are parallel ({\em i.e.}, they point in the same direction). Note that
Fig.~\ref{f13} illustrates an important point regarding vectors. Although vectors possess
both a magnitude (length) and a direction, they possess no intrinsic position information. 
Thus, since sides $AB$ and $DC$ are parallel  and of equal length, they can be represented
by the {\em same} vector ${\bf a}$, despite the fact that they are in different places on the
diagram.

The diagonal $BD$ in Fig.~\ref{f13} can be represented vectorially as ${\bf d} = {\bf b} - {\bf a}$.
Likewise, the diagonal $AC$ can be written ${\bf c} = {\bf a} + {\bf b}$. 
The displacement ${\bf x}$ (say) of the centroid $X$ from point $A$ can be written in one
of two different ways:
\begin{eqnarray}
{\bf x} &=& {\bf a} + \lambda\,{\bf d},\label{rh1}\\
{\bf x} &=& {\bf b} + {\bf a} - \mu \,{\bf c}.\label{rh2}
\end{eqnarray}
Equation~(\ref{rh1}) is interpreted as follows: in order to get from point $A$ to point $X$,
first move to point $B$ (along vector ${\bf a}$), then move along diagonal $BD$ (along
vector ${\bf d}$) for an unknown fraction $\lambda$ of its length. 
Equation~(\ref{rh2}) is interpreted as follows: in order to get from point $A$ to point
$X$, first move to point $D$ (along vector ${\bf b}$), then move to point $C$
(along vector ${\bf a}$), finally move along diagonal $CA$ (along vector
$-{\bf c}$) for an unknown fraction $\mu$ of its length. Since $X$ represents the
{\em same} point in Eqs.~(\ref{rh1}) and (\ref{rh2}), we can equate these two
expressions to give
\begin{equation}
{\bf a} + \lambda\,({\bf b} - {\bf a}) =  {\bf b} + {\bf a} - \mu \,({\bf a} + {\bf b}).
\end{equation}
Now vectors ${\bf a}$ and ${\bf b}$ point in {\em different} directions, so the only way
in which the above expression can be satisfied, in general,  is if the coefficients of
${\bf a}$ and ${\bf b}$ match on either side of the equality sign. Thus,
equating coefficients of ${\bf a}$ and ${\bf b}$, we obtain
\begin{eqnarray}
1 - \lambda &=& 1 - \mu,\\
\lambda &=& 1 -\mu.
\end{eqnarray}
It follows that $\lambda=\mu =1/2$. In other words, the centroid $X$ is located at
the halfway points of diagonals $BD$ and $AC$: {\em i.e.}, the diagonals
mutually bisect one another.

\subsection{Vector Velocity and Vector Acceleration}
Consider a body moving in 3 dimensions. Suppose that we know the Cartesian
coordinates, $x$, $y$, and $z$, of this body as  time, $t$, progresses. 
Let us consider how we can use this information to determine the body's instantaneous
velocity and acceleration as functions of time.

The vector displacement of the body is given by
\begin{equation}
{\bf r}(t) = [x(t), y(t), z(t)].
\end{equation}
By analogy with the 1-dimensional equation (\ref{e23}), the body's
vector velocity ${\bf v}=(v_x,v_y,v_z)$ is simply the {\em derivative} of ${\bf r}$ with respect to
$t$. In other words,
\begin{equation}
{\bf v}(t) = \lim_{{\mit\Delta} t\rightarrow 0}\frac{{\bf r}(t+{\mit\Delta} t)-{\bf r}(t)}{
{\mit\Delta} t} = \frac{d{\bf r}}{dt}.
\end{equation} 
When written in component form, the above definition yields
\begin{eqnarray}
v_x &=& \frac{dx}{dt},\\
v_y &=& \frac{dy}{dt},\\
v_z &=& \frac{dz}{dt}.
\end{eqnarray}
Thus, the $x$-component of velocity is simply the time derivative of the $x$-coordinate,
and so on.

By analogy with the 1-dimensional equation (\ref{e26}), the body's
vector acceleration ${\bf a}=(a_x,a_y,a_z)$ is simply the {\em derivative} of ${\bf v}$ with respect to
$t$. In other words,
\begin{equation}
{\bf a}(t) = \lim_{{\mit\Delta} t\rightarrow 0}\frac{{\bf v}(t+{\mit\Delta} t)-{\bf v}(t)}{
{\mit\Delta} t} = \frac{d{\bf v}}{dt}=\frac{d^2{\bf r}}{dt^2}.
\end{equation} 
When written in component form, the above definition yields
\begin{eqnarray}
a_x &=& \frac{dv_x}{dt}=\frac{d^2 x}{dt^2},\\
a_y &=& \frac{dv_y}{dt}=\frac{d^2 y}{dt^2},\\
a_z &=& \frac{dv_z}{dt}=\frac{d^2 z}{dt^2}.
\end{eqnarray}
Thus, the $x$-component of acceleration is simply the time derivative
of the $x$-component of velocity, and so on.

As an example, suppose that the coordinates of the body are given by
\begin{eqnarray}
x &=& \sin t,\\
y &=& \cos t,\\
z &=& 3\,t.
\end{eqnarray}
The corresponding components of the body's velocity are then simply
 \begin{eqnarray}
v_x &=& \frac{dx}{dt}=\cos t,\\
v_y &=&\frac{dy}{dt}= - \sin t,\\
v_z &=& \frac{dz}{dt}=3,
\end{eqnarray}
whilst the components of the body's acceleration are given by 
\begin{eqnarray}
a_x &=& \frac{dv_x}{dt}=-\sin t,\\
a_y &=& \frac{dv_y}{dt}=- \cos t,\\
a_z &=& \frac{dv_z}{dt}=0.
\end{eqnarray}


\subsection{Motion with Constant Velocity}
An object moving in 3 dimensions with constant velocity ${\bf v}$ possesses
a vector displacement of the form
\begin{equation}
{\bf r}(t) = {\bf r}_0 + {\bf v}\,t,
\end{equation}
where the constant vector ${\bf r}_0$ is the displacement at time $t=0$. Note
that $d{\bf r}/dt = {\bf v}$ and $d^2{\bf r}/dt^2={\bf 0}$, as expected.
As illustrated in Fig.~\ref{f14}, the object's trajectory  is a {\em straight-line}
which passes through point ${\bf r}_0$ at time $t=0$ and runs parallel to vector
${\bf v}$. 

\begin{figure}
\epsfysize=2.5in
\centerline{\epsffile{Chapter03/fig014.eps}}
\caption{\em Motion with constant velocity}\label{f14}   
\end{figure}

\subsection{Motion with Constant Acceleration}\label{s2ed}
An object moving in 3 dimensions with constant acceleration ${\bf a}$
possesses a vector displacement of the form
\begin{equation}
{\bf r}(t) = {\bf r}_0 + {\bf v}_0\,t + \frac{1}{2}\,{\bf a}\,t^2.
\end{equation}
Hence, the object's velocity is given by
\begin{equation}
{\bf v}(t) = \frac{d{\bf r}}{dt}= {\bf v}_0 + {\bf a}\,t.
\end{equation}
Note that $d{\bf v}/dt ={\bf a}$, as expected. In the above,
the constant vectors ${\bf r}_0$ and ${\bf v}_0$ are the object's displacement and velocity at time
$t=0$, respectively.

As is easily demonstrated, the vector equivalents of Eqs.~(\ref{e211})--(\ref{e214}) are:
\begin{eqnarray}
{\bf s} &=& {\bf v}_0\,t + \frac{1}{2}\,{\bf a}\,t^2,\label{e335}\\
{\bf v} &=& {\bf v}_0 + {\bf a}\,t,\label{e337}\\
v^2 &=& v_0^{\,2} + 2\,{\bf a}\!\cdot\!{\bf s}.\label{e338}
\end{eqnarray}
These equation fully characterize 3-dimensional motion with constant acceleration.
Here, ${\bf s} = {\bf r}-{\bf r}_0$ is the net displacement of the object
between times $t=0$ and $t$.

The quantity ${\bf a}\!\cdot\!{\bf s}$, appearing in Eq.~(\ref{e338}), is termed
the {\em scalar product} of vectors ${\bf a}$ and ${\bf s}$, and is defined
\begin{equation}
{\bf a}\!\cdot\!{\bf s} = a_x\,s_x+ a_y\,s_y+a_z\,s_z.\label{e339}
\end{equation}
The above formula has a simple geometric interpretation, which is illustrated
in Fig.~\ref{f15}. If $|{\bf a}|$ is the magnitude (or length) of vector ${\bf a}$,
$|{\bf s}|$ is the magnitude of vector ${\bf s}$, and $\theta$ is the angle
subtended between these two vectors, then
\begin{equation}
{\bf a} \! \cdot \! {\bf s} = |{\bf a}| \,|{\bf s}|\, \cos\theta.
\end{equation}
In other words, the scalar product of vectors ${\bf a}$ and ${\bf s}$ equals
the product of the length of vector ${\bf a}$ times the length of that component of
vector ${\bf s}$ which lies in the {\em same direction} as vector ${\bf a}$. 
It immediately follows that if two vectors are mutually perpendicular
({\em i.e.}, $\theta =90^\circ$) then their scalar product is zero. Furthermore, the
scalar product of a vector with itself is simply the magnitude squared of that vector [this
is immediately apparent from Eq.~(\ref{e339})]:
\begin{equation}
{\bf a} \!\cdot \!{\bf a} = |{\bf a}|^2 = a^2.
\end{equation}
It is also apparent from Eq.~(\ref{e339}) that ${\bf a}\!\cdot{\bf s}= {\bf s}\!\cdot{\bf a}$,
and ${\bf a}\!\cdot\!({\bf b} +{\bf c}) = {\bf a}\!\cdot\!{\bf b}+{\bf a}\!\cdot\!{\bf c}$,
and ${\bf a}\!\cdot\!(\lambda {\bf s}) = \lambda({\bf a}\!\cdot{\bf s})$.

\begin{figure}
\epsfysize=2in
\centerline{\epsffile{Chapter03/fig015.eps}}
\caption{\em The scalar product}\label{f15}   
\end{figure}

Incidentally, Eq.~(\ref{e338}) is obtained by taking the scalar product of Eq.~(\ref{e337}) with itself,
taking the scalar product of Eq.~(\ref{e335}) with ${\bf a}$, and then eliminating $t$.

\subsection{Projectile Motion}
As a simple illustration of the concepts introduced in the previous subsections,
let us examine
the following problem. Suppose that a projectile is launched upward
from ground level, with speed $v_0$, making an angle $\theta$
with the horizontal. {\em Neglecting the effect of air resistance}, what
is the subsequent trajectory of the projectile?

Our first task is to set up a suitable  Cartesian coordinate system.
A convenient system is illustrated in Fig.~\ref{f16}. The $z$-axis points vertically
upwards (this is a standard convention), whereas the $x$-axis points along the
projectile's
initial direction of horizontal motion. Furthermore, the origin of our
coordinate system  corresponds to the launch point. Thus, $z=0$ corresponds
to ground level.

Neglecting air resistance,
the projectile is subject to a constant acceleration $g=9.81\,{\rm m}\,{\rm s}^{-1}$,
due to gravity, which is directed {\em vertically downwards}. Thus, the projectile's
vector acceleration is written
\begin{equation}
{\bf a} = (0,0,-g).\label{e342}
\end{equation}
Here, the minus sign indicates that the acceleration is in the minus $z$-direction
({\em i.e.}, downwards), as opposed to the plus $z$-direction ({\em i.e.}, upwards).

\begin{figure}
\epsfysize=2in
\centerline{\epsffile{Chapter03/fig016.eps}}
\caption{\em Coordinates for the projectile problem}\label{f16}   
\end{figure}

What is the initial vector velocity ${\bf v}_0$ with which the projectile is launched
into the air at (say) $t=0$? As illustrated in Fig.~\ref{f16}, 
given that the magnitude of this velocity is $v_0$, its horizontal component
 is directed along the $x$-axis, and  its direction subtends an angle $\theta$
with this axis, the components of ${\bf v}_0$ take the form
\begin{equation}
{\bf v}_0 = (v_0\,\cos\theta,\,0,\,v_0\,\sin\theta).\label{e343}
\end{equation}
Note that ${\bf v}_0$ has zero component along the $y$-axis, which points
{\em into} the paper in Fig.~\ref{f16}.

Since the projectile moves with constant acceleration, its vector
displacement ${\bf s}=(x,y,z)$ from its launch point
satisfies [see Eq.~(\ref{e335})]
\begin{equation}
{\bf s} = {\bf v}_0\,t + \frac{1}{2}\,{\bf a}\,t^2.
\end{equation}
Making use of Eqs.~(\ref{e342}) and (\ref{e343}), the $x$-, $y$-, and $z$-components
of the above equation are written
\begin{eqnarray}
x &=& v_0\,\cos\theta\,\,t,\label{e345}\\[0.5ex]
y&=& 0,\\
z &=& v_0\,\sin\theta\,\,t - \frac{1}{2}\,g\,t^2,\label{e346}
\end{eqnarray}
respectively.
Note that the projectile moves with {\em constant velocity}, $v_x =dx/dt = v_0\,\cos\theta$,
in the $x$-direction ({\em i.e.}, horizontally). This is hardly surprising, since there
is zero component of the projectile's acceleration along the $x$-axis. Note, further,
that
since there is zero component of the projectile's acceleration along the $y$-axis, and
 the projectile's initial velocity also has zero component along this axis, 
the projectile never moves in the $y$-direction. In other words, the projectile's
trajectory is {\em 2-dimensional}, lying entirely within the $x$-$z$ plane. Note,
finally, that the projectile's vertical motion is entirely decoupled
from its horizontal motion. In other words, the projectile's vertical motion is
identical to that of a second projectile launched vertically upwards, at $t=0$,
with the initial velocity $v_0\,\sin\theta$ ({\em i.e.}, the
initial {\em vertical} velocity component of the first projectile)---both projectiles will reach the
same maximum altitude at the same time, and will subsequently  strike the ground simultaneously. 

Equations~(\ref{e345}) and (\ref{e346}) can be rearranged to give
\begin{equation}
z =x\, \tan\theta -\frac{1}{2}\frac{g\,x^2}{v_0^{\,2}}\,\sec^2\theta.
\end{equation}
As was first pointed out by Galileo, and is illustrated in Fig.~\ref{f17}, this is
the equation of a {\em parabola}. The horizontal range
$R$ of the projectile corresponds to its $x$-coordinate when it strikes the
ground ({\em i.e.}, when $z=0$). It follows from the above
expression (neglecting the trivial result $x=0$) that
\begin{equation}\label{range}
R = \frac{2\,v_0^{\,2}}{g}\,\sin\theta\,\cos\theta = 
\frac{v_0^{\,2}}{g}\,\sin 2\theta.
\end{equation}
Note that the range attains its maximum value,
\begin{equation}
R_{\rm max} =  \frac{v_0^{\,2}}{g},
\end{equation}
when $\theta=45^\circ$. In other words, neglecting air resistance, a projectile
travels furthest when it is launched into the air at $45^\circ$ to the horizontal. 

The maximum altitude $h$ of the projectile is attained when
$v_z=dz/dt=0$ ({\em i.e.}, when the projectile has just stopped rising and
is about to start falling). It follows from Eq.~(\ref{e346}) that the maximum altitude
occurs at time $t_0=v_0\,\sin\theta/g$. Hence,
\begin{equation}
h = z(t_0)= \frac{v_0^{\,2}}{2\,g}\,\sin^2\theta.
\end{equation}
Obviously, the largest value of $h$,
\begin{equation}
h_{\rm max} = \frac{v_0^{\,2}}{2\,g},
\end{equation}
is obtained when the projectile is launched vertically upwards
({\em i.e.}, $\theta =90^\circ$).

\begin{figure}
\epsfysize=2in
\centerline{\epsffile{Chapter03/fig017.eps}}
\caption{\em The parabolic trajectory of a projectile}\label{f17}   
\end{figure}

\subsection{Relative Velocity}
Suppose that, on a windy day, an airplane moves with constant velocity ${\bf v}_a$ with respect to the
air, and that the air moves with constant velocity ${\bf u}$ with
respect to the ground. What is the vector velocity ${\bf v}_g$ of the plane
with respect to the ground? In principle, the answer to this question is
very simple: 
\begin{equation}
{\bf v}_g = {\bf v}_a + {\bf u}.\label{e353}
\end{equation}
In other words, the velocity of the plane with respect to the ground is
the vector sum of the plane's velocity relative to the air and the air's
velocity relative to the ground. See Fig.~\ref{f18}. Note that, in general, ${\bf v}_g$ 
is parallel to neither  ${\bf v}_a$ nor ${\bf u}$. Let us now consider how we might
implement Eq.~(\ref{e353}) in practice.

\begin{figure}
\epsfysize=1.5in
\centerline{\epsffile{Chapter03/fig018.eps}}
\caption{\em Relative velocity}\label{f18}   
\end{figure}

As always, our first task is to set up a suitable Cartesian coordinate system.
A convenient system for dealing with  2-dimensional motion parallel to the Earth's surface
is illustrated in Fig.~\ref{f19}. The $x$-axis points northward, whereas the $y$-axis points
eastward. In this coordinate system, it is conventional to specify a vector ${\bf r}$ in
term of its magnitude, $r$, and its {\em compass bearing}, $\phi$. As illustrated in Fig.~\ref{f20},
a compass bearing is the angle subtended between the direction of a vector and the direction to
the North  pole: {\em i.e.}, the $x$-direction. By convention, compass bearings
run from $0^\circ$ to $360^\circ$. Furthermore, the compass bearings of North, East, South, and West
are $0^\circ$, $90^\circ$, $180^\circ$, and $270^\circ$, respectively.

\begin{figure}
\epsfysize=2in
\centerline{\epsffile{Chapter03/fig019.eps}}
\caption{\em Coordinates for relative velocity problem}\label{f19}   
\end{figure}

According to Fig.~\ref{f20}, the components of
a general vector ${\bf r}$, whose magnitude is $r$ and whose compass bearing is $\phi$, are simply
\begin{equation}\label{e354}
{\bf r} = (x,\,y) = (r\,\cos\phi, \,r\,\sin\phi).
\end{equation}
Note that we have suppressed the $z$-component of ${\bf r}$ (which is zero), for ease of
notation. Although, strictly speaking,  Fig.~\ref{f20} only justifies the above expression for
$\phi$ in the range $0^\circ$ to $90^\circ$, it turns out that this expression
is generally valid: {\em i.e.}, it is valid for $\phi$ in the full range $0^\circ$ to
$360^\circ$. 

\begin{figure}
\epsfysize=2.5in
\centerline{\epsffile{Chapter03/fig020.eps}}
\caption{\em A compass bearing}\label{f20}   
\end{figure}

As an illustration, suppose that the plane's velocity relative to the air
is $300\,{\rm km/h}$, at a compass bearing of $120^\circ$, and
 the air's velocity relative to the ground is $85\,{\rm km/h}$, at a compass
bearing of $225^\circ$. It follows that the components of ${\bf v}_a$ and
${\bf u}$ (measured in units of km/h) are 
\begin{eqnarray}
{\bf v}_a& =& (300\,\cos 120^\circ,\,300\,\sin 120^\circ) = (-1.500\times 10^2,\, 2.598\times 10^2),\\
{\bf u} &=& (85\,\cos 225^\circ,\,85\,\sin 225^\circ) =(-6.010\times 10^1,\,-6.010\times 10^1).
\end{eqnarray}
According to Eq.~(\ref{e353}), the components of the plane's velocity ${\bf v}_g$ relative to
the ground are simply the algebraic sums of the corresponding components
of ${\bf v}_a$ and ${\bf u}$. Hence,
\begin{eqnarray}
{\bf v}_g& =& (-1.500\times 10^2\,-\,6.010\times 10^1,\,2.598\times 10^2\,-\,6.010\times 10^1)\nonumber\\
&=& (-2.101\times 10^2, \, 1.997\times 10^2).
\end{eqnarray}

Our final task is to reconstruct the magnitude and compass bearing of vector ${\bf v}_g$,
given its components $(v_{g\,x}, v_{g\,y})$. The magnitude of ${\bf v}_g$ follows
from Pythagoras' theorem [see Eq.~(\ref{e36})]:
\begin{eqnarray}
v_g &=& \sqrt{(v_{g\,x})^2 + (v_{g\,y})^2}\nonumber \\
&=& \sqrt{(-2.101\times 10^2)^2 + (1.997\times 10^2)^2} = 289.9\,{\rm km/h}.
\end{eqnarray}
In principle, the compass bearing of ${\bf v}_g$ is given
by the following formula:
\begin{equation}
\phi = \tan^{-1} \left(\frac{v_{g\,y}}{v_{g\,x}}\right).
\end{equation}
This follows because $v_{g\,x}=v_g\,\cos\phi$ and $v_{g\,y}=v_g\,\sin\phi$ [see
Eq.~(\ref{e354})].
Unfortunately, the above expression becomes a little difficult to interpret
if $v_{g\,x}$ is negative. An unambiguous
pair of expressions for $\phi$ is given below:
\begin{equation}
\phi = \tan^{-1} \left(\frac{v_{g\,y}}{v_{g\,x}}\right),
\end{equation}
if $v_{g\,x}\geq 0$; or
\begin{equation}\label{pres}
\phi = 180^\circ - \tan^{-1} \left(\frac{v_{g\,y}}{|v_{g\,x}|}\right),
\end{equation}
if $v_{g\,x}< 0$.
These expressions can  be derived from simple
trigonometry.
 For the case in hand, Eq.~(\ref{pres}) is the
relevant expression, hence
\begin{equation}
\phi = 180^\circ - \tan^{-1} \left(\frac{1.997\times 10^2}{2.101\times 10^2}\right)
= 136.5^\circ.
\end{equation}
Thus, the plane's velocity relative to the ground is $289.9\,{\rm km/h}$ at a compass
bearing of $136.5^\circ$.

\subsection*{\em Worked Example 3.1: Broken Play}
{\em Question:} Major Applewhite receives the snap at the line
of scrimmage, takes a seven step drop ({\em i.e.}, runs backwards
9 yards), but is then flushed out of the pocket by a blitzing
linebacker. Major subsequently runs parallel to the line of scrimmage
for 12 yards and then gets off a forward pass, 36 yards straight downfield, to
Roy Williams, just prior to being creamed by the linebacker. What is the magnitude
of the football's resultant displacement (in yards)?\\
~\\
{\em Answer:} As illustrated in the diagram, the resultant displacement ${\bf r}$ of the
football is the  sum of vectors ${\bf a}$, ${\bf b}$, and ${\bf c}$, which
correspond to the seven step drop, the run parallel to the line of
scrimmage, and the forward pass, respectively.
\begin{figure*}[h]
\epsfysize=2in
\centerline{\epsffile{Chapter03/fig020a.eps}}
\end{figure*}
Using the coordinate system indicated in the diagram, the components of
vectors ${\bf a}$, ${\bf b}$, and ${\bf c}$ (measured in yards) are
\begin{eqnarray}
{\bf a} = (-9,0),\nonumber\\
{\bf b} = (0,12),\nonumber\\
{\bf c} = (36, 0),\nonumber
\end{eqnarray}
respectively. Hence the components of ${\bf r}$ are given by
$$
{\bf r} = (x,y)= (-9+0+36,\,0+12+0) = (27, 12).
$$
It follows that the magnitude of the football's resultant displacement is
$$
r = \sqrt{x^2+y^2} = \sqrt{27^2+12^2} = 29.55\,{\rm yd}.
$$

\subsection*{\em Worked Example 3.2: Gallileo's Experiment}
{\em Question:} Legend has it that Gallileo tested out his newly developed
theory of projectile motion by throwing weights from the top of the
Leaning Tower of
Pisa. (No wonder he eventually got into trouble with the authorities!)
Suppose that, one day, Gallileo simultaneously threw two equal weights off the tower from
a height of $100\,{\rm m}$ above the ground. Suppose, further, that he
dropped the  first weight straight down,
whereas he threw the second weight horizontally with a velocity of $5\,{\rm m/s}$. 
Which weight  struck the ground first? How long, after it was thrown, did it take
to do this? Finally, what horizontal distance was traveled by the
second weight before it hit the ground? Neglect the effect of air
resistance.\\
~\\
{\em Answer:} Since both weights start off traveling with the same initial velocities in the
vertical direction ({\em i.e.}, zero), and both accelerate vertically downwards
at the same rate, it follows that both weights strike the ground simultaneously. 
The time of flight of each weight is simply the time taken to fall
$h=100\,{\rm m}$, starting from rest, under the influence of gravity. From
Eq.~(\ref{e2.19}), this time is given by
$$
t = \sqrt{\frac{2\,h}{g}} = \sqrt{\frac{2\times 100}{9.81}} = 4.515\,{\rm s}.
$$
The horizontal distance $R$ traveled by the second weight is simply the
distance traveled by a body moving at a constant velocity $u=5\,{\rm m/s}$
(recall that gravitational acceleration does not affect horizontal motion)
during the time taken by the weight to drop 100\,m. Thus,
$$
R = u\,t = 5\times 4.515 = 22.58\,{\rm m}.
$$
\subsection*{\em Worked Example 3.3: Cannon Shot}
{\em Question:} A cannon placed on a 50\,{\rm m} high cliff 
fires a cannonball over the edge of the cliff at $v=200\,{\rm m}/{\rm s}$ making an angle
of $\theta = 30^\circ$ to the horizontal. How long is the cannonball in the air? Neglect
air resistance.\\
~\\
{\em Answer:} In order to answer this question we need only consider the cannonball's
vertical motion. At $t=0$ ({\em i.e.}, the time of firing) the cannonball's height off the
ground is $z_0=50$\,m and its velocity component in the vertical direction is
$v_0 = v\,\sin\theta = 200\times\sin 30^\circ = 100\,{\rm m}/{\rm s}$. Moreover,
the cannonball is accelerating vertically downwards at $g=9.91\,{\rm m}/{\rm s}^2$.
The equation of vertical motion of the cannonball is written
$$
z = z_0 + v_0\,t - \frac{1}{2}\,g\,t^2,
$$
where $z$ is the cannonball's height off the ground at time $t$. The time of flight
of the cannonball corresponds to the time $t$ at which $z=0$. In other words,
the time of flight is the solution of the quadratic equation
$$
0 = z_0 + v_0\,t -  \frac{1}{2}\,g\,t^2.
$$
Hence,
$$
t = \frac{v_0 +\sqrt{v_0^{\,2}+2\,g\,z_0}}{g} = 20.88\,{\rm s}.
$$
Here, we have neglected the unphysical negative root of our quadratic equation.

\subsection*{\em Worked Example 3.4: Hail Mary Pass}
{\em Question:} The Longhorns are down by 4 points with 5\,s left in the
fourth quarter. Chris Simms launches a Hail Mary pass into the end-zone,
60 yards away, where B.J.~Johnson is waiting to make the catch. Suppose
that Chris throws the ball at 55 miles per hour. At what angle to the horizontal
must the ball be launched in order for it to hit the receiver? Neglect the effect of air resistance.\\
~\\
{\em Answer:} The formula for the horizontal range $R$ of a projectile thrown
with initial velocity $v_0$ at an angle $\theta$ to the horizontal is [see
Eq.~(\ref{range})]:
$$
R= \frac{v_0^{\,2}}{g}\,\sin 2\theta.
$$
In this case, $R= 60\times 3\times 0.3048 = 54.86\,{\rm m}$ and 
$v_0 = 55\times 5280\times 0.3048/3600 = 24.59\,{\rm m/s}$. Hence,
$$
\theta = \frac{1}{2}\,\sin^{-1}\left(\frac{R\,g}{v_0^{\,2}}\right)
= \frac{1}{2} \,\sin^{-1}\left(\frac{54.86\times 9.81}{(24.59)^2}\right) =
31.45^\circ.
$$
Thus, the ball must be launched at $31.45^\circ$ to the horizontal. (Actually,
$58.56^\circ$ would  work just as well. Why is this?)

\subsection*{\em Worked Example 3.5: Flight UA\,589}
{\em Question:}~United Airlines flight UA\,589 from Chicago is 20\,miles due
North of Austin's Bergstrom airport. Suppose that the plane is flying
at $200\,{\rm mi/h}$ relative to the air. Suppose, further, that there
is a wind blowing due East at $60\,{\rm mi/h}$. Towards which compass bearing must
the plane steer in order to land at the airport?\\
~\\
{\em Answer:} The problem in hand is illustrated in the diagram.
\begin{figure*}[h]
\epsfysize=2in
\centerline{\epsffile{Chapter03/fig020b.eps}}
\end{figure*}
The plane's velocity ${\bf v}_g$ relative to the ground is the
vector sum of its velocity ${\bf v}_a$ relative to the air, and the
velocity ${\bf u}$ of the wind relative to the ground. We know that
${\bf u}$ is directed due East, and we require ${\bf v}_g$ to be directed
due South. We also know that $|{\bf v}_a|=200\,{\rm mi/h}$ and $|{\bf u}|  = 60\,{\rm mi/h}$.
Now, from simple trigonometry, 
$$
\cos\alpha = \frac{|{\bf u}|}{|{\bf v}_a|} = \frac{60}{200} =0.3.
$$
Hence,
$$
\alpha = 72.54^\circ.
$$
However, it is clear from the diagram that the compass bearing $\phi$ of the
plane is given by
$$
\phi = 270^\circ -\alpha = 270^\circ -  72.54^\circ = 197.46^\circ.
$$
Thus, in order to land at Bergstrom airport the plane must fly towards compass
bearing $197.46^\circ$.
