\section{Newton's Laws of Motion}
\subsection{Introduction}
In his {\em Principia}, Newton reduced the basic principles of
mechanics to three laws:
\begin{enumerate}{\sf
\item Every body continues in its state of rest, or uniform motion in a straight line, unless
compelled to change that state by forces impressed upon it.
\item The change of motion of an object is proportional to the force impressed upon
it, and is made in the direction of the straight line in which the force is impressed.
\item To every action there is always opposed an equal reaction; or, the mutual actions
of two bodies upon each other are always equal and directed to contrary parts.  }
\end{enumerate}
These laws are known as Newton's first law of motion, Newton's second law of motion,
and Newton's third law of motion, respectively. In this section, we shall examine each of
these laws in detail, and then give some simple illustrations of their use.

\subsection{Newton's First Law of Motion}
Newton's first law was actually discovered by Galileo and perfected by
Descartes (who added the crucial proviso ``in a straight line''). This law
states that if the motion of a given body is not disturbed by external influences
then
that body moves with {\em constant} velocity. In other words, the displacement ${\bf r}$
of the body as a function of time $t$ can be written
\begin{equation}
{\bf r} = {\bf r}_0 + {\bf v}\,t,
\end{equation}
where ${\bf r}_0$ and ${\bf v}$ are {\em constant} vectors. As illustrated in
Fig.~\ref{f14}, the body's trajectory is a {\em straight-line} which
passes through point ${\bf r}_0$ at time $t=0$ and runs parallel to ${\bf v}$.
In the special case in which ${\bf v}={\bf 0}$  the body simply remains at rest.  

Nowadays, Newton's first law strikes us as almost a statement of the obvious. However,
in Galileo's time this was far from being the case. From the time of the ancient
Greeks, philosophers---observing that objects set into motion on the Earth's
surface eventually come to rest---had concluded that the natural state of motion
of objects was that they should remain at rest. Hence, they reasoned, any object which moves
does so under the influence of an external influence, or {\em force}, exerted on it by some other object.
It took the genius of Galileo to realize that an object set into motion
on the Earth's surface eventually comes to rest under the influence of frictional forces, and that if these
forces could somehow be abstracted from the motion then it would continue forever. 

\subsection{Newton's Second Law of Motion}\label{moment}
Newton used the word ``motion'' to mean what we nowadays call {\em momentum}. The
momentum ${\bf p}$ of a body is simply defined as the product of its mass $m$ and its
velocity ${\bf v}$: {\em i.e.}, 
\begin{equation}
{\bf p} = m\,{\bf v}.
\end{equation}
Newton's second law of motion is summed up in the equation
\begin{equation}
\frac{d{\bf p}}{dt} = {\bf f},
\end{equation}
where the vector ${\bf f}$ represents the net influence, or force, exerted on the object, whose
motion is under investigation,
by other objects.
For the case of a object with {\em constant} mass, the above law reduces to its more conventional
form
\begin{equation}\label{en2}
{\bf f} = m\,{\bf a}.
\end{equation}
In other words, the net force exerted on a given object  by other objects equals the product of that object's
mass and its acceleration. Of course, this law is entirely devoid of content unless we have
some independent means of quantifying the forces exerted between different objects. 

\subsection{Hooke's Law}
One method of quantifying the force exerted on an object is via {\em Hooke's law}. 
This law---discovered by the English scientist Robert Hooke in 1660---states that the
force $f$ exerted by a coiled spring is directly proportional to its {\em extension}  
${\mit\Delta}x$. The extension of the spring is the difference between its actual
length and its natural length ({\em i.e.}, its length when it is exerting no force).
The force acts parallel to the axis of the spring. Obviously, Hooke's law
only holds if the extension of the spring is sufficiently small. If the extension
becomes too large then the spring  deforms permanently, or even breaks. Such behaviour
lies beyond the scope of Hooke's law. 

\begin{figure}
\epsfysize=1.5in
\centerline{\epsffile{Chapter04/fig021.eps}}
\caption{\em Hooke's law}\label{f21}   
\end{figure}

Figure~\ref{f21} illustrates how we might use Hooke's law to quantify the force we exert
on a body of mass $m$ when we pull on the handle of
a spring attached to it. The magnitude $f$ of the force is proportional to the extension
of the spring: twice the extension means twice the force. As shown, the direction of the force
is towards the spring, parallel to its axis (assuming that the extension is positive). 
The magnitude of the force can be quantified in terms of the critical extension required to
impart  a unit acceleration
({\em i.e.}, $1\,{\rm m/s^2}$) to a body of unit mass ({\em i.e.}, $1\,{\rm kg}$).
 According to Eq.~(\ref{en2}), the force corresponding
to this extension is  1 {\em newton}. Here, a newton (symbol N) is
equivalent to a kilogram-meter per second-squared, and is the mks unit of force. Thus,
if the critical extension corresponds to a force of $1\,{\rm N}$ then half the critical
extension corresponds to a force of $0.5\,{\rm N}$, and so on. In this manner, we
can quantify both the direction and magnitude of the force we exert, by means of a spring,
on a given body.

Suppose that we apply two forces, ${\bf f}_1$ and ${\bf f}_2$ (say), acting in different directions,
to a body of mass $m$ by means of two springs. As illustrated in Fig.~\ref{f22}, the body accelerates
as if it were subject to a single force ${\bf f}$ which is the {\em vector sum} of the individual
forces ${\bf f}_1$ and ${\bf f}_2$. It follows that the force ${\bf f}$ appearing in
Newton's second law of motion, Eq.~(\ref{en2}), is the {\em resultant} of all the external forces to which
the body whose motion is under investigation is subject.

\begin{figure}
\epsfysize=2.5in
\centerline{\epsffile{Chapter04/fig022.eps}}
\caption{\em Addition of forces}\label{f22}   
\end{figure}

Suppose that the resultant of all the forces acting on a given body is {\em zero}. In other words,
suppose that the forces acting on the body exactly balance one another. According to Newton's second law of motion,
Eq.~(\ref{en2}), the body does not accelerate: {\em i.e.}, it either remains at rest or
moves with uniform velocity in a straight line. It follows that Newton's first law
of motion applies not only to bodies which have no forces acting upon them but also to
bodies acted upon by exactly balanced forces.

\subsection{Newton's Third Law of Motion}
Suppose, for the sake of argument, that there are only two bodies in the Universe. Let
us label these bodies $a$ and $b$. Suppose that body $b$ exerts a force ${\bf f}_{ab}$ on body $a$.
According to to Newton's third law of motion,  body $a$ must exert an
{\em equal and opposite} force ${\bf f}_{ba}=-{\bf f}_{ab}$ on body $b$. See Fig.~\ref{f22}. 
Thus, if we label ${\bf f}_{ab}$ the ``action'' then, in Newton's language, ${\bf f}_{ba}$
is the equal and opposed ``reaction''.

Suppose, now, that there are many objects in the Universe (as is, indeed, the case). According
to Newton's third law, if object $j$ exerts a force ${\bf f}_{ij}$ on object $i$ then
object $i$ must exert an equal and opposite force ${\bf f}_{ji}=-{\bf f}_{ij}$ on object $j$.
It follows that all of the forces acting in the Universe can ultimately be grouped into equal and opposite
action-reaction pairs. Note, incidentally, that an action and its associated reaction
always act on {\em different} bodies.

\begin{figure}
\epsfysize=1in
\centerline{\epsffile{Chapter04/fig023.eps}}
\caption{\em Newton's third law}\label{f23}   
\end{figure}

Why do we need Newton's third law? Actually, it is almost a matter of common sense. 
Suppose that  bodies $a$ and $b$ constitute an {\em isolated} system. If
${\bf f}_{ba}\neq -{\bf f}_{ab}$ then this system  exerts a {\em non-zero net force}
${\bf f} = {\bf f}_{ab}+{\bf f}_{ba}$ on itself,
without the aid of any external agency. It will, therefore, accelerate forever under its
own steam. We know, from experience, that this sort of behaviour does not occur in real
life. For instance, I cannot grab hold of my shoelaces and, thereby, pick myself up off the ground.
In other words, I cannot self-generate a force which will spontaneously lift me into the air:
I need to exert forces on other objects around me in order to achieve this. Thus,
Newton's third law essentially acts as a guarantee against the absurdity of self-generated forces. 

\subsection{Mass and Weight}
The terms {\em mass} and {\em weight} are often confused with one another. However, in physics
their meanings are quite distinct. 

A body's mass is a measure of its
{\em inertia}: {\em i.e.}, its reluctance to deviate from uniform straight-line
motion under the influence of external forces. According to Newton's second law, Eq.~(\ref{en2}),
if two objects of differing masses are acted upon by forces of the same magnitude
then the resulting acceleration of the larger mass is less than that of the smaller
mass. In other words, it is more difficult to force the larger mass to deviate from
its preferred state of uniform motion in a straight line. Incidentally, the mass of
a body is an intrinsic property of that body, and, therefore, does not change if the
body is moved to a different place.

\begin{figure}
\epsfysize=1.6in
\centerline{\epsffile{Chapter04/fig024.eps}}
\caption{\em Weight}\label{f24}   
\end{figure}

Imagine a block of granite resting on the surface of the Earth. 
See Fig.~\ref{f24}. The block experiences
a downward force ${\bf f}_g$ due to the gravitational attraction of the Earth. This
force is of magnitude $m\,g$, where $m$ is the mass of the block and $g$ is the acceleration
due to gravity at the surface of the Earth. The block transmits this force to the
ground below it, which is supporting it, and, thereby, preventing it from accelerating 
downwards.  In other words, the block exerts a downward force ${\bf f}_W$, of magnitude $m\,g$,
on the ground immediately beneath it. We usually refer to this force (or the
magnitude of this force) as the {\em weight} of the block. According to Newton's third
law, the ground below the block exerts an upward reaction force ${\bf f}_R$ on the block.
This force is also of magnitude $m\,g$. Thus, the net force acting on the block
is ${\bf f}_g + {\bf f}_R = {\bf 0}$, which accounts for the fact that the block
remains stationary.

Where, you might ask, is the equal and opposite reaction to the force of gravitational
attraction
${\bf f}_g$ exerted by the Earth on the block of granite? It turns out that
this reaction is exerted at the centre of the Earth. In other words, the Earth attracts the
block of granite, and the block of granite attracts the Earth by an equal amount. However,
since the Earth is far more massive than the block, the force
exerted  by the granite block at the centre of the Earth has no observable consequence.

So far, we have established that the weight $W$ of a body is the magnitude of the downward
force it exerts on any object which supports it.  Thus, $W=m\,g$, where $m$ is the
mass of the body and $g$ is the local acceleration due to gravity. Since weight is a force,
it is measured in newtons. A body's weight is location dependent, and is not, therefore, 
an intrinsic property of that body. For instance, a body weighing 10\,N on the surface of
the Earth will only weigh about $3.8\,{\rm N}$ on the surface of Mars, due to the
weaker surface gravity of Mars relative to the Earth.

\begin{figure}
\epsfysize=3in
\centerline{\epsffile{Chapter04/fig025.eps}}
\caption{\em Weight in an elevator}\label{f25}   
\end{figure}

Consider a block of mass $m$ resting on the floor of an elevator, as shown in Fig.~\ref{f25}. 
Suppose that the elevator is accelerating upwards with acceleration $a$. How does this
acceleration affect the weight of the block?
Of course, the block
experiences a downward force $m\,g$ due to gravity. Let $W$ be the weight of the block:
by definition, this is the size of the downward force exerted by the block on the
floor of the elevator. From Newton's third law, the floor of the elevator exerts an upward
reaction force of magnitude $W$ on the block. Let us apply Newton's
second law, Eq.~(\ref{en2}), to the motion of the block. The mass of the block is $m$,
and its upward acceleration is $a$. Furthermore, the block is subject to two forces:
a downward force $m\,g$ due to gravity, and an upward reaction force $W$. Hence,
\begin{equation}
W -m\,g = m\,a.
\end{equation}
This equation
can be rearranged to give
\begin{equation}
W = m\,(g+a).
\end{equation}
Clearly, the upward acceleration of the elevator has the effect of increasing the
weight $W$ of the block: for instance, if the elevator accelerates upwards at $g=9.81\,{\rm m/s^2}$
then the weight of the block is doubled. Conversely, if the elevator accelerates
downward ({\em i.e.}, if $a$ becomes negative) then the weight of the block
is reduced: for instance, if the elevator accelerates downward at $g/2$ then the
weight of the block is halved. Incidentally, these weight changes could easily be measured
by placing some scales between the block and the floor of the elevator.

Suppose that the downward acceleration of the elevator matches the acceleration
due to gravity: {\em i.e.}, $a=-g$. In this case, $W=0$. In other words, the
block becomes weightless! This is the principle behind the so-called
``Vomit Comet'' used by NASA's Johnson Space Centre to train prospective astronauts in
the effects of weightlessness. The ``Vomit Comet'' is actually a KC-135 (a predecessor
of the Boeing 707 which is typically used for refueling military aircraft). The plane
typically ascends to 30,000\,ft and then accelerates downwards at $g$ ({\em i.e.},
drops like a stone) for about 20\,s, allowing its passengers to feel the effects
of weightlessness during this period. All of the weightless scenes
in the film {\em Apollo~11} were shot in this manner.

Suppose, finally, that the downward acceleration of the elevator exceeds
the acceleration due to gravity: {\em i.e.}, $a<-g$. In this case, the block acquires
a negative weight! What actually happens is that the block flies off the
floor of the elevator and slams into the ceiling: when things have settled down, the
block exerts an upward force (negative weight) $|W|$ on the ceiling of the elevator.

\subsection{Strings, Pulleys, and Inclines}\label{sspi}
Consider a block of mass $m$ which is suspended from a fixed beam by means
of a string, as shown in Fig.~\ref{f26}. The string is assumed to be
light ({\em i.e.}, its mass is negligible compared to that of the block) and
inextensible ({\em i.e.}, its length increases by a negligible amount because of the
weight of the block). The string is clearly being stretched, since it is being pulled at
both ends by the block and the beam. Furthermore, the string must be being pulled
by oppositely directed forces of the same magnitude, otherwise it would 
accelerate greatly (given that it has negligible inertia).
By Newton's third law, the string
exerts  oppositely directed
forces of equal magnitude, $T$ (say),  on both the block and the beam.
These forces act so as to 
oppose the stretching of the string:
{\em i.e.}, the beam experiences a downward force of magnitude $T$, whereas the
block experiences an upward force of magnitude $T$. Here, $T$ is termed the
{\em tension} of the string. Since $T$ is a force, it is measured in newtons. 
Note that, unlike a coiled spring, a string can never possess a negative tension, since
this would imply that the string is trying to push its supports apart, rather
than pull them together.

\begin{figure}
\epsfysize=2in
\centerline{\epsffile{Chapter04/fig026.eps}}
\caption{\em Block suspended by a string}\label{f26}   
\end{figure}

Let us apply Newton's second law to the block. The mass of the block is $m$, and
its acceleration is zero, since the block is assumed to be in equilibrium.
The block is subject to two forces, a downward force $m\,g$ due to gravity, and
an upward force $T$ due to the tension of the string. It follows that
\begin{equation}
T - m\,g = 0.
\end{equation}
In other words, in equilibrium, the tension $T$ of the string equals the weight
$m\,g$ of the block.

Figure~\ref{f27} shows a slightly more complicated example in which a block of mass $m$ is
suspended by three strings. The question is what are the tensions, $T$, $T_1$, and $T_2$, in these
strings, assuming that the block is in equilibrium? Using analogous arguments to
the previous case, we can easily  demonstrate that the tension $T$ in the lowermost string is
$m\,g$. The tensions in the two uppermost strings are obtained by applying Newton's
second law of motion to the knot where all three strings meet. See Fig.~\ref{f28}.

\begin{figure}
\epsfysize=2in
\centerline{\epsffile{Chapter04/fig027.eps}}
\caption{\em Block suspended by three strings}\label{f27}   
\end{figure}

There are three forces acting on the knot: the downward force $T$ due to the
tension in the lower string, and the forces $T_1$ and $T_2$ due to the tensions
in the upper strings. The latter two forces act along their respective strings,
as indicate in the diagram.
Since the knot is in equilibrium, the vector sum of all the forces acting
on it must be zero. 

Consider the horizontal components of the
forces acting on the knot. Let components acting to the right be positive, and
{\em vice versa}. 
The horizontal component of tension $T$ is
zero, since this tension acts straight down. The horizontal
component of tension $T_1$ is $T_1\,\cos 60^\circ = T_1/2$, since this
force  subtends an angle of $60^\circ$ with respect to the horizontal (see Fig.~\ref{f16}). 
Likewise, the
horizontal component of tension $T_2$ is $-T_2\,\cos 30^\circ = -\sqrt{3}\,T_2/2$. 
Since the knot does not accelerate in the horizontal direction, we can equate
the sum of these components to zero:
\begin{equation}\label{eaa}
\frac{T_1}{2} -\frac{\sqrt{3}\,T_2}{2} = 0.
\end{equation}

Consider the vertical components of the forces acting on the knot. Let
components acting upward be positive, and
{\em vice versa}. The vertical component of tension $T$ is
$-T=-m\,g$, since this tension acts straight down. The vertical
component of tension $T_1$ is $T_1\,\sin 60^\circ = \sqrt{3}\,T_1/2$, since this
force  subtends an angle of $60^\circ$ with respect to the horizontal (see Fig.~\ref{f16}). 
Likewise, the
vertical component of tension $T_2$ is $T_2\,\sin 30^\circ = T_2/2$. 
Since the knot does not accelerate in the vertical direction, we can equate
the sum of these components to zero:
\begin{equation}\label{ebb}
-m\,g+\frac{\sqrt{3}\,T_1}{2} + \frac{T_2}{2} = 0.
\end{equation}

Finally, Eqs.~(\ref{eaa}) and (\ref{ebb}) yield
\begin{eqnarray}
T_1 &=& \frac{\sqrt{3}\,m\,g}{2},\\[0.5ex]
T_2 &=& \frac{m\,g}{2}.
\end{eqnarray}

\begin{figure}
\epsfysize=2in
\centerline{\epsffile{Chapter04/fig028.eps}}
\caption{\em Detail of Fig.~\ref{f27}}\label{f28}   
\end{figure}

Consider a block of mass $m$ sliding down a smooth frictionless incline
which subtends an angle $\theta$ to the horizontal, as shown in Fig~\ref{f29}.
The weight $m\,g$ of the block is directed vertically downwards. However,
this force can be resolved into components $m\,g\,\cos\theta$, acting
perpendicular (or normal) to the incline, and  $m\,g\,\sin\theta$,
acting parallel to the incline. Note that the reaction of the incline
to the weight of the block acts {\em normal}\/ to the incline, and only
matches the {\em normal component}\/ of the weight: {\em i.e.}, it is
of magnitude $m\,g\,\cos\theta$. This is a general result: the reaction
of any unyielding surface is always locally
normal to that surface, directed outwards (away from the surface),
and matches the normal component of any inward force applied to the surface. 
The block is clearly in equilibrium in the direction normal to
the incline, since the normal component of the block's weight is balanced by
the reaction of the incline. However, the block is subject
to the unbalanced force $m\,g\,\sin\theta$ in the direction parallel
to the incline, and, therefore, accelerates down the slope.
Applying Newton's second law to this problem (with the coordinates shown
in the figure), we obtain
\begin{equation}
m\,\frac{d^2 x}{dt^2} = m\,g\,\sin\theta,
\end{equation}
which can be solved to give
\begin{equation}
x = x_0 + v_0\,t + \frac{1}{2}\,g\,\sin\theta\,t^2.
\end{equation}
In other words, the block accelerates down the slope with
acceleration $g\,\sin\theta$. Note that this acceleration
is {\em less} than the full acceleration due to gravity, $g$. In fact,
if the incline is fairly gentle ({\em i.e.}, if $\theta$ is small) then
the acceleration of the block can be made {\em much less} than $g$. 
This was the technique used by Galileo in his pioneering studies
of motion under gravity---by diluting the acceleration due to gravity,
using inclined planes, he was able to obtain motion sufficiently
slow for him to make accurate measurements using the crude time-keeping
devices available in the 17th Century.

\begin{figure}
\epsfysize=2in
\centerline{\epsffile{Chapter04/fig029.eps}}
\caption{\em Block sliding down an incline}\label{f29}   
\end{figure}

Consider two masses, $m_1$ and $m_2$, connected by a light
inextensible string. Suppose that the first mass slides
over a smooth, frictionless, horizontal table, whilst the
second is suspended over the edge of the table by means
of a light frictionless pulley. See Fig.~\ref{f30}. Since the
pulley is light, we can neglect its rotational inertia in our analysis.
Moreover,  no force is required to turn a frictionless pulley, so we can
assume that the  tension
$T$ of the string is the same on either side of the pulley. Let us
apply Newton's second law of motion to each mass in turn. The first mass
is subject to a downward force $m_1\,g$, due to gravity. However, this force
is completely canceled out by the upward reaction force due to the table.
The mass $m_1$ is also subject to a horizontal force $T$, due to
the tension in the string, which causes it to move {\em rightwards}
with acceleration
\begin{equation}
a = \frac{T}{m_1}.
\end{equation}
The second mass is subject to a downward force $m_2\,g$, due to gravity,
plus an upward force $T$ due to the tension in the string. These forces
cause the mass to move {\em downwards} with acceleration
\begin{equation}
a =g - \frac{T}{m_2}.
\end{equation}
Now, the rightward acceleration of the first mass must match the downward
acceleration of the second, since the string which connects them is
inextensible. Thus, equating the previous two expressions, we obtain
\begin{eqnarray}
T&=& \frac{m_1\,m_2}{m_1+m_2}\,g,\\[0.5ex]
a &=& \frac{m_2}{m_1+m_2}\,g.
\end{eqnarray}
Note that the acceleration of the two coupled masses is {\em less}
than the full acceleration due to gravity, $g$, since the first mass
contributes to the inertia of the system, but does not contribute
to the downward gravitational force which sets the system in motion.

\begin{figure}
\epsfysize=2in
\centerline{\epsffile{Chapter04/fig030.eps}}
\caption{\em Block sliding over a smooth table, pulled by a
second block}\label{f30}   
\end{figure}

Consider two masses, $m_1$ and $m_2$, connected by a light
inextensible string which is suspended from a light frictionless
pulley, as shown in Fig.~\ref{f31}. Let us again apply Newton's
second law to each mass in turn. Without being given the values
of $m_1$ and $m_2$, we cannot determine beforehand which mass is going to
move upwards. Let us  {\em assume} that mass $m_1$ is going to move upwards:
if we are wrong in this assumption then we will simply obtain a negative
acceleration for this mass. The first mass is subject to an
upward force $T$, due to the tension in the string, and a downward
force $m_1\,g$, due to gravity. These forces cause the mass to
move {\em upwards} with acceleration
\begin{equation}
a = \frac{T}{m_1} - g.
\end{equation}
The second mass is subject to a downward force $m_2\,g$, due to gravity,
and an upward force $T$, due to the tension in the string. These forces
cause the mass to move {\em downward} with acceleration
\begin{equation}
a = g - \frac{T}{m_2}.
\end{equation}
Now, the upward acceleration of the first mass must match the downward
acceleration of the second, since they are connected by an inextensible
string. Hence, equating the previous two expressions, we obtain
\begin{eqnarray}
T&=& \frac{2\,m_1\,m_2}{m_1+m_2}\,g,\\[0.5ex]
a &=& \frac{m_2-m_1}{m_1+m_2}\,g.
\end{eqnarray}
As expected, the first mass accelerates upward ({\em i.e.}, $a>0$) if $m_2>m_1$,
and {\em vice versa}. Note that the acceleration of the system is
{\em less} than the full acceleration due to gravity, $g$, since both masses
contribute to the inertia of the system, but their weights partially
cancel one another out. In particular, if the  two
masses are almost equal then the acceleration of the system becomes
very much less than $g$. 

\begin{figure}
\epsfysize=3.5in
\centerline{\epsffile{Chapter04/fig031.eps}}
\caption{\em An Atwood machine}\label{f31}   
\end{figure}

Incidentally, the device pictured in Fig.~\ref{f31} is called an {\em Atwood machine},
after the eighteenth Century English scientist George Atwood, who used it
to ``slow down'' free-fall sufficiently to make accurate observations of this
phenomena using the primitive time-keeping devices available in his day.

\subsection{Friction}
When a body slides over a rough surface a frictional force generally develops which
acts to impede the motion. Friction, when
viewed at the microscopic level, is actually a very complicated phenomenon.
Nevertheless, physicists and engineers have managed to develop
a relatively simple empirical law of force which allows the effects
of friction to be incorporated into their calculations. This law of force
was first proposed by Leonardo da Vinci (1452--1519), and later extended
by Charles Augustin de Coulomb (1736--1806) (who is more famous for discovering the
law of electrostatic attraction). The frictional
force exerted on a body sliding over a rough surface is proportional to
the {\em normal reaction} $R_n$ at that surface, the constant of proportionality
depending on the nature of the surface. In other words,
\begin{equation}
f = \mu\, R_n,
\end{equation}
where $\mu$ is termed the {\em coefficient of (dynamical) friction}. For ordinary
surfaces, 
$\mu$ is generally of order unity. 

Consider a block of mass $m$ being dragged over a horizontal
surface, whose coefficient of friction is $\mu$, by a horizontal
force $F$. See Fig.~\ref{f32}. The weight $W=m\,g$ of the block
acts vertically downwards, giving rise to a reaction $R=m\,g$ acting
vertically upwards. The magnitude of the frictional force $f$, which
impedes the motion of the block, is simply $\mu$ times the
normal reaction $R=m\,g$. Hence, $f=\mu\,m\,g$. The acceleration of
the block is, therefore,
\begin{equation}
a = \frac{F - f}{m} = \frac{F}{m}-\mu\,g,
\end{equation}
assuming that $F>f$. What happens if $F<f$: {\em i.e.}, if the applied
force $F$ is less than the frictional force $f$? In this case, common
sense suggests that the block
simply remains at rest (it certainly does not
accelerate backwards!). Hence, $f=\mu\,m\,g$ is actually the {\em maximum}
force which friction can generate in order to impede the motion of the
block. If the applied force $F$ is less than this maximum value then the 
applied force is canceled out by an equal and opposite frictional force, and
the block remains stationary. Only if the applied force exceeds the
maximum frictional force does the block start to move.

\begin{figure}
\epsfysize=2in
\centerline{\epsffile{Chapter04/fig032.eps}}
\caption{\em Friction}\label{f32}   
\end{figure}

Consider a block of mass $m$ sliding down a rough incline (coefficient of
friction $\mu$)
which subtends an angle $\theta$ to the horizontal, as shown in Fig~\ref{f33}.
The weight $m\,g$ of the block can be resolved into components $m\,g\,\cos\theta$, acting
 normal to the incline, and  $m\,g\,\sin\theta$,
acting parallel to the incline. The reaction of the incline
to the weight of the block acts {\em normally}\/ outwards from  the incline, and  is
of magnitude $m\,g\,\cos\theta$.
 Parallel to the incline, 
the block is subject
to the downward gravitational force $m\,g\,\sin\theta$,
and the upward frictional force $f$ (which acts to prevent the block sliding down the
incline). In order for the block to move, the magnitude of the former force must
exceed the maximum value of the latter, which is $\mu$ time the magnitude
of the normal reaction, or $\mu\,m\,g\,\cos\theta$. Hence, the condition
for the weight of the block to overcome friction, and, thus, to cause the block
to slide down the incline, is
\begin{equation}
m\,g\,\sin\theta > \mu\,m\,g\,\cos\theta,
\end{equation}
or 
\begin{equation}
\tan\theta > \mu.
\end{equation}
In other words, if the slope of the incline exceeds a certain critical value, which
depends on $\mu$, then the block will start to slide. Incidentally, the above formula
suggests a fairly simple way of determining the coefficient of friction for
a given object sliding over a particular surface. Simply tilt the surface gradually
until the object just starts to move: the coefficient of friction is simply the tangent of
the critical tilt angle (measured with respect to the horizontal).

\begin{figure}
\epsfysize=2in
\centerline{\epsffile{Chapter04/fig033.eps}}
\caption{\em Block sliding down a rough slope}\label{f33}   
\end{figure}

Up to now, we have implicitly suggested that the coefficient of friction between an
object and a surface is the same whether the object remains stationary or slides over the surface.
In fact, this is generally not the case. Usually, the coefficient of friction
when the object is stationary is slightly {\em larger} than the coefficient when the object is sliding. We call the former coefficient the
{\em coefficient of static friction}, $\mu_s$, whereas the latter coefficient
is usually termed the {\em coefficient of kinetic (or dynamical) friction}, $\mu_k$. 
The fact that $\mu_s>\mu_k$ simply implies that objects have a
tendency to ``stick'' to rough surfaces when placed upon them. The force required to
unstick a given object, and, thereby, set it in motion, is $\mu_s$ times the
normal reaction at the surface. Once the object has been set in motion,
the frictional force acting to impede this motion falls somewhat to $\mu_k$ times
the normal reaction.

\subsection{Frames of Reference}\label{frames}
As discussed in Sect.~\ref{s1}, the laws of physics are assumed to possess {\em objective reality}. 
In other words, it is assumed that two independent observers, studying the same physical phenomenon,
would eventually formulate {\em identical}\/ laws of physics in order to account for their
observations. Now, two completely independent observers are likely to choose different systems of units
with which to quantify physical measurements. However, as we have seen in Sect.~\ref{s1}, the
dimensional consistency of valid laws of physics renders them {\em invariant} under transformation from
one system of units to another. Independent observers are also likely to choose
different coordinate systems. For instance, the origins of their separate coordinate systems might
differ, as well as the orientation of the various coordinate axes. Are the laws of physics also
{\em invariant} under transformation between coordinate systems possessing different origins,
or a different orientation of the various coordinate axes?

Consider the vector equation
\begin{equation}\label{evec}
{\bf r}  = {\bf r}_1 + {\bf r}_2,
\end{equation}
which is represented diagrammatically in Fig.~\ref{f12}. Suppose that we shift the origin
of our coordinate system, or rotate the coordinate axes. Clearly, in general, the {\em components}
of vectors ${\bf r}$, ${\bf r}_1$, and ${\bf r}_2$ are going to be modified by this change
in our coordinate scheme.
However, Fig.~\ref{f12} still remains valid. Hence, we conclude that the vector
equation (\ref{evec}) also  remains valid. In other words, although the individual
components of vectors  ${\bf r}$, ${\bf r}_1$, and ${\bf r}_2$ are modified by the change in
coordinate scheme, the interrelation between these components expressed in Eq.~(\ref{evec}) remains
invariant.
This observation suggests that the independence of the laws of
physics from the arbitrary choice of  the location of  the underlying coordinate system's origin,
or the equally arbitrary choice of the orientation of the various coordinate axes, can be made
manifest by simply writing these laws as interrelations between vectors. 
In particular, Newton's second law of motion,
\begin{equation}
{\bf f} = m\,{\bf a},
\end{equation}
is clearly invariant under shifts in the origin of our coordinate system, or changes
in the orientation of the various coordinate axes. Note that the quantity $m$
({\em i.e.}, the mass of the body whose motion is under investigation),
appearing in the above equation, is invariant under any changes in the coordinate
system, since measurements
of mass are completely independent of measurements of distance. We refer to
such a quantity as a {\em scalar} (this is an improved definition). 
We conclude that valid laws
of physics must consist of combinations of scalars and vectors, otherwise
they would retain an unphysical dependence on the details of the chosen coordinate system.

Up to now, we have implicitly assumed that all of our observers are {\em stationary}
({\em i.e.}, they are all standing still on the surface of the Earth). Let us, now,
relax this assumption. Consider two observers, $O$ and $O'$, whose coordinate systems
coincide momentarily at $t=0$. Suppose that observer $O$ is stationary (on the surface of the
Earth), whereas observer $O'$ moves (with respect to observer $O$) with
{\em uniform} velocity ${\bf v}_0$.   As illustrated in Fig.~\ref{f34}, if ${\bf r}$ represents
the displacement of some body $P$ in the stationary observer's frame of reference, at time $t$, then
the corresponding displacement in the moving observer's frame of reference is simply
\begin{equation}
{\bf r}' = {\bf r} - {\bf v}_0\,t.
\end{equation}
The velocity of body $P$ in the stationary observer's frame of reference is defined as
\begin{equation}
{\bf v} = \frac{d{\bf r}}{dt}.
\end{equation}
Hence, the corresponding velocity in the moving observer's frame of reference takes the form
\begin{equation}
{\bf v}' = \frac{ d{\bf r}'}{dt} = {\bf v} - {\bf v}_0.
\end{equation}
Finally, the acceleration of body $P$ in  stationary observer's frame of reference is
defined as
\begin{equation}
{\bf a} = \frac{d{\bf v}}{dt},
\end{equation}
whereas the corresponding acceleration in the moving observer's frame of reference takes the form
\begin{equation}
{\bf a}' = \frac{ d{\bf v}'}{dt} = {\bf a}.
\end{equation}
Hence, the acceleration of body $P$ is {\em identical} in both frames of reference.

 \begin{figure}
\epsfysize=2in
\centerline{\epsffile{Chapter04/fig034.eps}}
\caption{\em A moving observer}\label{f34}   
\end{figure}

It is clear that if observer $O$ concludes that body $P$ is moving with constant
velocity, and, therefore, subject to zero net force, then observer $O'$ will agree
with this conclusion. Furthermore, if observer $O$ concludes that body $P$
is accelerating, and, therefore, subject to a force ${\bf a}/m$, then observer
$O'$ will remain in agreement. It follows that Newton's laws of motion
are equally valid in the frames of reference of the moving and the stationary observer. 
Such frames are termed {\em inertial frames of reference}. There are infinitely many
inertial frames of reference---within which Newton's laws of motion are equally valid---all
moving with {\em constant} velocity with respect to one another. Consequently, there is no universal
standard of rest in physics. Observer $O$ might claim to be at rest compared to observer $O'$,
and {\em vice versa}: however, both points of view are equally valid. Moreover, there is 
absolutely no physical
experiment which observer $O$ could perform in order to demonstrate that he/she is at
rest whilst observer $O'$ is moving. This, in essence, is the principle of {\em special
relativity}, first formulated by Albert Einstein in 1905. 

\subsection*{\em Worked Example 4.1: In Equilibrium}
{\em Question:} Consider the diagram. If the system is in equilibrium, and the
tension in string 2 is $50\,{\rm N}$, determine the mass $M$.

\begin{figure*}[h]
\epsfysize=1.5in
\centerline{\epsffile{Chapter04/fig034a.eps}}
\end{figure*}

\noindent {\em Answer:} It follows from symmetry that the tensions in strings 1 and 3 are equal. 
Let $T_1$ be the tension in string 1, and  $T_2$  the
tension in string 2. Consider the equilibrium of the knot above the leftmost mass.
As shown below, this knot is subject to three forces: the downward force $T_4=M\,g$
due to the tension in the string which directly supports the leftmost mass, the rightward force
$T_2$ due to the tension in string 2, and the upward and leftward force
$T_1$ due to the tension in string 1. The resultant of all these forces must be zero, otherwise
the system would not be in equilibrium. Resolving in the horizontal direction (with rightward
forces positive), we obtain
$$
T_2 - T_1\,\sin 40^\circ = 0.
$$
Likewise, resolving in the vertical direction (with upward forces positive) yields
$$
T_1\,\cos 40^\circ - T_4 = 0.
$$
Combining the above two expressions, making use of the fact that
$T_4 = M\,g$, gives
$$
M = \frac{T_2}{g\,\tan 40^\circ}.
$$
Finally, since $T_2 =50\,{\rm N}$ and $g=9.81\,{\rm m/s^2}$, 
we obtain
$$
M = \frac{50}{9.81\times 0.8391} = 6.074\,{\rm kg}.
$$

\begin{figure*}[h]
\epsfysize=1.5in
\centerline{\epsffile{Chapter04/fig034b.eps}}
\end{figure*}

\subsection*{\em Worked Example 4.2: Block Accelerating up a Slope}
{\em Question:} Consider the diagram. Suppose that the block, mass
$m= 5\,{\rm kg}$, is subject to a horizontal force $F= 27\,{\rm N}$. 
What is the acceleration of the block up the (frictionless) slope?


\begin{figure*}[h]
\epsfysize=1.5in
\centerline{\epsffile{Chapter04/fig034c.eps}}
\end{figure*}

\noindent{\em Answer:} Only that component of the applied force which is
parallel to the incline has any influence on the block's motion: the normal
component of the applied force is canceled out by the normal reaction
of the incline. The component of the applied force acting up the
incline is $F\,\cos 25^\circ$. Likewise, the component of the block's
weight acting down the incline is $m\,g\,\sin 25^\circ$. Hence, using
Newton's second law to determine the acceleration $a$ of the block
up the incline, we obtain
$$
a = \frac{F\,\cos 25^\circ - m\,g\,\sin 25^\circ}{m}.
$$
Since $m = 5\,{\rm kg}$ and $F= 27\,{\rm N}$, we have
$$
a = \frac{27\times 0.9063 - 5\times 9.81\times 0.4226}{5} = 0.7483\,{\rm m/s^2}.
$$

\subsection*{\em Worked Example 4.3: Raising a Platform}
{\em Question:} Consider the diagram. The platform and the attached frictionless
pulley weigh a total of 34\,N. With what force $F$ must the (light) rope be pulled
in order to lift the platform at $3.2\,{\rm m/s^2}$? 

\begin{figure*}[h]
\epsfysize=2.5in
\centerline{\epsffile{Chapter04/fig034d.eps}}
\end{figure*}

\noindent{\em Answer:} Let $W$ be the weight of the platform, $m=W/g$ the mass
of the platform, and $T$ the tension in the  rope. From Newton's third law, it is clear that $T=F$.
 Let us apply Newton's second law to the upward motion
of the platform. The platform is subject to two vertical forces: a downward
force $W$ due to its  weight, and an upward force $2\,T$ due to
the tension in the rope (the force is $2\,T$, rather than $T$, because both
the leftmost and rightmost sections of the rope, emerging from the pulley,
are in tension and exerting an upward force on the pulley). Thus, the
upward acceleration $a$ of the platform is
$$
a = \frac{2\,T-W}{m}.
$$
Since $T=F$ and $m=W/g$, we obtain
$$
F = \frac{W\,(a/g+1)}{2}.
$$
Finally, given that $W=34\,{\rm N}$ and $a=3.2\,{\rm m/s^2}$, we have
$$
F = \frac{34\,(3.2/9.81+1)}{2} = 22.55\,{\rm N}.
$$

\subsection*{\em Worked Example 4.4: Suspended Block}
{\em Question:} Consider the diagram. The mass of block $A$ is $75\,{\rm kg}$ and
the mass of block $B$ is $15\,{\rm kg}$. The coefficient of static friction
between the two blocks is $\mu = 0.45$. The horizontal surface is frictionless. What 
minimum force
$F$ must be exerted on block $A$ in order to prevent block $B$ from
falling?

\begin{figure*}[h]
\epsfysize=1.5in
\centerline{\epsffile{Chapter04/fig034e.eps}}
\end{figure*}

\noindent {\em Answer:} Suppose that block $A$ exerts a rightward force $R$
on block $B$. By Newton's third law, block $B$ exerts an equal and opposite force
on block $A$. Applying Newton's second law of motion to the rightward
acceleration $a$ of block $B$, we obtain
$$
a = \frac{R}{m_B},
$$
where $m_B$ is the mass of block $B$. The normal reaction at the interface between
the two blocks is $R$. Hence, the maximum frictional force that block $A$
can exert on block $B$ is $\mu\,R$. In order to prevent block $B$ from falling,
this maximum frictional force (which acts upwards) must exceed the downward acting weight, $m_B\, g$,
of the block. Hence, we require
$$
\mu\,R > m_B\,g,
$$
or 
$$
a > \frac{g}{\mu}.
$$
Applying Newton's second law to the rightward acceleration $a$ of both blocks (remembering that the
equal and opposite forces exerted between the blocks cancel one another out), we obtain
$$
a = \frac{F}{m_A+m_B},
$$
where $m_A$ is the mass of block $A$. It follows that
$$
F > \frac{(m_A+m_B)\,g}{\mu}.
$$
Since $m_A=75\,{\rm kg}$, $m_B = 15\,{\rm kg}$, and $\mu= 0.45$, we have
$$
F > \frac{(75+15)\times 9.81}{0.45} = 1.962\times 10^3\,{\rm N}.
$$
