\section{Conservation of Energy}\label{senergy}
\subsection{Introduction}
Nowadays, the {\em conservation of energy} is undoubtedly the single most important idea in physics.
Strangely enough, although the basic idea of energy conservation was familiar to
scientists from the time of Newton onwards,
 this crucial concept only moved to centre-stage  in physics in about 1850 ({\em i.e.}, when
scientists first realized that heat was a form of energy).

According to the ideas of modern physics, {\em energy} is the substance from which
all things in the Universe are made up. Energy can take many different forms: {\em e.g.},
potential energy, kinetic energy, electrical energy, thermal energy, chemical
energy, nuclear energy, {\em etc}. In fact, everything that we observe in the
world around us represents one of the multitudinous manifestations of energy.
Now, there exist processes in the Universe which transform energy from one form
into another: {\em e.g.}, mechanical processes (which are the focus of this course), thermal
processes, electrical processes, nuclear processes, {\em etc}. However, all of
these processes leave the {\em total} amount of energy in the Universe {\em invariant}. In other words,
whenever, and however, energy is transformed from one form into another, it is always
{\em conserved}. For a closed system ({\em i.e.}, a system which does not exchange
energy with the rest of the Universe), the above law of universal energy conservation implies that the
total energy of the system in question must remain constant in time.

\subsection{Energy Conservation During Free-Fall}
Consider a mass $m$ which is falling vertically under the influence of gravity.
We already know how to analyze the motion of such a mass. Let us employ this knowledge to
search for
an expression for the conserved energy during  this process. ({\em N.B.}, This is
clearly an example of a closed system, involving only the mass and the gravitational
field.)
The physics of
free-fall under gravity is summarized by the three equations (\ref{e215})--(\ref{e218}).
Let us examine the last of these equations: $v^2 = v_0^{\,2} - 2\,g\,s$.
Suppose that the mass falls from height $h_1$ to $h_2$,  its initial velocity
is $v_1$, and its final velocity is $v_2$. It follows that the net vertical displacement of the
mass is $s=h_2-h_1$. Moreover, $v_0=v_1$ and $v=v_2$. Hence, the previous expression
can be rearranged to give
\begin{equation}\label{e51}
\frac{1}{2}\,m\,v_1^{\,2} + m\,g\,h_1 = \frac{1}{2}\,m\,v_2^{\,2} + m\,g\,h_2.
\end{equation}
The above equation clearly represents a conservation law, of some description,
since the left-hand side only contains quantities evaluated at the initial height,
whereas the right-hand side only contains quantities evaluated at the final height. 
In order to clarify the meaning of Eq.~(\ref{e51}), let us define the
{\em kinetic energy} of the mass,
\begin{equation}\label{e52}
K = \frac{1}{2}\,m\,v^2,
\end{equation}
and the {\em gravitational potential energy} of the mass,
\begin{equation}
U = m\,g\,h.\label{e53}
\end{equation}
Note that kinetic energy represents energy the mass possesses by virtue of
its {\em motion}. Likewise, potential energy represents energy the mass
possesses by virtue of its {\em position}. 
It follows that Eq.~(\ref{e51}) can be written
\begin{equation}
E = K+ U= {\rm constant}.
\end{equation}
Here, $E$ is the {\em total} energy of the mass: {\em i.e.}, the sum of its
kinetic and potential energies. It is clear that $E$ is a conserved quantity:
{\em i.e.}, although the kinetic and potential energies of the mass vary
as it falls, its total energy  remains the same.

Incidentally, the expressions (\ref{e52}) and (\ref{e53}) for kinetic and gravitational potential energy,
respectively, are quite general, and do not just apply to free-fall under gravity. 
The mks unit of energy is  called the {\em joule} (symbol J). In fact, 1 joule
is equivalent to 1 kilogram meter-squared per second-squared, or 1 newton-meter. Note that
all forms of energy are measured in the {\em same} units (otherwise the idea of energy conservation
would make no sense). 

One of the most important lessons which   students learn during their studies
is that there are generally many different paths to the same
result in physics. Now, we have already analyzed free-fall under gravity using Newton's
laws of motion. However, it is illuminating to re-examine this problem from the
point of view of energy conservation. Suppose that a mass $m$ is dropped from rest and
falls a distance $h$. What is the final velocity $v$ of the mass? Well,
according to Eq.~(\ref{e51}), if energy is conserved then
\begin{equation}\label{e5.5}
{\mit\Delta}K = - {\mit\Delta}U:
\end{equation}
{\em i.e.}, any increase in the kinetic energy of the mass must be offset
by a corresponding  decrease in its potential energy. Now, the change in potential
energy of the mass is simply ${\mit\Delta}U=m\,g\,s=-m\,g\,h$, where $s=-h$ is its net
vertical displacement. The change in kinetic energy is simply
${\mit\Delta}K=(1/2)\,m\,v^2$, where $v$ is the final velocity. This follows because the
initial kinetic energy of the mass is zero (since it is initially at rest). Hence,
the above expression yields
\begin{equation}
\frac{1}{2}\,m\,v^2 = m\,g\,h,
\end{equation}
or
\begin{equation}
v = \sqrt{2\,g\,h}.
\end{equation}

Suppose that the same mass is thrown upwards with initial velocity $v$. What is the
maximum height $h$ to which it rises? Well, it is clear from Eq.~(\ref{e53}) that as
the mass rises its potential energy {\em increases}. It, therefore,  follows from
energy conservation that its kinetic energy must {\em decrease} with height. Note, however,
from Eq.~(\ref{e52}), that kinetic energy can never be negative (since it
is the product of the two positive definite quantities, $m$ and $v^2/2$). Hence, once the
mass has risen to a height $h$ which is such that its kinetic energy is reduced to {\em zero} 
it can rise no further, and must, presumably, start to fall. The change in potential
energy of the mass in moving from its initial height to its maximum height is
$m\,g\,h$. The corresponding change in kinetic energy is $-(1/2)\,m\,v^2$; since
$(1/2)\,m\,v^2$ is the initial kinetic energy, and the final kinetic energy is zero.
It follows from Eq.~(\ref{e5.5}) that $-(1/2)\,m\,v^2 = -m\,g\,h$, which can be
rearranged to give
\begin{equation}
h = \frac{v^2}{2\,g}.
\end{equation}

It should be noted that the idea of energy conservation---although extremely useful---is
{\em not} a replacement for Newton's laws of motion. For instance, in the previous example, there
is no way in which we can deduce {\em how long} it takes the mass to rise to its
maximum height from energy conservation alone---this information can only come from the 
direct application of Newton's laws.

\subsection{Work}
We have seen that when a mass free-falls under the influence of gravity some
of its kinetic energy is transformed into potential energy, or {\em vice versa}. 
Let us now investigate, in detail, how this transformation is effected. 
The mass falls because it is subject to a downwards gravitational force of magnitude
$m\,g$. It stands to reason, therefore, that the transformation of kinetic into potential
energy is a direct consequence of the action of this force. 

This is, perhaps,
an appropriate point at which to note that the concept of gravitational potential energy---although 
extremely useful---is, strictly speaking, {\em fictitious}.
To be more exact, the potential energy of a body is not 
an intrinsic property of that body (unlike its kinetic energy). In fact, the
gravitational  potential energy of
a given body is stored in the {\em gravitational field} which surrounds it. Thus,
when the body rises, and its potential energy consequently increases by an amount
${\mit\Delta}U$; in reality, it is 
the energy of the gravitational field surrounding the body which increases by this amount.
Of course, the increase in energy of the gravitational field is offset by a corresponding decrease in the
body's kinetic energy. Thus, when we speak of a body's kinetic energy being transformed
into potential energy, we are really talking about a flow of energy {\em from} the body {\em to}
the surrounding gravitational field. This energy flow is mediated by the
gravitational force exerted by the field on the body in question. 

Incidentally,
according to Einstein's general theory of relativity (1917), the gravitational field
of a mass consists of the local distortion that mass induces in the fabric of space-time. Fortunately,
however,
we do not need to understand general relativity in order to talk about gravitational fields or
gravitational potential energy. All we need to know is that a gravitational field stores
energy {\em without loss}: {\em i.e.}, if a given mass rises a certain distance, and, thereby,
gives up a certain amount of energy to the surrounding gravitational field, then that
 field will return this energy to the mass---without
loss---if the
mass falls by the same distance. In physics, we term such a field a {\em conservative field}
(see later). 

Suppose that a mass $m$ falls a distance $h$. During this process, the energy of
the gravitational field decreases by a certain amount ({\em i.e.}, the fictitious
potential energy of the mass decreases by a certain  amount), and the body's
kinetic energy increases by a corresponding amount. This transfer of
energy, from the field to the mass,
 is, presumably, mediated by the gravitational force $-m\,g$ (the minus sign
indicates that the force is directed downwards) acting on the mass. In fact,
given that $U=m\,g\,h$, it follows from Eq.~(\ref{e5.5}) that
\begin{equation}\label{e59}
{\mit\Delta}K = f\,{\mit\Delta h}.
\end{equation}
In other words, the amount of energy transferred to the mass ({\em i.e.}, the increase in the
mass's kinetic energy) is equal to the product of the force acting on the
mass and the distance moved by the mass {\em in the direction of that force}. 

In physics, we generally refer to the
amount of energy transferred to a body, when a force acts upon it, as
the amount of {\em work} $W$ performed by that force on the body in question. It follows from
Eq.~(\ref{e59}) that when a gravitational force $f$ acts on a body,
causing it to displace a distance $x$ {\em in the direction of that force},
then the net work done on the body is
\begin{equation}\label{e510}
W = f\,x.
\end{equation}
It turns out that this equation is quite general, and does not just apply to
gravitational forces. If $W$ is positive then energy is transferred to the body, and
its intrinsic energy consequently increases by an amount $W$. This situation occurs whenever
a body moves in the {\em same} direction as the force acting upon it.
 Likewise,
if $W$ is negative then energy is transferred from the body, and its 
intrinsic energy consequently decreases  by an amount $|W|$. This situation occurs whenever
a body moves in the {\em opposite} direction to the force acting upon it.
Since an amount of work
is equivalent to a transfer of energy, the mks unit of work is the same as the
mks unit of energy: namely, the joule.

In deriving equation (\ref{e510}), we have made two assumptions which are not universally
valid. Firstly, we have assumed that the motion of the body upon which the force acts is
both 1-dimensional and parallel to the line of action of the force.
 Secondly, we have assumed that the force does not vary with position. 
Let us attempt to relax these two assumptions, so as to obtain an expression for the
work $W$ done by a general force ${\bf f}$. 

Let us start by relaxing the first assumption. Suppose, for the sake of argument, that
we have a mass $m$ which moves under gravity in 2-dimensions. Let us adopt
the coordinate system shown in Fig.~\ref{f35}, with $z$ representing vertical
distance, and $x$ representing horizontal distance. The vector
acceleration of the mass is simply ${\bf a}=(0,-g)$. Here, we are neglecting the redundant
$y$-component, for the sake of simplicity. The physics of motion under gravity in more
than 1-dimension is summarized by the three equations (\ref{e335})--(\ref{e338}). Let us examine
the last of these equations:
\begin{equation}
v^2 = v_0^{\,2} + 2\,{\bf a}\!\cdot\!{\bf s}.
\end{equation}
Here, $v_0$ is the speed at $t=0$, $v$ is the speed at $t=t$, and
${\bf s} = ({\mit\Delta}x$, ${\mit\Delta}z)$ is the net displacement of the mass
during this time interval. Recalling the definition of a scalar product 
[{\em i.e.}, ${\bf a}\!\cdot\!{\bf b}= (a_x\,b_x+ a_y\,b_y+a_z\,b_z)$], the above equation
can be rearranged to give
\begin{equation}\label{e511}
\frac{1}{2}\,m\,v^2 - \frac{1}{2}\,m\,v_0^{\,2} = -m\,g\,{\mit\Delta}z.
\end{equation}
Since the right-hand side of the above expression is manifestly the increase in the
kinetic energy of the mass between times $t=0$ and $t=t$, the left-hand side
must equal the decrease in the mass's potential energy during the same time interval. 
Hence, we arrive at the following expression for the gravitational potential energy
of the mass:
\begin{equation}
U = m\,g\,z.
\end{equation}
Of course, this expression is entirely equivalent to our previous expression for
gravitational potential energy, Eq.~(\ref{e53}). The above expression merely
makes manifest a point which should have been obvious anyway: namely, that the
gravitational potential energy of a mass only depends on its height above
the ground, and is quite independent of its horizontal displacement.

\begin{figure}
\epsfysize=1.5in
\centerline{\epsffile{Chapter05/fig035.eps}}
\caption{\em Coordinate system for 2-dimensional motion under gravity}\label{f35}   
\end{figure}

Let us now try to relate the flow of energy between the gravitational field and the mass
to the action of the gravitational force, ${\bf f} = (0,-m\,g)$. Equation~(\ref{e511})
can be rewritten
\begin{equation}\label{e514}
{\mit\Delta} K = W = {\bf f}\!\cdot {\bf s}.
\end{equation}
In other words, the work $W$ done by the force ${\bf f}$ is equal to the scalar product
of ${\bf f}$ and the vector displacement ${\bf s}$ of the body upon which the force 
acts. It turns out that this result is quite general, and does not just apply
to gravitational forces. 

Figure~\ref{f36} is a visualization of the definition (\ref{e514}). The work $W$
performed by a force ${\bf f}$ when the object upon which it acts is subject to a displacement
${\bf s}$ is
\begin{equation}
W = |{\bf f}|\,\,|{\bf s}|\,\cos\theta.\,
\end{equation}
where $\theta$ is the angle subtended between the directions of ${\bf f}$ and ${\bf s}$. 
In other words, the work performed is the product of the magnitude of the force, $|{\bf f}|$,
and the displacement of the object {\em in the direction of that force}, $|{\bf s}|\,\cos\theta$.
It follows that any component of the displacement in a direction perpendicular to the force
generates zero work. Moreover, if the displacement is entirely perpendicular to the direction
of the force ({\em i.e.}, if $\theta=90^\circ$) then no work is performed, irrespective of
the nature of the force. As before, if the displacement has a  component in the same direction
as the force ({\em i.e.}, if $\theta<90^\circ$) then  positive work is performed
Likewise, if the 
displacement has a  component in the opposite direction
to the force ({\em i.e.}, if $\theta>90^\circ$) then negative work is performed.

\begin{figure}
\epsfysize=2in
\centerline{\epsffile{Chapter05/fig036.eps}}
\caption{\em Definition of work}\label{f36}   
\end{figure}

Suppose, now, that an object is subject to a force ${\bf f}$ which varies with position. What
is the total work done by the force  when the object moves along some
general trajectory in space between points $A$ and $B$ (say)? See Fig.~\ref{f37}.
Well, one way in which we could approach this problem would be to approximate the trajectory
as a series of $N$ straight-line segments, as shown in Fig.~\ref{f38}. Suppose that the
vector displacement of the $i$th segment is ${\mit\Delta}{\bf r}_i$. Suppose, further,
that $N$ is sufficiently large that the force ${\bf f}$ does not vary much
along each segment. In fact, let the average force along the $i$th segment be
${\bf f}_i$. We shall assume that formula (\ref{e514})---which is valid for constant
forces and straight-line displacements---holds good for each segment. It follows that
the net work done on the body, as it moves from point $A$ to
point $B$, is approximately
\begin{equation}
W \simeq \sum_{i=1}^{N} {\bf f}_i\!\cdot\!{\mit\Delta}{\bf r}_i.
\end{equation}
We can always improve the level of our approximation by increasing the number $N$ of the
straight-line segments which we use to approximate the  body's trajectory
between points $A$ and $B$. In fact, if we take the limit $N\rightarrow\infty$
then the above expression becomes exact:
\begin{equation}\label{e517}
W = \lim_{N\rightarrow\infty}  \sum_{i=1}^{N} {\bf f}_i\!\cdot\!{\mit\Delta}{\bf r}_i
= \int_A^B {\bf f}({\bf r})\!\cdot\!d{\bf r}.
\end{equation}
Here, ${\bf r}$ measures vector displacement from the origin of our
coordinate system, and the mathematical
construct $\int_A^B {\bf f}({\bf r})\!\cdot\!d{\bf r}$ is termed a {\em line-integral}.

\begin{figure}
\epsfysize=2in
\centerline{\epsffile{Chapter05/fig037.eps}}
\caption{\em Possible trajectory of an object in a variable force-field}\label{f37}   
\end{figure}

\begin{figure}
\epsfysize=2in
\centerline{\epsffile{Chapter05/fig038.eps}}
\caption{\em Approximation to the previous trajectory using straight-line segments}\label{f38}   
\end{figure}

The meaning of Eq.~(\ref{e517}) becomes a lot clearer if we restrict our
attention to 1-dimensional motion. Suppose, therefore, that an object moves
in 1-dimension, with displacement $x$, and is subject to a varying force $f(x)$
(directed along the $x$-axis). What is the work done by this force when
the object moves from $x_A$ to $x_B$? Well, a straight-forward
application of Eq.~(\ref{e517}) [with ${\bf f} = (f,0,0)$ and $d{\bf r}=(dx,0,0)$]
yields
\begin{equation}\label{eaax}
W = \int_{x_A}^{x_B} f(x)\,dx.
\end{equation}
In other words, the net work done by the force  as the object moves
from displacement $x_A$ to $x_B$ is simply the area under the $f(x)$ curve
between these two points, as illustrated in Fig.~\ref{f39}.

\begin{figure}
\epsfysize=2in
\centerline{\epsffile{Chapter05/fig039.eps}}
\caption{\em Work performed by a 1-dimensional force}\label{f39}   
\end{figure}

Let us, finally, round-off this discussion by re-deriving the so-called {\em work-energy theorem},
Eq.~(\ref{e514}), in 1-dimension, allowing for a non-constant force. 
According to Newton's second law of motion,
\begin{equation}\label{ebbx}
f = m\,\frac{d^2x}{dt^2}.
\end{equation}
Combining Eqs.~(\ref{eaax}) and (\ref{ebbx}), we obtain
\begin{equation}
W = \int_{x_A}^{x_B} m\,\frac{d^2 x}{dt^2}\,dx = \int_{t_A}^{t_B} m\,\frac{d^2 x}{dt^2}\frac{dx}{dt}\,dt
= \int_{t_A}^{t_B} \frac{d}{dt}\!\left[\frac{m}{2}\!\left(\frac{dx}{dt}\right)^2\right]\,dt,
\end{equation}
where $x(t_A)=x_A$ and $x(t_B)=x_B$.
It follows that 
\begin{equation}
W = \frac{1}{2}\,m\,v_B^{\,2} -\frac{1}{2}\,m\,v_A^{\,2} = {\mit\Delta}K,
\end{equation}
where $v_A = (dx/dt)_{t_A}$ and $v_B = (dx/dt)_{t_B}$. Thus, the net work performed on a
body by a non-uniform force,  as it moves from point $A$ to point $B$, is equal
to the net  increase in that body's kinetic energy  between these two points. This result
is completely general (at least, for conservative force-fields---see later), and does not just apply to 1-dimensional motion.

Suppose, finally, that an object is subject to more than one force. How do we calculate the
net work $W$ performed by all these forces as the object moves from point $A$ to
point $B$? One approach would be to calculate the work done by each force, taken
in isolation, and then to sum the results. In other words, defining
\begin{equation}
W_i = \int_A^B {\bf f}_i({\bf r})\!\cdot\!d{\bf r}
\end{equation}
as the work done by the $i$th force, the net work is given by
\begin{equation}
W = \sum_i W_i.
\end{equation}
An alternative approach would be to take the vector sum of all the forces to find the
resultant force,
\begin{equation}
{\bf f} = \sum_i {\bf f}_i,
\end{equation}
and then to calculate the work done by the resultant force:
\begin{equation}
W = \int_A^B {\bf f}({\bf r})\!\cdot\!d{\bf r}.
\end{equation}
It should, hopefully, be clear that these two approaches are entirely equivalent.

\subsection{Conservative and Non-Conservative Force-Fields}
Suppose that a non-uniform force-field ${\bf f}({\bf r})$  acts upon
an object which moves along a curved trajectory, labeled path 1,  from point $A$ to
point $B$. See Fig.~\ref{f40}. As we have seen, the work $W_1$ performed by the force-field
on the object can be written as a line-integral along this trajectory:
\begin{equation}\label{line1}
W_1 = \int_{A\rightarrow B: {\rm path} 1} {\bf f}\!\cdot\!d{\bf r}.
\end{equation}
Suppose that the same object moves along a different trajectory, labeled path
2, between the same two points. In this case, the work $W_2$ performed by the force-field is
\begin{equation}\label{line2}
W_2 = \int_{A\rightarrow B:{\rm path} 2} {\bf f}\!\cdot\!d{\bf r}.
\end{equation}
Basically, there are  two possibilities. Firstly, the line-integrals (\ref{line1}) and
(\ref{line2}) might depend on the end points, $A$ and $B$, but {\em not} on the path
taken between them, in which case $W_1=W_2$. Secondly, the line-integrals (\ref{line1}) and
(\ref{line2}) might depend both on the end points, $A$ and $B$, and the path
taken between them, in which case $W_1\neq W_2$ (in general). The first possibility
corresponds to what physicists term a {\em conservative} force-field, whereas the
second possibility corresponds to a {\em non-conservative}  force-field.

\begin{figure}
\epsfysize=2.5in
\centerline{\epsffile{Chapter05/fig040.eps}}
\caption{\em Two alternative paths between points $A$ and  $B$}\label{f40}   
\end{figure}

What is the {\em physical} distinction between a conservative and a non-conservative
force-field? Well, the easiest way of answering this question is to
slightly modify the problem discussed above. Suppose, now, that  the object moves from
point $A$ to point $B$ along path 1, and then from point $B$ back to point $A$
along path 2. What is the total work done on the object by the force-field as
it executes this closed circuit? Incidentally, one fact which should be clear from the
definition of a line-integral is that if we simply reverse the path of a given
integral then the value of that integral picks up a minus sign: in other
words,
\begin{equation}
\int_A^B {\bf f}\!\cdot\!d{\bf r} = - \int_B^A {\bf f}\!\cdot\!d{\bf r},
\end{equation}
where it is understood that both the above integrals are taken in {\em opposite} directions
along the {\em same} path. Recall that conventional 1-dimensional integrals
obey an analogous rule: {\em i.e.}, if we swap the limits of
integration then the integral picks up a minus sign. 
It follows that the total work done on the object as it executes
the circuit is simply
\begin{equation}
{\mit\Delta}W = W_1 - W_2,
\end{equation}
where $W_1$ and $W_2$ are defined in Eqs.~(\ref{line1}) and (\ref{line2}), respectively.
There is a minus sign in front of $W_2$ because we are moving from point $B$ to point
$A$, instead of the other way around. For the case of a conservative field, we have
$W_1=W_2$. Hence, we conclude that
\begin{equation}
{\mit\Delta}W = 0.
\end{equation}
In other words, the net work done by a conservative field on an object taken around
a closed loop is {\em zero}. This is just another way of saying that a conservative
field stores energy without loss: {\em i.e.}, if an object gives up a
certain amount of energy to a conservative field in traveling from point $A$ to point $B$, then the
field returns this energy to the object---without loss---when it travels back
to point $B$. For the case of a non-conservative field, $W_1\neq W_2$. Hence, we
conclude that
\begin{equation}
{\mit\Delta}W \neq 0.
\end{equation}
In other words, the net work done by a non-conservative field on an object taken around
a closed loop is non-zero. In practice, the net work is invariably {\em negative}. 
This is just another way of saying that a non-conservative field {\em dissipates energy}: 
{\em i.e.}, if an object gives up a
certain amount of energy to a non-conservative field in traveling from point $A$ to point $B$, then the
field only returns part, or, perhaps, none, of this energy to the object when it travels back
to point $B$. The remainder is usually dissipated as heat.

What are typical examples of conservative and non-conservative fields? Well, a gravitational
field is probably the most well-known example of a conservative field (see later).
A typical example of a non-conservative field might consist of an
object moving over a rough horizontal surface. Suppose, for the sake of simplicity,
that the object executes a closed circuit on the surface which is made
up entirely of straight-line segments, as shown in Fig.~\ref{f41}.
Let ${\mit\Delta}{\bf r}_i$ represent the vector displacement of the $i$th leg
of this circuit. Suppose that the frictional force acting on the object as it executes this
leg is ${\bf f}_i$. One thing that we know about a frictional force
is that it is always directed in  the opposite direction to the instantaneous
direction of motion of the object upon which it acts. Hence, ${\bf f}_i\propto
-{\mit\Delta}{\bf r}_i$. It follows that ${\bf f}_i\!\cdot \! {\mit\Delta}{\bf r}_i
= - |{\bf f}_i|\,|{\mit\Delta}{\bf r}_i|$. Thus, the net work
performed by the frictional force on the object, as it executes the circuit, is
given by
\begin{equation}
{\mit\Delta}W = \sum_i {\bf f}_i\!\cdot\!{\mit\Delta}{\bf r}_i = 
-\sum_i |{\bf f}_i| \,|{\mit\Delta}{\bf r}_i| < 0.
\end{equation}
The fact that the net work is negative indicates that the frictional force continually
drains energy from the object as it moves over the surface. This energy
is actually dissipated as heat (we all know that if we rub two rough surfaces
together, sufficiently vigorously, then they will eventually heat up: this is
how mankind first made fire) and is, therefore, lost to the
system. (Generally speaking, the laws of {\em thermodynamics} forbid energy which has been
converted into heat from being converted back to its original form.)
Hence, friction is an example of a non-conservative force, because it dissipates
energy rather than storing it.

\begin{figure}
\epsfysize=2.5in
\centerline{\epsffile{Chapter05/fig041.eps}}
\caption{\em Closed circuit over a rough horizontal surface}\label{f41}   
\end{figure}

\subsection{Potential Energy}\label{spotn}
Consider a body moving in a conservative force-field ${\bf f}({\bf r})$. Let us
arbitrarily pick some point $O$ in this field. We can define a
function $U({\bf r})$ which possesses a unique value at every point in
the field. The value of this function associated with some
general point $R$ is simply
\begin{equation}\label{epot}
U(R) = -\int_O^R {\bf f}\!\cdot\!d{\bf r}.
\end{equation}
In other words, $U(R)$ is just the energy transferred to the field ({\em i.e.}, minus
the work done by the field) when the
body moves from point $O$ to point $R$. Of course, the value of $U$ at point
$O$ is zero: {\em i.e.}, $U(O)=0$. Note that the above definition {\em uniquely} specifies $U(R)$, since
the work done when a body moves between two points in a conservative
force-field is {\em independent} of the path taken between these points. Furthermore,
the above definition would make no sense in a non-conservative field, since
the work done when a body moves between two points in such a
field is dependent on the chosen path: hence, $U(R)$ would have an infinite number
of different values corresponding to the infinite number of different paths the body could take
between points $O$ and $R$.

According to the work-energy theorem,
\begin{equation}\label{ewe}
{\mit\Delta} K = \int_O^R {\bf f}\!\cdot\!d{\bf r}.
\end{equation}
In other words, the net change in the kinetic energy of the body, as it
moves from point $O$ to point $R$, is equal to the work done
on the body by the force-field during this process.
However, comparing with  Eq.~(\ref{epot}), we can see that
\begin{equation}
{\mit\Delta} K = U(O) - U(R) = -{\mit\Delta} U.
\end{equation}
In other words, the increase in the  kinetic energy of the body, as it moves
from point $O$ to point $R$, is equal to the decrease in the
function $U$ evaluated between these same two points. Another way
of putting this is
\begin{equation}
E = K + U = {\rm constant}:
\end{equation}
{\em i.e.}, the sum of the kinetic energy and the function $U$ remains
constant as the body moves around in the force-field. It should be clear, by now, that the
function $U$ represents some form of {\em potential energy}. 

The above discussion leads to the following important conclusions. Firstly,
it should be possible to associate a potential energy ({\em i.e.}, an energy
a body possesses by virtue of its position) with {\em any} conservative
force-field. Secondly, any force-field for which we can define
a potential energy must necessarily be conservative. For instance, the existence
of gravitational potential energy is proof that gravitational fields
are conservative.
Thirdly, the concept of potential energy is meaningless
in a non-conservative force-field (since the potential energy at a
given point cannot be uniquely defined). Fourthly, potential energy is
only defined to within an arbitrary additive constant. In other words,
the point in space at which we set the potential energy to zero
can be chosen at will. This implies that only {\em differences} in potential
energies between different points in space have any physical
significance. For instance, we have seen that the definition of
gravitational potential energy is $U=m\,g\,z$, where $z$ represents height above
the ground. However, we could just as well write $U=m\,g\,(z-z_0)$, where
$z_0$ is the height of some arbitrarily chosen reference point
({\em e.g.}, the top of Mount Everest, or the bottom of the
Dead Sea). Fifthly, the difference in potential energy between two points
represents the net energy transferred to the associated force-field when a body moves
between these two points. In other words, potential energy is not, strictly speaking,  a
property of the body---instead, it is a property of the force-field within which
the body  moves.

\subsection{Hooke's Law}\label{s56}
Consider a mass $m$ which slides over a horizontal frictionless surface. Suppose that
the mass is attached
to a light horizontal spring whose other end is anchored to an immovable object. See
Fig.~\ref{f42}. Let $x$ be the extension of the spring: {\em i.e.}, the difference between
the spring's actual length and its unstretched length. Obviously, $x$ can also be used as
a coordinate to determine the horizontal displacement of the mass. 
According to Hooke's law, the force
$f$ that the spring exerts on the mass is directly proportional to its extension, and always acts to reduce
this extension. 
Hence, we can write
\begin{equation}
f = -k\,x,
\end{equation}
where the positive quantity $k$ is called the {\em force constant} of the spring. Note that
the minus sign in the above equation ensures that the force always acts to reduce the 
spring's extension:
{\em e.g.}, if the extension is positive then the force acts to the left, so as to
shorten the spring. 

\begin{figure}
\epsfysize=2.5in
\centerline{\epsffile{Chapter05/fig042.eps}}
\caption{\em Mass on a spring}\label{f42}   
\end{figure}

According to Eq.~(\ref{eaax}), the work performed by the spring force on the mass
as it moves from displacement $x_A$ to $x_B$ is
\begin{equation}
W = \int_{x_A}^{x_B} f(x)\,dx = -k\int_{x_A}^{x_B} x\,dx = -\left[
\frac{1}{2}\,k\,x_B^{\,2} -\frac{1}{2}\,k\,x_A^{\,2} \right].
\end{equation}
Note that the right-hand side of the above expression consists of the difference
between two factors: the first only depends on the final state of the mass, whereas the
second only depends on its initial state. This is a sure sign  that it is
possible to associate a {\em potential energy} with the spring force. 
Equation~(\ref{epot}), which is the basic definition of potential
energy, yields
\begin{equation}\label{epotn}
U(x_B) - U(x_A) = -  \int_{x_A}^{x_B} f(x)\,dx = 
\frac{1}{2}\,k\,x_B^{\,2} -\frac{1}{2}\,k\,x_A^{\,2}.
\end{equation}
Hence, the potential energy of the mass takes the form
\begin{equation}
U(x) = \frac{1}{2}\,k\,x^{2}.
\end{equation}
Note that the above potential energy actually represents  energy stored {\em by the
spring}---in
 the form of mechanical stresses---when it is either
stretched or compressed. Incidentally, this energy must be stored {\em without loss}, otherwise
the concept of potential energy would be meaningless. It follows that the spring
force is another  example of a conservative force.

It is reasonable to suppose that the form of the spring potential energy is somehow related to the
form of the spring force. Let us now explicitly investigate this
relationship. If we let $x_B\rightarrow x$ and $x_A\rightarrow 0$ then Eq.~(\ref{epotn})
gives
\begin{equation}
U(x) = -  \int_0^x f(x')\,dx'.
\end{equation}
We can differentiate this expression to obtain
\begin{equation}\label{egrad}
f = -\frac{dU}{dx}.
\end{equation}
Thus, in 1-dimension, a conservative force is equal to minus the derivative (with respect
to displacement) of its associated potential energy. This is a quite general result.
For the case of a
spring force: $U=(1/2)\,k\,x^2$, so $f = - dU/dx = - k\,x$. 

As is easily demonstrated, the 3-dimensional equivalent to Eq.~(\ref{egrad}) is
\begin{equation}
{\bf f} = -\left(\frac{\partial U}{\partial x}, \frac{\partial U}{\partial y},
\frac{\partial U}{\partial z}\right).
\end{equation}
For example, we have seen that the gravitational potential energy of
a mass $m$ moving above the Earth's surface is $U=m\,g\,z$,
where $z$ measures height off the ground. It follows that the
associated gravitational force is
\begin{equation}
{\bf f} = (0,0,-m\,g).
\end{equation}
In other words, the force is of magnitude $m\, g$, and is directed
vertically downward.

The total energy of the mass shown in Fig.~\ref{f42} is the sum of its kinetic and
potential energies:
\begin{equation}
E = K + U = K + \frac{1}{2}\,k\,x^2.
\end{equation}
Of course, $E$ remains constant during the
mass's motion. Hence, the above expression can be rearranged
to give
\begin{equation}\label{epotn1}
K = E - \frac{1}{2}\,k\,x^2.
\end{equation}
 Since it is impossible
for a kinetic energy
to be  negative, the above expression
suggests that $|x|$ can never exceed the value 
\begin{equation}
x_0 = \sqrt{\frac{2\,E}{k}}.
\end{equation}
Here, $x_0$ is termed the {\em amplitude} of the mass's motion. Note that when
$x$ attains its maximum value $x_0$, or its minimum value $-x_0$, the 
kinetic energy is momentarily zero ({\em i.e.}, $K=0$).

\subsection{Motion in a General One-Dimensional Potential}\label{gpotn}
Suppose that the curve $U(x)$ in Fig.~\ref{f43} represents the potential
energy of some mass $m$ moving in a 1-dimensional conservative force-field.
For instance, $U(x)$ might represent the gravitational potential energy 
of a cyclist freewheeling in a hilly region. Note that we have set the
potential energy at infinity to zero. This is a useful, and quite common,  convention (recall that
potential energy is undefined to within an arbitrary additive constant).
What can we deduce about the motion of the mass in this potential?

\begin{figure}
\epsfysize=2.5in
\centerline{\epsffile{Chapter05/fig043.eps}}
\caption{\em General 1-dimensional potential}\label{f43}   
\end{figure}

Well, we know that the total energy, $E$---which is the sum of the kinetic
energy, $K$, and the potential energy, $U$---is a {\em constant} of the motion.
Hence, we can write
\begin{equation}\label{e555}
K(x) = E - U(x).
\end{equation}
Now, we also know that a kinetic energy can never be negative, so the above
expression tells us that the motion of the mass is restricted to the
region (or regions) in which the potential energy curve $U(x)$ falls
below the value $E$. This idea is illustrated in Fig.~\ref{f43}.
Suppose that the total energy of the system is $E_0$. It is clear, from
the figure, that the mass is trapped inside one or other of the two dips
in the potential---these dips are
generally referred to as {\em potential wells}. 
Suppose that we now raise the energy to $E_1$. In this
case, the mass is free to enter or leave each of the potential wells, but
its motion is still {\em bounded} to some extent, since it clearly cannot move off to
infinity. Finally, let us raise the energy to $E_2$. Now the
mass is {\em unbounded}: {\em i.e.}, it can move off to infinity. In systems
in which it makes sense to adopt the convention that the potential
energy at infinity is zero, bounded systems are characterized
by $E<0$, whereas unbounded systems are characterized by  $E>0$. 

The above discussion suggests that the motion of a mass moving in a potential
generally becomes less bounded as the total energy $E$ of the system increases. 
Conversely, we would expect the motion to become more bounded as $E$ decreases.
In fact, if the energy becomes sufficiently small, it appears likely that the
system will settle down in some {\em equilibrium state} in which the mass is stationary.
Let us try  to identify any  prospective equilibrium states in Fig.~\ref{f43}.
If the mass remains stationary then it must be subject to zero force (otherwise
it would accelerate). Hence, according to Eq.~(\ref{egrad}), an equilibrium
state is characterized by
\begin{equation}
\frac{dU}{dx} = 0.
\end{equation}
In other words, a equilibrium state corresponds to either a {\em maximum}
or a {\em minimum} of the potential energy curve $U(x)$. It can
be seen that the $U(x)$ curve shown in Fig.~\ref{f43} has
three associated equilibrium states: these are located at
$x=x_0$, $x=x_1$, and $x=x_2$. 

Let us now make a distinction between {\em stable equilibrium} points
and {\em unstable equilibrium} points. When the system is slightly
perturbed from a stable equilibrium point then the resultant force $f$
should always be such as to attempt to return the system to this point.
In other words, if $x=x_0$ is an equilibrium point, then we require
\begin{equation}
\left.\frac{df}{dx}\right|_{x_0} <0
\end{equation}
for stability: {\em i.e.}, if the system is perturbed to the right, so that $x-x_0>0$,
then the force must act to the left, so that $f <0$, and {\em vice versa}.
Likewise, if 
\begin{equation}
\left.\frac{df}{dx}\right|_{x_0} >0
\end{equation}
then the equilibrium point $x=x_0$ is unstable. It follows, from
 Eq.~(\ref{egrad}), that stable equilibrium points are
characterized by
\begin{equation}
\frac{d^2 U}{dx^2}>0.
\end{equation}
In other words, a stable equilibrium point corresponds to a {\em minimum}
of the potential energy curve $U(x)$. Likewise, an unstable
equilibrium point corresponds to a {\em maximum} of the $U(x)$ curve. Hence,
we conclude that $x=x_0$ and $x=x_2$ are stable equilibrium points,
in Fig.~\ref{f43}, whereas $x=x_1$ is an unstable equilibrium point. 
Of course, this makes perfect sense if we think of $U(x)$ as
a gravitational potential energy curve, in which case $U$ is
directly proportional to height. All we are saying is that it is
easy to confine a low energy mass at the bottom of a valley,
but very difficult to balance the same mass on the top of
a hill (since any slight perturbation to the mass will cause it
to fall down the hill). Note, finally, that if
\begin{equation}
\frac{dU}{dx}=\frac{d^2 U}{dx^2}=0
\end{equation}
at any point (or in any region) then we have what is known as a {\em neutral equilibrium}
point. We can move the mass slightly off such a point and it will still
remain in equilibrium ({\em i.e.}, it will neither attempt to return to
its initial state,  nor will it continue to move). A neutral equilibrium point
corresponds to a {\em flat spot} in a $U(x)$ curve. See Fig.~\ref{f44}.

\begin{figure}
\epsfysize=1.8in
\centerline{\epsffile{Chapter05/fig044.eps}}
\caption{\em Different types of equilibrium}\label{f44}   
\end{figure}
 
\subsection{Power}\label{spower}
Suppose that an object moves in a general force-field ${\bf f}({\bf r})$. 
We now know how to calculate how much energy flows from the force-field
to the object as it moves along a given path between two
points. Let us now consider the {\em rate} at which this energy flows. 
If $dW$ is the amount of work that the force-field performs on the mass
in a time interval $dt$ then the rate of working is given by
\begin{equation}\label{e577}
P = \frac{dW}{dt}.
\end{equation}
In other words, the rate of working---which is usually referred to as the
{\em power}---is simply the time derivative of the work performed. 
Incidentally, the mks unit of power is called the {\em watt} (symbol W). In
fact, 1 watt equals 1 kilogram meter-squared per second-cubed, or
1 joule per second.

Suppose that the object displaces by $d{\bf r}$ in the time interval
$dt$. By definition, the amount of work done on the object
during this time interval is given by
\begin{equation}
d W = {\bf f}\!\cdot\!d{\bf r}.
\end{equation}
It follows from Eq.~(\ref{e577}) that
\begin{equation}
P = {\bf f}\!\cdot\!{\bf v},
\end{equation}
where ${\bf v} = d{\bf r}/dt$ is the object's instantaneous velocity.
Note that power can be positive or negative, depending on the relative
directions of the vectors ${\bf f}$ and ${\bf v}$. If these
two vectors are mutually perpendicular then the power is zero. 
For the case of 1-dimensional motion, the above expression  reduces to
\begin{equation}
P = f\,v.
\end{equation}
In other words, in 1-dimension, power simply equals force times velocity. 

\subsection*{\em Worked Example 5.1: Bucket Lifted from a Well}
\noindent{\em Question:} A man lifts a $30$\,kg bucket from a well whose depth
is $150$\,m. Assuming that the man lifts the bucket at a constant rate,
how much work does he perform?

\noindent{\em Answer:} Let $m$ be the mass of the bucket and $h$ the depth of
the well. The gravitational force $f'$ acting on the bucket is of magnitude $m\,g$ and
is directed vertically downwards. Hence, $f'=-m\,g$ (where upward is defined to be positive).
The net upward displacement of the bucket is $h$. Hence, the work $W'$ performed by the
gravitational force is the product of the (constant) force and the displacement of the
bucket along the line of action of that force:
$$
W' = f'\,h= - m\,g\,h.
$$
Note that $W'$ is negative, which implies that the gravitational field surrounding the
bucket gains energy as the bucket is lifted.
In order to lift the bucket at a constant rate, the man must exert a force $f$ on the
bucket which balances (and very slightly exceeds) the force due to gravity.
Hence, $f=-f'$. It follows that the work $W$ done by the man is
$$
W = f\,h = m\,g\,h = 30\times 150\times 9.81 = 4.415\times 10^4\,{\rm J}.
$$
Note that the work is positive, which implies that the man expends energy whilst
lifting the bucket. Of course, since $W=-W'$,  the energy expended by the
man equals the energy gained by the gravitational field.

\subsection*{\em Worked Example 5.2: Dragging a Treasure Chest}
\noindent{\em Question:} A pirate drags a 50\,kg treasure chest over the rough
surface of a dock by exerting a constant force of 95\,N acting at an angle
of $15^\circ$ above the horizontal. The chest moves 6\,m in  a straight line,
and the coefficient of kinetic friction between the chest and the dock is $0.15$. 
How much work does the pirate perform? How much energy is dissipated as heat via
friction? What is the final velocity of the chest?
\begin{figure*}[h]
\epsfysize=2in
\centerline{\epsffile{Chapter05/fig044a.eps}}
\end{figure*}

\noindent{\em Answer:} Referring to the  diagram, the force $F$ exerted by the
pirate can be resolved into a horizontal component $F\,\cos\theta$ and a vertical
component $F\,\sin\theta$. Since the chest only moves horizontally, the vertical
component of $F$ performs zero work. The work $W$ performed by the horizontal
component is simply the magnitude of this component times the horizontal distance
$x$ moved by the chest:
$$
W = F\,\cos\theta\,x = 95\times \cos 15^\circ\times 6 = 550.6\,{\rm J}.
$$

The chest is subject to the following forces in the vertical direction:
the downward force $m\,g$ due to gravity, the upward reaction force $R$
due to the dock, and the upward component $F\,\sin\theta$ of the force
exerted by the pirate. Since the chest does not accelerate in the
vertical direction, these forces must balance. Hence,
$$
R = m\,g - F\,\sin\theta = 50\times 9.81 - 95\times\sin 15^\circ
= 465.9\,{\rm N}.
$$
The frictional force $f$ is the product of the coefficient of
kinetic friction $\mu_k$ and the normal reaction $R$, so
$$
f = \mu_k\,R = 0.15\times 465.9 = 69.89\,{\rm N}.
$$
The work $W'$ done by the frictional force is
$$
W' = - f\,x = - 69.89\times 6 = -419.3\,{\rm J}.
$$
Note that there is a minus sign in front of the $f$ because the
displacement of the chest is in the opposite direction to
the frictional force. The fact that $W'$ is negative indicates
a loss of energy by the chest: this energy is dissipated as
heat via friction. Hence, the dissipated energy
is $ 419.3\,{\rm J}$. 

The final kinetic energy $K$ of the chest (assuming that it
is initially at rest) is the difference between the work $W$
done by the pirate and the energy $-W'$ dissipated as heat. 
Hence,
$$
K = W + W' =  550.6 - 419.3 = 131.3\,{\rm J}.
$$
Since $K= (1/2)\,m\,v^2$,  the final velocity of the chest
is
$$
v = \sqrt{\frac{2\,K}{m}} = \sqrt{\frac{2\times 131.3}{50}} = 2.29\,{\rm m/s}.
$$

\subsection*{\em Worked Example 5.3: Stretching a Spring}
\noindent{\em Question:} The force required to slowly stretch a spring
varies from 0\,N to 105\,N as the spring is extended by 13\,cm
from its unstressed length. What is the force constant of the spring?
What work is done in stretching the spring? Assume that the spring
obeys Hooke's law.

\noindent{\em Answer:}
The force $f$ that the spring exerts on whatever is stretching it
is $f=-k\,x$, where $k$ is the force constant, and $x$ is the extension
of the spring. The minus sign indicates that the force acts in the opposite
direction to the extension. Since the spring is stretched slowly, the force
$f'$ which must be exerted on it is (almost) equal and opposite to $f$.
Hence, $f' = -f = k\,x$. We are told that $f'= 105$\,N when $x=0.13$\,m. It
follows that
$$
k = \frac{105}{0.13} = 807.7\,{\rm N/m}.
$$
The work $W'$ done by the external force in extending the spring from 0 to $x$ is
$$
W' = \int_0^x f'\,dx = k\int_0^x x\,dx = \frac{1}{2}\,k\,x^2.
$$
Hence, 
$$ 
W' = 0.5\times 807.7\times 0.13^2 = 6.83\,{\rm J}.
$$

\subsection*{\em Worked Example 5.4: Roller Coaster Ride}
\noindent{\em Question:} A roller coaster cart of mass $m= 300\,{\rm kg}$ starts
at rest at point $A$, whose height off the ground is $h_1=25\,{\rm m}$, and a
little while later reaches point $B$, whose height off the ground is $h_2=7\,{\rm m}$.
What is the potential energy of the cart relative to the ground at point $A$?
What is the speed of the cart at point $B$, neglecting the effect of friction?
\begin{figure*}[h]
\epsfysize=2in
\centerline{\epsffile{Chapter05/fig044b.eps}}
\end{figure*}

\noindent{\em Answer:} The gravitational potential energy of the cart with
respect to the ground at point $A$ is
$$
U_A = m\,g\,h_1 = 300\times 9.81 \times 25 = 7.36\times 10^4\,{\rm J}.
$$
Likewise, the potential energy of the cart at point $B$ is
$$
U_B = m\,g\,h_2 = 300\times 9.81 \times 7 = 2.06\times 10^4\,{\rm J}.
$$
Hence, the change in the cart's potential energy in moving from point
$A$ to point $B$ is
$$
{\mit\Delta} U = U_B-U_A = 2.06\times 10^4-7.36\times 10^4 = -5.30\times 10^4\,{\rm J}.
$$
By energy conservation, ${\mit\Delta}K = -{\mit\Delta} U$, where $K$ represents kinetic
energy. However, since the initial kinetic energy is zero, the change in kinetic energy
${\mit\Delta} K$ is equivalent to the final kinetic energy $K_B$. Thus,
$$
K_B = -{\mit\Delta}U = 5.30\times 10^4\,{\rm J}.
$$
Now, $K_B = (1/2)\,m\,v_B^{\,2}$, where $v_B$ is the final speed. Hence,
$$
v_B = \sqrt{\frac{2\,K_B}{m}} = \sqrt{\frac{2\times 5.30\times 10^4}{300}} = 18.8\,{\rm m/s}.
$$

\subsection*{\em Worked Example 5.5: Sliding Down a Plane}
\noindent{\em Question:} A block of mass $m=3\,{\rm kg}$ starts at rest at a height of
$h=43\,{\rm cm}$ on a plane that has an angle of inclination of $\theta=35^\circ$ with
respect to the horizontal. The block slides down the
plane, and, upon reaching the bottom, then slides along a horizontal
surface. The coefficient of kinetic friction of the block on both surfaces is $\mu=0.25$. 
How far does the block slide along the horizontal surface before coming to rest?

\noindent{\em Answer:}
The normal reaction of the plane to the block's weight is
$$
R = m\,g\,\cos\theta.
$$
Hence, the frictional force acting on the block when it is sliding down the plane
is 
$$
f = \mu\,R = 0.25\times 3\times 9.81 \times \cos 35^\circ = 6.03\,{\rm N}.
$$

The change in gravitational potential energy of the block as it slides down the
plane is
$$
{\mit\Delta}U = -m\,g\,h = - 3\times 9.81 \times 0.43 = -12.65\,{\rm J}.
$$
The work $W$ done on the block  by the 
frictional force during this process is
$$
W = -f\,x,
$$
where $x= h/\sin\theta$ is the distance the block slides.
The minus sign indicates that $f$ acts in the opposite direction to the displacement of the block. 
Hence,
$$
W = -\frac{6.03 \times 0.43 }{\sin 35^\circ} =- 4.52\,{\rm J}.
$$
Now, by energy conservation, the kinetic energy $K$ of the block at the bottom of the plane
equals the decrease in the block's potential energy plus the amount
of work done on the block:
$$
K = - {\mit\Delta}U + W = 12.65 - 4.52 = 8.13\,{\rm J}.
$$

The frictional force acting on the block when it slides over the horizontal
surface is
$$
f' = \mu\,m\,g = 0.25\times 3\times 9.81 = 7.36\,{\rm N}.
$$
The work done on the block as it slides a distance $y$ over this
surface is
$$
W' = -f'\,y.
$$
By energy conservation, the block comes to rest when the action of the frictional force has
drained all of the kinetic energy from the block: {\em i.e.}, when $W'=-K$. It follows
that
$$
y = \frac{K}{f'} = \frac{8.13}{7.36} = 1.10\,{\rm m}.
$$

\subsection*{\em Worked Example 5.6: Driving up an Incline}
\noindent{\em Question:} A car of weight 3000\,N possesses an engine whose maximum
power output is 160\,kW. The maximum speed of this car on a
level road is 35\,m/s. Assuming that the
resistive force (due to a combination of friction and
air resistance) remains constant, what is the car's maximum
speed on an incline of 1 in 20 ({\em i.e.}, if $\theta$
is the angle of the incline with respect to the horizontal, then
$\sin\theta = 1/20$)?

\noindent{\em Answer:}
When the car is traveling on a level road at its maximum speed, $v$, then all
of the power output, $P$, of its engine is used to overcome the power
dissipated by the resistive force, $f$. Hence,
$$
P = f\,v
$$
where the left-hand side is the power output of the engine, and the right-hand side
is the power dissipated by the resistive force ({\em i.e.}, minus the rate at which
this force does work on the car). It follows that
$$
f = \frac{P}{v} = \frac{160\times 10^3}{35} = 4.57\times 10^3\,{\rm N}.
$$

When the car, whose weight is $W$, is traveling up an incline, whose angle with respect
to the horizontal is $\theta$, it is subject to the additional
force $f'= W\,\sin\theta$, which acts to impede its motion. 
Of course, this force is
just the component of the car's weight acting down the incline. Thus, the
new power balance equation is written
$$
P = f\,v' + W\,\sin\theta\,v',
$$
where $v'$ is the maximum velocity of the car up the incline. Here, the
left-hand side represents the power output of the car, whereas the right-hand
side represents the sum of the  power dissipated by the resistive force and the power expended to
overcome the component of the car's weight acting down the incline. It follows that
$$
v' = \frac{P}{f+W\,\sin\theta} = \frac{160\times 10^3}{4.57\times 10^3 + 3000/20} = 33.90\,{\rm m/s}.
$$
\begin{figure*}
\epsfysize=1.5in
\centerline{\epsffile{Chapter05/fig044c.eps}}
\end{figure*}
