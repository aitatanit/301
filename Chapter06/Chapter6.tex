\section{Conservation of Momentum}
\subsection{Introduction}
Up to now, we have  only analyzed the behaviour of dynamical systems
which consist of {\em single} point masses ({\em i.e.}, objects whose spatial extent is either
negligible or plays no role in their motion) or arrangements of point masses which are constrained to
move {\em together} because they are connected via inextensible cables. Let us now
broaden our approach somewhat  in order to take into account systems of point masses which exert forces on
one another, but are not necessarily constrained to move together. The classic example of
such a multi-component point mass
system is one in which two (or more) freely moving  masses {\em collide} with one another.
The physical concept which plays the central role in the  dynamics
of multi-component point mass systems is the {\em conservation of momentum}. 

\subsection{Two-Component Systems}
The simplest imaginable multi-component dynamical system consists of two point mass objects
which are both constrained to move along the same straight-line. See Fig.~\ref{f45}.
Let $x_1$ be the displacement of the first object, whose mass is $m_1$. Likewise, let
$x_2$ be the displacement of the second object, whose mass is $m_2$. Suppose that the
first object exerts a force $f_{21}$ on the second object, whereas the second
object exerts a force $f_{12}$ on the first. From Newton's third law of motion, we
have
\begin{equation}\label{e61}
f_{12} = - f_{21}.
\end{equation}
Suppose, finally, that the first object is subject to an external force  ({\em i.e.}, a force
which originates {\em outside} the system) $F_1$, whilst the second object is subject to an
external force $F_2$.

\begin{figure}[b]
\epsfysize=1in
\centerline{\epsffile{Chapter06/fig045.eps}}
\caption{\em A one-dimensional dynamical system consisting of two point mass objects}\label{f45}  
\end{figure}

Applying Newton's second law of motion to each object in turn, we obtain
\begin{eqnarray}
m_1\,\ddot{x}_1 &=& f_{12} + F_1,\label{e62}\\[0.5ex]
m_2\,\ddot{x}_2 &=& f_{21} + F_2.\label{e63}
\end{eqnarray}
Here, $\dot{~}$ is a convenient shorthand for $d/dt$. Likewise,
$\ddot{~}$ means $d^2/dt^2$.

At this point, it is helpful to introduce the concept of the {\em centre of mass}.
The centre of mass is an imaginary point whose displacement $x_{cm}$ is defined to be
the {\em mass weighted average}
of the displacements of the two objects which constitute the system. In
other words,
\begin{equation}
x_{cm} = \frac{m_1\,x_1 + m_2\,x_2}{m_1 + m_2}.\label{e64}
\end{equation}
Thus, if the two masses are equal then the centre of mass lies half way between them;
if the second mass is three times larger than the first then the centre
of mass lies three-quarters of the way along the line linking the first and second
masses, respectively; if the
second mass is much larger than the first then the centre of mass is almost
coincident with the second mass; and so on.

Summing Eqs.~(\ref{e62}) and (\ref{e63}), and then making use of Eqs.~(\ref{e61}) and (\ref{e64}),
we obtain
\begin{equation}
m_1\,\ddot{x}_1+m_2\,\ddot{x}_2 = (m_1+m_2)\,\ddot{x}_{cm} = F_1 + F_2.
\end{equation}
Note that the {\em internal forces}, $f_{12}$ and $f_{21}$, have canceled out.
The physical significance of this equation becomes clearer if we write it in the following
form:
\begin{equation}\label{e66}
M\,\ddot{x}_{cm} = F,
\end{equation}
where $M=m_1+m_2$ is the total mass of the system, and $F=F_1+F_2$ is the net
{\em external} force acting on the system. Thus, the motion of the centre of mass
is equivalent to that which would occur if all the mass contained in the system were collected
at the centre of mass, and this conglomerate mass were then acted upon by the
net external force. In general, this suggests that the motion of the centre
of mass is {\em  simpler} than the motions of the component masses, $m_1$ and
$m_2$. This is particularly the case if the internal forces, $f_{12}$ and $f_{21}$,
are complicated in nature.

Suppose that there are {\em no} external forces acting on the system ({\em i.e.}, $F_1=F_2=0$), or, 
equivalently, suppose
that the sum of all the external forces is {\em zero} ({\em i.e.}, $F=F_1+F_2=0$). 
In this case, according to Eq.~(\ref{e66}), the motion of the centre of mass is governed by
Newton's first law of motion: {\em i.e.}, it consists of uniform motion in a straight-line.
Hence, in the absence of a net external force, the motion of the centre of mass is
almost certainly {\em far simpler} than that of the component masses. 

Now, the velocity
of the centre of mass is written
\begin{equation}
v_{cm} = \dot{x}_{cm} = \frac{m_1\,\dot{x}_1+m_2\,\dot{x}_2}{m_1+m_2}.
\end{equation}
We have seen that in the absence of external forces $v_{cm}$ is a constant of the motion
({\em i.e.}, the centre of mass does not accelerate). It follows that, in this case,
 \begin{equation}
m_1\,\dot{x}_1+m_2\,\dot{x}_2
\end{equation}
is also a constant of the motion. Recall, however, from Sect.~\ref{moment}, that
{\em momentum}  is defined as the product of mass and velocity. Hence, the momentum
of the first mass is written $p_1=m_1\,\dot{x}_1$, whereas the momentum of the
second mass takes the form $p_2=m_2\,\dot{x}_2$. It follows that the above expression
corresponds to the {\em total momentum} of the system:
 \begin{equation}
P = p_1 + p_2.
\end{equation}
Thus, the total momentum is a conserved quantity---provided there is no net external force acting on the system.
This is true {\em irrespective} of the nature of the internal forces. 
More generally, Eq.~(\ref{e66}) can be written
\begin{equation}
\frac{dP}{dt} = F.
\end{equation}
In other words, the time derivative of the total momentum is equal to
the net external force acting on the system---this is just Newton's second law of
motion applied to the system as a whole.

\begin{figure}
\epsfysize=3in
\centerline{\epsffile{Chapter06/fig046.eps}}
\caption{\em An example two-component system}\label{f46}  
\end{figure}

Let us now try to apply some of the concepts discussed above.
Consider the simple two-component system shown in Fig.~\ref{f46}. A
gondola of mass $m_g$ hangs from a hot-air balloon whose mass
is negligible compared to that of the gondola. A sandbag of mass $m_w$ is
suspended from the gondola by means of a light inextensible cable. The system
is in equilibrium. 
Suppose, for the sake of consistency with our other examples, that the
$x$-axis runs vertically upwards. 
Let $x_g$ be the height of the gondola, and $x_w$  the height of the
sandbag. Suppose that the upper end of the cable is attached to a winch inside the gondola, and
that this winch is used to {\em slowly} shorten the cable, so that the sandbag is lifted upwards
 a distance ${\mit\Delta}x_w$. The question is this: does the height of the gondola
also change as the cable is reeled in? If so, by how much?

Let us identify all of the forces acting on the system shown in Fig.~\ref{f46}. The internal forces are
the upward force exerted by the gondola on the sandbag, and the downward force exerted by the
sandbag on the gondola. These forces are transmitted via the cable, and
are equal and opposite (by Newton's third law of motion). The external forces are the
net downward force due to the combined weight of the gondola and the sandbag, and the
upward force due to the buoyancy of the balloon. Since the system is
in equilibrium, these forces are equal and opposite (it is assumed that the cable is reeled in
sufficiently slowly that the equilibrium is not upset). Hence, there is {\em zero}
 net external force acting on the system. It follows, from the previous discussion, that the
centre of mass of the system is subject to Newton's first law. In particular, since the
centre of mass is clearly {\em stationary} before the winch is turned on, it must {\em remain
stationary} both during and after the period in which the
 winch is operated. Hence, the height of the centre of mass,
\begin{equation}
x_{cm} = \frac{m_g\,x_g + m_w\,x_w}{m_g+m_w},
\end{equation}
is a conserved quantity. 

Suppose that the operation of the winch causes the height of the sandbag to change
by ${\mit\Delta}x_w$, and that of the gondola to simultaneously change by ${\mit\Delta}x_g$. If
$x_{cm}$ is a conserved quantity, then we must have
\begin{equation}
0 = m_g\,{\mit\Delta}x_g+m_w\,{\mit\Delta}x_w,
\end{equation}
or
\begin{equation}
{\mit\Delta}x_g =- \frac{m_w}{m_g} \,{\mit\Delta}x_w.
\end{equation}
Thus, if the winch is used to {\em raise} the sandbag a distance ${\mit\Delta}x_w$ then the gondola
is simultaneously pulled {\em downwards}  a distance  $(m_w/m_g) \,{\mit\Delta}x_w$.
It is clear that we could use a suspended sandbag as a mechanism for
adjusting a hot-air  balloon's altitude: the balloon descends as the sandbag is raised, and ascends
as it is lowered. 

\begin{figure}
\epsfysize=1.5in
\centerline{\epsffile{Chapter06/fig047.eps}}
\caption{\em Another example two-component system}\label{f47}  
\end{figure}

Our next example is pictured in Fig.~\ref{f47}. Suppose that a cannon of mass $M$ propels a cannonball
of mass $m$
horizontally with velocity $v_b$. What is the recoil velocity $v_r$ of the cannon? Let us
first identify all of the forces acting on the system. The internal forces are
the force exerted by the cannon on the cannonball, as the cannon is fired, and the
equal and opposite force exerted by the cannonball on the cannon. These forces are
extremely large, but only last for a short instance in time: in physics, we call these
 {\em impulsive} forces.
There are no external forces acting
in the horizontal direction (which is the only direction that we are considering in this
example).
 It follows that the total (horizontal) momentum $P$ of the system is a
conserved quantity. Prior to the firing of the cannon, the total momentum is {\em zero} (since
momentum is mass times velocity, and nothing is initially moving). After the cannon is
fired, the total momentum of the system takes the form
\begin{equation}
P = m\,v_b + M\,v_r.
\end{equation}
Since $P$ is a conserved quantity, we can set $P=0$. Hence,
\begin{equation}
v_r = - \frac{m}{M}\,v_b.
\end{equation}
Thus, the recoil velocity of the cannon is in the {\em opposite} direction to
the velocity of the cannonball (hence, the minus sign in the above equation), and
is of magnitude $(m/M)\,v_b$. Of course, if the cannon is far more massive that
the cannonball ({\em i.e.}, $M\gg m$), which is  usually the case, then the recoil velocity of the cannon
is far smaller in magnitude than the velocity of the cannonball. Note, however,
that the {\em momentum} of the cannon is equal in magnitude to that of the cannonball.
It follows that it takes the same effort ({\em i.e.}, force applied for a certain period
of time) to slow down and stop the cannon as it
does to slow down and stop the cannonball.

\subsection{Multi-Component Systems}\label{scm}
\begin{figure}
\epsfysize=3in
\centerline{\epsffile{Chapter06/fig048.eps}}
\caption{\em A three-dimensional dynamical system consisting of many point mass objects.}\label{f49}  
\end{figure}

Consider a system of $N$ mutually interacting point mass objects which move in 3-dimensions.
See Fig.~\ref{f49}.
Let the $i$th object, whose mass is $m_i$, be located at vector displacement ${\bf r}_i$.
Suppose that this object exerts a force ${\bf f}_{ji}$ on the $j$th object. By Newton's third
law of motion, the force ${\bf f}_{ij}$ exerted  by the $j$th object on the $i$th  is
given by
\begin{equation}
{\bf f}_{ij} = - {\bf f}_{ji}.
\end{equation}
Finally, suppose that the $i$th object is subject to an external force ${\bf F}_i$. 

Newton's second law of motion applied to the $i$th object yields
\begin{equation}
m_i\,\ddot{\bf r}_i = \sum_{j=1,N}^{j\neq i}\! {\bf f}_{ij} + {\bf F}_i.
\end{equation}
Note that the summation on the right-hand side of the above equation excludes the case
$j=i$, since the $i$th object cannot exert a force on itself. Let us now take the above
equation and sum it over all objects. We obtain
\begin{equation}
\sum_{i=1,N} \!m_i\,\ddot{\bf r}_i =\sum_{i,j=1,N}^{j\neq i}\! {\bf f}_{ij} + \sum_{i=1,N}\!
{\bf F}_i.
\end{equation}
Consider the sum  over all internal forces: {\em i.e.}, the first term on the right-hand side.
Each element of this sum---${\bf f}_{ij}$, say---can be paired with another element---${\bf f}_{ji}$,
in this case---which is equal and opposite. In other words,
 the elements of the sum all cancel out in pairs. Thus, the net value of the sum is {\em zero}.  
It follows that the above equation can be written
\begin{equation}\label{e6yy}
M\,\ddot{\bf r}_{cm}= {\bf F},
\end{equation}
where $M = \sum_{i=1}^N m_i$ is the total mass, and ${\bf F} = \sum_{i=1}^N {\bf F}_i$ is the
net external force. The quantity ${\bf r}_{cm}$ is the vector displacement of the centre of mass.
As before, the centre of mass is an imaginary point whose coordinates are the mass weighted
averages of the coordinates of the objects which constitute  the system. Thus,
\begin{equation}
{\bf r}_{cm} = \frac{\sum_{i=1}^N m_i\,{\bf r}_i}{\sum_{i=1}^N m_i}.
\end{equation}
According to Eq.~(\ref{e6yy}), the motion of the centre of mass is equivalent to that which
would be obtained if all the mass contained in the system were collected at the centre of mass, and
this conglomerate mass were then acted upon by the net external force. As before, the motion
of the centre of mass is likely to be far simpler than the motions of the component masses. 

Suppose that there is zero net external force acting on the system, so that ${\bf F} = {\bf 0}$. 
In this case, Eq.~(\ref{e6yy}) implies that the centre of mass moves with  uniform
velocity in a straight-line. In other words, the velocity of the centre of mass,
\begin{equation}\label{e6xy}
\dot{\bf r}_{cm} = \frac{\sum_{i=1}^N m_i\,\dot{\bf r}_i}{\sum_{i=1}^N m_i},
\end{equation}
is a constant of the motion. Now, the momentum of the $i$th object takes the
form ${\bf p}_i = m_i\,\dot{\bf r}_i$. Hence, the total momentum of the
system is written
\begin{equation}\label{e6yx}
{\bf P} = \sum_{i=1}^N m_i\,\dot{\bf r}_i.
\end{equation}
A comparison of Eqs.~(\ref{e6xy}) and (\ref{e6yx}) suggests that ${\bf P}$ is also
a constant of the motion when zero net external force acts on the system. Finally, 
Eq.~(\ref{e6yy}) can be rewritten
\begin{equation}
\frac{d{\bf P}}{dt} = {\bf F}.
\end{equation}
In other words, the time derivative of the total momentum
is equal to the net external force acting on the system.

It is clear, from the above discussion, that most of the important results obtained in the
previous section, for the case of a two-component system moving
in 1-dimension, also apply to a multi-component system moving in 3-dimensions.

\begin{figure}
\epsfysize=2in
\centerline{\epsffile{Chapter06/fig049.eps}}
\caption{\em The unfortunate history of the planet Krypton.}\label{f50}  
\end{figure}

As an illustration of the points raised in the above discussion, let us consider the unfortunate
history of the planet Krypton. As you probably all know, Krypton---Superman's home planet---eventually
 exploded. 
Note, however, that before, during, and after this explosion the net external force acting
on Krypton, or the fragments of Krypton---namely, the gravitational attraction to Krypton's sun---remained
the same. In other words, the forces responsible for the explosion can be thought of as
large, transitory, {\em internal} forces. We conclude that  the motion of the centre of mass of
Krypton, or the fragments of Krypton, was {\em unaffected} by the explosion. This follows, from
Eq.~(\ref{e6yy}), since the motion of the centre of mass is independent of internal forces. 
Before the explosion, the planet Krypton presumably executed a standard elliptical orbit
around Krypton's sun. We conclude that,
 after the explosion, the fragments of Krypton (or, to be more exact, the
centre of mass of these fragments) continued to execute {\em exactly the same} orbit. See
Fig.~\ref{f50}.

\subsection{Rocket Science}
\begin{figure}
\epsfysize=2in
\centerline{\epsffile{Chapter06/fig050.eps}}
\caption{\em A rocket.}\label{f51}  
\end{figure}
A rocket engine is the only type of propulsion device that operates
effectively in outer space. As shown in Fig.~\ref{f51}, a rocket works by ejecting a propellant at
high velocity from its rear end. The rocket exerts a backward force on the
propellant, in order to eject it, and, by Newton's third law, the propellant
exerts an equal and opposite force on the rocket, which propels it forward.

Let us attempt to find the equation of motion of a rocket. Let $M$ be the fixed mass of the
rocket engine and the payload, and  $m(t)$  the total mass of the propellant contained
in the rocket's fuel tanks at time $t$. Suppose that the rocket engine ejects the propellant
at some fixed velocity $u$ {\em relative to the rocket}. Let us examine the rocket at two
closely spaced instances in time. Suppose that at time $t$ the rocket and propellant, whose
total mass is $M+m$, are traveling with instantaneous velocity $v$. Suppose, further, that
between times $t$ and $t+dt$ the rocket  ejects a quantity of propellant of mass $-dm$ ({\em n.b.}, $dm$ 
is understood to be negative,
so this represents a positive mass) which travels with velocity $v-u$ ({\em i.e.}, velocity
$-u$ in the instantaneous rest frame of the rocket). As a result of the fuel
ejection, the velocity of the rocket
at time $t+dt$ is boosted to $v+dv$, and its total
mass becomes $M+m+dm$. See Fig.~\ref{f52}.

\begin{figure}
\epsfysize=2in
\centerline{\epsffile{Chapter06/fig051.eps}}
\caption{\em Derivation of the rocket equation.}\label{f52}  
\end{figure}

Now, there is zero external force acting on the system, since the rocket is assumed to be in outer space.
It follows that the total momentum of the system is a constant of the motion. Hence, we
can equate the momenta evaluated at times $t$ and $t+dt$:
\begin{equation}
(M+m)\,v = (M+m+dm)\,(v+dv) + (-dm)\,(v-u).
\end{equation}
Neglecting second order quantities ({\em i.e.}, $dm\,dv$), the above expression
yields
\begin{equation}
0 = (M+m)\,dv + u\,dm.
\end{equation}
Rearranging, we obtain
\begin{equation}
\frac{dv}{u} =- \frac{dm}{M+m}.
\end{equation}
Let us integrate the above equation between an initial time at which the rocket is
fully fueled---{\em i.e.}, $m=m_p$, where $m_p$ is the maximum mass of propellant that the
rocket can carry---but stationary, and a final time at which the mass of the fuel is $m$
and the velocity of the rocket is $v$. Hence,
\begin{equation}
\int_0^v\frac{dv}{u}= -\int_{m_p}^m\frac{dm}{M+m}.
\end{equation}
It follows that
\begin{equation}
\left[\frac{v}{u}\right]_{v=0}^{v=v}= -\left[\ln(M+m)\right]_{m=m_p}^{m=m},
\end{equation} 
which yields
\begin{equation}
v = u\,\ln\left(\frac{M+m_p}{M+m}\right).
\end{equation}
The final velocity of the rocket ({\em i.e.}, the velocity attained by the time
the rocket has exhausted its fuel, so that  $m=0$) is
\begin{equation}
v_f = u\, \ln\left(1+\frac{m_p}{M}\right).
\end{equation}
Note that, unless the initial mass of the fuel  exceeds the fixed mass of the
rocket by many orders of magnitude
(which is highly unlikely), the final velocity $v_f$ of the rocket is similar to
the velocity $u$ with which  fuel is ejected from the rear of the rocket in 
its instantaneous rest frame. This follows because $\ln x \sim O(1)$, unless $x$ becomes
extremely large.

Let us now consider the factors which might influence the design of a rocket for use in interplanetary or
interstellar travel. Since the distances involved in such travel are vast, it is 
important that the rocket's final velocity be made as large as possible, otherwise the journey
is going to take an unacceptably long time. However, as we have just seen, the factor which
essentially determines the final velocity $v_f$ of a rocket is the speed of ejection $u$ of
the propellant relative to the rocket. Broadly speaking, $v_f$ can never significantly
exceed $u$. It follows that a rocket suitable for interplanetary or
interstellar travel should have as high an ejection speed as practically possible.
Now, ordinary chemical rockets (the kind which powered the Apollo moon program)
can develop enormous thrusts, but are limited to ejection velocities below
about $5000\,{\rm m/s}$. Such rockets are ideal for lifting payloads out of the Earth's
gravitational field, but their relatively low ejection velocities render them unsuitable for
long distance space travel. A new type of rocket engine, called an {\em ion thruster}, is currently
under development: ion thrusters operate by accelerating ions electrostatically to great
velocities, and then ejecting them. Although ion thrusters only generate very small thrusts, compared to
chemical rockets, their much larger ejection velocities (up to 100 times those
of chemical rockets) makes them far more suitable for interplanetary or interstellar space travel.
The first spacecraft to employ an ion thruster was the
Deep Space 1 probe, which was launched from Cape Canaveral on October 24, 1998: this
probe  successfully encountered the  asteroid 9969 Braille   in July, 1999. 

\subsection{Impulses}\label{s65}
\begin{figure}
\epsfysize=2.5in
\centerline{\epsffile{Chapter06/fig052.eps}}
\caption{\em A ball bouncing off a wall.}\label{f53}  
\end{figure}

Suppose that a ball of mass $m$ and {\em speed} $u_i$ strikes an immovable wall normally and
rebounds with {\em speed} $u_f$. See Fig.~\ref{f53}. Clearly, the momentum of the ball is changed
by the collision with the wall, since the direction of the ball's velocity is reversed.
It follows that the wall must exert a force on the ball, since force is the rate of change
of momentum. This force is generally very large, but is only exerted for the short instance in time
during which the ball is in physical contact with the wall. As we have already mentioned, physicists
generally refer to such a force as an {\em impulsive} force. 

\begin{figure}
\epsfysize=2.5in
\centerline{\epsffile{Chapter06/fig053.eps}}
\caption{\em An impulsive force.}\label{f54}  
\end{figure}

Figure~\ref{f54} shows the typical time history of an impulsive force, $f(t)$. It can be seen that
the force is only non-zero in the short time interval $t_1$ to $t_2$. It is helpful
to define a quantity known as the net {\em impulse}, $I$, associated with $f(t)$:
\begin{equation}\label{e6uu}
I = \int_{t_1}^{t_2} f(t)\,dt.
\end{equation}
In other words, $I$ is the total area under the $f(t)$ curve shown in Fig.~\ref{f54}. 

Consider a object subject to the impulsive force pictured in Fig.~\ref{f54}. Newton's
second law of motion yields
\begin{equation}
\frac{dp}{dt} = f,
\end{equation}
where $p$ is the momentum of the object. Integrating the above equation, making use of
the definition (\ref{e6uu}), we obtain
\begin{equation}
{\mit\Delta}p = I.
\end{equation}
Here, ${\mit\Delta} p = p_f-p_i$, where $p_i$ is the momentum before the impulse, and
$p_f$ is the momentum after the impulse. We conclude that the net change in momentum of
an object subject to an impulsive force is equal to the  total impulse associated with that
force. For instance, the net change in momentum of the ball bouncing off the wall in
Fig.~\ref{f53} is ${\mit\Delta} p = m\,u_f -m\,(-u_i)=
m\,(u_f+u_i)$. [Note: The initial {\em velocity} is $-u_i$, since the ball is initially
moving in the negative direction.]
It follows that the net impulse imparted
to the ball by the wall is $I = m\,(u_f+u_i)$. Suppose that we know the ball was only in physical
contact with the wall for the short time interval ${\mit\Delta} t$. We conclude that the
{\em average force} $\bar{f}$ exerted on the ball during this time interval
was
\begin{equation}
\bar{f} = \frac{I}{{\mit\Delta} t}.
\end{equation}

The above discussion is only relevant to 1-dimensional motion. However, the generalization
to 3-dimensional motion is fairly straightforward. Consider an impulsive force
${\bf f}(t)$, which is only non-zero in the short time interval $t_1$ to $t_2$.
The vector impulse associated with this force is simply
\begin{equation}
{\bf I} = \int_{t_1}^{t_2} {\bf f}(t)\,dt.
\end{equation}
The net change in momentum of an object subject to  ${\bf f}(t)$ is
\begin{equation}
{\mit\Delta} {\bf p} = {\bf I}.
\end{equation}
Finally, if $t_2-t_1={\mit\Delta} t$, then the average force experienced by the
object in the time interval $t_1$ to $t_2$ is 
\begin{equation}
\bar{\bf f} = \frac{{\bf I}}{{\mit\Delta}t}.
\end{equation}

\subsection{Collisions in One-Dimension}
\begin{figure}
\epsfysize=1.5in
\centerline{\epsffile{Chapter06/fig054.eps}}
\caption{\em A collision in one-dimension.}\label{f55}  
\end{figure}
Consider two objects of mass $m_1$ and $m_2$, respectively, which are free
to move in 1-dimension. Suppose that these two objects collide.
Suppose, further, that both objects are subject to zero net force when they are
not in contact with one another. This situation is illustrated in Fig.~\ref{f55}.

Both before and after the collision, the two objects move with {\em constant velocity}.
Let $v_{i1}$ and $v_{i2}$ be the velocities of the first and second objects, respectively,
before the collision. Likewise, let  $v_{f1}$ and $v_{f2}$ be the velocities of 
the first and second objects, respectively,
after the collision. During the collision itself, the first object exerts a large
transitory force $f_{21}$ on the second, whereas the second object exerts an
equal and opposite force $f_{12}=-f_{21}$ on the first. In fact, we can model the collision
as equal and opposite {\em impulses} given to the two objects at the instant in time
when they come together.

We are clearly considering a system in which there is zero net external force (the forces
associated with the collision are internal in nature). Hence, the total momentum of the
system is a conserved quantity. Equating the total momenta before and after the
collision, we obtain
\begin{equation}\label{e6tty}
m_1\,v_{i1} + m_2\,v_{i2} = m_1\,v_{f1} + m_2\,v_{f2}.
\end{equation}
This equation is valid for {\em any} 1-dimensional
collision, irrespective its nature. Note that, assuming
we know the masses of the colliding objects, the above equation only fully describes the
collision if we are given the initial velocities of both objects, and the final velocity of
at least  one of the objects. (Alternatively, we could be given both final velocities and only
one of the initial velocities.)

There are many different types of collision. An {\em elastic} collision is one in which
the total kinetic energy of the two colliding objects is the same
before and after the collision. Thus, for an elastic collision we can write
\begin{equation}\label{e6ttt}
\frac{1}{2}\,m_1\,v_{i1}^{\,2} +\frac{1}{2}\,m_2\,v_{i2}^{\,2}=
\frac{1}{2}\,m_1\,v_{f1}^{\,2} +\frac{1}{2}\,m_2\,v_{f2}^{\,2},
\end{equation}
in addition to Eq.~(\ref{e6tty}). Hence, in this case, the collision is fully specified
once we are given the two initial velocities of the colliding objects. (Alternatively,
we could be given the two final velocities.)

The majority of collisions occurring in real life are not elastic in nature.
 Some fraction of the initial kinetic
energy of the colliding objects is usually converted into some other form of energy---generally
heat energy, or energy associated with the mechanical deformation of the objects---during the
collision. Such collisions are termed {\em inelastic}. For instance, a large fraction of
the initial kinetic energy of a typical automobile accident is converted into mechanical
energy of deformation of the two vehicles. Inelastic collisions also occur during squash/racquetball/handball
games: in each case,  the ball becomes warm to the touch after a long game, because some
fraction  of the
ball's kinetic energy of collision  with the walls of the court has been converted into
heat energy. Equation~(\ref{e6tty}) remains valid for inelastic collisions---however, Eq.~(\ref{e6ttt})
is invalid. Thus, generally speaking, an inelastic collision is only fully characterized
when we are given the initial velocities of both objects, and the final velocity of
at least  one of the objects.
There is, however,
 a special case of an inelastic collision---called a {\em totally inelastic} collision---which
is fully characterized once we are given the  initial velocities of the colliding objects.
In a totally inelastic collision, the two objects {\em stick together} after the collision, so
that $v_{f1}=v_{f2}$. 

Let us, now, consider elastic collisions in more detail. Suppose that we transform to a frame of
reference which co-moves with the centre of mass of the system. The motion of a multi-component system 
often looks particularly simple when viewed
in such a frame. Since the system is subject to zero net external force, the velocity of the
centre of mass is {\em invariant}, and is given by
\begin{equation}
v_{cm} = \frac{m_1\,v_{i1} + m_2\,v_{i2}}{m_1+m_2}=  \frac{m_1\,v_{f1} + m_2\,v_{f2}}{m_1+m_2}.
\end{equation}
An object which possesses a velocity $v$ in our original frame of reference---hence\-forth, termed
the {\em laboratory frame}---possesses a velocity $v'=v-v_{cm}$ in the centre of mass
frame. It is easily demonstrated that
\begin{eqnarray}
v_{i1}' &=& - \frac{m_2}{m_1+m_2}\,(v_{i2}-v_{i1}),\\[0.5ex]
v_{i2}' &=& + \frac{m_1}{m_1+m_2}\,(v_{i2}-v_{i1}),\\[0.5ex]
v_{f1}' &=& - \frac{m_2}{m_1+m_2}\,(v_{f2}-v_{f1}),\\[0.5ex]
v_{f2}' &=& + \frac{m_1}{m_1+m_2}\,(v_{f2}-v_{f1}).
\end{eqnarray}

The above equations yield
\begin{eqnarray}
-p_{i1}' = p_{i2}' = \mu\,(v_{i2}-v_{i1}),\label{e6qqq}\\[0.5ex]
-p_{f1}' = p_{f2}' = \mu\,(v_{f2}-v_{f1}),\label{e6qq}
\end{eqnarray}
where $\mu = m_1\,m_2/(m_1+m_2)$ is the
so-called {\em reduced mass}, and $p_{i1}' = m_1\,v_{i1}'$ is the initial momentum of the first object in the
centre of mass frame, {\em etc}. In other words, when viewed in the centre of mass frame, the 
two objects approach one another with {\em equal and opposite momenta} before the collision,
and diverge from one another with equal and opposite momenta after the collision.
Thus, the centre of mass momentum conservation equation,
\begin{equation}
p_{i1}' + p_{i2}' = p_{f1}' + p_{f2}',
\end{equation}
is trivially satisfied, because both the left- and right-hand sides are zero. Incidentally,
this result is valid for both elastic {\em and} inelastic collisions.

The centre of mass kinetic energy conservation equation takes the form
\begin{equation}
\frac{p_{i1}'^{~2}}{2\,m_1} +  \frac{p_{i2}'^{~2}}{2\,m_2} = 
\frac{p_{f1}'^{~2}}{2\,m_1} +  \frac{p_{f2}'^{~2}}{2\,m_2}.\label{e6q}
\end{equation}
Note, incidentally, that if energy and momentum are conserved in the laboratory frame then
they must also be conserved in the centre of mass frame. A comparison
of Eqs.~(\ref{e6qqq}), (\ref{e6qq}), and (\ref{e6q}) yields
\begin{equation}
(v_{i2}-v_{i1}) = - (v_{f2}-v_{f1}).\label{e6p}
\end{equation}
In other words, the {\em relative velocities} of the colliding objects are {\em equal and
opposite} before and after the collision. This is true in {\em all} frames of
reference, since relative velocities are frame invariant. Note, however, that this
result {\em only} applies to fully elastic collisions. 

Equations~(\ref{e6tty}) and (\ref{e6p}) can be combined to give the following
pair of equations which fully specify the final velocities (in the laboratory frame) of two objects
which collide elastically, given their initial velocities:
\begin{eqnarray}
v_{f1} &=&  \frac{(m_1-m_2)}{m_1+m_2}\,v_{i1} + \frac{2\,m_2}{m_1+m_2}\,v_{i2},\label{e6hh}\\[0.5ex]
v_{f2} &=& \frac{2\,m_1}{m_1+m_2}\,v_{i1} -  \frac{(m_1-m_2)}{m_1+m_2}\,v_{i2}.\label{e6hhh}
\end{eqnarray}

Let us, now, consider some special cases. Suppose that two {\em equal mass} objects collide elastically.
If $m_1=m_2$ then Eqs.~(\ref{e6hh}) and (\ref{e6hhh}) yield
\begin{eqnarray}
v_{f1} = v_{i2},\\[0.5ex]
v_{f2} = v_{i1}.
\end{eqnarray}
In other words, the two objects simply {\em exchange velocities} when they collide. 
For instance, if the second object is stationary and the first object strikes it head-on
with velocity $v$ then the first object is brought to a halt whereas the second object
moves off with velocity $v$. It is possible to reproduce this effect in pool
by striking the cue ball with great force in such a manner that it {\em slides}, rather
that rolls, over the table---in this case, when the cue ball strikes another ball head-on
it comes to a {\em complete halt}, and the other ball is propelled forward very rapidly.
Incidentally, it is necessary to prevent the cue ball from rolling, because rolling motion
is not taken into account in our analysis, and actually changes the answer.

Suppose that the second object is much more massive than the first ({\em i.e.}, $m_2\gg m_1$)
and is initially at rest ({\em i.e.}, $v_{i2} =0$). In this case, Eqs.~(\ref{e6hh}) and (\ref{e6hhh}) yield
\begin{eqnarray}
v_{f1} &\simeq & - v_{i1},\\[0.5ex]
v_{f2} &\simeq & 0.
\end{eqnarray}
In other words, the velocity of the light object is effectively {\em reversed} during the collision, whereas
the massive object remains approximately at rest. Indeed, this is the sort of behaviour we
expect when an object collides elastically with an immovable obstacle: {\em e.g.}, when an
elastic ball bounces off a brick wall.

Suppose, finally, that the second object is much lighter than the first ({\em i.e.}, $m_2\ll m_1$)
and is initially at rest ({\em i.e.}, $v_{i2} =0$). In this case, Eqs.~(\ref{e6hh}) and (\ref{e6hhh}) yield
\begin{eqnarray}
v_{f1} &\simeq &  v_{i1},\\[0.5ex]
v_{f2} &\simeq & 2\,v_{i1}.
\end{eqnarray}
In other words, the motion of the massive object is essentially unaffected by the collision,
whereas the light object ends up going {\em twice} as fast as the massive one.

Let us, now, consider totally inelastic collisions in more detail. In a totally inelastic
collision the two objects stick together after colliding, so they
 end up moving with the same final velocity 
$v_f= v_{f1}=v_{f2}$. In this case, Eq.~(\ref{e6tty}) reduces to
\begin{equation}
v_f = \frac{m_1\,v_{i1} + m_2\,v_{i2}}{m_1+m_2}= v_{cm}.
\end{equation}
In other words, the common final velocity of the two objects is equal to the
centre of mass velocity of the system. This is hardly a surprising result. We have already seen
that in the centre of mass frame the two objects must diverge with {\em equal and opposite momenta} after
the collision. However, in a totally inelastic collision these two momenta must
also be {\em equal} (since the two objects stick together). The only way in which this is possible
is if the two objects remain {\em stationary} in the centre of mass frame after the collision.
Hence, the two objects move with the centre of mass velocity in the laboratory frame.

Suppose that the second object is initially at rest ({\em i.e.}, $v_{i2}=0$). In this
special case, the common final velocity of the two objects is
\begin{equation}
v_f =  \frac{m_1}{m_1+m_2}\,v_{i1}.
\end{equation}
Note that the first object is slowed down by the collision.
The fractional loss in kinetic energy of the system due to the collision is given
by
\begin{equation}
f = \frac{K_i-K_f}{K_i} = \frac{m_1\,v_{i1}^{\,2} - (m_1+m_2)\,v_f^{\,2}}{m_1\,v_{i1}^{\,2}}
= \frac{m_2}{m_1+m_2}.
\end{equation}
The loss in kinetic energy is small if the (initially) stationary object
is much lighter than the  moving object ({\em i.e.}, if $m_2\ll m_1$), and almost $100\%$ if the moving
object is much lighter than the stationary one ({\em i.e.}, if $m_2\gg m_1$). 
Of course, the lost kinetic energy of the
system is converted into some other form of energy: {\em e.g.}, heat energy. 

\begin{figure}
\epsfysize=2.5in
\centerline{\epsffile{Chapter06/fig055.eps}}
\caption{\em A collision in two-dimensions.}\label{f56}  
\end{figure}

\subsection{Collisions in Two-Dimensions}
Suppose that an object of mass $m_1$, moving with initial speed $v_{i1}$, strikes a
second object, of mass $m_2$, which is initially at rest. Suppose, further, that the collision
is not head-on, so that after the collision the first object moves off at an angle $\theta_1$ to
its initial direction of motion, whereas the second object moves off at an angle $\theta_2$ to
this direction. Let the final speeds of the two objects be
$v_{f1}$ and $v_{f2}$, respectively. See Fig.~\ref{f56}. 

We are again considering a system in which there is zero net external force (the forces associated
with the collision are internal in nature). It follows that the total momentum of
the system is a conserved quantity. However, unlike before, we must now treat the total
momentum as a {\em vector} quantity, since we are no longer dealing with 1-dimensional
motion. Note that if the collision takes place wholly within the $x$-$y$ plane, as indicated in Fig.~\ref{f56},
then it is sufficient to equate the $x$- and $y$- components of the total momentum before and after the collision. 

Consider the $x$-component of the system's total momentum. Before the collision, the
total $x$-momentum is simply $m_1\,v_{i1}$, since the second object is initially
stationary, and the first object is initially moving along the $x$-axis with
speed $v_{i1}$. After the collision, the $x$-momentum of the first object is
$m_1\,v_{f1}\,\cos\theta_1$: {\em i.e.}, $m_1$ times the $x$-component of the
first object's final velocity. Likewise, the final $x$-momentum of the second
object is $m_2\,v_{f2}\,\cos\theta_2$. Hence, momentum conservation in the $x$-direction
yields
\begin{equation}\label{epx}
m_1\,v_{i1} = m_1\,v_{f1}\,\cos\theta_1+ m_2\,v_{f2}\,\cos\theta_2.
\end{equation}

Consider the $y$-component of the system's total momentum. Before the collision, the
total $y$-momentum is zero, since there is initially no  motion along the $y$-axis.
After the collision, the $y$-momentum of the first object is
$-m_1\,v_{f1}\,\sin\theta_1$: {\em i.e.}, $m_1$ times the $y$-component of the
first object's final velocity. Likewise, the final $y$-momentum of the second
object is $m_2\,v_{f2}\,\sin\theta_2$. Hence, momentum conservation in the $y$-direction
yields
\begin{equation}\label{epy}
  m_1\,v_{f1}\,\sin\theta_1= m_2\,v_{f2}\,\sin\theta_2.
\end{equation}

For the special case of an {\em elastic} collision, we can equate the
total kinetic energies of the two objects before and after the collision. Hence,
we obtain
\begin{equation}\label{eke}
 \frac{1}{2}\,m_1\,v_{i1}^{\,2} =\frac{1}{2}\,m_1\,v_{f1}^{\,2}+\frac{1}{2}\,m_2\,v_{f2}^{\,2}. 
\end{equation}
Given the initial conditions ({\em i.e.}, $m_1$, $m_2$, and $v_{i1}$), we have a system
of {\em  three} equations [{\em i.e.}, Eqs.~(\ref{epx}), (\ref{epy}), and (\ref{eke})] and
{\em four} unknowns ({\em i.e.}, $\theta_1$, $\theta_2$, $v_{f1}$, and $v_{f2}$). Clearly,
we cannot uniquely  solve such a system without being given additional information:
{\em e.g.}, the direction of motion or speed of one of the objects after the collision.

\begin{figure}
\epsfysize=2.5in
\centerline{\epsffile{Chapter06/fig056.eps}}
\caption{\em A totally inelastic collision in two-dimensions.}\label{f57}  
\end{figure}

Figure~\ref{f57} shows a 2-dimensional totally inelastic collision. In this
case, the first object, mass $m_1$, initially moves along the $x$-axis
with speed $v_{i1}$. On the other hand, the second object, mass $m_2$, initially moves at an
angle $\theta_i$ to the $x$-axis with speed $v_{i2}$. After the collision, the two
objects stick together and move off at an angle $\theta_f$ to the $x$-axis with
speed $v_f$. Momentum conservation along the $x$-axis yields
\begin{equation}\label{tin1}
m_1\,v_{i1} + m_2\,v_{i2}\,\cos\theta_i = (m_1+m_2)\,v_f\,\cos\theta_f.
\end{equation}
Likewise, momentum conservation along the $y$-axis gives
\begin{equation}\label{tin2}
 m_2\,v_{i2}\,\sin\theta_i = (m_1+m_2)\,v_f\,\sin\theta_f.
\end{equation}
Given the initial conditions ({\em i.e.}, $m_1$, $m_2$, $v_{i1}$, $v_{i2}$, and
$\theta_i$), we have a system of {\em two} equations [{\em i.e.}, Eqs.~(\ref{tin1}) and
(\ref{tin2})]  and {\em two} unknowns ({\em i.e.}, $v_f$ and $\theta_f$). 
Clearly, we should be able to find a unique solution for such a system. 

\subsection*{\em Worked Example 6.1: Cannon in a Railway Carriage}
\noindent{\em Question:} A cannon is bolted to the floor of a railway carriage, which is
free to move without friction along a straight track. The combined mass of the cannon and
the carriage is $M=1200\,{\rm kg}$. The cannon fires a cannonball, of mass $m=1.2\,{\rm kg}$,
horizontally with velocity $v=115\,{\rm m/s}$. The cannonball travels the length of the
carriage, a distance $L=85\,{\rm m}$,  and then becomes embedded in the carriage's end wall. What is the recoil
speed of the carriage right after the cannon is fired? What is the velocity of the
carriage after the cannonball strikes the far wall? What net distance, and in what
direction, does the carriage move as a result of the firing of the cannon?

\begin{figure*}[h]
\epsfysize=1in
\centerline{\epsffile{Chapter06/fig057.eps}}
\end{figure*}

\noindent{\em Answer:} Conservation of momentum implies that the net horizontal momentum
of the system is the same before and after the cannon is fired. The momentum before
the cannon is fired is zero, since nothing is initially moving. Hence, we can also set the momentum after
the cannon is fired to zero, giving
$$
0 = M\,u + m\,v,
$$
where $u$ is the recoil velocity of the carriage. It follows that
$$
u = - \frac{m}{M}\,v = -\frac{1.2\times 115}{1200} = -0.115\,{\rm m/s}.
$$
The minus sign indicates that the recoil velocity of the carriage is in the
opposite direction to the direction of motion of the cannonball. Hence, the
recoil {\em speed} of the carriage is $|u|= 0.115\,{\rm m/s}$.

Suppose that, after the cannonball strikes the far wall of the carriage, both
the cannonball and the carriage move with common velocity $w$. Conservation
of momentum implies that the net horizontal momentum of the system is the same
before and after the collision. Hence, we can write
$$
M\,u + m\,v= (M+m)\,w.
$$
However, we have already seen that $M\,u + m\,v=0$. It follows that $w=0$: in other
words, the carriage is brought to a complete halt when the cannonball strikes its far wall.

In the frame of reference of the carriage, the cannonball moves with velocity
$v-u$ after the cannon is fired. Hence, the time of flight of the cannonball is
$$
t = \frac{L}{v-u} = \frac{85}{115+0.115} = 0.738\,s.
$$
The distance moved by the carriage in this time interval is
$$
d = u\,t = -0.115\times 0.738 = -0.0849\,{\rm m}.
$$
Thus, the carriage moves $8.49\,{\rm cm}$ in the opposite direction to
the direction of motion of the cannonball.

\subsection*{\em Worked Example 6.2: Hitting a Softball}
\noindent{\em Question:} A softball of mass $m=0.35\,{\rm kg}$  is pitched
at a speed of $u=12\,{\rm m/s}$. The batter hits the ball directly back to
the pitcher at a speed of $v=21\,{\rm m/s}$. The bat acts on the ball
for $t=0.01\,{\rm s}$. What  impulse is imparted by the bat to the ball?
What  average force is exerted by the bat on the ball?

\noindent{\em Answer:} The initial momentum of the softball is $-m\,u$, whereas its
final momentum is $m\,v$. Here, the final direction of motion of the softball
is taken to be positive. Thus, the net change in momentum of the softball due to
its collision with the bat is
$$
{\mit\Delta} p = m\,v - (-)\,m\,u = 0.35\times(21+12) = 11.55\,{\rm N\,s}.
$$
By definition, the net momentum change is equal to the impulse imparted
by the bat, so
$$
I = {\mit\Delta} p = 11.55\,{\rm N\,s}.
$$

The average force exerted by the bat on the ball is simply the net impulse divided by the
time interval over which the ball is in contact with the bat. Hence,
$$
\bar{f} = \frac{I}{t}= \frac{11.55}{0.01} = 1155.0\,{\rm N}.
$$

\subsection*{\em Worked Example 6.3: Skater and Medicine Ball}
\noindent{\em Question:} A skater of  mass $M=120\,{\rm kg}$ is skating across a pond
with uniform velocity $v=8\,{\rm m/s}$. One of the skater's friends, who is
standing at the edge of the pond, throws a
medicine ball of mass $m=20\,{\rm kg}$ with velocity $u=3\,{\rm m/s}$ to the skater, who catches
it. The direction of motion of the ball is perpendicular to the initial direction of motion
of the skater. 
What is the final speed of the skater? What is the final direction
of motion of the skater relative to his/her initial direction of motion? Assume that the
skater moves without friction. 

\begin{figure*}[h]
\epsfysize=1.5in
\centerline{\epsffile{Chapter06/fig057a.eps}}
\end{figure*}

\noindent{\em Answer:} Suppose that the skater is initially moving along the $x$-axis, whereas
the initial direction of motion of the medicine ball is along the $y$-axis. The skater's
initial momentum is
$$
{\bf p}_1 = (M\,v,\,0) = (120\times 8,\,0 ) = (960,\, 0)\,{\rm N\,s}.
$$
Likewise, the initial momentum of the medicine ball is
$$
{\bf p}_2 = (0,\,m\,u) = (0,\,20\times 3) = (0, \,60) \,{\rm N\,s}.
$$
After the skater catches the ball, the combined momentum of the skater
and the ball is
$$
{\bf p}_3 = {\bf p}_1 + {\bf p}_2 = (960, \,60) \,{\rm N\,s}.
$$
This follows from momentum conservation. The final speed of the skater (and the ball)
is
$$
v' = \frac{|{\bf p}_3|}{M+m} = \frac{\sqrt{960^2+60^2}}{120+20} = 6.87\,{\rm m/s}.
$$
The final direction of motion of the skater is parameterized by the angle $\theta$
(see the above diagram), where
$$
\theta = \tan^{-1}\!\left(\frac{|{\bf p}_2|}{|{\bf p}_1|}
\right) =\tan^{-1}\!\left(\frac{60}{960}\right)= 3.58^\circ.
$$

\subsection*{\em Worked Example 6.4: Bullet and Block}
\noindent{\em Question:} A bullet of mass $m=12\,{\rm g}$ strikes a stationary wooden block
of mass $M=5.2\,{\rm kg}$ standing on a frictionless surface. The block, with
the bullet embedded in it, acquires a velocity of $v=1.7\,{\rm m/s}$. What
was the velocity of the bullet before it struck the block? What fraction of the
bullet's initial kinetic energy is lost ({\em i.e.}, dissipated) due to
the collision with the block?


\noindent{\em Answer:} Let $u$ be the initial velocity of the bullet.
Momentum conservation requires the total horizontal momentum of the
system to be the same before and after the bullet strikes the block.
The initial momentum of the system is simply $m\,u$, since the
block is initially at rest. The final momentum is $(M+m)\,v$, since
both the block and the bullet end up moving with velocity $v$. Hence,
$$
m\,u = (M+m)\,v,
$$
giving
$$
u = \frac{M+m}{m}\,v = \frac{(0.012+5.2)\times 1.7}{0.012} = 738.4\,{\rm m/s}.
$$

The initial kinetic energy of the bullet is
$$
K_i = \frac{1}{2}\,m\,u^2 = 0.5\times 0.012 \times 738.4^2 = 3.2714\,{\rm kJ}.
$$
The final kinetic energy of the system is
$$
K_f = \frac{1}{2}\,(M+m)\,v^2 = 0.5\times(0.012+5.2)\times 1.7^2 = 7.53\,{\rm J}.
$$
Hence, the fraction of the initial kinetic energy which is dissipated is
$$
f = \frac{K_i-K_f}{K_i} = \frac{3.2714\times 10^3-7.53}{3.2714\times 10^3} = 0.9977.
$$

\subsection*{\em Worked Example 6.5: Elastic Collision}
\noindent{\em Question:} An object of mass $m_1=2\,{\rm kg}$, moving
with velocity $v_{i1}= 12\,{\rm m/s}$,  collides head-on with
a stationary object whose mass is $m_2=6\,{\rm kg}$. Given that the collision
is elastic, what are the final velocities of the two objects. Neglect friction.

\noindent{\em Answer:} Momentum conservation yields
$$
m_1\,v_{i1} = m_1\,v_{f1} + m_2\,v_{f2},
$$
where $v_{f1}$ and $v_{f2}$ are the final velocities of the first and second
objects, respectively. Since the collision is elastic, the total kinetic
energy must be the same before and after the collision. Hence,
$$
\frac{1}{2}\,m_1\,v_{i1}^2 = \frac{1}{2}\,m_1\,v_{f1}^{\,2} + \frac{1}{2}\,m_2 \,v_{f2}^{\,2}.
$$

Let $x=v_{f1}/v_{i1}$ and $y=v_{f2}/v_{i1}$. Noting that $m_2/m_1=3$, the above two
equations reduce to
$$
1 = x + 3\,y,
$$
and
$$
1 = x^2 + 3\,y^2.
$$
Eliminating $x$ between the previous two expressions, we obtain
$$
1 = (1-3\,y)^2 + 3\,y^2,
$$
or
$$
6\,y\,(2\,y-1)=0,
$$
which has the non-trivial solution $y=1/2$. The corresponding solution for $x$ is $x=
(1-3\,y)=-1/2$.

It follows that the final velocity of the first object is
$$
v_{f1} = x\,v_{i1} = -0.5\times12 = -6\,{\rm m/s}.
$$
The minus sign indicates that this object reverses direction as a
result of the collision. Likewise, the final velocity of the second object is
$$
v_{f2} = y\,v_{i1} = 0.5 \times 12 = 6\,{\rm m/s}.
$$ 

\subsection*{\em Worked Example 6.6: Two-Dimensional Collision}
\noindent{\em Question:} Two objects slide over a frictionless
horizontal surface. The first object, mass $m_1=5\,{\rm kg}$, is propelled with
speed $v_{i1}  = 4.5\,{\rm m/s}$ toward the second object, mass
$m_2 =2.5\,{\rm kg}$, which is initially at rest. After the collision, both
objects have velocities which are directed $\theta=30^\circ$ on either side of the
original line of motion of the first object. What are the
final speeds of the two objects? Is the collision elastic or  inelastic?

\begin{figure*}[ht]
\epsfysize=2in
\centerline{\epsffile{Chapter06/fig057b.eps}}
\end{figure*}

\noindent{\em Answer:} Let us adopt the coordinate system shown in the diagram. Conservation
of momentum along the $x$-axis yields
$$
m_1\,v_{i1}  = m_1\,v_{f1}\,\cos\theta + m_2\,v_{f2}\,\cos\theta.
$$
Likewise, conservation of momentum along the $y$-axis yields
$$
m_1\,v_{f1}\,\sin\theta = m_2\,v_{f2}\,\sin\theta.
$$
The above pair of equations can be combined to give
$$
v_{f1} = \frac{v_{i1}}{2\,\cos\theta} = \frac{4.5}{2\times\cos 30^\circ} = 2.5981\,{\rm m/s},
$$
and
$$
v_{f2} = \frac{m_1}{m_2}\,v_{f1} = \frac{5\times 2.5981}{2.5} = 5.1962\,{\rm m/s}.
$$

The initial kinetic energy of the system is
$$
K_i = \frac{1}{2}\,m_1\,v_{i1}^{\,2} = 0.5\times 5\times 4.5^2 = 50.63\,{\rm J}.
$$
The final kinetic energy of the system is
$$
K_f= \frac{1}{2}\,m_1\,v_{f1}^{\,2}+ \frac{1}{2}\,m_2\,v_{f2}^{\,2}=
0.5\times 5\times 2.5981^2 + 0.5\times 2.5\times 5.1962^2 = 50.63\,{\rm J}.
$$
Since $K_i=K_f$, the collision is {\em elastic}.
