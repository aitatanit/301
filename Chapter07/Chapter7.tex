\section{Circular Motion}
\subsection{Introduction}
Up to now, we have basically only considered {\em rectilinear} motion: {\em i.e.}, motion
in a straight-line. Let us now broaden our approach so as to take into account
the most important type of non-rectilinear motion: namely, {\em circular} motion.

\subsection{Uniform Circular Motion}
Suppose that an object executes a circular orbit of radius
$r$ with {\em uniform} tangential speed $v$. The instantaneous
position of the object is most conveniently specified in terms of an
angle $\theta$.  See Fig.~\ref{f58}. For instance, we could decide that $\theta=0^\circ$ corresponds
to the object's location at $t=0$, in  which case we would write
\begin{equation}
\theta(t) = \omega\,t,
\end{equation}
where $\omega$ is termed the {\em angular velocity} of the object. For a uniformly
rotating object, the angular velocity is simply the angle through which the object
turns in one second.

\begin{figure}[h]
\epsfysize=1.7in
\centerline{\epsffile{Chapter07/fig058.eps}}
\caption{\em Circular motion.}\label{f58}  
\end{figure}

Consider the motion of the object in the time interval between
$t=0$ and $t=t$. In this interval, the object rotates through an angle
$\theta$, and traces out a circular arc of length $s$. See Fig.~\ref{f58}. 
It is fairly obvious that the arc length $s$ is directly proportional to the angle
$\theta$: but, what is the constant of proportionality? Well, an angle of
$360^\circ$ corresponds to an arc length of $2\pi\,r$. Hence, an angle $\theta$
must correspond to an arc length of
\begin{equation}\label{e71}
s = \frac{2\pi}{360^\circ} \,r\,\theta(^\circ).
\end{equation}
At this stage, it is convenient to define a new angular unit known as a {\em radian} (symbol rad.). An angle
measured in radians is related to an angle measured in degrees via the
following simple formula:
\begin{equation}
\theta({\rm rad.}) = \frac{2\pi}{360^\circ}\,\theta(^\circ).
\end{equation}
Thus, $360^\circ$ corresponds to $2\,\pi$ radians, $180^\circ$ corresponds to $\pi$ radians,
$90^\circ$ corresponds to $\pi/2$ radians, and $57.296^\circ$ corresponds
to 1 radian. When $\theta$ is measured in radians, Eq.~(\ref{e71}) simplifies
greatly to give
\begin{equation}
s = r\,\theta.
\end{equation}
Henceforth, in this course, all angles are measured in radians {\em by default}.

Consider the motion of the object in the short  interval between times $t$ and $t+\delta t$.
In this interval, the object turns through a small angle $\delta\theta$ and
traces out a short arc of length $\delta s$, where
\begin{equation}\label{e72}
\delta s = r\,\delta\theta.
\end{equation}
Now $\delta s/\delta t$ ({\em i.e.}, distance moved per unit time) 
is simply the tangential velocity $v$, whereas $\delta\theta /\delta t$ 
({\em i.e.}, angle turned through  per unit time) is simply the angular velocity
$\omega$. Thus, dividing  Eq.~(\ref{e72}) by $\delta t$, we obtain 
\begin{equation}
v = r\,\omega.\label{e76}
\end{equation}
Note, however, that this formula is only valid if the angular velocity 
$\omega$ is measured in {\em radians per second}. From now on, in this
course, all angular velocities are measured in radians per second {\em by default}.

An object that rotates with uniform angular velocity $\omega$ turns through
$\omega$ radians in 1 second. Hence, the object turns through $2\,\pi$ radians
({\em i.e.}, it executes a complete circle) in 
\begin{equation}
T = \frac{2\,\pi}{\omega}
\end{equation}
seconds. Here, $T$ is the {\em  repetition period} of the circular motion. If the object
executes a complete cycle ({\em i.e.}, turns through $360^\circ$) in $T$ seconds,
then the number of cycles executed per second is
\begin{equation}
f = \frac{1}{T} = \frac{\omega}{2\,\pi}.
\end{equation}
Here, the {\em repetition frequency}, $f$, of the motion is measured in
{\em cycles per second}---otherwise known as {\em hertz} (symbol Hz).

As an example, suppose that an object executes uniform circular motion,  radius
$r=1.2\,{\rm m}$, at a frequency of $f=50\,{\rm Hz}$ ({\em i.e.}, the object executes a complete
rotation 50 times a second). The repetition period of this motion is simply
\begin{equation}
T = \frac{1}{f} = 0.02\,{\rm s}.
\end{equation}
Furthermore, the angular frequency of the motion is given by
\begin{equation}
\omega = 2\pi\,f = 314.16\,{\rm rad. / s}.
\end{equation}
Finally, the tangential velocity of the object is
\begin{equation}
v = r\,\omega = 1.2\times 314.16 =376.99\,{\rm m/s}.
\end{equation}

\subsection{Centripetal Acceleration}
An object executing a circular orbit of radius $r$ with uniform tangential
speed $v$ possesses a velocity vector ${\bf v}$ whose magnitude is constant, but
whose direction is continuously changing. It follows that the object must
be {\em accelerating}, since (vector) acceleration is the rate of change of (vector)
velocity, and the (vector) velocity is indeed varying in time.

\begin{figure}
\epsfysize=3in
\centerline{\epsffile{Chapter07/fig059.eps}}
\caption{\em Centripetal acceleration.}\label{f59}  
\end{figure}

Suppose that the object moves from point $P$ to point $Q$ between times $t$ and $t+\delta t$, as
shown in Fig.~\ref{f59}. Suppose, further, that the object  rotates
through $\delta\theta$ radians in this time interval. The vector 
 $\stackrel{\displaystyle \rightarrow}{PX}$, 
shown in the diagram, is identical
to the vector $\stackrel{\displaystyle \rightarrow}{QY}$.  Moreover, the angle subtended between vectors
$\stackrel{\displaystyle \rightarrow}{PZ}$ and  $\stackrel{\displaystyle \rightarrow}{PX}$
is simply $\delta\theta$. The vector $\stackrel{\displaystyle \rightarrow}{ZX}$ represents
the change in vector velocity, $\delta{\bf v}$, between times $t$ and $t+\delta t$. It can be seen
that this vector is directed {\em towards the centre of the circle}. From standard trigonometry,
the length of vector $\stackrel{\displaystyle \rightarrow}{ZX}$ is 
\begin{equation}
\delta v = 2\,v\,\sin(\delta\theta/2).
\end{equation}
However, for small angles $\sin\theta \simeq \theta$, provided that $\theta$ 
is measured in {\em radians}. Hence, 
\begin{equation}
\delta v\simeq v\,\delta\theta.
 \end{equation}
It follows that
\begin{equation}
a = \frac{\delta v}{\delta t} = v\,\frac{\delta\theta}{\delta t} = v\,\omega,
\end{equation}
where $\omega = \delta \theta/\delta t$ is the angular velocity of the object, measured
in {\em radians per second}.
In summary,  an object executing a circular orbit, radius $r$, with uniform tangential
velocity $v$, and uniform angular velocity $\omega=v/r$, possesses an acceleration
directed towards the centre of the circle---{\em i.e.}, a {\em centripetal} acceleration---of
magnitude
\begin{equation}
a = v\,\omega = \frac{v^2}{r} = r\,\omega^2.\label{e715}
\end{equation}

\begin{figure}
\epsfysize=2.5in
\centerline{\epsffile{Chapter07/fig060.eps}}
\caption{\em Weight on the end of a cable.}\label{f60}  
\end{figure}

Suppose that a weight, of mass $m$, is attached to the end of a cable, of
length $r$, and whirled around such that the weight executes a horizontal circle,
 radius $r$, with uniform tangential velocity $v$. As we have just learned,
the weight is subject to a centripetal acceleration of magnitude $v^2/r$. Hence,
the weight experiences a centripetal force
\begin{equation}
f = \frac{m\,v^2}{r}.
\end{equation}
What provides this force? Well, in the present example, the force is provided by the
{\em tension} $T$ in the cable. Hence, $T=m\,v^2/r$. 

Suppose that the cable is such that it  snaps whenever the tension in it
exceeds a certain critical value $T_{\rm max}$. It follows that there is a
maximum velocity with which the weight can be whirled around: namely,
\begin{equation}
v_{\rm max} = \sqrt{\frac{r\,T_{\rm max}}{m}}.
\end{equation}
If $v$ exceeds $v_{\rm max}$ then the cable  will  break. As soon as the cable snaps,
the weight will cease to be subject to a centripetal force, so it will fly off---with 
velocity $v_{\rm max}$---along the {\em straight-line}
which is {\em tangential} to the circular orbit it was previously executing.

\subsection{Conical Pendulum}
Suppose that an object, mass $m$, is attached to the end of a light inextensible
string whose other end is attached to a rigid beam. Suppose, further, that the
object is given an initial horizontal velocity such that it executes a
horizontal circular orbit of radius $r$ with angular velocity $\omega$. See
Fig.~\ref{f61}. Let $h$ be the vertical distance between the beam and the plane of
the circular orbit, and let $\theta$ be the angle subtended by the string with
the downward vertical.

\begin{figure}
\epsfysize=2.5in
\centerline{\epsffile{Chapter07/fig061.eps}}
\caption{\em A conical pendulum.}\label{f61}  
\end{figure}

The object is subject to two forces: the gravitational force $m\,g$ which acts vertically
downwards, and the tension force $T$ which acts upwards along the string. The tension
force can be resolved into a component $T\,\cos\theta$ which acts vertically upwards, and
a component $T\,\sin\theta$ which acts towards the centre of the circle. Force balance
in the vertical direction yields
\begin{equation}\label{con1}
T\,\cos\theta = m\,g.
\end{equation}
In other words, the vertical component of the tension force balances the weight of the object.

Since the object is executing a circular orbit, radius $r$, with angular velocity $\omega$,
it experiences a centripetal acceleration $\omega^2\,r$. Hence, it is subject to
a centripetal force $m\,\omega^2\,r$. This force is provided by the component of
the string tension which acts towards the centre of the circle. In other words,
\begin{equation}\label{con2}
T\,\sin\theta = m\,\omega^2\,r.
\end{equation}

Taking the ratio of Eqs.~(\ref{con1}) and (\ref{con2}), we obtain
\begin{equation}
\tan\theta = \frac{\omega^2\,r}{g}.
\end{equation}
However, by simple trigonometry,
\begin{equation}
\tan\theta = \frac{r}{h}.
\end{equation}
Hence, we find
\begin{equation}
\omega = \sqrt{\frac{g}{h}}.
\end{equation}
Note that if $l$ is the length of the string then $h=l\,\cos\theta$. It follows that
\begin{equation}
\omega = \sqrt{\frac{g}{l\,\cos\theta}}.
\end{equation}

For instance, if the length of the string is $l=0.2\,{\rm m}$ and the conical angle is $\theta=30^\circ$
then the angular velocity of rotation is given by
\begin{equation}
\omega = \sqrt{\frac{9.81}{0.2\times \cos 30^\circ}} = 7.526\,{\rm rad./s}.
\end{equation}
This translates to a rotation frequency in cycles per second of
\begin{equation}
f = \frac{\omega}{2\,\pi} = 1.20\,{\rm Hz}.
\end{equation}

\subsection{Non-Uniform Circular Motion}\label{s75}
Consider an object which executes {\em non-uniform} circular motion, as shown in Fig.~\ref{f62}.
Suppose that the motion is confined to a 2-dimensional plane. We can specify the
instantaneous position of the object in terms of its {\em polar coordinates} $r$ and
$\theta$. Here, $r$ is the radial distance of the object from the origin, whereas $\theta$
is the angular bearing of the object from the origin, measured with respect to some arbitrarily chosen
direction. We imagine that both $r$ and $\theta$ are changing in time. As an example of
non-uniform circular motion, consider the motion of the Earth around the Sun. Suppose that 
the origin of our coordinate system corresponds to the position of the Sun. As the Earth rotates, its
angular bearing $\theta$, relative to the Sun, obviously changes in time. 
However, since the
Earth's orbit is slightly {\em elliptical}, its radial distance $r$ from the
Sun also varies in time. Moreover, as the Earth moves closer to the Sun, its rate of
rotation speeds up, and {\em vice versa}. Hence, the rate of change of $\theta$ with
time is non-uniform.

\begin{figure}
\epsfysize=2.5in
\centerline{\epsffile{Chapter07/fig062.eps}}
\caption{\em Polar coordinates.}\label{f62}  
\end{figure}

Let us define two unit vectors, ${\bf e}_r$ and ${\bf e}_\theta$. Incidentally,
a unit vector simply a vector whose length is unity. 
As shown in Fig.~\ref{f62}, the {\em radial}
unit vector ${\bf e}_r$ always points from the origin to the instantaneous position of the object.
Moreover, the {\em tangential} unit vector ${\bf e}_\theta$ is always {\em normal}
to  ${\bf e}_r$, in the direction of increasing $\theta$. The position vector ${\bf r}$
of the object can be written
\begin{equation}
{\bf r} = r\,{\bf e}_r.\label{ec33}
\end{equation}
In other words, vector ${\bf r}$ points in the same direction as the radial unit vector 
${\bf e}_r$, and is of
length $r$. We can write the object's velocity in the form
\begin{equation}\label{ec56}
{\bf v} = \dot{\bf r} = v_r\,{\bf e}_r + v_\theta\,{\bf e}_\theta,
\end{equation}
whereas the acceleration is written
 \begin{equation}\label{ec57}
{\bf a} = \dot{\bf v} = a_r\,{\bf e}_r + a_\theta\,{\bf e}_\theta.
\end{equation}
Here, $v_r$ is termed the object's {\em radial velocity}, whilst $v_\theta$ is
termed the {\em tangential velocity}. Likewise, $a_r$ is the {\em radial acceleration},
and $a_\theta$ is the {\em tangential acceleration}. But, how do we express these
quantities in terms of the object's polar coordinates $r$ and $\theta$? It turns out that this
is a far from straightforward task. For instance, if we simply differentiate Eq.~(\ref{ec33})
with respect to time, we obtain
\begin{equation}
{\bf v}=\dot{r}\,{\bf e}_r + r\,\dot{\bf e}_r,
\end{equation}
where $\dot{\bf e}_r$ is the time derivative of the radial unit vector---this quantity
is non-zero because ${\bf e}_r$ {\em changes direction} as the object moves. Unfortunately,
it is not entirely clear how to evaluate $\dot{\bf e}_r$. In the following, we
outline a famous trick for calculating $v_r$, $v_\theta$, {\em etc.} without
ever having to evaluate the time derivatives of the unit vectors
${\bf e}_r$ and ${\bf e}_\theta$. 

Consider a general {\em complex number},
\begin{equation}
z = x +  i\,y,
\end{equation}
where $x$ and $y$ are real, and $i$ is the square root of $-1$ ({\em i.e.}, $i^2=-1$).
Here, $x$ is the real part of $z$, whereas $y$ is the imaginary part.
We can visualize $z$ as a point in the so-called {\em complex plane}: {\em i.e.}, a 2-dimensional
plane
in which the real parts of complex numbers are plotted along one Cartesian axis, whereas the
corresponding imaginary parts are plotted along the other axis.
 Thus, the coordinates
of $z$ in the complex plane are simply ($x$, $y$). See Fig.~\ref{f63}. In other words,
we can use a complex number to represent a position vector in a 2-dimensional plane.
Note that the length of the vector is equal to the {\em modulus} of the corresponding complex
number. Incidentally, the modulus of $z = x + i\,y$ is defined
\begin{equation}
|z| = \sqrt{x^2 + y^2}.
\end{equation}

\begin{figure}
\epsfysize=2.5in
\centerline{\epsffile{Chapter07/fig063.eps}}
\caption{\em Representation of a complex number in the complex plane.}\label{f63}  
\end{figure}

Consider the complex number ${\rm e}^{\,i\,\theta}$, where $\theta$ is real.
A famous result in complex analysis---known as {\em de Moivre's theorem}---allows us
to split this number into its real and imaginary components:
\begin{equation}
{\rm e}^{\,i\,\theta} = \cos\theta + i\,\sin\theta.\label{ec44}
\end{equation}
Now, as we have just discussed, we can think of  ${\rm e}^{\,i\,\theta}$ as representing a vector
in the complex plane: the real and imaginary parts of ${\rm e}^{\,i\,\theta}$
form the coordinates of the head of the vector, whereas the tail of the vector
corresponds to the origin. What are the properties of this vector? Well,
the length of the vector is given by
\begin{equation}
\left|{\rm e}^{\,i\,\theta}\right| = \sqrt{\cos^2\theta + \sin^2\theta } = 1.
\end{equation}
In other words, ${\rm e}^{\,i\,\theta}$ represents a {\em unit vector}.
In fact, it is clear from Fig.~\ref{f64} that ${\rm e}^{\,i\,\theta}$ represents
the radial unit vector ${\bf e}_r$ for an object whose angular polar
coordinate (measured anti-clockwise from the real axis) is $\theta$. 
Can we also find a complex representation of the corresponding tangential
unit vector ${\bf e}_\theta$? Actually, we can. The
complex number $i\,{\rm e}^{\,i\,\theta}$ can be written
\begin{equation}
i\,{\rm e}^{\,i\,\theta} = -\sin\theta + i\,\cos\theta.
\end{equation}
Here, we have just multiplied Eq.~(\ref{ec44}) by $i$, making use of the fact
that $i^2 = -1$. This number again represents a unit vector, since
\begin{equation}
\left|i\,{\rm e}^{\,i\,\theta}\right| = \sqrt{\sin^2\theta + \cos^2\theta } = 1.
\end{equation}
Moreover, as is clear from Fig.~\ref{f64}, this vector is normal to ${\bf e}_r$, in the
direction of increasing $\theta$. In other words, $i\,{\rm e}^{\,i\,\theta}$ represents
the tangential unit vector ${\bf e}_\theta$. 

\begin{figure}
\epsfysize=2.5in
\centerline{\epsffile{Chapter07/fig064.eps}}
\caption{\em Representation of the unit vectors ${\bf e}_r$ and
${\bf e}_\theta$ in  the  complex plane.}\label{f64}  
\end{figure}

Consider an object executing non-uniform circular motion in the complex plane. By analogy
with Eq.~(\ref{ec33}), we can represent the instantaneous position vector of
this object via the complex number
\begin{equation}
z = r\, {\rm e}^{\,i\,\theta}.\label{ec55}
\end{equation}
Here, $r(t)$ is the object's radial distance from the origin, whereas $\theta(t)$
is its angular bearing relative to the real axis. Note that, in the above formula, we are using
${\rm e}^{\,i\,\theta}$ to represent the radial unit vector ${\bf e}_r$. Now,
if $z$ represents the position vector of the object, then $\dot{z} = dz/dt$ must
represent the object's velocity vector. Differentiating Eq.~(\ref{ec55}) with
respect to time, using the standard rules of calculus, we obtain
\begin{equation}
\dot{z} = \dot{r}\,{\rm e}^{\,i\,\theta} + r\,\dot{\theta}\,i\,{\rm e}^{\,i\,\theta}.
\end{equation}
Comparing with Eq.~(\ref{ec56}), recalling that ${\rm e}^{\,i\,\theta}$ represents ${\bf e}_r$
and $i\,{\rm e}^{\,i\,\theta}$ represents ${\bf e}_\theta$, we obtain
\begin{eqnarray}
v_r &=& \dot{r},\label{e738}\\[0.5ex]
v_\theta &=& r\,\dot{\theta}=r\,\omega,\label{e739}
\end{eqnarray}
where $\omega = d\theta/dt$ is the object's instantaneous angular velocity. Thus,
as desired,
we have obtained expressions for the radial and tangential velocities of the
object in terms of its polar coordinates, $r$ and $\theta$.
We can go further. Let us differentiate $\dot{z}$ with respect to time,
in order to obtain a complex number representing the object's vector
acceleration. Again, using the standard rules of calculus, we obtain
\begin{equation}
\ddot{z} = (\ddot{r}-r\,\dot{\theta}^2)\,{\rm e}^{\,i\,\theta} + (r\,\ddot{\theta}+2\,\dot{r}\,
\dot{\theta})\,i\,{\rm e}^{\,i\,\theta}.
\end{equation}
Comparing with Eq.~(\ref{ec57}), recalling that ${\rm e}^{\,i\,\theta}$ represents ${\bf e}_r$
and $i\,{\rm e}^{\,i\,\theta}$ represents ${\bf e}_\theta$, we obtain
\begin{eqnarray}
a_r &=& \ddot{r}-r\,\dot{\theta}^2 = \ddot{r}-r\,\omega^2,\label{e741}\\[0.5ex]
a_\theta &=& r\,\ddot{\theta}+2\,\dot{r}\,
\dot{\theta} = r\,\dot{\omega} + 2\,\dot{r}\,\omega.\label{e742}
\end{eqnarray}
Thus, we now have expressions for the object's radial and tangential accelerations in terms
of $r$ and $\theta$. The beauty of this derivation is that the complex analysis
has automatically taken care of the fact that the unit vectors
${\bf e}_r$ and ${\bf e}_\theta$ change direction as the object moves.

Let us now consider the commonly occurring special case in which an object executes a circular
orbit at {\em fixed radius}, but varying angular velocity. Since the radius is fixed, it follows
that $\dot{r}=\ddot{r}=0$. According to Eqs.~(\ref{e738}) and (\ref{e739}), the radial
velocity of the object is zero, and the tangential velocity takes the form
\begin{equation}
v_\theta = r\,\omega.
\end{equation}
Note that the above equation is exactly the same as Eq.~(\ref{e76})---the only difference
is that we have now proved that this relation holds for non-uniform, as well as uniform,
circular motion.
According to Eq.~(\ref{e741}), the radial acceleration is given by
\begin{equation}
a_r = -r\,\omega^2.
\end{equation}
The minus sign indicates that this acceleration is directed towards the centre of the
circle. Of course, the above equation is equivalent to Eq.~(\ref{e715})---the only difference
is that we have now proved that this relation holds for non-uniform, as well as uniform,
circular motion. Finally, according to Eq.~(\ref{e742}), the tangential acceleration
takes the form
\begin{equation}
a_\theta = r\,\dot{\omega}.
\end{equation}
The existence of a non-zero tangential acceleration (in the former case) is the one difference between
non-uniform and uniform circular motion (at constant radius).

\subsection{Vertical Pendulum}
Let us now examine an example of non-uniform circular motion. 
Suppose that an object of mass $m$ is attached to the end of a light rigid rod, or light string, of length
$r$. The other end of the rod, or string, is attached to a stationary pivot
 in such a manner that the object is free to execute a vertical circle about this pivot. Let $\theta$
measure the angular position of the object, measured with
respect to the downward vertical. Let $v$ be the velocity of the object at $\theta=0^\circ$.
 How large do we have to make $v$ in order for the object to execute a complete
vertical circle?

\begin{figure}
\epsfysize=3in
\centerline{\epsffile{Chapter07/fig065.eps}}
\caption{\em Motion in a vertical circle.}\label{f65}  
\end{figure}

Consider Fig.~\ref{f65}. Suppose that the object moves from point $A$, where its tangential 
velocity is $v$, to
point $B$, where its tangential velocity is $v'$. Let us, first of all, obtain the relationship
between $v$ and $v'$. This is most easily achieved by considering energy conservation.
At point $A$, the object is situated a vertical distance $r$ below the pivot, whereas at point $B$ the
vertical distance below the pivot has been reduced to $r\,\cos\theta$. Hence, in moving
from $A$ to $B$ the object gains potential energy $m\,g\,r\,(1-\cos\theta)$. This gain
in potential energy must be offset by a corresponding loss in kinetic energy. Thus,
\begin{equation}
\frac{1}{2}\,m\,v^2 - \frac{1}{2}\,m\,{v'}^{\,2} = m\,g\,r\,(1-\cos\theta),
\end{equation}
which reduces to
\begin{equation}\label{e747}
{v'}^{\,2} = v^2 - 2\,r\,g\,(1-\cos\theta).
\end{equation}

Let us now examine the radial acceleration of the object at point $B$. The radial forces
acting on the object are the tension $T$ in the rod, or string, which acts towards
the centre of the circle, and the component $m\,g\,\cos\theta$ of the object's weight,
which acts away from the centre of the circle. Since the object is executing
circular motion with instantaneous tangential velocity $v'$, it must experience
an instantaneous acceleration ${v'}^{\,2}/r$ towards the centre of the circle. Hence,
Newton's second law of motion yields
\begin{equation}\label{e748}
\frac{m\,{v'}^{\,2}}{r} = T - m\,g\,\cos\theta.
\end{equation}
Equations~(\ref{e747}) and (\ref{e748}) can be combined to give
\begin{equation}\label{e799}
T = \frac{m\,v^2}{r} + m\,g\,(3\,\cos\theta -2).
\end{equation}

Suppose that the object is, in fact, attached to the end of a piece of string, rather than a
rigid rod. One important property of strings is that, unlike rigid rods, they cannot
support negative tensions. In other words, a string can only pull objects attached to its
two ends together---it cannot push them apart. Another way of putting this is that if the
tension in a string ever becomes negative then the string will become slack and collapse.
Clearly, if our object is to execute a full vertical circle then then tension $T$ in the
string must
remain positive for all values of $\theta$. It is clear from Eq.~(\ref{e799}) that the
tension attains its minimum value when $\theta =180^\circ$ (at which point $\cos\theta=-1$). This
is hardly surprising, since $\theta=180^\circ$ corresponds to the point at which the
object attains its maximum height, and, therefore, its minimum tangential velocity. It is certainly
the case that if the string tension is positive at this point then it must be positive
at all other points. Now, the tension at $\theta=180^\circ$ is given by
\begin{equation}
T_0 =  \frac{m\,v^2}{r} -5\,m\,g.
\end{equation}
Hence, the condition for the object to execute a complete vertical circle without the
string becoming slack is $T_0>0$, or
\begin{equation}\label{e788}
v^2 > 5\,r\,g.
\end{equation}
Note that this condition is independent of the mass of the object.

Suppose that the object is attached to the end of a rigid rod, instead of a
piece of string. There is now no constraint on the tension, since a rigid rod can
quite easily support a negative tension ({\em i.e.}, it can push, as well as pull, on objects
attached to its two ends). 
However, in order for the object to execute a complete vertical circle the square of its
tangential velocity ${v'}^{\,2}$ must remain positive at all values of $\theta$. 
It is clear from Eq.~(\ref{e747}) that 
${v'}^{\,2}$ attains its minimum value when $\theta =180^\circ$. This
is, again, hardly surprising. Thus, if ${v'}^{\,2}$ is positive at this point then it must be positive
at all other points. Now, the expression for ${v'}^{\,2}$ at $\theta=180^\circ$ is
\begin{equation}
({v'}^{\,2})_0 = v^2 - 4\,r\,g.
\end{equation}
Hence, the condition for the object to execute a complete vertical circle is $({v'}^{\,2})_0 >0$, or
\begin{equation}
v^2 > 4\,r\,g.
\end{equation}
Note that this condition is slightly easier to satisfy than the condition (\ref{e788}). In other words,
it is slightly easier to cause an object attached to the end of a rigid rod to execute a
vertical circle than it is to cause an object attached to the end of a string
to execute the same circle. The reason for this is that the rigidity of the rod
helps support the object when it is situated above the pivot point.

\subsection{Motion on Curved Surfaces}
Consider a smooth rigid vertical hoop of internal radius $r$, as shown in Fig.~\ref{f66}.
Suppose that an object of mass $m$ slides without friction around the inside of this hoop.
What is the motion of this object? Is it possible for the object to execute a complete
vertical circle?

\begin{figure}
\epsfysize=3in
\centerline{\epsffile{Chapter07/fig066.eps}}
\caption{\em Motion on the inside of a vertical hoop.}\label{f66}  
\end{figure}

 Suppose that the object moves from point $A$ to
point $B$ in Fig.~\ref{f66}.  In doing so, it gains potential energy $m\,g\,r\,(1-\cos\theta)$,
where $\theta$ is the angular coordinate of the object measured with
respect to the downward vertical. This gain
in potential energy must be offset by a corresponding loss in kinetic energy. Thus,
\begin{equation}
\frac{1}{2}\,m\,v^2 - \frac{1}{2}\,m\,{v'}^{\,2} = m\,g\,r\,(1-\cos\theta),
\end{equation}
which reduces to
\begin{equation}\label{e747a}
{v'}^{\,2} = v^2 - 2\,r\,g\,(1-\cos\theta).
\end{equation}
Here, $v$ is the velocity at point $A$ ($\theta=0^\circ$), and $v'$ is the velocity
at point $B$ ($\theta =\theta^\circ$). 

Let us now examine the radial acceleration of the object at point $B$. The radial forces
acting on the object are the reaction $R$ of the vertical hoop, which acts towards
the centre of the hoop, and the component $m\,g\,\cos\theta$ of the object's weight,
which acts away from the centre of the hoop. Since the object is executing
circular motion with instantaneous tangential velocity $v'$, it must experience
an instantaneous acceleration ${v'}^{\,2}/r$ towards the centre of the hoop. Hence,
Newton's second law of motion yields
\begin{equation}\label{e748a}
\frac{m\,{v'}^{\,2}}{r} = R - m\,g\,\cos\theta.
\end{equation}

Note, however, that there is a constraint on the reaction $R$ that the hoop
can exert on the object. This reaction must always be {\em positive}. In other
words, the hoop can push the object away from itself, but it can never pull
it towards itself. Another way of putting this is that if the reaction ever becomes
negative then the object will fly off the surface of the hoop, since it is no longer
being pressed into this surface. It should be clear, by now, that the problem
we are considering is exactly analogous to the earlier problem of an object attached to the end
of a piece of string which is executing a vertical circle, with the reaction $R$ of the
hoop playing the role of the tension $T$ in the string.

Let us imagine that the hoop under consideration is a ``loop the loop'' segment in a
fairground roller-coaster. The object sliding around the inside of the loop then becomes
the roller-coaster train. Suppose that the fairground operator can vary the
velocity $v$ with which the train is sent into the bottom of the loop
 ({\em i.e.}, the velocity at $\theta=0^\circ$). What is the safe
range of $v$? Now, if the train starts at $\theta=0^\circ$ with velocity $v$ then
there are only {\em three} possible outcomes. Firstly, the train can execute a complete
circuit of the loop. Secondly, the train can slide part way up the loop, come to a halt,
reverse direction, and then slide back down again. Thirdly, the train can slide part way up the loop,
but then fall off the loop. Obviously, it is the third possibility that the fairground
operator would wish to guard against. 

Using the analogy between this problem and the problem of a mass on the end of a piece
of string executing a vertical circle, the condition for the roller-coaster train to
execute a complete circuit is
\begin{equation}\label{e755}
v^2 > 5\,r\,g.
\end{equation}
Note, interestingly enough, that this condition is independent of the mass of the train.

Equation~(\ref{e748a}) yields
\begin{equation}
{v'}^{\,2} = \frac{r\,R}{m} - r\,g\,\cos\theta.
\end{equation}
Now, the condition for the train to reverse direction without falling off the loop
is ${v'}^{\,2}=0$ with $R>0$. Thus, the train reverses direction when
\begin{equation}
R=   m\,g\,\cos\theta.
\end{equation}
Note that this equation can only be satisfied for positive $R$ when $\cos\theta>0$.
In other words, the train can only turn around without falling off the loop
if the turning point lies in the {\em lower half} of the loop ({\em i.e.},
$-90^\circ<\theta<90^\circ$).
The condition for the train to fall off the loop is
\begin{equation}
{v'}^{\,2} =- r\,g\,\cos\theta.
\end{equation}
Note that this equation can only be satisfied for positive ${v'}^{\,2}$ when $\cos\theta<0$.
In other words, the train can only fall off the loop when it is situated in the {\em upper half} of the
loop. It is fairly clear that if the train's initial velocity is not sufficiently large for
it to execute a complete circuit of the loop, and not sufficiently small for it to turn around before
entering the upper half of the loop, then it must inevitably fall off the loop somewhere in the
loop's upper half. The critical value of $v^2$ above which the train
executes a complete circuit is $5\,r\,g$ [see Eq.~(\ref{e755})]. The critical value of $v^2$ at which the train just
turns around before entering the upper half of the loop is $2\,r\,g$ [this is obtained
from Eq.~(\ref{e747a}) by setting $v'=0$ and $\theta=90^\circ$]. Hence, the
dangerous range of $v^2$ is
\begin{equation}
2\,r\,g < v^2< 5\,r\,g.
\end{equation}
For $v^2<2\,r\,g$, the train turns around in the lower half of the loop.
For $v^2 >  5\,r\,g$, the train executes a complete circuit around the loop. However, for
$2\,r\,g < v^2< 5\,r\,g$, the train falls off the loop somewhere in its upper half.

\begin{figure}
\epsfysize=1.75in
\centerline{\epsffile{Chapter07/fig067.eps}}
\caption{\em A skier on a hemispherical mountain.}\label{f67}  
\end{figure}

Consider a skier of mass $m$ skiing down a hemispherical mountain of radius $r$, as
shown in Fig.~\ref{f67}. Let $\theta$ be the angular coordinate of the skier, measured
with respect to the upward vertical. Suppose that the skier starts at rest ($v=0$) on top
of the mountain ($\theta=0^\circ$), and slides down the mountain without friction.
At what point does the skier fly off the surface of the mountain?

Suppose that the skier has reached angular coordinate $\theta$. At this stage, the
skier has fallen though a height $r\,(1-\cos\theta)$. Thus, the tangential
velocity $v$ of the skier is given by energy conservation:
\begin{equation}\label{e7aa}
\frac{1}{2}\,m\,v^2 = m\,g\,r\, (1-\cos\theta).
\end{equation}
Let us now consider the skier's radial acceleration. The radial forces acting
on the skier are the reaction $R$ exerted by the mountain, which acts radially
outwards, and the component of the skier's weight $m\,g\,\cos\theta$, which
acts radially inwards. Since the skier is executing circular motion, radius $r$,
with instantaneous tangential velocity $v$, he/she experiences an instantaneous
inward radial acceleration $v^2/r$. Hence, Newton's second law of motion
yields
\begin{equation}\label{e7bb}
m\,\frac{v^2}{r} = m\,g\,\cos\theta - R.
\end{equation}

Equations (\ref{e7aa}) and (\ref{e7bb}) can be combined to give
\begin{equation}
R = m\,g\,(3\,\cos\theta-2).
\end{equation}
As before, the reaction $R$ is constrained to be {\em  positive}---the
mountain can push outward on the skier, but it cannot pull the skier inward.
In fact, as soon as the reaction becomes negative, the skier flies of
the surface of the mountain. This occurs when $\cos\theta_0 = 2/3$, or
$\theta_0 = 48.19^\circ$. The height through which the skier falls
before becoming a ski-jumper is $h=r\,(1-\cos\theta_0) = a/3$.

\subsection*{\em Worked Example 7.1: Banked Curve}
\noindent{\em Question:} Civil engineers generally {\em bank} curves on roads in such a manner
that a car going around the curve at the recommended speed does not have to
rely on  friction between its tires and the road surface in order to round the curve.
Suppose that the radius of curvature of a given curve is $r=60\,{\rm m}$, and that
the recommended speed is $v=40\,{\rm km/h}$. At what angle $\theta$ should the
curve be banked?

\begin{figure*}
\epsfysize=1.75in
\centerline{\epsffile{Chapter07/fig067a.eps}}
\end{figure*}

\noindent{\em Answer:} Consider a car of mass $m$ going around the curve. The car's
weight, $m\,g$, acts vertically downwards. The road surface exerts an upward normal reaction $R$
on the car.
The vertical component of the reaction must balance the downward weight of the car, so
$$
R\,\cos\theta = m\,g.
$$
The horizontal component of the reaction, $R\,\sin\theta$, acts towards the centre of curvature of
the road.  This 
component provides the force $m\,v^2/r$ towards the centre of the curvature which the
car experiences as it rounds the curve. In other words,
$$
 R\,\sin\theta = m\,\frac{v^2}{r},
$$
which yields
$$
\tan\theta = \frac{v^2}{r\,g}, 
$$
or
$$
\theta =\tan^{-1}\left(\frac{v^2}{r\,g}\right).
$$
Hence,
$$
\theta = \tan^{-1}\left(\frac{(40\times 1000/3600)^2}{60\times9.81}\right)= 11.8^\circ.
$$
Note that if the car attempts to round the curve at the wrong speed then
$m\,v^2/r\neq m\,g\,\tan\theta$, and the difference has to be made up by a
sideways friction force exerted between the car's tires and the road surface. Unfortunately,
this does not always work---especially if the road surface is wet!

\subsection*{\em Worked Example 7.2: Circular Race Track}
\noindent{\em Question:} A car of mass $m=2000\,{\rm kg}$ travels around a flat
circular race track of radius $r = 85\,{\rm m}$. The car starts at rest, and its
speed increases at the constant rate $a_\theta = 0.6\,{\rm m/s}$. What is the speed
of the car at the point when its centripetal and tangential accelerations are equal?

\noindent{\em Answer:} The tangential acceleration of the car is $a_\theta = 0.6\,{\rm m/s}$.
When the car travels with tangential velocity $v$ its centripetal acceleration is
$a_r=v^2/r$. Hence, $a_r = a_\theta$ when
$$
\frac{v^2}{r} = a_\theta,
$$
or
$$
v = \sqrt{r\, a_\theta} = \sqrt{85\times 0.6} = 7.14\,{\rm m/s}.
$$

\subsection*{\em Worked Example 7.3: Amusement Park Ride}
\noindent{\em Question:} An amusement park ride consists of a vertical cylinder that
spins about a vertical axis. When the cylinder spins sufficiently fast, any person
inside it is held up against the wall. Suppose that the coefficient of static
friction between a typical person and the wall is $\mu=0.25$. Let the mass of an typical
person be $m=60\,{\rm kg}$, and let $r=7\,{\rm m}$ be the radius of the cylinder. 
Find the critical angular velocity of the cylinder above which a typical person
will not slide down the wall. How many revolutions per second
is the cylinder executing at this critical velocity?

\begin{figure*}[h]
\epsfysize=2in
\centerline{\epsffile{Chapter07/fig067b.eps}}
\end{figure*}

\noindent{\em Answer:} In the vertical direction, the person is subject to
a downward force $m\,g$ due to gravity, and a maximum upward force $f=\mu\,R$ due
to friction with the wall. Here, $R$ is the normal reaction between the person
and the wall. In order for the person not to slide down the wall, we require
$f>m\,g$. Hence, the critical case corresponds to 
$$
f=\mu\,R=m\,g.
$$

In the radial direction, the person is subject to a single force: namely, the reaction
$R$ due to the wall, which acts radially inwards. If the cylinder (and, hence, the person)
rotates with angular velocity $\omega$, then this force must provided the acceleration
$r\,\omega^2$ towards the axis of rotation. Hence,
$$
R = m\,r\,\omega^2.
$$
It follows that, in the critical case,
$$
\omega = \sqrt{\frac{g\,}{\mu\,r}} = \sqrt{\frac{9.81}{0.25\times 7}} = 2.37\,{\rm rad/s}.
$$
The corresponding number of revolutions per second is
$$
f = \frac{\omega}{2\,\pi} = \frac{2.37}{2\times 3.1415} = 0.38\,{\rm Hz}.
$$

\subsection*{\em Worked Example 7.4: Aerobatic Maneuver} 

\begin{figure*}[h]
\epsfysize=2in
\centerline{\epsffile{Chapter07/fig067c.eps}}
\end{figure*}

\noindent{\em Question:} A stunt pilot experiences weightlessness
momentarily at the top of a ``loop the loop'' maneuver. Given that the
speed of the stunt plane is $v=500\,{\rm km/h}$, what is the radius $r$ of the
loop?

\noindent{\em Answer:} Let $m$ be the mass of the pilot.
Consider the radial acceleration of the pilot at the
top of the loop. The pilot is subject to two radial forces: the gravitational
force $m\,g$, which acts towards the centre of the loop, and the reaction force $R$,
due to the plane, which acts away from the centre of the loop. Since the pilot
experiences an acceleration $v^2/r$ towards the centre of the loop, Newton's second
law of motion yields
$$
m\,\frac{v^2}{r} = m\,g- R.
$$
Now, the reaction $R$ is equivalent to the apparent weight of the pilot. 
In particular, if the pilot is ``weightless'' then he/she exerts no force on the plane, and,
therefore, the plane exerts no reaction force on the pilot. Hence, if the pilot is 
weightless at the top of the loop then $R=0$, giving
$$
r = \frac{v^2}{g} = \frac{(500\times 1000/3600)^2}{9.81} = 1.97\,{\rm km}.
$$

\subsection*{\em Worked Example 7.5: Ballistic Pendulum} 
\noindent{\em Question:} A bullet of mass $m=10\,{\rm g}$ strikes a pendulum bob
of mass $M=1.3\,{\rm kg}$ horizontally with speed $v$,  and then becomes embedded in the bob. The bob
is initially at rest, and is suspended by a stiff rod of length $l=0.6\,{\rm m}$
and negligible mass. The bob is free to rotate in the vertical direction.
What is the minimum value of $v$ which causes the bob to execute a
complete vertical circle? How does the answer change if the bob is suspended from 
a light flexible rod (of the same length), instead of a stiff rod?

\noindent{\em Answer:} When the bullet strikes the bob, and then sticks to it, the bullet and
bob move off with a velocity  $v'$ which is given by momentum conservation:
$$
m\,v = (M+m)\,v'.
$$
Hence,
$$
v' = \frac{m\,v}{M+m}.
$$

Consider the case where the bob is suspended by a rigid rod.
If the bob and bullet  only just manage to execute a vertical loop, then their initial kinetic
energy $(1/2)\,(M+m)\,{v'}^{\,2}$ must only just be sufficient to lift them from the bottom 
to the top of the loop---a distance $2\,l$. Hence, in this critical case,
energy conservation yields
$$
\frac{1}{2}\,(M+m)\,{v'}^{\,2} = (M+m)\,2\,g\,l,
$$
which implies
$$
{v'}^{\,2} = 4\,g\,l,
$$
or
$$
v =  \frac{ (M+m)\,\sqrt{4\,g\,l} }{ m }  = 
\frac{1.31\times\sqrt{4\times 9.81\times 0.6}}{0.01}
=635.6\,{\rm m/s}.
$$

Consider the case where the bob is suspended by a flexible rod.
The velocity $v''$ of the bob and bullet at the top of the loop is obtained from energy
conservation:
$$
\frac{1}{2}\,(M+m)\,{v''}^{\,2} = \frac{1}{2}\,(M+m)\,{v'}^{\,2} - (M+m)\,2\,g\,l.
$$
If the bob and bullet  only just manage to execute a vertical loop, then the tension
in the rod is zero at the top of the loop. Hence, the acceleration due to
gravity $g$ must account exactly for the required acceleration ${v''}^2/l$ towards the centre of the loop:
$$
\frac{{v''}^{\,2}}{l} = g.
$$
It follows that, in this critical case,
$$
{v'}^{\,2} = 5\,g\,l,
$$
or
$$
v =  \frac{(M+m)\,\sqrt{5\, g\,l} }{ m }  = 
\frac{1.31\times\sqrt{5\times 9.81\times 0.6}}{0.01}
=710.7\,{\rm m/s}.
$$
