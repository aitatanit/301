\section{Rotational Motion}
\subsection{Introduction}
Up to now, we have  only analyzed the dynamics
of {\em point masses} ({\em i.e.}, objects whose spatial extent is either
negligible or plays no role in their motion). Let us now
broaden our approach   in order to take {\em extended objects} into account.
Now, the only type of motion which a point mass object can exhibit is {\em translational
motion}: {\em i.e.}, motion by which the object moves from one point in space
to another. However, an extended object can exhibit another, quite distinct, type of
motion by which it remains located (more or less) at the same spatial position,
but constantly changes its orientation with respect to other fixed points in space.
This new type of motion is called {\em rotation}. Let us investigate rotational motion.

\subsection{Rigid Body Rotation}
Consider a rigid body executing pure rotational motion ({\em i.e.}, rotational motion which
has no translational component). It is possible to define an {\em axis of rotation}
(which, for the sake of simplicity, is assumed to pass through the body)---this
axis corresponds to   the straight-line
which is the locus of all points inside the body which remain stationary as the body rotates. A general point
located inside the body executes {\em circular motion} which is centred on the rotation axis, and orientated
in the plane perpendicular to this axis. In the following, we tacitly assume that the axis
of rotation remains fixed.

\begin{figure}
\epsfysize=3in
\centerline{\epsffile{Chapter08/fig068.eps}}
\caption{\em Rigid body rotation.}\label{f68}  
\end{figure}

Figure~\ref{f68} shows a typical rigidly 
rotating body. The axis of rotation is the line $AB$. A general point
$P$ lying within the body executes a circular orbit, centred on $AB$, in the plane perpendicular to
$AB$. Let the line $QP$ be a radius of this orbit which links the axis of rotation to the
instantaneous position of  $P$ at time $t$. Obviously, this implies that $QP$ is normal
to $AB$. Suppose that at time $t+\delta t$ point $P$ has moved to $P'$, and the radius
$QP$ has rotated through an angle $\delta\phi$. The instantaneous
{\em angular velocity}  of the body $\omega(t)$ is defined
\begin{equation}
\omega = \lim_{\delta t\rightarrow 0}\frac{\delta\phi}{\delta t}=\frac{d\phi}{dt}.
\end{equation}
Note that if the body is indeed rotating rigidly, then the calculated value of $\omega$ should
be the same for all possible points $P$ lying within the body (except for those points lying
exactly on the axis of rotation, for which $\omega$ is ill-defined). 
The rotation speed $v$ of point $P$ is related to the angular velocity $\omega$
of the body via
\begin{equation}
v = \sigma\,\omega,
\end{equation}
where $\sigma$ is the {\em perpendicular distance}  from the axis of rotation to
point $P$. Thus,
in a rigidly rotating body, the rotation speed increases {\em linearly} with (perpendicular) distance
from the axis of rotation.

It is helpful to introduce the {\em angular acceleration} $\alpha(t)$ of a rigidly rotating body:
this quantity is defined as the 
time derivative of the angular velocity. Thus,
\begin{equation}
\alpha  = \frac{d\omega}{dt}= \frac{d^2\phi}{dt^2},
\end{equation}
where $\phi$ is the angular coordinate of some arbitrarily chosen point reference
within the body, measured
with respect to the rotation axis.
Note that angular velocities are conventionally measured in radians per second, whereas angular
accelerations are measured in radians per second squared.
 
 For a body rotating with constant angular velocity, $\omega$,
the angular acceleration is zero, and the rotation angle $\phi$ increases linearly with time:
\begin{equation}
\phi(t) = \phi_0 + \omega\,t,
\end{equation}
where $\phi_0 = \phi(t=0)$. Likewise, for a body rotating with constant angular acceleration,
$\alpha$, the angular velocity increases linearly with time, so that
\begin{equation}
\omega(t) = \omega_0 + \alpha\,t,
\end{equation}
and the rotation angle satisfies
\begin{equation}
\phi(t) = \phi_0 + \omega_0\,t + \frac{1}{2}\,\alpha\,t^2.
\end{equation}
Here, $\omega_0=\omega(t=0)$. Note that there is a clear analogy between the above
equations, and the equations of rectilinear motion at constant acceleration introduced in 
Sect.~\ref{caccn}---rotation angle plays the role of displacement, angular velocity plays the role of (regular) velocity, and
angular acceleration plays the role of (regular) acceleration.

\subsection{Is Rotation a Vector?}\label{srotn}
Consider a rigid body which rotates through an angle $\phi$ about a given
axis.
It is tempting to try to define a rotation ``vector'' $\bphi$ which describes this
motion. For example, suppose that $\bphi$ is defined as the ``vector'' whose magnitude
is the angle of  rotation, $\phi$, and whose direction runs parallel to the axis of 
rotation. Unfortunately, this  definition is ambiguous, since there are
two possible directions which run parallel to the rotation axis. However, we can
resolve this problem by adopting the following convention---the
rotation ``vector'' runs parallel to the axis of rotation in the sense indicated by the
thumb of the right-hand, when the fingers of this hand circulate around the axis in the
direction of  rotation. This convention is known as the {\em right-hand grip} rule.
The right-hand grip rule is illustrated in Fig.~\ref{f69}.

\begin{figure}
\epsfysize=3in
\centerline{\epsffile{Chapter08/fig069.eps}}
\caption{\em The right-hand grip rule.}\label{f69}  
\end{figure}

The  rotation ``vector'' $\bphi$  now has a well-defined magnitude and
 direction. But, is this quantity really a vector? 
This may seem like a strange question to ask, but it turns out that not all
quantities which have well-defined magnitudes and directions are necessarily
vectors. Let us review some properties of vectors. If ${\bf a}$ and ${\bf b}$
are two general vectors, then it is certainly the case that
\begin{equation}
{\bf a} + {\bf b} = {\bf b} + {\bf a}.
\end{equation}
In other words, the addition of vectors is necessarily {\em commutative} ({\em i.e.}, it is
independent of the order of addition). Is this true for ``vector'' rotations, as we have 
just defined them?
Figure~\ref{f70}  shows the effect of applying two successive $90^\circ$ rotations---one about the $x$-axis, and the other about the $z$-axis---to a six-sided die. In the
left-hand case, the $z$-rotation is applied before the $x$-rotation, and {\em vice
versa} in the right-hand case. It can be seen that the die ends up in two completely
different states. Clearly, the $z$-rotation plus the
$x$-rotation {\em does not} equal
the  $x$-rotation plus the $z$-rotation. This non-commutative algebra cannot be
represented by vectors. We conclude that, although rotations have  well-defined magnitudes and
directions, they are {\em not}, in general, vector quantities. 

\begin{figure}
\epsfysize=4in
\centerline{\epsffile[-300 -40 860 771]{Chapter08/fig070.eps}}
\caption{\em The addition of rotation is non-commutative.}\label{f70}
\end{figure}

There is a direct analogy between rotation and motion over the Earth's surface. After
all, the  motion of a pointer along the Earth's equator from longitude $0^\circ$W to
longitude $90^\circ$W could  just as well be achieved by keeping the pointer fixed and
rotating the Earth through $90^\circ$ about a North-South axis. The non-commutative nature
of rotation ``vectors'' is a direct consequence of the non-planar ({\em i.e.}, curved)
nature of the Earth's surface. 
 For instance, suppose we start off at ($0^\circ$\,N, $0^\circ$\,W), which is just off the Atlantic
coast of equatorial Africa, and rotate $90^\circ$ northwards and then  $90^\circ$ eastwards.
We end up at ($0^\circ$\,N, $90^\circ$\,E), which is in the middle of the Indian Ocean. However,
if we start at the same point, and  rotate $90^\circ$ eastwards and then $90^\circ$ northwards,
we end up at the North pole. Hence, large rotations over the Earth's surface do
not commute.
 Let us now repeat this experiment on a far smaller
scale. Suppose that we walk 10\,m northwards and then 10\,m eastwards.
Next, suppose  that---starting from
the same initial position---we walk 10\,m eastwards and then 10\,m northwards. In this case, few people
would need much convincing that  the two end points are essentially identical. The
crucial  point
is that for sufficiently small displacements the Earth's surface is approximately planar, and
vector displacements on a plane surface commute with one another. This observation immediately
suggests that rotation ``vectors'' which correspond to rotations through {\em small angles} 
must also commute with
one another. In other words, although the quantity $\bphi$, defined above, is not a true
vector, the infinitesimal quantity $\delta\bphi$, which is defined in a similar manner but
corresponds to a rotation through an infinitesimal angle $\delta\phi$, is a perfectly good
vector.

We have just established that it is possible to define a true  vector $\delta \bphi$ which
describes a rotation through a {\em small} angle $\delta\phi$ about a fixed axis. But, how is this
definition
useful? Well, suppose that vector $\delta\bphi$ describes the small rotation that a given
object executes in the infinitesimal time interval between $t$ and $t+\delta t$. We can 
then define the quantity
\begin{equation}
\bomega = \lim_{\delta t\rightarrow 0} \frac{\delta\bphi}{\delta t} = \frac{d\bphi}{dt}.
\end{equation}
This quantity is clearly a true vector, since it is simply the ratio of a true vector and
a scalar. Of course, $\bomega$ represents an {\em angular velocity vector}. The
magnitude of this vector, $\omega$, specifies the instantaneous angular velocity of the
object, whereas the direction of the vector indicates the axis of rotation. The sense
of rotation is given by the right-hand grip rule: if the thumb of the right-hand points along
the direction of the vector, then the fingers of the right-hand indicate the sense of rotation.
We conclude that, although rotation can only be thought of as a vector quantity under certain
very special circumstances, we can safely treat angular velocity as a vector
quantity under all circumstances.

Suppose, for example, that a rigid body rotates at constant angular velocity $\bomega_1$.
Let us now combine this motion with rotation about a {\em different axis} at constant
angular velocity $\bomega_2$. What is the subsequent motion of the body? Since we know
that angular velocity is a vector, we can be certain that the combined motion simply
corresponds to rotation about a third axis at constant angular velocity
\begin{equation}
\bomega_3 = \bomega_1 + \bomega_2,\label{e89}
\end{equation}
where the sum is performed according to the standard rules of vector addition. [Note, however,
the following important proviso. In order for Eq.~(\ref{e89}) to be valid, the rotation
axes corresponding to $\bomega_1$ and $\bomega_2$ must {\em cross} at a certain
point---the rotation axis corresponding to $\bomega_3$ then passes through this point.]
 Moreover,
a  constant angular velocity
\begin{equation}
\bomega = \omega_x\,\hat{\bf x} + \omega_y\,\hat{\bf y} + \omega_z\,\hat{\bf z}
\end{equation}
can be thought of as representing rotation about the $x$-axis at angular velocity $\omega_x$,
combined with rotation about the $y$-axis at angular velocity $\omega_y$, combined with
rotation about the $z$-axis at angular velocity $\omega_z$. [There is, again, a proviso---namely,
that the rotation axis corresponding to $\bomega$ must pass through the origin. Of course,
we can always shift the origin such that this is the case.]
 Clearly, the knowledge that
angular velocity is vector quantity can be extremely useful.

\subsection{The Vector Product}\label{svp}
We saw earlier, in Sect.~\ref{s2ed}, that it is possible to combine two vectors  multiplicatively, 
by means of a {\em scalar product}, to
form a scalar. Recall that the scalar product ${\bf a}\!\cdot\!{\bf b}$
of two vectors ${\bf a}=(a_x,a_y,a_z)$ and ${\bf b}=(b_x,b_y,b_z)$ is defined
\begin{equation}\label{dsp}
{\bf a} \!\cdot\!{\bf b} = a_x\,b_x+ a_y\,b_y + a_z\,b_z = |{\bf a}|\,|{\bf b}|\,\cos\theta,
\end{equation}
where $\theta$ is the angle subtended between the directions of ${\bf a}$ and ${\bf b}$. 

Is it also possible to combine two vector multiplicatively to form a third (non-coplanar) vector?
 It turns out that this goal can be achieved via the use of the so-called
{\em vector product}. By definition, the vector product, 
${\bf a}\times{\bf b}$, of two vectors ${\bf a}=(a_x,a_y,a_z)$ and ${\bf b}=(b_x,b_y,b_z)$ 
is of 
magnitude
\begin{equation}
|{\bf a}\times{\bf b} | 
=|{\bf a}|\,|{\bf b}|\,\sin\theta.
\end{equation}
The direction of ${\bf a}\times{\bf b}$ is  {\em mutually perpendicular} to ${\bf a}$ and
${\bf b}$, in the sense given by the right-hand grip rule when vector ${\bf a}$ is rotated onto
vector ${\bf b}$ (the direction of rotation being
 such that the angle of rotation is less than $180^\circ$).
See Fig.~\ref{f71}. In coordinate form,
\begin{equation}\label{e899}
{\bf a}\times{\bf b}  = (a_y\,  b_z-a_z\, b_y, a_z \,b_x - a_x\, b_z, a_x\, b_y - a_y\, b_x).
\end{equation}

\begin{figure}
\epsfysize=2in
\centerline{\epsffile{Chapter08/fig071.eps}}
\caption{\em The vector product.}\label{f71}  
\end{figure}

There are a number of fairly obvious consequences of the above definition. Firstly, if vector
${\bf b}$ is parallel to vector ${\bf a}$, so that we can write ${\bf b} = \lambda\,{\bf a}$,
then the vector product ${\bf a}\times {\bf b}$ has {\em zero magnitude}. The easiest way of seeing
this is to note that if ${\bf a}$ and ${\bf b}$ are parallel then the angle $\theta$ subtended between
them is zero, hence the magnitude of the vector product, $|{\bf a}|\,|{\bf b}|\,\sin\theta$, must
also be zero (since $\sin 0^\circ=0$). Secondly, the order of multiplication matters.
Thus, ${\bf b}\times {\bf a}$ is {\em not} equivalent to ${\bf a}\times {\bf b}$. In fact,
as can be seen from Eq.~(\ref{e899}), 
\begin{equation}
{\bf b}\times {\bf a} = - {\bf a}\times {\bf b}.
\end{equation}
In other words, ${\bf b}\times {\bf a}$ has the same magnitude as ${\bf a}\times {\bf b}$, but
points in diagrammatically the opposite direction.

Now that we have defined the vector product of two vectors, let us find a use for this concept.
Figure~\ref{f72} shows a rigid body rotating with angular velocity $\bomega$. For the
sake of simplicity, the axis of rotation, which runs parallel to $\bomega$, is
assumed to pass through the origin $O$ of our coordinate system. Point $P$, whose position
vector is ${\bf r}$, represents a general point inside the body. What is the velocity
of rotation ${\bf v}$ at point $P$? Well, the magnitude of this velocity is simply
\begin{equation}
v=\sigma\,\omega = \omega\,r\,\sin\theta,
\end{equation}
where $\sigma$ is the perpendicular distance of point $P$ from the axis of rotation, and 
$\theta$ is the angle subtended between the directions of $\bomega$ and ${\bf r}$. 
The direction of the velocity is {\em into} the page. Another way of saying this, is that the
direction of the velocity is mutually perpendicular to the directions of $\bomega$ and
${\bf r}$, in the sense indicated by the right-hand grip rule when $\bomega$ 
is rotated onto ${\bf r}$ (through an angle less than $180^\circ$). It follows that
we can write
\begin{equation}
{\bf v} = \bomega\times {\bf r}.
\end{equation}
Note, incidentally, 
 that the direction of the angular velocity vector $\bomega$ indicates the orientation
of the axis of rotation---however, nothing actually moves in this direction; in fact, all of the motion
is {\em perpendicular} to the direction of $\bomega$.

\begin{figure}
\epsfysize=3in
\centerline{\epsffile{Chapter08/fig072.eps}}
\caption{\em Rigid rotation.}\label{f72}  
\end{figure}

\subsection{Centre of Mass}\label{s85}
The {\em centre of mass}---or centre of gravity---of an extended object is defined in much
the same manner as we earlier defined the  centre of mass of a set of mutually
interacting point mass objects---see Sect.~\ref{scm}. To be more exact, the
coordinates of the centre of mass of an extended object are the mass weighted
averages of the coordinates of the elements which make up that object. Thus,
if  the object has net mass $M$, and is composed of $N$ elements,
such that the $i$th element has mass $m_i$ and position vector ${\bf r}_i$, then the
position vector of the centre of mass is given by
\begin{equation}\label{e8cm1}
{\bf r}_{cm} = \frac{1}{M} \sum_{i=1,N} m_i\,{\bf r}_i.
\end{equation}
If the object under consideration is {\em  continuous}, then
\begin{equation}\label{e8cm2}
m_i = \rho({\bf r}_i)\,V_i,
\end{equation}
where $\rho({\bf r})$ is the mass density of the object, and $V_i$ is the volume occupied by the
$i$th element. Here, it is assumed that this volume is small compared to the
total volume of the object. Taking the limit that the number of elements goes to
infinity, and the volume of each element goes to zero, Eqs.~(\ref{e8cm1}) and (\ref{e8cm2})
yield the following integral formula for the position vector of the centre of mass:
\begin{equation}
{\bf r}_{cm} = \frac{1}{M} \int\!\int\!\int \rho\,{\bf r}\,dV.
\end{equation}
Here, the integral is taken over the whole volume of the object, and $dV = dx\,dy\,dz$ is an element
of that volume. Incidentally, the triple integral sign indicates a volume integral: {\em i.e.},
a simultaneous integral over three independent Cartesian coordinates.
Finally, for an object whose mass density is {\em constant}---which is the only type of
object that we shall be considering in this course---the above expression
reduces to
\begin{equation}\label{e8cm3}
{\bf r}_{cm} = \frac{1}{V} \int\!\int\!\int {\bf r}\,dV,
\end{equation}
where $V$ is the volume of the object. According to Eq.~(\ref{e8cm3}), 
the centre of mass of a body of uniform density is located at the {\em geometric centre} of that body.

\begin{figure}[h]
\epsfysize=1.5in
\centerline{\epsffile{Chapter08/fig073.eps}}
\caption{\em Locating the geometric centre of a cube.}\label{f73}  
\end{figure}

For many solid objects, the location of the geometric centre follows from symmetry.
For instance, the geometric centre of a 
cube is the point of intersection of the cube's diagonals. See Fig.~\ref{f73}. Likewise,
the geometric centre of a right cylinder is located on the axis, half-way up the
cylinder. See Fig.~\ref{f74}.

\begin{figure}[h]
\epsfysize=2in
\centerline{\epsffile{Chapter08/fig074.eps}}
\caption{\em Locating the geometric centre of a right cylinder.}\label{f74}  
\end{figure}

As an illustration of the use of formula (\ref{e8cm3}), let us calculate
the geometric centre of a regular square-sided pyramid. Figure~\ref{f75} shows such a
pyramid. Let $a$ be the length of each side. It follows, from simple
trigonometry,  that the height of
the pyramid is $h=a/\sqrt{2}$. Suppose that the base of the pyramid lies on the $x$-$y$ plane,
and  the apex is aligned with the $z$-axis, as shown in the diagram. 
It follows, from symmetry, that the geometric centre of the pyramid lies on the $z$-axis.
It only remains to calculate the perpendicular distance, $z_{cm}$, between the geometric
centre and the base of the pyramid. This quantity is obtained from the $z$-component
of Eq.~(\ref{e8cm3}):
\begin{equation}\label{e8py}
z_{cm} = \frac{\int\!\int\!\int z\,dx\,dy\,dz}{\int\!\int\!\int\,dx\,dy\,dz},
\end{equation}
where the integral is taken over the volume of the pyramid.

\begin{figure}
\epsfysize=2.5in
\centerline{\epsffile{Chapter08/fig075.eps}}
\caption{\em Locating the geometric centre of a regular square-sided pyramid.}\label{f75}  
\end{figure}

In the above integral, the limits of integration for $z$ are $z=0$ to $z=h$, respectively
({\em i.e.}, from the base  to the apex of the pyramid). 
The corresponding limits of integration for $x$ and $y$ are $x,y=-a\,(1-z/h)/2$ to
$x,y=+a\,(1-z/h)/2$, respectively ({\em i.e.}, the limits are $x,y=\pm a/2$ at the
base of the pyramid, and $x,y=\pm 0$ at the apex). Hence, Eq.~(\ref{e8py}) can
be written more explicitly as
\begin{equation}
z_{cm} = \frac{\int_0^h\,z\,dz\!\int_{-a\,(1-z/h)/2}^{+a\,(1-z/h)/2}\,dy\!
\int_{-a\,(1-z/h)/2}^{+a\,(1-z/h)/2}\,dx}{\int_0^h\,dz\!\int_{-a\,(1-z/h)/2}^{+a\,(1-z/h)/2}\,dy\!
\int_{-a\,(1-z/h)/2}^{+a\,(1-z/h)/2}\,dx}.
\end{equation}
As indicated above, it makes sense to perform the $x$- and $y$- integrals before the $z$-integrals,
since the limits of integration for the $x$- and $y$- integrals are $z$-dependent.
Performing the $x$-integrals, we obtain
\begin{equation}
z_{cm} = \frac{\int_0^h\,z\,dz\!\int_{-a\,(1-z/h)/2}^{+a\,(1-z/h)/2}\,a\,(1-z/h)\,dy}{\int_0^h\,dz\!\int_{-a\,(1-z/h)/2}^{+a\,(1-z/h)/2}\,
a\,(1-z/h)\,dy}.
\end{equation}
Performing the $y$-integrals, we obtain
\begin{equation}
z_{cm} = \frac{\int_0^h\,a^2\,z\,(1-z/h)^2\,dz}{\int_0^h\,a^2\,(1-z/h)^2\,dz}.
\end{equation}
Finally, performing the $z$-integrals, we obtain
\begin{equation}
z_{cm} = \frac{ a^2\left[z^2/2 - 2\,z^3/(3\,h) + z^4/(4\,h^2)\right]_0^h}
{a^2\left[z - z^2/(h) + z^3/(3\,h)\right]_0^h} = \frac{a^2\,h^2/12}{a^2\,h/3} = \frac{h}{4}.
\end{equation}
Thus, the geometric centre of a regular square-sided pyramid is located on the 
symmetry axis, one quarter of the way from the base to the apex.

\subsection{Moment of Inertia}\label{smoi}
Consider an extended object which is made up of $N$ elements. Let the $i$th element
possess mass $m_i$, position vector ${\bf r}_i$, and velocity ${\bf v}_i$. The
total kinetic energy of the object is written
\begin{equation}
K = \sum_{i=1,N} \frac{1}{2}\,m_i\,v_i^{\,2}.
\end{equation}
Suppose that the motion of the object consists merely of rigid rotation at angular
velocity $\bomega$. It follows, from Sect.~\ref{svp}, that
\begin{equation}
{\bf v}_i = \bomega\times {\bf r}_i.
\end{equation}
Let us write
\begin{equation}
\bomega = \omega\,{\bf k},
\end{equation}
where ${\bf k}$ is a unit vector aligned along the axis of rotation (which
is assumed to pass through the origin of our coordinate system). It follows from
the above equations that the kinetic energy of rotation of the object takes the
form 
\begin{equation}
K = \sum_{i=1,N} \frac{1}{2}\,m_i\,|{\bf k}\times {\bf r}_i|^2\,\omega^2,
\end{equation}
or
\begin{equation}\label{e8xx}
K = \frac{1}{2}\,I\,\omega^2.
\end{equation}
Here, the quantity $I$ is termed the {\em moment of inertia} of the object, and
is written
\begin{equation}
I = \sum_{i=1,N} m_i\,|{\bf k}\times {\bf r}_i|^2= \sum_{i=1,N} m_i\,\sigma_i^{\,2},
\end{equation}
where $\sigma_i=|{\bf k}\times {\bf r}_i|$ is the perpendicular distance from the $i$th element to the axis of
rotation. Note that for translational motion we usually write
\begin{equation}\label{e8yy}
K = \frac{1}{2}\,M\,v^2,
\end{equation}
where $M$ represents  mass and $v$ represents  speed. A comparison of
Eqs.~(\ref{e8xx}) and (\ref{e8yy}) suggests that moment of inertia plays the
same role in rotational motion that mass plays in translational motion.

For a continuous object, analogous arguments to those employed in Sect.~\ref{s85}
yield
\begin{equation}
I = \int\!\int\!\int \rho\,\sigma^2\,dV,
\end{equation}
where $\rho({\bf r})$ is the mass density of the object, $\sigma = |{\bf k}\times {\bf r}|$ is
the perpendicular distance from the axis of rotation,
and $dV$ is a volume element.
Finally, for an object of constant density, the above expression reduces to
\begin{equation}\label{e8bb}
I = M\,\frac{\int\!\int\!\int \sigma^2\,dV}
{\int\!\int\!\int \,dV}.
\end{equation}
Here, $M$ is the total mass of the object. Note that the integrals are taken over the whole
volume of the object. 

The moment of inertia of a uniform  object depends not only on the size and shape of that
object but on the location of the axis about which the object is rotating. In particular,
the same object can have  different moments of inertia when rotating about 
different axes. 

Unfortunately, the evaluation of the moment of inertia of a given body about a given axis invariably
involves the performance of a nasty volume integral. In fact, there is only
one trivial moment of inertia calculation---namely, the moment of inertia of a thin
circular ring about a symmetric axis    which runs perpendicular
to the plane of the ring. See Fig.~\ref{f76}. Suppose that $M$ is the mass of the ring, and
$b$ is its radius. Each element of the ring shares a common perpendicular distance from
the axis of rotation---{\em i.e.}, $\sigma=b$. Hence, Eq.~(\ref{e8bb})
reduces to 
\begin{equation}
I = M\,b^2.
\end{equation}

\begin{figure}
\epsfysize=2in
\centerline{\epsffile{Chapter08/fig076.eps}}
\caption{\em The moment of inertia of a ring about a perpendicular symmetric axis.}\label{f76}  
\end{figure}

In general, moments of inertia are rather tedious to calculate. Fortunately, there exist 
two powerful theorems which enable us to simply relate the moment of inertia of a given body
about a given axis to the moment of inertia of the same body about another axis. The first of
these theorems is called the {\em perpendicular axis theorem}, and only applies to
uniform {\em laminar} objects. Consider a laminar object ({\em i.e.}, a thin, planar object)
of uniform density. Suppose, for the sake of simplicity,
 that the object lies in the $x$-$y$ plane. The moment of inertia of the object about the
$z$-axis is given by
\begin{equation}
I_z = M\,\frac{\int\,\int\, (x^2+y^2)\,dx\,dy}{\int\,\int\,dx\,dy},
\end{equation}
where we have suppressed the trivial $z$-integration, and the integral is taken
over the extent of the object in the $x$-$y$ plane.  Incidentally, the
above expression follows from the observation that $\sigma^2 = x^2+y^2$
when the axis of rotation is coincident with the $z$-axis. Likewise, the moments of 
inertia of the object about the $x$- and $y$- axes take the form
\begin{eqnarray}
I_x & =& M\,\frac{\int\,\int\, y^2\,dx\,dy}{\int\,\int\,dx\,dy},\\[0.5ex]
I_y & =& M\,\frac{\int\,\int\, x^2\,dx\,dy}{\int\,\int\,dx\,dy},
\end{eqnarray}
respectively.
Here, we have made use of the fact that $z=0$ inside the object. It follows by inspection
of the previous three equations that
\begin{equation}
I_z = I_x + I_y.
\end{equation}
See Fig.~\ref{f77}. 

\begin{figure}
\epsfysize=2in
\centerline{\epsffile{Chapter08/fig077.eps}}
\caption{\em The perpendicular axis theorem.}\label{f77}  
\end{figure}

Let us use the perpendicular axis theorem to find the moment of inertia of a thin ring about
a symmetric axis which lies in the plane of the ring. Adopting the coordinate system shown in
Fig.~\ref{f78}, it is clear, from symmetry, that $I_x=I_y$. 
Now, we already know that $I_z=M\,b^2$,
where $M$ is the mass of the ring, and $b$ is its radius. Hence, the perpendicular axis
theorem tells us that
\begin{equation}
2\,I_x = I_z,
\end{equation}
or
\begin{equation}
I_x = \frac{I_z}{2} = \frac{1}{2}\,M\,b^2.
\end{equation}
Of course, $I_z>I_x$, because when the ring spins about the $z$-axis its elements are, on average,
farther from the axis of rotation than when it spins about the $x$-axis.

\begin{figure}
\epsfysize=2in
\centerline{\epsffile{Chapter08/fig078.eps}}
\caption{\em The moment of inertia of a ring about a coplanar symmetric axis.}\label{f78}  
\end{figure}

The second useful theorem regarding moments of inertia is called the {\em parallel
axis theorem}. The parallel axis theorem---which is quite general---states that if $I$
is the moment of inertia of a given body about an axis passing through the centre of mass
of that body, then the moment of inertia $I'$ of the same body about a second axis
which is parallel to the first is
\begin{equation}
I' = I + M\,d^2,
\end{equation}
where $M$ is the mass of the body, and $d$ is the perpendicular distance between the
two axes. 

In order to prove the parallel axis theorem, let us choose the origin of our
coordinate system to coincide with the centre of mass of the body in question. 
Furthermore, let us orientate the axes of our coordinate system such that
the $z$-axis coincides with the first axis of rotation, whereas the second
axis pieces the $x$-$y$ plane at $x=d, y=0$. From Eq.~(\ref{e8cm3}), the fact that
the centre of mass is located at the origin implies that
\begin{equation}\label{e8nn}
\int\!\int\!\int\, x\,dx\,dy\,dz = \int\!\int\!\int\,y\,dx\,dy\,dz=\int\!\int\!\int\, z\,dx\,dy\,dz=0,
\end{equation}
where the integrals are taken over the volume of the body. From Eq.~(\ref{e8bb}),
the expression for the first moment of inertia is
\begin{equation}\label{e8mm}
I = M\,\frac{\int\!\int\!\int (x^2+y^2)\,\,dx\,dy\,dz}
{\int\!\int\!\int \,dx\,dy\,dz},
\end{equation}
since $x^2+y^2$ is the perpendicular distance of a general point $(x,y,z)$ from the $z$-axis.
Likewise, the expression for the second moment of inertia takes the
form 
\begin{equation}
I' = M\,\frac{\int\!\int\!\int [(x-d)^2+y^2]\,\,dx\,dy\,dz}
{\int\!\int\!\int \,dx\,dy\,dz}.
\end{equation}
The above equation can be expanded to give
\begin{eqnarray}
I' &=& M\,\frac{\int\!\int\!\int [(x^2+y^2) - 2\,d\,x + d^2]\,dx\,dy\,dz}
{\int\!\int\!\int \,dx\,dy\,dz}\nonumber\\[0.5ex]
&=& M\frac{\int\!\int\!\int (x^2+y^2)\,dx\,dy\,dz}
{\int\!\int\!\int \,dx\,dy\,dz}-2\,d\,M\frac{\int\!\int\!\int x\,dx\,dy\,dz}
{\int\!\int\!\int \,dx\,dy\,dz}\nonumber\\[0.5ex]&&
 + d^2\,M \frac{\int\!\int\!\int dx\,dy\,dz}
{\int\!\int\!\int \,dx\,dy\,dz}.
\end{eqnarray}
It follows from Eqs.~(\ref{e8nn}) and (\ref{e8mm}) that
\begin{equation}
I' = I + M\,d^2,
\end{equation}
which proves the theorem.

Let us use the parallel axis theorem to calculate the moment of inertia, $I'$, of a thin
ring about an axis which runs perpendicular to the plane of the ring, and passes
through the circumference of the ring. We know that the moment of inertia of a ring of mass
$M$ and radius $b$ about an axis which runs perpendicular to the plane of the ring, and passes
through the centre of the ring---which coincides with the centre
of mass of the ring---is $I=M\,b^2$. Our new axis is parallel to this original axis, but shifted
sideways by the perpendicular distance $b$. Hence, the parallel
axis theorem tells us that
\begin{equation}
I' = I + M\,b^2 = 2\,M\,b^2.
\end{equation}
See Fig.~\ref{f79}.

\begin{figure}
\epsfysize=2in
\centerline{\epsffile{Chapter08/fig079.eps}}
\caption{\em An application of the parallel axis theorem.}\label{f79}  
\end{figure}

As an illustration of the direct application  of formula (\ref{e8bb}), let us
calculate the moment of inertia of a thin circular disk, of mass $M$ and radius $b$,
about an axis which passes through the centre of the disk, and runs perpendicular to
the plane of the disk. Let us choose our coordinate system such that the disk
lies in the $x$-$y$ plane with its centre at the origin. The axis of rotation is, therefore,
coincident with the $z$-axis. Hence, formula (\ref{e8bb}) reduces to
\begin{equation}
I = M\,\frac{\int\!\int\, (x^2+y^2)\,dx\,dy}{\int\!\int\, dx\,dy},
\end{equation}
where the integrals are taken over the area of the disk, and the redundant $z$-integration
has been suppressed. Let us divide the disk up into thin annuli. Consider an annulus
of radius $\sigma=\sqrt{x^2+y^2}$ and radial thickness $d\sigma$. The area of this annulus is simply
$2\pi\,\sigma\,d\sigma$. Hence, we can replace $dx\,dy$ in the above integrals by
$2\pi\,\sigma\,d\sigma$, so as to give
\begin{equation}
I = M\,\frac{\int_0^b 2\pi\,\sigma^3\,d\sigma}{\int_0^b 2\pi\,\sigma\,d\sigma}.
\end{equation}
The above expression yields
\begin{equation}
I = M\,\frac{\left[2\,\pi\,\sigma^4/4\right]_0^b}{\left[2\,\pi\,\sigma^2/2\right]_0^b}=
\frac{1}{2}\,M\,b^2.
\end{equation}

Similar calculations to the above yield the following standard results:
\begin{itemize}
\item The moment of inertia of a thin rod of mass $M$ and length $l$ about an axis
passing through the centre of the rod and perpendicular to its length is 
$$
I = \frac{1}{12}\,M\,l^2.
$$
\item The moment of inertia of a thin rectangular sheet of mass $M$ and dimensions $a$ and $b$
about a perpendicular axis passing through the centre of the sheet is
$$
I = \frac{1}{12}\,M\,(a^2+b^2).
$$
\item The moment of inertia of a
solid cylinder of mass $M$ and radius $b$ about the cylindrical axis is
$$
I = \frac{1}{2}\,M\,b^2.
$$
\item The moment of inertia of a
thin spherical shell of mass $M$ and radius $b$ about a diameter is
$$
I = \frac{2}{3}\,M\,b^2.
$$
\item The moment of inertia of a
solid sphere of mass $M$ and radius $b$ about a diameter is
$$
I = \frac{2}{5}\,M\,b^2.
$$
\end{itemize}

\subsection{Torque}
We have now identified the rotational equivalent of velocity---namely, angular velocity---and
the rotational equivalent of mass---namely, moment of inertia. But, what is the
rotational equivalent of force? 

Consider a bicycle wheel of radius $b$ which is free to rotate around a perpendicular
axis passing through its centre. Suppose that we apply a force ${\bf f}$, which is coplanar with
the wheel, to a
point $P$ lying on its circumference. See Fig.~\ref{f80}.
What is the wheel's subsequent motion?

Let us choose the origin $O$ of our coordinate system to coincide with the pivot
point of the wheel---{\em i.e.}, the point of intersection between the wheel and the axis
of rotation. Let ${\bf r}$ be the position vector of point $P$, and let $\theta$
be the angle subtended between the directions of ${\bf r}$ and 
${\bf f}$. We can resolve ${\bf f}$ into two components---namely, a component $f\,\cos\theta$
which acts radially, and a  component $f\,\sin\theta$ which acts tangentially. The radial
component of ${\bf f}$ is canceled out by a reaction at the pivot, since the wheel
is assumed to be mounted in such a manner that it  can only rotate, and is prevented from
displacing sideways. The tangential component of ${\bf f}$ causes the wheel to
accelerate tangentially. Let $v$ be the instantaneous rotation velocity of the wheel's
circumference. Newton's second law of motion, applied to the tangential motion of the
wheel, yields
\begin{equation}
M\,\dot{v} = f\,\sin\theta,
\end{equation}
where $M$ is the mass of the wheel (which is assumed to be concentrated in the wheel's rim).

\begin{figure}
\epsfysize=2in
\centerline{\epsffile{Chapter08/fig080.eps}}
\caption{\em A rotating bicycle wheel.}\label{f80}  
\end{figure}

Let us now  convert the above expression into a rotational equation
of motion. If $\omega$ is the instantaneous angular velocity of the wheel, then
the relation between $\omega$ and $v$ is simply
\begin{equation}
v = b\,\omega.
\end{equation}
Since the wheel is basically a ring of radius $b$, rotating about a perpendicular
symmetric axis, its moment of inertia is 
\begin{equation}
I = M\,b^2. 
\end{equation}
Combining the previous three equations, we obtain
\begin{equation}\label{e8ang}
I \,\dot{\omega} = \tau,
\end{equation}
where 
\begin{equation}\label{e8tor}
\tau = f\,b\,\sin\theta.
\end{equation}

Equation~(\ref{e8ang}) is the {\em angular equation of motion} of the wheel. It relates the
wheel's angular velocity, $\omega$,  and moment of inertia, $I$, to a quantity, $\tau$, which is known
as
the {\em torque}. Clearly, if $I$ is analogous to mass, and $\omega$ is analogous to velocity, then
torque must be analogous to force. In other words, torque is the rotational equivalent of
force. 

It is clear, from Eq.~(\ref{e8tor}), that a torque is the product of the magnitude
of the applied force, $f$, and some distance $l=b\,\sin\theta$. The physical interpretation
of $l$ is illustrated in Fig.~\ref{f81}. If can be seen that $l$ is the 
perpendicular distance of the line of action of the force from the axis of rotation.
We usually refer to this distance as the length of the {\em lever arm}. 

In summary,
a torque measures the propensity of a given force to cause the object upon which
it acts to twist about a certain axis. The torque, $\tau$, is simply the product of the magnitude
of the applied force, $f$,  and the length of the lever arm, $l$:
\begin{equation}\label{etdef}
\tau = f\,l.
\end{equation}
Of course, this definition makes a lot of sense. We all know that it is far easier to turn a
rusty bolt using a long, rather than a short, wrench. Assuming that we exert
the same force on the end of each wrench, the torque we apply to the bolt is larger in the
former case, since the perpendicular distance between the line of action
of the force and the bolt ({\em i.e.}, the length of the wrench) is greater.

\begin{figure}
\epsfysize=3in
\centerline{\epsffile{Chapter08/fig081.eps}}
\caption{\em Definition of the length of the level arm, $l$.}\label{f81}  
\end{figure}

Since force is a vector quantity, it stands to reason that torque must also be a vector
quantity. It follows that Eq.~(\ref{etdef}) defines the magnitude, $\tau$, of some torque vector, $\btau$.
But, what is the direction of this vector? By convention, if a torque is such as to cause the
object upon which it acts to twist about a certain axis, then the direction of that
torque runs along the direction of the axis in the sense given by the right-hand grip rule.
In other words, if the fingers of the right-hand circulate around the axis of rotation  in the sense
in which the torque twists the object, then the thumb of the right-hand
points along the axis in the direction of the torque. It follows that we can rewrite our
rotational equation of motion, Eq.~(\ref{e8ang}), in vector form:
\begin{equation}\label{eqtvec}
I\,\frac{d\bomega}{dt} = I\,\balpha= \btau,
\end{equation}
where $\balpha = d\bomega/dt$ is the vector angular acceleration. Note that the
direction of $\balpha$ indicates the direction of the rotation axis about which the object
accelerates (in the sense given by the right-hand grip rule), whereas the direction
of $\btau$ indicates the direction of the  rotation axis about which the torque attempts
to twist the object (in the sense given by the right-hand grip rule). Of course, these two
 rotation axes are identical. 

Although Eq.~(\ref{eqtvec}) was derived for the special case of a torque
applied to a ring rotating about a perpendicular symmetric axis, it is, nevertheless, completely
general.

It is important to appreciate that the directions  we 
ascribe to angular velocities, angular
accelerations, and torques are merely {\em conventions}. There is actually no physical motion
in the direction of the angular velocity vector---in fact, all of the motion is in the
plane perpendicular to this vector. Likewise, there is  no physical acceleration
in the direction of the angular acceleration vector---again, all of the acceleration is in the
plane perpendicular to this vector. Finally, no physical forces act
in the direction of the torque vector---in fact, all of the forces act in the plane
perpendicular to this vector.

Consider a rigid body which is free to pivot in any direction about some fixed point $O$.
Suppose that a force ${\bf f}$ is applied to the body at some point $P$ whose position
vector relative to $O$ is ${\bf r}$. See Fig.~\ref{f82}.
 Let $\theta$ be the angle subtended between
the directions of ${\bf r}$ and ${\bf f}$. What is the vector torque $\btau$ acting
on the object about an axis passing through the pivot point? The magnitude of this torque is simply
\begin{equation}
\tau = r\,f\,\sin\theta.
\end{equation}
In Fig.~\ref{f82}, 
the conventional 
direction of the torque is  out of the page. Another way of saying this is that the direction
of the torque is mutually perpendicular to both ${\bf r}$ and ${\bf f}$, in the sense given by the
right-hand grip rule when vector ${\bf r}$ is rotated onto vector ${\bf f}$ (through an
angle less than $180^\circ$ degrees). 
It follows that we can write
\begin{equation}\label{etdeff}
\btau = {\bf r}\times {\bf f}.
\end{equation}
In other words, the torque exerted by a force acting  on a rigid body which pivots about some fixed
point is the vector product of the displacement of the point of application of the force from
the pivot point with the force itself. Equation~(\ref{etdeff}) specifies both the magnitude
of the torque, and the axis of rotation about which the torque twists the body upon which
it acts. This axis runs parallel to the direction of $\btau$, and passes through the
pivot point.

\begin{figure}
\epsfysize=3in
\centerline{\epsffile{Chapter08/fig082.eps}}
\caption{\em Torque about a fixed point.}\label{f82}  
\end{figure}

\subsection{Power and Work}
Consider a mass $m$ attached to the end of a light rod of length $l$ whose
other end is attached to a fixed pivot. Suppose that the pivot is such that the rod
is free to rotate in any direction. Suppose, further, that a force ${\bf f}$ is
applied to the mass, whose instantaneous angular velocity about an axis of rotation passing
through the pivot is $\bomega$. 

Let ${\bf v}$ be the instantaneous velocity of the mass. We know that the
rate at which the force ${\bf f}$ performs work on the mass---otherwise known
as the power---is given by (see Sect.~\ref{spower})
\begin{equation}\label{epowt}
P = {\bf f}\!\cdot\!{\bf v}.
\end{equation}
However, we also know that (see Sect.~\ref{svp})
\begin{equation}
{\bf v} = \bomega\times {\bf r},
\end{equation}
where ${\bf r}$ is the vector displacement of the mass from the pivot. 
Hence, we can write
\begin{equation}\label{dpow}
P = \bomega\times {\bf r} \cdot  {\bf f}
\end{equation}
(note that  ${\bf a}\!\cdot\!{\bf b}={\bf b}\!\cdot\!{\bf a}$).

Now, for any three vectors, ${\bf a}$, ${\bf b}$, and ${\bf c}$, we can write
\begin{equation}
{\bf a}\times {\bf b} \cdot  {\bf c}= {\bf a}\cdot{\bf b} \times {\bf c}.
\end{equation}
This theorem is  easily proved by expanding the vector and scalar products in component
form using the definitions (\ref{dsp}) and (\ref{e899}). It follows
that Eq.~(\ref{dpow}) can be rewritten
\begin{equation}
P = \bomega\cdot {\bf r} \times {\bf f}.
\end{equation}
However,
\begin{equation}
\btau = {\bf r}\times{\bf f},
\end{equation}
where $\btau$ is the torque associated with force ${\bf f}$ about an axis of rotation passing
through the pivot. Hence, we obtain
\begin{equation}\label{epowr}
P = \btau\!\cdot\!\bomega.
\end{equation}
In other words, the rate at which a torque performs work on the object upon which
it acts is the scalar product of the torque and the angular velocity of the object.
Note the great similarity between Eq.~(\ref{epowt}) and Eq.~(\ref{epowr}).

Now the relationship between work, $W$, and power, $P$, is simply
\begin{equation}
P = \frac{dW}{dt}.
\end{equation}
Likewise, the relationship between angular velocity, $\bomega$, and angle
of rotation, $\bphi$, is
\begin{equation}
\bomega = \frac{d\bphi}{dt}.
\end{equation}
It follows that Eq.~(\ref{epowr}) can be rewritten
\begin{equation}
dW = \btau\!\cdot\!d\bphi.
\end{equation}
Integration yields
\begin{equation}\label{eworr}
W = \int \btau\!\cdot\!d \bphi.
\end{equation}
Note that this is a good definition, since it only involves an infinitesimal
rotation vector, $d\bphi$. Recall, from Sect.~\ref{srotn}, that it is impossible
to define a finite rotation vector. For the case of translational motion, the analogous
expression to the above is
\begin{equation}
W = \int {\bf f}\!\cdot\!d {\bf r}.
\end{equation}
Here, ${\bf f}$ is the force, and $d{\bf r}$ is an element of displacement of the body upon which
the force acts.

Although Eqs.~(\ref{epowr}) and (\ref{eworr}) were derived for the special case of
the rotation of a mass attached to the end of a light rod, they are, nevertheless,
completely general.

Consider, finally, the special case in which the torque is aligned with the
angular velocity, and both are constant in time. In this case, the
rate at which the torque performs work is
simply
\begin{equation}
P = \tau\,\omega.
\end{equation}
Likewise, the net work performed by the torque in twisting the body upon which
it acts through an angle ${\mit\Delta}\phi$ is just
\begin{equation}
W = \tau\,{\mit\Delta}\phi.
\end{equation}

\subsection{Translational Motion Versus Rotational Motion}
It should be clear, by now, that there is a strong analogy between rotational motion
and standard translational motion. Indeed, each physical concept used to analyze rotational
motion has its translational concomitant. Likewise, every law of physics governing
rotational motion has a translational equivalent. The analogies between rotational
and translational motion are summarized in Table~\ref{tt1}.

\begin{table}
\centering
\begin{tabular}{lclc}\hline
{\em Translational motion} & & {\em Rotational motion} &\\[0.5ex] \hline
Displacement & $d{\bf r}$ & Angular displacement & $d\bphi$ \\[0.5ex]
Velocity     & ${\bf v} = d{\bf r}/dt$ & Angular velocity & $\bomega = d\bphi/dt$ \\[0.5ex]
Acceleration & ${\bf a} = d{\bf v}/dt$ & Angular acceleration & $\balpha = d\bomega/dt$ \\[0.5ex]
Mass & $M$ & Moment of inertia & $I = 
{\scriptstyle\int \rho\, |\hat{\bomega}\times {\bf r}|^2\,dV}$ \\[0.5ex]
Force & ${\bf f} = M\,{\bf a}$ & Torque & $\btau \equiv{\bf r}\times {\bf f}= I\,\balpha$ \\[0.5ex]
Work & $W = \int {\bf f}\!\cdot\!d{\bf r}$ & Work & $W = \int \btau\!\cdot\!d\bphi$ \\[0.5ex]
Power & $P = {\bf f}\!\cdot\!{\bf v}$ & Power & $P = \btau\!\cdot\!\bomega$ \\[0.5ex]
Kinetic energy & $K = M\,v^2/2$ & Kinetic energy & $K = I\,\omega^2/2$ \\[0.5ex]
 \hline
\end{tabular}
\caption{\em The analogies between translational and rotational motion.}\label{tt1}
\end{table}

\subsection{The Physics of Baseball}
Baseball players know from experience that there is a ``sweet spot'' on a baseball bat, about
17\,cm from the end of the barrel, where the shock of impact with the ball, as felt
by the hands, is minimized. In fact, if the ball strikes the bat exactly on the ``sweet spot'' then the
hitter is almost unaware of the collision. Conversely, if the ball strikes the bat well
away from the ``sweet spot'' then the impact is felt as a painful jarring of the hands.

The existence of a ``sweet spot'' on a baseball bat is just a consequence of rotational dynamics.
Let us analyze this problem. Consider the schematic baseball bat shown in Fig.~\ref{f83}. Let $M$
be the mass of the bat, and let $l$ be its length. Suppose that the bat pivots about a fixed
point located at one of its ends. Let the centre of mass of the bat be located a
distance $b$ from the pivot point. Finally, suppose that the ball strikes the
bat a distance $h$ from the pivot point. 

The collision between the bat and the ball can be modeled as equal and opposite {\em impulses}, $J$,
applied to each object at the time of the  collision (see Sect.~\ref{s65}). At the same time,
equal and opposite impulses $J'$ are applied to the pivot and the bat,
as shown in Fig.~\ref{f83}. If the pivot  actually corresponds to
a hitter's hands then the latter
impulse gives rise to the painful jarring sensation felt when the ball is not struck
properly.

\begin{figure}
\epsfysize=3in
\centerline{\epsffile{Chapter08/fig083.eps}}
\caption{\em A schematic baseball bat.}\label{f83}  
\end{figure}

We saw earlier that in a general multi-component system---which includes an extended
body such as a baseball bat---the motion of the
centre of mass takes a particularly simple form (see Sect.~\ref{scm}).
To be more exact, the motion of the centre of mass is equivalent to that
of the point particle obtained by concentrating the whole mass of the system at the centre of mass, and
then
allowing all of the external forces acting on the system to act upon that mass. Let us
use this idea to analyze the effect of the collision with the ball on the motion
of the bat's centre of mass. The centre of mass of the bat acts like a
point particle of mass $M$ which is subject to the two impulses, $J$ and $J'$ (which
are applied simultaneously). If $v$ is the instantaneous velocity of the centre
of mass then the change in momentum of this point due to the action
of the two impulses is simply
\begin{equation}\label{ebata2}
M\,{\mit\Delta} v = -J - J'.
\end{equation}
The minus signs on the right-hand side of the above equation follow
from the fact that the impulses are oppositely directed to $v$ in Fig.~\ref{f83}.

Note that in order to specify the instantaneous state of an extended body
we must do more than just specify the location of the body's centre of mass.
Indeed, since the body can rotate about its centre of mass, we must also specify
its orientation in space. Thus, in order to follow the motion of an extended body,
we must not only follow the translational motion of its centre of mass, but also
the body's rotational motion about this point (or any other convenient reference point
located within the body).

Consider the rotational motion of the bat shown in Fig.~\ref{f83} about a perpendicular
(to the bat) axis passing through the pivot point. This motion satisfies
\begin{equation}\label{ebata}
I\,\frac{d\omega}{dt} = \tau,
\end{equation}
where $I$ is the moment of inertia of the bat, $\omega$ is its instantaneous angular velocity,
and $\tau$ is the applied torque. The bat is actually  subject to an
impulsive torque ({\em i.e.}, a torque which only lasts for a short period in time) at the
time of the collision with the ball. Defining the angular impulse $K$ associated with
an impulsive torque $\tau$ in much the same manner as we earlier defined
the impulse associated
with an impulsive force  (see Sect.~\ref{s65}), we obtain
\begin{equation}
K = \int^t \tau\,dt.
\end{equation}
It follows that we can integrate Eq.~(\ref{ebata}) over the time of the collision
to find
\begin{equation}\label{ebata1}
I\,{\mit\Delta}\omega = K,
\end{equation}
where ${\mit\Delta}\omega$ is the change in angular velocity of the bat due to the collision
with the ball.

Now, the torque associated with a given force is equal to the magnitude of the force times the 
length of the lever arm. Thus, it stands to reason that the angular impulse, $K$, associated
with an impulse, $J$,  is simply
\begin{equation}
K = J\,x,
\end{equation}
where $x$ is the perpendicular distance from the line of action of the impulse
to the axis of rotation. 
Hence, the angular impulses associated with the
two impulses, $J$ and $J'$, to which the bat is subject when it collides with the
ball, are $J\,h$ and $0$, respectively. The latter angular impulse is zero since the
point of application of the associated impulse coincides with the pivot point, and so
the length of the lever arm is zero. It follows that Eq.~(\ref{ebata1}) can be written
\begin{equation}\label{e8cx}
I\,{\mit\Delta}\omega = -J\,h.
\end{equation}
The minus sign comes from the fact that the
impulse $J$ is oppositely directed to the angular velocity in Fig.~\ref{f83}.

Now, the relationship between the instantaneous velocity of the bat's centre of mass
 and the bat's instantaneous angular velocity  is simply
\begin{equation}
v = b\, \omega.
\end{equation}
Hence, Eq.~(\ref{ebata2}) can be rewritten
\begin{equation}\label{e8cy}
M\,b\,{\mit\Delta} \omega = -J - J'.
\end{equation}

Equations~(\ref{e8cx}) and (\ref{e8cy}) can be combined to yield
\begin{equation}
J' = -\left(1 - \frac{M\,b\,h}{I}\right) J.
\end{equation}'
The above expression specifies the magnitude of the impulse $J'$ applied to
the hitter's hands terms of the magnitude of the impulse $J$ applied to the ball.

Let us crudely model the bat as a uniform rod of length $l$ and
mass $M$. It follows, by symmetry, that the centre of mass of the bat
lies at its half-way point: {\em i.e.}, 
\begin{equation}
b = \frac{l}{2}.
\end{equation}
Moreover, the moment of inertia of the bat about a perpendicular axis passing
through one of its ends is
\begin{equation}
I = \frac{1}{3}\,M\,l^2
\end{equation}
(this is a standard result). 
Combining the previous three equations, we obtain 
\begin{equation}
J' = -\left(1 - \frac{3\,h}{2\,l}\right) J=-\left(1 - \frac{h}{h_0}\right) J,
\end{equation}
where 
\begin{equation}
h_0 = \frac{2}{3}\,l.
\end{equation}
Clearly, if $h=h_0$ then no matter how hard the ball is hit ({\em i.e.}, no matter how
large we make $J$) {\em zero} impulse is applied to the hitter's hands. 
We conclude that the ``sweet spot''---or, in scientific terms, the {\em centre of percussion}---of
a uniform baseball bat lies two-thirds of the way down the bat from the hitter's end. If we
adopt a more realistic model of a baseball bat, in which the bat is tapered such that the
majority of its weight is located at its hitting end, we can easily demonstrate that the
centre of percussion is shifted further away from the hitter ({\em i.e.}, it is
more that two-thirds of the way along the bat).

\subsection{Combined Translational and Rotational Motion}\label{sroll}
In Sect.~\ref{sspi}, we analyzed the motion of a block sliding down a
frictionless incline. We found that the block accelerates down the
slope with uniform acceleration $g\,\sin\theta$, where $\theta$ is
the angle subtended by the incline with the horizontal. In this case,
all of the potential energy lost by the block, as it slides down
the slope, is converted into translational kinetic energy (see
Sect.~\ref{senergy}). In particular, no energy
is dissipated.

There is, of course, no way in which a block can  slide over a {\em frictional}
surface without dissipating energy. However, we know from experience that
a round object can {\em roll} over such a  surface with hardly
any  dissipation. For instance, it is far easier to drag a heavy suitcase across the
concourse of an airport
if the suitcase has wheels on the bottom. Let us investigate the physics of round objects rolling
over rough surfaces, and, in particular, rolling
down rough inclines.

Consider a uniform cylinder of radius $b$ rolling over a horizontal, frictional surface.
See Fig.~\ref{f84}. Let $v$ be the translational velocity of the cylinder's centre of
mass, and let $\omega$ be the angular velocity of the cylinder about an axis running along
its length, and passing through its centre of mass. Consider the point
of contact between the cylinder and the surface. The velocity $v'$ of this point
is made up of two components: the translational velocity $v$, which is common to all
elements of the cylinder, and the tangential velocity $v_t=-b\,\omega$, due to the
cylinder's rotational motion. Thus,
\begin{equation}
v' = v + v_t = v - b\,\omega.
\end{equation}
Suppose that the cylinder rolls {\em without slipping}. In other words, suppose that
there is no frictional energy dissipation as the cylinder moves over the surface.
This is only possible if there is zero net motion between the surface and the 
bottom of the cylinder, which
implies $v'=0$, or
\begin{equation}\label{erolls}
v = b\,\omega.
\end{equation}
It follows that when a cylinder, or any other round object,  rolls across a rough surface without
slipping---{\em i.e.}, without dissipating energy---then the cylinder's translational and
rotational velocities are not independent, but satisfy a particular relationship (see the above equation).
Of course, if the cylinder slips as it rolls across the surface then this relationship
no longer holds.

\begin{figure}
\epsfysize=2in
\centerline{\epsffile{Chapter08/fig084.eps}}
\caption{\em A cylinder rolling over a rough surface.}\label{f84}  
\end{figure}

Consider, now, what happens when the cylinder shown in Fig.~\ref{f84} rolls,
without slipping, down a rough slope whose angle of inclination, with respect to the horizontal, is
$\theta$. If the cylinder starts from rest, and rolls down the slope a vertical distance
$h$, then its gravitational potential energy decreases by $-{\mit\Delta}P = M\,g\,h$,
where $M$ is the mass of the cylinder. This decrease in potential energy must be
offset by a corresponding increase in kinetic energy. (Recall that when a
cylinder rolls without slipping there is no frictional energy loss.) However, a rolling
cylinder can possesses two different types of kinetic energy. Firstly, {\em translational}
kinetic energy: $K_t = (1/2)\,M\,v^2$, where $v$ is the cylinder's translational
velocity; and, secondly, {\em rotational} kinetic energy: $K_r = (1/2)\,I\,\omega^2$, where $\omega$
is the cylinder's angular velocity, and $I$ is its moment of inertia. Hence,
energy conservation yields
\begin{equation}\label{e890}
M\,g\,h = \frac{1}{2}\,M\,v^2 + \frac{1}{2}\,I\,\omega^2.
\end{equation}
Now, when the cylinder rolls without slipping, its translational and rotational
velocities are related via Eq.~(\ref{erolls}). It follows from Eq.~(\ref{e890}) that
\begin{equation}
v^2 = \frac{2\,g\,h}{1+ I/M\,b^2}.
\end{equation}
Making use of the fact that the moment of inertia of a uniform cylinder about its
axis of symmetry is $I=(1/2)\,M\,b^2$, we can write the
above equation more explicitly as
\begin{equation}\label{e8555}
v^2 = \frac{4}{3}\, g\,h.
\end{equation}
Now, if the same cylinder were to slide down a {\em frictionless} slope,
such that it fell from rest through a vertical distance $h$, then
its final translational velocity would satisfy
\begin{equation}\label{e8666}
v^2 = 2\,g\,h.
\end{equation}
A comparison of Eqs.~(\ref{e8555}) and (\ref{e8666}) reveals that when a uniform
cylinder {\em rolls} down an incline without slipping, its final translational
velocity is {\em less} than that obtained when the cylinder {\em slides} down the same
incline without friction. The reason for this is that, in the former case,
some of the potential energy released as the cylinder falls is converted into
rotational kinetic energy, whereas, in the latter case, all of the
released potential energy is converted into translational kinetic energy.
Note that, in both cases, the cylinder's {\em total} kinetic energy 
at the bottom of the incline is equal to the released potential
energy.

\begin{figure}
\epsfysize=3in
\centerline{\epsffile{Chapter08/fig085.eps}}
\caption{\em A cylinder rolling down a rough incline.}\label{f85}  
\end{figure}

Let us examine the equations of motion of a cylinder, of mass $M$ and radius $b$, rolling down a
rough slope without slipping. As shown in Fig.~\ref{f85}, there are three
forces acting on the cylinder. Firstly, we have the cylinder's weight, $M\,g$, which acts
vertically downwards. Secondly, we have the reaction, $R$, of the slope, which acts
normally outwards from the surface of the slope. Finally, we have the frictional force, $f$,
which acts up the slope, parallel to its surface.

As we have already discussed, we can most easily describe the translational
motion of an extended body by following the motion of its centre of mass.
This motion is equivalent to that of a point particle, whose mass equals that
of the body, which is subject to the same external forces as those that act
on the body. Thus, applying the three forces, $M\,g$, $R$, and $f$, to
the cylinder's centre of mass, and resolving in the direction normal to the surface of the
slope, we obtain
\begin{equation}
R = M\,g\,\cos\theta.
\end{equation}
Furthermore, Newton's second law,  applied to the motion of the centre of mass
parallel to the slope, yields
\begin{equation}\label{e8123}
M\,\dot{v} = M\,g\,\sin\theta - f,
\end{equation}
where $\dot{v}$ is the cylinder's translational acceleration down the slope.

Let us, now, examine the cylinder's rotational equation of motion.
First, we must evaluate the torques associated with the three forces
acting on the cylinder. Recall, that the torque associated with
a given force is the product of the magnitude of that force and the
length of the level arm---{\em i.e.}, the
perpendicular distance between the line of action of the force and the
axis of rotation. Now, by definition, the weight of an extended
object acts at its centre of mass. However, in this case, the axis of
rotation passes through the centre of mass. Hence, the length of the lever
arm associated with the weight $M\,g$ is zero. It follows
that the associated torque is also zero. It is clear, from Fig.~\ref{f85}, that
the line of action of the reaction force, $R$, passes through the centre
of mass of the cylinder, which coincides with the axis of rotation.
Thus, the length of the lever
arm associated with $R$ is zero, and so is the associated torque.
Finally, according to Fig.~\ref{f85}, the perpendicular distance between the line
of action of the friction force, $f$, and the axis of rotation is just
the radius of the cylinder, $b$---so the associated torque is $f\,b$.
 We conclude that the net torque acting on the
cylinder is simply
\begin{equation}
\tau = f\,b.
\end{equation}
It follows that the rotational equation of motion of the cylinder takes the form,
\begin{equation}\label{e8321}
I\,\dot{\omega} = \tau=f\,b,
\end{equation}
where $I$ is its moment of inertia, and $\dot{\omega}$ is its rotational acceleration.

Now, if the cylinder  rolls, without slipping, such that the constraint (\ref{erolls})
is satisfied at all times, then the time derivative of this constraint implies the
following relationship between the cylinder's translational and rotational accelerations:
\begin{equation}
\dot{v} = b\,\dot{\omega}.
\end{equation}
It follows from Eqs.~(\ref{e8123}) and (\ref{e8321}) that
\begin{eqnarray}\label{e8333}
\dot{v} &=& \frac{g\,\sin\theta}{1 + I/M\,b^2},\\[0.5ex]
f &=& \frac{M\,g\,\sin\theta}{1+ M\,b^2/I}.
\end{eqnarray}
Since the moment of inertia of the cylinder is actually $I=(1/2)\,M\,b^2$, the above
expressions simplify to give
\begin{equation}
\dot{v} = \frac{2}{3}\,g\,\sin\theta,
\end{equation}
and
\begin{equation}\label{e8frict}
f = \frac{1}{3}\,M\,g\,\sin\theta.
\end{equation}
Note that the acceleration of a uniform cylinder as it rolls down a slope, without
slipping, is only {\em two-thirds}
of the value obtained when the  cylinder slides down the same slope without friction.
It is clear from Eq.~(\ref{e8123}) that, in the former case, the acceleration
of the cylinder down the slope is retarded by friction. Note, however, that
the frictional force merely acts to convert translational kinetic energy into rotational
kinetic energy, and does not dissipate  energy.

Now, in order for the slope to exert the frictional force specified in Eq.~(\ref{e8frict}),
 without any slippage between the slope and cylinder, this force must
be less than the maximum allowable static frictional force, $\mu\,R (=\mu\,M\,g\,\cos\theta)$, where $\mu$ is
the coefficient of static friction. In other words, the condition for the
cylinder to roll down the slope without slipping is $f< \mu\,R$, or
\begin{equation}
\tan\theta < 3\,\mu.
\end{equation}
This condition is easily satisfied for gentle slopes, but may well be violated for
extremely steep slopes (depending on the size of  $\mu$). Of course, the above condition
is always violated for frictionless slopes, for which $\mu=0$.

Suppose, finally, that we place two cylinders, side by side and at rest, at the top of a
frictional  slope
of inclination $\theta$. Let the two cylinders possess the same mass, $M$, and the
same radius, $b$. However, suppose that the first cylinder is uniform, whereas the
second is a hollow shell. Which cylinder reaches the bottom of the slope first, assuming that they are
both released simultaneously, and both roll without slipping? 
The acceleration of each cylinder down the slope is given by Eq.~(\ref{e8333}).
For the case of the solid cylinder, the moment of inertia is $I=(1/2)\,M\,b^2$,
and so
\begin{equation}
\dot{v}_{\rm solid} = \frac{2}{3}\,g\,\sin\theta.
\end{equation}
For the case of the hollow cylinder, the moment of inertia is  $I=M\,b^2$ ({\em i.e.},
the same as that of a ring with a similar mass, radius, and axis of rotation),
and so
\begin{equation}
\dot{v}_{\rm hollow} = \frac{1}{2}\,g\,\sin\theta.
\end{equation}
It is clear that the solid cylinder reaches the bottom of the slope {\em before} the
hollow one (since it possesses the greater acceleration). Note that the
accelerations of the two cylinders are independent of their sizes or masses. This
suggests that a solid cylinder will always roll down a frictional incline faster
than a hollow one, irrespective of their relative dimensions (assuming that they
both roll without slipping). In fact, Eq.~(\ref{e8333}) suggests that whenever two
different objects roll (without slipping) down the same slope,
then the {\em most compact} object---{\em i.e.}, the object
with the smallest $I/M\,b^2$ ratio---always wins the race.

\subsection*{\em Worked Example 8.1: Balancing Tires}
\noindent{\em Question:} A tire placed on a balancing machine in a
service station starts from rest and turns through $5.3$ revolutions
in $2.3\,{\rm s}$ before reaching its final angular speed. What is the
angular acceleration of the tire (assuming that this quantity remains constant)?
What is the final angular speed of the tire?

\noindent{\em Answer:} The tire turns through $\phi=5.3\times 2\,\pi = 33.30\,{\rm rad.}$
in $t=2.3\,{\rm s}$. The relationship between $\phi$ and $t$ for the case of rotational
motion, starting from
rest,
with uniform angular acceleration $\alpha$ is
$$
\phi = \frac{1}{2}\,\alpha\,t^2.
$$
Hence,
$$
\alpha = \frac{2\,\phi}{t^2} = \frac{2\times 33.30}{2.3^2} = 12.59\, {\rm rad./s^2}.
$$

Given that the tire starts from rest, its angular velocity after $t$ seconds takes the form
$$
\omega = \alpha\,t = 12.59\times 2.3 =28.96\,{\rm rad./s}.
$$

\subsection*{\em Worked Example 8.2: Accelerating a Wheel}
\noindent{\em Question:} The net work done in accelerating a wheel from rest to
an angular speed of $30\, {\rm rev./min.}$ is $W=5500\,{\rm J}$. What is the moment
of inertia of the wheel?

\noindent{\em Answer:} The final angular speed of the wheel
is
$$
\omega = 30\times 2\,\pi/60 = 3.142\,{\rm rad./s}.
$$
Assuming that all of the work $W$ performed on the wheel goes to increase its
rotational kinetic energy, we have
$$
W = \frac{1}{2}\,I\,\omega^2,
$$
where $I$ is the wheel's moment of inertia. It follows that
$$
I = \frac{2\,W}{\omega^2} = \frac{2\times 5500}{3.142^2}= 1114.6\,{\rm kg\,m^2}.
$$

\subsection*{\em Worked Example 8.3: Moment of Inertia of a Rod}
\noindent{\em Question:} A rod of mass $M=3\,{\rm kg}$ and length $L=1.2\,{\rm m}$ pivots
about an axis, perpendicular to its length, which passes through one of its ends. What
is the moment of inertia of the rod? Given that the rod's instantaneous angular velocity is
$60\,{\rm deg./s}$, what is its rotational kinetic energy?

\noindent{\em Answer:} The moment of inertia of a rod of mass $M$ and length $L$ about
an axis,   perpendicular to its length, which passes through its midpoint is
$I=(1/12)\,M\,L^2$. This is a standard result.
Using the parallel axis theorem, the moment of inertia
about a parallel axis passing through one of the ends of the rod
is
$$
I' = I + M\,\left(\frac{L}{2}\right)^2 = \frac{1}{3}\,M\,L^2,
$$
so
$$
I' = \frac{3\times 1.2^2}{3} = 1.44\,{\rm kg\,m^2}.
$$

The instantaneous angular velocity of the rod is
$$
\omega = 60 \times \frac{\pi}{180} = 1.047\,{\rm rad./s}.
$$
Hence, the rod's rotational kinetic energy is written
$$
K = \frac{1}{2}\,I'\,\omega^2 = 0.5\times 1.44\times 1.047^2 = 0.789\,{\rm J}.
$$

\subsection*{\em Worked Example 8.4: Weight and Pulley}
\noindent{\em Question:} A weight of mass $m=2.6\,{\rm kg}$ is suspended via
a light inextensible cable which is wound around a pulley of mass $M=6.4\,{\rm kg}$
and radius $b=0.4\,{\rm m}$. Treating the pulley as a uniform disk, find the downward
acceleration of the weight and the tension in the cable. Assume that the cable
does not slip with respect to the pulley.

\begin{figure*}[h]
\epsfysize=2.5in
\centerline{\epsffile{Chapter08/fig085a.eps}}
\end{figure*}

\noindent{\em Answer:} Let $v$ be the instantaneous downward velocity of the weight, $\omega$ the
instantaneous angular velocity of the pulley, and $T$ the tension in the cable.
Applying Newton's second law to the vertical motion of the weight, we obtain
$$
m\, \dot{v} = m\,g-T.
$$
The angular equation of motion of the pulley is written
$$
I\,\dot{\omega} = \tau,
$$
where $I$ is its moment of inertia, and $\tau$ is the torque acting
on the pulley. Now, the only force acting on the pulley (whose
line of action does not pass through the pulley's axis of rotation) is the tension in the cable. The
torque associated with this force is the product of the tension, $T$, and the perpendicular
distance from the line of action of this force to the rotation axis, which is equal to the
radius, $b$, of the pulley. Hence,
$$
\tau = T\,b.
$$
If the cable does not slip with respect to the pulley, then its
downward velocity, $v$, must match the tangential velocity of the outer surface of the
pulley, $b\,\omega$. Thus,
$$
v = b\,\omega.
$$
It follows that
$$
\dot{v} = b\,\dot{\omega}.
$$

The above equations can be combined to give
\begin{eqnarray}
\dot{v} &=& \frac{g}{1+I/m\,b^2},\nonumber\\[0.5ex]
T &=& \frac{m\,g}{1+ m\,b^2/I}.\nonumber
\end{eqnarray}
Now, the moment of inertia of the pulley is $I=(1/2)\,M\,b^2$. Hence,
the above expressions reduce to
\begin{eqnarray}
\dot{v} &=& \frac{g}{1+M/2\,m} = \frac{9.81}{1+6.4/2\times 2.6}=4.40\,{\rm m/s^2},\nonumber\\[0.5ex]
T &=& \frac{m\,g}{1+ 2\,m/M}= \frac{2.6\times 9.81}{1+2\times 2.6/6.4}
=14.07\,{\rm N}.\nonumber
\end{eqnarray}

\subsection*{\em Worked Example 8.5: Hinged Rod}
\noindent{\em Question:} A uniform rod of mass $m=5.3\,{\rm kg}$ and length $l=1.3\,{\rm m}$
rotates about a fixed frictionless pivot located at one of its ends. The rod is released from
rest at an angle $\theta=35^\circ$ beneath the horizontal. What is the angular acceleration
of the rod immediately after it is released?

\begin{figure*}[h]
\epsfysize=2.5in
\centerline{\epsffile{Chapter08/fig085b.eps}}
\end{figure*}

\noindent{\em Answer:} The moment of inertia of a rod of mass $m$ and length $l$ about
an axis, perpendicular to its length, which passes through one of its ends is
$I= (1/3)\,m\,l^2$ (see question 8.3). Hence,
$$
I = \frac{5.3\times 1.3^2}{3} = 2.986\,{\rm kg\,m^2}.
$$
The angular equation of motion of the rod is
$$
I\,\alpha = \tau,
$$
where $\alpha$ is the rod's angular acceleration, and $\tau$ is the net torque exerted on the
rod. Now, the only force acting on the rod (whose line of action does not pass through
the pivot) is the rod's weight, $m\,g$. This force acts at the centre of mass of the rod,
which is situated at the rod's midpoint. The perpendicular distance $x$ between the 
line of action of the weight and the pivot point is simply
$$
x = \frac{l}{2}\,\cos\theta = \frac{1.3\times \cos 35^\circ}{2} = 0.532\,{\rm m}.
$$
Thus, the torque acting on the rod is 
$$
\tau = m\,g\,x.
$$
It follows that the rod's angular acceleration is written
$$
\alpha = \frac{\tau}{I} = \frac{m\,g\,x}{I} = \frac{5.3\times 9.81\times 0.532}{2.986} = 9.26\,{\rm
rad./s^2}.
$$

\subsection*{\em Worked Example 8.6: Horsepower of Engine}
\noindent{\em Question:} A car engine develops a torque
of $\tau=500\,{\rm N\,m}$ and rotates at $3000\,{\rm rev./min.}$.
What horsepower does the engine generate? ($1\,{\rm hp} = 746\,{\rm W}$).

\noindent{\em Answer:} The angular speed of the engine is
$$
\omega = 3000\times 2\,\pi/ 60= 314.12\,{\rm rad./s}.
$$
Thus, the power output of the engine is
$$
P = \omega\,\tau = 314.12\times 500 = 1.57\times 10^5\,{\rm W}.
$$
In units of horsepower, this becomes
$$
P = \frac{1.57\times 10^5}{746} = 210.5\,{\rm hp}.
$$

\subsection*{\em  Worked Example 8.7: Rotating Cylinder}
\noindent{\em Question:} A uniform cylinder of radius $b=0.25\,{\rm m}$ is given
an angular speed of $\omega_0=35\,{\rm rad./s}$ about an axis, parallel
to its length, which passes through its centre. The cylinder is gently lowered onto
a horizontal frictional surface, and released. The coefficient of friction
of the surface is $\mu=0.15$. How long does it take before the 
cylinder starts to roll without slipping? What distance does the cylinder
travel between its release point and the point at which it commences to
roll without slipping?

\begin{figure*}[h]
\epsfysize=2in
\centerline{\epsffile{Chapter08/fig085c.eps}}
\end{figure*}

\noindent{\em Answer:} Let $v$ be the velocity of the cylinder's centre of mass, 
$\omega$ the cylinder's angular velocity, $f$ the frictional force exerted by the
surface on the cylinder, $M$ the cylinder's mass, and $I$ the cylinder's moment of inertia. 
The cylinder's translational equation of motion is written
$$
M\,\dot{v} =  f.
$$
Note that the friction force acts to accelerate the cylinder's translational motion.
Likewise, the cylinder's rotational equation of motion takes the form
$$
I\,\dot{\omega} = - f\,b,
$$
since the perpendicular distance between the line of action of $f$ and the axis
of rotation is the radius, $b$, of the cylinder.
Note that the friction force acts to decelerate the cylinder's rotational
motion.
If the cylinder is slipping with respect to the surface, then the friction force,
$f$, is equal to the coefficient of friction, $\mu$, times the normal reaction,
$M\,g$, at the surface:
$$
f = \mu\,M\,g.
$$
Finally, the moment of inertia of the cylinder is
$$
I = \frac{1}{2}\,M\,b^2.
$$

The above equations can be solved to give
\begin{eqnarray}
\dot{v} &=& \mu\,g,\nonumber\\[0.5ex]
b\,\dot{\omega} &=& - 2\,\mu\,g.\nonumber
\end{eqnarray}
Given that $v=0$ ({\em i.e.}, the cylinder
is initially at rest) and $\omega=\omega_0$ at time $t=0$, the above
expressions can be integrated to give
\begin{eqnarray}
v &=& \mu\,g\,t,\nonumber\\[0.5ex]
b\,\omega &=& b\,\omega_0 - 2\,\mu\,g\,t,\nonumber
\end{eqnarray}
which yields
$$
v - b\,\omega = -(b\,\omega_0 - 3\,\mu\,g\,t).
$$

Now, the cylinder stops slipping as soon as the ``no slip'' condition,
$$
v = b\,\omega,
$$
is  satisfied. This occurs when
$$
t = \frac{b\,\omega_0}{3\,\mu\,g}=\frac{0.25\times 35}{3\times 0.15\times 9.81} = 1.98\,{\rm s}.
$$
Whilst it is slipping, the cylinder travels a distance
$$
x = \frac{1}{2}\,\mu\,g\,t^2 = 0.5\times 0.15 \times 9.81\times 1.98^2=2.88\,{\rm m}.
$$
