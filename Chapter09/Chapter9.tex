\section{Angular Momentum}\label{sangm}
\subsection{Introduction}
Two physical quantities are noticeable by their absence in Table~\ref{tt1}. Namely,
momentum, and its rotational concomitant {\em angular momentum}. It turns
out that angular momentum is a sufficiently important concept to merit a separate discussion.

\subsection{Angular Momentum of a Point Particle}\label{sang}
Consider a particle of mass $m$, position vector ${\bf r}$, and instantaneous velocity ${\bf v}$,
which rotates about an axis 
 passing through the origin of our coordinate system. We know that the
particle's linear momentum  is written
\begin{equation}
{\bf p} = m\,{\bf v},
\end{equation}
and satisfies
\begin{equation}\label{e90}
\frac{d{\bf p}}{dt} = {\bf f},
\end{equation}
where ${\bf f}$ is the force acting on the particle. Let us search for the rotational
equivalent of ${\bf p}$. 

Consider the quantity
\begin{equation}\label{e91}
{\bf l} = {\bf r}\times {\bf p}.
\end{equation}
This quantity---which is known as {\em angular momentum}---is a vector of magnitude
\begin{equation}
l = r\,p\,\sin\theta,
\end{equation}
where $\theta$ is the angle subtended between the directions of ${\bf r}$ and ${\bf p}$. 
The direction of ${\bf l}$ is defined to be mutually perpendicular to the directions
of ${\bf r}$ and ${\bf p}$, in the sense given by the right-hand grip rule. In other
words, if vector ${\bf r}$ rotates onto vector ${\bf p}$ (through an angle less
than $180^\circ$), and the fingers of the right-hand are aligned with this rotation, then
the thumb of the right-hand indicates the direction of ${\bf l}$. See Fig.~\ref{f86}.

\begin{figure}
\epsfysize=2.5in
\centerline{\epsffile{Chapter09/fig086.eps}}
\caption{\em Angular momentum of a point particle about the origin.}\label{f86}  
\end{figure}

Let us differentiate Eq.~(\ref{e91}) with respect to time. We obtain
\begin{equation}
\frac{d{\bf l}}{dt} = \dot{\bf r} \times {\bf p} + {\bf r} \times \dot{\bf p}.
\end{equation}
Note that the derivative of a vector product is formed in much the same manner as
the derivative of an ordinary product, except that the order of the various terms is
 preserved. Now, we know that $\dot{\bf r} = {\bf v} = {\bf p}/m$ and $\dot{\bf p} = {\bf f}$.
Hence, we obtain
\begin{equation}
\frac{d{\bf l}}{dt} = \frac{{\bf p} \times {\bf p}}{m} + {\bf r} \times {\bf f}.
\end{equation}
However, ${\bf p} \times {\bf p} = {\bf 0}$, since the vector product of two parallel
vectors is zero. Also,
\begin{equation}
{\bf r} \times {\bf f} = \btau,
\end{equation}
where $\btau$ is the torque acting on the particle about an axis passing through the
origin. We conclude that
\begin{equation}
\frac{d{\bf l}}{dt} = \btau.
\end{equation}
Of course, this equation is analogous to Eq.~(\ref{e90}), which suggests that angular
momentum, ${\bf l}$,  plays the role of linear momentum, ${\bf p}$, in rotational
dynamics.

For the special case of a particle of mass $m$ executing a {\em circular} orbit of radius $r$,
with instantaneous velocity $v$ and instantaneous angular velocity $\omega$, the
magnitude of the particle's angular momentum is simply
\begin{equation}
l = m\,v\,r = m\,\omega\,r^2.
\end{equation}

\subsection{Angular Momentum of an Extended Object}
Consider a rigid object rotating about some fixed axis with angular velocity $\bomega$.
Let us model this object as a swarm of $N$ particles. Suppose that the $i$th particle
has mass $m_i$, position vector ${\bf r}_i$, and velocity ${\bf v}_i$. 
Incidentally, it is
assumed that the object's axis of rotation passes through the origin of our
coordinate system. The total angular momentum of the object, ${\bf L}$, is simply the
vector sum of the angular momenta of the $N$ particles from which it is made up. Hence,
\begin{equation}
{\bf L} = \sum_{i=1,N} m_i\,{\bf r}_i\times {\bf v}_i.
\end{equation}
Now, for a rigidly rotating object we can write (see Sect.~\ref{svp})
\begin{equation}
{\bf v}_i = \bomega\times {\bf r}_i.
\end{equation}
Let 
\begin{equation}
\bomega = \omega\,{\bf k},
\end{equation}
where ${\bf k}$ is a unit vector pointing along the object's axis of rotation (in the sense
given by the right-hand grip rule).
It follows that
\begin{equation}
{\bf L} = \omega \,\sum_{i=1,N} m_i\,{\bf r}_i\times ({\bf k}\times {\bf r}_i).
\end{equation}

Let us calculate the component of ${\bf L}$ along the object's rotation axis---{\em i.e.},
the component along the ${\bf k}$ axis. We can write
\begin{equation}
L_k = {\bf L}\cdot {\bf k} = \omega \,\sum_{i=1,N} m_i\,
{\bf k}\cdot {\bf r}_i\times ({\bf k}\times {\bf r}_i).
\end{equation}
However, since ${\bf a}\cdot {\bf b}\times {\bf c} = {\bf a}\times {\bf b} \cdot {\bf c}$,
the above expression can be rewritten
\begin{equation}
L_k  = \omega \,\sum_{i=1,N} m_i\,
({\bf k}\times {\bf r}_i)\cdot({\bf k}\times {\bf r}_i)= \omega \,\sum_{i=1,N} m_i\,
|{\bf k}\times {\bf r}_i|^2.
\end{equation}
Now,
\begin{equation}
\sum_{i=1,N} m_i\,
|{\bf k}\times {\bf r}_i|^2 = I_k,
\end{equation}
where $I_k$ is the moment of inertia of the object about the ${\bf k}$ axis.
(see Sect.~\ref{smoi}). Hence, it follows that
\begin{equation}
L_k = I_k\,\omega.
\end{equation}

According to the above formula, the component of a rigid body's angular
momentum vector along its axis of rotation  is simply the product
of the body's moment of inertia about this axis and the body's angular velocity.
Does this result  imply that we can automatically write
\begin{equation}
{\bf L} = I \,\bomega?
\end{equation}
Unfortunately, in general, the answer to the above question is {\em no}! This 
conclusion follows because the
body may possess non-zero angular momentum components about axes perpendicular to
its axis of rotation. Thus, in general, the angular momentum vector of a rotating body
is {\em not} parallel to its angular velocity vector. This is a major difference from
translational motion, where linear momentum is always found to be parallel
to linear velocity. 

For a rigid object rotating with angular velocity $\bomega = (\omega_x$, $\omega_y$,
$\omega_z)$, we can write the object's angular momentum ${\bf L} = (L_x$, $L_y$,
$L_z)$ in the form
\begin{eqnarray}
L_x &=& I_x\,\omega_x,\\[0.5ex]
L_y &=& I_y\,\omega_y,\\[0.5ex]
L_z &=& I_z\,\omega_z,
\end{eqnarray}
where $I_x$ is the moment of inertia of the object about the $x$-axis, {\em etc.} Here, it is
again assumed that the origin of our coordinate system lies on the object's
axis of rotation. Note that the above equations are only valid when the $x$-, $y$-, and
$z$-axes are aligned in a certain very special manner---in fact, they must be aligned along the
so-called {\em principal axes} of the object (these axes invariably coincide with the object's
main symmetry axes). Note that it is always possible to find three, mutually perpendicular,
principal axes of rotation which pass through a given point in a rigid body.
 Reconstructing ${\bf L}$ from its components, we obtain
\begin{equation}\label{e93}
{\bf L} = I_x\,\omega_x\,\hat{\bf x} + I_y\,\omega_y\,\hat{\bf y} + I_z\,\omega_z\,\hat{\bf z},
\end{equation}
where $\hat{\bf x}$ is a unit vector pointing along the $x$-axis, {\em etc.}
It is clear, from the above equation, that the reason ${\bf L}$ is not generally parallel
to $\bomega$ is because the moments of inertia of a rigid object about
its  different
possible axes of rotation are {\em not generally the same}. In other words, if $I_x=I_y=I_z=I$ then
${\bf L} = I\,\bomega$, and the angular momentum and angular velocity vectors
are always parallel. However, if $I_x\neq I_y\neq I_z$, which is usually the case,
 then ${\bf L}$ is not, in general, 
parallel to $\bomega$. 

Although Eq.~(\ref{e93}) suggests that the angular momentum of a rigid object
is not generally parallel to its angular velocity, this equation also implies that there
are, at least, {\em three} special axes of rotation  for which this is the case. 
Suppose, for instance, that the object rotates about the $z$-axis, so that
$\bomega = \omega_z\,\hat{\bf z}$. It follows from Eq.~(\ref{e93}) that
\begin{equation}
{\bf L} = I_z\,\omega_z\hat{\bf z} = I_z\,\bomega.
\end{equation}
Thus, in this case, the angular momentum vector is parallel to the angular
velocity vector. The same can be said for rotation about the $x$- or $y$- axes. 
We conclude that when a rigid object rotates about one of its principal axes
then its angular momentum is parallel to its angular velocity, but not, in general,
otherwise.

How can we identify a principal axis of a rigid object? At the simplest level,
a principal axis is one about which the object possesses {\em axial symmetry}. 
The required type of symmetry is illustrated in Fig.~\ref{f87}. Assuming that
the object can be modeled as a swarm of particles---for every
particle of mass $m$, located a distance $r$ from the origin, and subtending an
angle $\theta$ with the rotation axis, there must be an identical particle located on
diagrammatically the opposite side of the rotation axis. As shown in the diagram,
the angular momentum vectors of such a matched pair of particles can be added together to form
a resultant angular momentum vector which is {\em parallel} to the axis of rotation. Thus, if
the object is composed entirely of matched particle pairs then its angular momentum
vector must be parallel to its angular velocity vector. The generalization of this
argument to deal with continuous objects is fairly straightforward.  For instance, symmetry implies
that any axis of rotation which passes through the centre of a uniform sphere is
a principal axis of that object. Likewise, a perpendicular axis which passes through the
centre of a uniform disk is a principal axis. Finally, a perpendicular axis which passes
through the centre of a uniform rod is  a principal axis. 

\begin{figure}
\epsfysize=2.5in
\centerline{\epsffile{Chapter09/fig087.eps}}
\caption{\em A principal axis of rotation.}\label{f87}  
\end{figure}

\subsection{Angular Momentum of a Multi-Component System}\label{sam}
Consider a system consisting of $N$ mutually interacting point particles.
Such a system might represent a true multi-component system, such as an asteroid cloud,
or it might represent an extended body.
Let the $i$th particle, whose mass is $m_i$, be located at vector displacement ${\bf r}_i$.
Suppose that this particle exerts a force ${\bf f}_{ji}$ on the $j$th particle. By Newton's third
law of motion, the force ${\bf f}_{ij}$ exerted  by the $j$th particle on the $i$th  is
given by
\begin{equation}\label{e966}
{\bf f}_{ij} = - {\bf f}_{ji}.
\end{equation}
Let us assume that the internal forces acting within the system are {\em central forces}---{\em i.e.},
the force ${\bf f}_{ij}$, acting between particles $i$ and $j$, is directed along the
{\em line of centres} of these particles. See Fig.~\ref{f88}.
In other words,
\begin{equation}
{\bf f}_{ij} \propto ({\bf r}_i - {\bf r}_j).
\end{equation}
Incidentally, this is not a particularly restrictive assumption, since most forces occurring in nature are
central forces. For instance, gravity is a central force, electrostatic forces are central,
 and the internal stresses acting within a rigid body are approximately central.
Suppose, finally, that the $i$th particle is subject to an external force ${\bf F}_i$. 

\begin{figure}
\epsfysize=3in
\centerline{\epsffile{Chapter09/fig088.eps}}
\caption{\em A multi-component system with central internal forces.}\label{f88}  
\end{figure}

The equation of motion of the $i$th particle can be written
\begin{equation}
\dot{\bf p}_i = \sum_{j=1,N}^{j\neq i} {\bf f}_{ij} + {\bf F}_i.
\end{equation}
Taking the vector product of this equation with the position vector ${\bf r}_i$, we
obtain
\begin{equation}\label{e988}
{\bf r}_i\times \dot{\bf p}_i = \sum_{j=1,N}^{j\neq i} {\bf r}_i\times {\bf f}_{ij} + 
{\bf r}_i\times {\bf F}_i.
\end{equation}
Now, we have already seen that
\begin{equation}
{\bf r}_i\times \dot{\bf p}_i = \frac{d ({\bf r}_i\times {\bf p}_i)}{dt}.
\end{equation}
We also know that the total angular momentum, ${\bf L}$, of the system (about the origin) can
be written in the form
\begin{equation}
{\bf L} = \sum_{i=1,N} {\bf r}_i\times {\bf p}_i.
\end{equation}
Hence, summing Eq.~(\ref{e988}) over all particles, we obtain
\begin{equation}\label{e977}
\frac{d {\bf L}}{dt} = \sum_{i,j = 1,N}^{i\neq j} {\bf r}_i\times {\bf f}_{ij} + 
\sum_{i=1,N} {\bf r}_i\times {\bf F}_i.
\end{equation}

Consider the first expression on the right-hand side of Eq.~(\ref{e977}). A general term,
${\bf r}_i\times {\bf f}_{ij}$, in this sum can always be paired with a
matching term, ${\bf r}_j\times {\bf f}_{ji}$, in which the indices have been swapped.
Making use of Eq.~(\ref{e966}), the sum of a general matched pair  can be written
\begin{equation}
{\bf r}_i\times {\bf f}_{ij}+{\bf r}_j\times {\bf f}_{ji} = 
({\bf r}_i-{\bf r}_j)\times {\bf f}_{ij}.
\end{equation}
However, if the internal forces are central in nature then ${\bf f}_{ij}$ is parallel
to $({\bf r}_i-{\bf r}_j)$. Hence, the vector product of these two vectors is zero.
We conclude that
\begin{equation}
{\bf r}_i\times {\bf f}_{ij}+{\bf r}_j\times {\bf f}_{ji} = {\bf 0},
\end{equation}
for any values of $i$ and $j$. Thus, the first expression on the right-hand side of Eq.~(\ref{e977})
sums to zero. We are left with
\begin{equation}\label{e955}
\frac{d {\bf L}}{dt} = \btau,
\end{equation}
where
\begin{equation}
\btau = \sum_{i=1,N} {\bf r}_i\times {\bf F}_i
\end{equation}
is the net external torque acting on the system (about an axis
passing through the origin). Of course, Eq.~(\ref{e955})
is simply the rotational equation of motion for the system taken as a whole.

Suppose that the system is {\em isolated}, such that it is subject to {\em zero net external torque}. 
It follows from Eq.~(\ref{e955}) that, in this case, {\em the total angular momentum of the
system is a conserved quantity}. To be more exact, the components of the
total angular momentum taken about any three independent axes are individually conserved quantities. 
Conservation of angular momentum  is an extremely useful concept which greatly simplifies the
analysis of  a wide range of rotating systems. Let us consider some
examples.

\begin{figure}
\epsfysize=2.5in
\centerline{\epsffile{Chapter09/fig089.eps}}
\caption{\em Two movable weights on a rotating rod.}\label{f89}  
\end{figure}

Suppose that two identical weights of mass $m$ are attached to a light rigid rod which
rotates without friction about a perpendicular axis passing through its mid-point. Imagine that
the two weights are equipped with small motors which allow them to travel along the
rod: the motors are synchronized in such a manner  that the distance  of the two
weights from the axis of rotation is always the same. Let us call this common distance
$d$, and let $\omega$ be the angular velocity of the rod. See Fig.~\ref{f89}. How
does the angular velocity $\omega$ change as the distance $d$ is varied?

Note that there are no external torques acting on the system. It follows that the
system's angular momentum  must remain constant as the weights move along the rod. 
Neglecting the contribution of the rod, the moment of inertia of the system is
written
\begin{equation}
I = 2\,m\,d^2.
\end{equation}
Since the system is rotating about a principal axis, its angular momentum takes the form
\begin{equation}
L = I\,\omega =  2\,m\,d^2\,\omega.
\end{equation}
If $L$ is a constant of the motion then we obtain
\begin{equation}
\omega\,d^2 = {\rm constant}.
\end{equation}
In other words,  the
system spins {\em faster} as the  weights move {\em inwards}
 towards the axis of rotation, and {\em vice versa}.
This effect is familiar from figure skating. When a skater spins about a vertical axis,
her angular momentum is approximately a conserved quantity, since the ice
exerts very little torque on her. Thus, if the skater starts spinning with
outstretched arms, and then draws her arms inwards, then her rate of rotation
will spontaneously increase in order to conserve angular momentum. The skater can
slow her rate of rotation by simply pushing her arms outwards again.

Suppose that a bullet of mass $m$ and velocity $v$ strikes, and becomes
embedded in, a stationary rod of mass $M$ and length $2\,b$ which
pivots about a frictionless perpendicular axle passing through its mid-point.
Let the bullet strike the rod normally a distance $d$ from its
axis of rotation. See Fig.~\ref{f90}. What is the instantaneous angular
velocity $\omega$ of the rod (and bullet) immediately after the collision?

\begin{figure}
\epsfysize=2.5in
\centerline{\epsffile{Chapter09/fig090.eps}}
\caption{\em A bullet strikes a pivoted rod.}\label{f90}  
\end{figure}

Taking the bullet and the rod as a whole, this is again a system upon which no
external torque acts. Thus, we expect the system's net angular momentum to
be the same before and after the collision. Before the collision, only the
bullet possesses angular momentum, since the rod is at rest. As is easily demonstrated, the
bullet's angular momentum about the pivot point is
\begin{equation}
l = m\,v\,d:
\end{equation}
{\em i.e.}, the product of its mass, its velocity, and its distance of closest approach
to the point about which the angular momentum is measured---this is a general result
(for a point particle). After the collision, the bullet lodges a distance $d$
from the pivot, and is forced to co-rotate with the
rod. Hence, the angular momentum of the bullet after the collision is given by
\begin{equation}
l' = m\,d^2\,\omega,
\end{equation}
where $\omega$ is the angular velocity of the rod. The angular momentum of the
rod after the collision is
\begin{equation}
L = I\,\omega,
\end{equation}
where $I= (1/12)\,M\,(2\,b)^2= (1/3)\,M\,b^2$ is the rod's moment of inertia (about a perpendicular
axis passing through its mid-point). Conservation of angular
momentum yields
\begin{equation}
l = l' + L,
\end{equation}
or
\begin{equation}
\omega = \frac{m\,v\,d}{I + m\,d^2}.
\end{equation}

\subsection*{\em Worked Example 9.1: Angular Momentum of a Missile}
\noindent{\em Question:} A missile of mass $m=2.3\times 10^4\,{\rm kg}$ flies
level to the ground at an
altitude of $d=10,000\,{\rm m}$ with constant speed $v= 210\,{\rm m/s}$. What is the magnitude of the missile's angular momentum
relative to a point on the ground directly below its flight
path?

\begin{figure*}[h]
\epsfysize=2in
\centerline{\epsffile{Chapter09/fig090a.eps}}
\end{figure*}

\noindent{\em Answer:} The missile's angular momentum about point $O$ is
$$
L = m\,v\,r\,\sin\theta,
$$
where $\theta$ is the angle subtended between the missile's velocity vector
and its position vector relative to $O$. However,
$$
r\,\sin\theta = d,
$$
where $d$ is the distance of closest approach of the missile to point $O$. 
Hence,
$$
L = m\,v\,d = (2.3\times 10^4)\times  210\times (1\times  10^4)= 4.83\times 10^{10}\,{\rm kg\,m^2/s}.
$$


\subsection*{\em Worked Example 9.2: Angular Momentum of a Sphere}
\noindent{\em Question:} A uniform sphere of mass $M=5\,{\rm kg}$ and radius $a=0.2\,{\rm m}$
spins about an axis passing through its centre with period $T=0.7\,{\rm s}$. What is the
angular momentum of the sphere?

\noindent{\em Answer:} The angular velocity of the sphere is
$$
\omega = \frac{2\,\pi}{T} = \frac{2\,\pi}{0.7}= 8.98\,{\rm rad./s}.
$$
The moment of inertia of the sphere is
$$
I = \frac{2}{5}\,M\,a^2 = 0.4\times 5\times(0.2)^2 = 0.08\,{\rm kg\,m^2}.
$$
Hence, the angular momentum of the sphere is
$$
L = I\,\omega = 0.08\times 8.98 = 0.718\,{\rm kg\,m^2/s}.
$$

\subsection*{\em Worked Example 9.3: Spinning Skater}
\noindent{\em Question:} A skater spins at an initial angular velocity of $\omega_1=
11\,{\rm rad./s}$ with her arms outstretched. The  skater then lowers her arms, thereby
decreasing her moment of inertia by a factor $8$. What is the skater's final
angular velocity? Assume that any friction between the skater's skates and the
ice is negligible.

\noindent{\em Answer:} Neglecting any friction between the skates and the ice, we
expect the skater to spin with constant angular momentum. The skater's initial
angular momentum is
$$
L_1 = I_1\,\omega_1,
$$
where $I_1$ is the skater's initial moment of inertia. The skater's final angular
momentum is
$$
L_2 = I_2\,\omega_2,
$$
where $I_2$ is the skater's final moment of inertia, and $\omega_2$ is her final
angular velocity. Conservation of angular momentum yields $L_1=L_2$, or
$$
\omega_2 = \frac{I_1}{I_2}\,\omega_2.
$$
Now, we are told that $I_1/I_2 = 8$. Hence,
$$
\omega_2 = 8\times 11 = 88\,{\rm rad./s}.
$$