\section{Statics}
\subsection{Introduction}
Probably the most useful application of the laws of mechanics is the study
of situations in which nothing moves---this discipline is known as {\em statics}. 
The principles of statics are employed by engineers whenever they design 
stationary structures, such as buildings,
bridges, and tunnels, in order to ensure that these
structures do not collapse.

\subsection{Principles of Statics}
Consider a general extended body which is subject to a number of external forces. 
Let us model this body as a swarm of $N$ point particles. In the limit that
$N\rightarrow\infty$, this model becomes a fully accurate representation of the body's
dynamics.

In Sect.~\ref{scm} we determined that the overall translational equation
of motion of a general $N$-component system  can
be written in the form
\begin{equation}\label{e101}
\frac{d{\bf P}}{dt} = {\bf F}.
\end{equation}
Here, ${\bf P}$ is the total linear momentum of the system, and
\begin{equation}
{\bf F} =  \sum_{i=1,N} {\bf F}_i
\end{equation}
is the resultant of all the external forces acting on the system. Note that ${\bf F}_i$
is the external force acting on the $i$th component of the system.

Equation~(\ref{e101}) effectively determines the {\em translational motion} of the system's centre of mass.
Note, however, that in order to fully determine the motion of the system we must also follow its
{\em rotational motion}  about its centre of mass (or any other convenient reference point).
In Sect.~\ref{sam} we determined that the overall rotational equation of motion
of a general $N$-component system (with central internal forces) can
be written in the form
\begin{equation}\label{e102}
\frac{d {\bf L}}{dt} = \btau.
\end{equation}
Here, ${\bf L}$ is the total angular momentum of the system (about the origin of
our coordinate scheme), and
\begin{equation}
\btau = \sum_{i=1,N} {\bf r}_i\times {\bf F}_i
\end{equation}
is the resultant of all the external torques acting on the system (about the
origin of our coordinate scheme). In the above, ${\bf r}_i$ is the vector
displacement of the $i$th component of the system.

What conditions must be satisfied by the various external forces and torques acting
on the system if it  is to remain stationary in time? Well,
if the system does not evolve in time then its net linear momentum, ${\bf P}$,
and its net angular momentum, ${\bf L}$, must both remain constant. 
In other words, $d{\bf P} /dt = d{\bf L}/dt = {\bf 0}$. 
It
follows from Eqs.~(\ref{e101}) and (\ref{e102}) that
\begin{eqnarray}
{\bf F} &=& {\bf 0},\\[0.5ex]
\btau &=& {\bf 0}.
\end{eqnarray}
In other words, the net external force acting on system must be zero, and the
net external torque acting on the system must be zero. To be more exact:
\begin{quote}
{\sf The components of the net external force acting along any three independent directions
must all be zero};
\end{quote}
 and 
\begin{quote}
{\sf The magnitudes of the net external torques acting about any three independent axes
(passing through the origin of the coordinate system) must all be zero}. 
\end{quote}
In a nutshell,
these are the principles of statics.

It is clear that the above principles are {\em necessary} conditions for a general physical
system not to evolve in time. But, are they also {\em sufficient} conditions? In other
words, is it necessarily true that a general system which satisfies these conditions does not
exhibit any time variation? The answer to this question is as follows: if the
system under investigation is a {\em rigid body}, such that the motion of any
component of the body necessarily implies the motion of the whole body, then the
above principles are necessary and sufficient conditions for the existence of an equilibrium
state. On the other hand, if the system is not a rigid body, so
that some components of the body can move independently of others, then
the above conditions only guarantee that the system remains static in an average
sense.

Before we attempt to apply the principles of statics, there are a couple
of important points which need clarification. Firstly, does it matter about
which point we calculate the net torque acting on the system? To be more
exact, if we determine that the net torque acting about a given point
is zero does this necessarily imply that the net torque acting about any
other point is also zero? Well, 
\begin{equation}
\btau = \sum_{i=1,N} {\bf r}_i\times {\bf F}_i
\end{equation}
is the net torque acting on the system about the origin of our coordinate scheme.
The net torque about some general point ${\bf r}_0$ is simply
\begin{equation}
\btau' = \sum_{i=1,N} ({\bf r}_i-{\bf r}_0)\times {\bf F}_i.
\end{equation}
However, we can rewrite the above expression as
\begin{equation}
\btau' = \sum_{i=1,N} {\bf r}_i\times {\bf F}_i - {\bf r}_0\times \left(\sum_{i=1,N}{\bf F}_i\right)
=\btau + {\bf r}_0\times {\bf F} .
\end{equation}
Now, if the system is in equilibrium then ${\bf F} = \btau ={\bf 0}$. Hence, it follows
from the above equation that
\begin{equation}
\btau' = {\bf 0}.
\end{equation}
In other words, for a system in {\em equilibrium}, the determination that the net torque acting
about a given point is zero necessarily implies that the net torque acting about
any other point is also zero. Hence, we can choose the point about which we calculate the
net torque at will---this choice is usually made so as to simplify the calculation.

Another question which needs clarification is as follows. At which point should
we assume that the weight of the system acts in order to calculate the contribution
of the weight to the net torque acting about a given point? Actually, in Sect.~\ref{sroll},
we effectively answered this question by assuming that the weight 
acts at the centre of mass of the system. Let
us now justify this assumption. The external force acting on the $i$th component
of the system due to its weight is
\begin{equation}
{\bf F}_i = m_i\,{\bf g},
\end{equation}
where ${\bf g}$ is the acceleration due to gravity (which is assumed to be uniform
throughout the system). Hence, the net gravitational torque acting on the system about the
origin of our coordinate scheme is
\begin{equation}
\btau = \sum_{i=1,N} {\bf r}_i\times m_i\,{\bf g} =
\left(\sum_{i=1,N} m_i\,{\bf r}_i\right)\times {\bf g} = {\bf r}_{cm}\times M\,{\bf g},
\end{equation}
where $M=\sum_{i=1,N}m_i$ is the total mass of the system, and
${\bf r}_{cm} = \sum_{i=1,N} m_i\,{\bf r}_i/M$ is the position vector of its centre of mass.
It follows, from the above equation, that the net gravitational torque acting on the system
about a given point can be calculated by assuming that the total mass of the system is
concentrated at its centre of mass.

\subsection{Equilibrium of a Laminar Object}\label{s103}
Consider a general laminar object which is free to pivot about a fixed perpendicular axis.
 Assuming that the object is placed in a uniform
gravitational field (such as that on the surface of the Earth), what is the object's equilibrium
configuration  in this field? 

Let $O$ represent  the pivot point, and let $C$ be the centre of mass of the object. See Fig.~\ref{f91}.
Suppose that $r$ represents the distance between points $O$ and $C$, whereas $\theta$ is
the angle subtended between the line $OC$ and the
downward vertical. There are two external forces acting on the object. First, there is the
downward force, $M\,g$, due to gravity, which acts at the centre of mass. Second, there is
the reaction, $R$, due to the pivot, which acts at the pivot point. Here, $M$ is the mass of the object, and $g$ is the
acceleration due to gravity. 

\begin{figure}
\epsfysize=2.5in
\centerline{\epsffile{Chapter10/fig091.eps}}
\caption{\em A laminar object pivoting about a fixed point in a gravitational field.}\label{f91}  
\end{figure}

Two conditions must be satisfied in order for a given configuration of the
object shown in Fig.~\ref{f91} to represent
 an equilibrium configuration. First, there must be zero net external
force acting on the object. This implies that the reaction, $R$, is equal and
opposite to the gravitational force, $M\,g$. In other words, the reaction is of
magnitude $M\,g$ and is directed vertically upwards. The second condition
 is that there must be zero net torque acting about the pivot point. Now,
the reaction, $R$, does not generate a torque, since it acts at the pivot point.
Moreover, the torque associated with the gravitational force, $M\,g$, is simply the magnitude
of this force times the length of the lever arm, $d$ (see Fig.~\ref{f91}). Hence, the net torque
acting on the system about the pivot point is
\begin{equation}
\tau = M\,g\, d = M\,g\,r\,\sin\theta.
\end{equation}
Setting this torque to zero, we obtain $\sin\theta =0$, which implies that
$\theta = 0^\circ$. In other words, the equilibrium configuration of a general laminar
object (which is free to rotate about a fixed perpendicular axis in a uniform
gravitational field) is that in which  the centre of mass of the object is aligned {\em vertically
below} the pivot point.

Incidentally, we can use the above result to experimentally determine the centre of mass of a given
laminar object. We would need to suspend the object from two different pivot points,
successively. In each equilibrium configuration, we would  mark a line running vertically
downward from the pivot point, using a plumb-line. The crossing point of these
two lines would  indicate the position of the centre of mass.

Our discussion of the equilibrium configuration of the laminar object
shown in Fig.~\ref{f91} is not quite complete. We have determined that the
condition which must be satisfied by an equilibrium state is $\sin\theta =0$. 
However, there are, in fact, {\em two}
physical roots
of this equation. The first, $\theta = 0^\circ$, corresponds to the case
where the centre of mass of the object is aligned vertically {\em below} the
pivot point. The second, $\theta =180^\circ$, corresponds to the 
case where the centre of mass  is aligned vertically {\em above} the
pivot point. Of course, the former root is far more important than the latter, since the
former root corresponds to a {\em stable equilibrium}, whereas the latter 
corresponds to an {\em unstable equilibrium}. We recall, from Sect.~\ref{gpotn}, that
when a system is slightly disturbed from a stable equilibrium  then the
forces and torques which act upon it tend to return it to this equilibrium, and
{\em vice versa} for an unstable equilibrium. The easiest way to distinguish between
stable and unstable equilibria, in the present case, is to evaluate the gravitational
potential energy of the system. The potential energy of the object shown
in Fig.~\ref{f91}, calculated using the height of the pivot as the reference height,
is simply
\begin{equation}
U = - M\,g\,h = -M\,g\,r\,\cos\theta.
\end{equation}
(Note that the gravitational potential energy of an extended object
can be calculated by imagining that all of the mass of the object is
concentrated at its centre of mass.)
It can be seen that $\theta=0^\circ$ corresponds to a minimum of this potential, 
whereas $\theta =180^\circ$ corresponds to a maximum. This is in accordance with 
 Sect.~\ref{gpotn}, where it was demonstrated that whenever an object moves
in a conservative force-field (such as a gravitational field),  the stable equilibrium
points correspond to {\em minima} of the potential energy associated with this field, whereas the
unstable equilibrium points correspond to {\em maxima}.

\subsection{Rods and Cables}
Consider a uniform rod of mass $M$ and length $l$ which is suspended horizontally
via two vertical cables. Let the points of attachment of the two cables be located
distances $x_1$ and $x_2$ from one of the ends of the rod, labeled $A$. It is assumed that $x_2>x_1$.
See Fig.~\ref{f92}.
What are the tensions, $T_1$ and $T_2$, in the cables?

Let us first locate the centre of mass of the rod, which is situated at the rod's mid-point,
 a distance $l/2$ from  reference point $A$ (see Fig.~\ref{f92}). There are
three forces acting on the rod: the gravitational force, $M\,g$, and the two tension forces,
$T_1$ and $T_2$. Each of these forces is directed vertically. Thus, the condition that zero net force
acts on the system reduces to the condition that the net vertical force is zero, which yields
\begin{equation}
T_1 + T_2 - M\,g = 0.
\end{equation}

Consider the torques exerted by the three above-mentioned forces about point $A$. Each of these
torques attempts to twist the rod about an axis perpendicular to the plane of the  diagram.
Hence, the condition that zero net torque acts on the system reduces to the condition that
the net torque at point $A$, about an axis perpendicular to the plane of the diagram, is zero. The contribution
of each force to this torque is simply the product of the magnitude of the force and the
length of the associated lever arm. In each case, the length of the lever arm is equivalent to
the distance of the point of action of the force from $A$, measured along the length of the rod. 
Hence, setting the net torque to zero, we obtain
\begin{equation}
x_1\,T_1 + x_2\,T_2 - \frac{l}{2}\,M\,g = 0.
\end{equation}
Note that the torque associated with the gravitational force, $M\,g$, has a minus sign in front, because
this torque obviously attempts to twist the rod in the opposite direction to the torques
associated with the tensions in the cables.

\begin{figure}
\epsfysize=2.5in
\centerline{\epsffile{Chapter10/fig092.eps}}
\caption{\em A horizontal rod suspended by two vertical cables.}\label{f92}  
\end{figure}

The previous two equations can be solved to give
\begin{eqnarray}
T_1 &=& \left(\frac{x_2-l/2}{x_2-x_1}\right)M\,g,\\[0.5ex]
T_2 &=& \left(\frac{l/2-x_1}{x_2-x_1}\right)M\,g.
\end{eqnarray}
Recall that tensions in flexible cables can never be negative, since this would imply that
the cables in question were being compressed. Of course, when cables are compressed they simply
collapse.
It is clear, from the above expressions, that 
in order for the tensions $T_1$ and $T_2$ to remain positive (given that $x_2>x_1$), the following
conditions must be satisfied:
\begin{eqnarray}
x_1 &<& \frac{l}{2},\\[0.5ex]
x_2 &>& \frac{l}{2}.
\end{eqnarray}
In other words, the attachment points of the two cables must {\em straddle} the centre of
mass of the rod.

Consider a uniform rod of mass $M$ and length $l$ which is free to rotate in the vertical
plane about a fixed pivot attached to one of its ends. The other end of the rod is
attached to a fixed cable. We can imagine that both the pivot and the cable are anchored
in the same vertical wall. See Fig.~\ref{f93}.
Suppose that the rod is level, and that the cable subtends an angle $\theta$
with the horizontal.
Assuming that the rod is in equilibrium, what is the magnitude
of the tension, $T$, in the cable, and what is the direction and magnitude of the
reaction, $R$, at the pivot? 

\begin{figure}
\epsfysize=3in
\centerline{\epsffile{Chapter10/fig093.eps}}
\caption{\em A rod suspended by a fixed pivot and a cable.}\label{f93}  
\end{figure}

As usual, the centre of mass of the rod lies at  its mid-point.
There are three forces acting on the rod: the reaction, $R$; the weight, $M\,g$; and the tension,
$T$. The reaction acts at the pivot.  Let $\phi$ be
the angle subtended by the reaction with the horizontal, as shown in Fig.~\ref{f93}. 
The weight acts at the centre of mass of the rod, and is directed
vertically downwards. Finally, the tension acts at the  end of the rod, and is directed along
the cable.

Resolving horizontally, and setting the net horizontal force acting on the rod to zero,
we obtain
\begin{equation}
R\,\cos\phi - T\,\cos\theta = 0.\label{e1099}
\end{equation}
Likewise, resolving vertically, and setting  the net vertical force acting on the rod to zero,
we obtain
\begin{equation}\label{e1088}
R\,\sin\phi + T\,\sin\theta - M\,g = 0.
\end{equation}
The above constraints are sufficient to ensure that zero net force acts on the rod. 

Let us evaluate the net torque acting
at the pivot point (about an axis perpendicular to the plane of the diagram).
The reaction, $R$, does not contribute to this torque, since it acts at the pivot point. 
The length of the lever arm associated with the weight, $M\,g$,
is  $l/2$. 
Simple trigonometry reveals that the length of the  lever
arm associated with the tension, $T$, is $l\,\sin\theta$. Hence, setting the net
torque about the pivot point to zero, we obtain
\begin{equation}
M\,g\,\frac{l}{2} - T\,l\,\sin\theta = 0.\label{e1066}
\end{equation}
Note that there is a minus sign in front of the second torque, since this torque
clearly attempts to twist the rod in the opposite sense to the first.

Equations~(\ref{e1099}) and (\ref{e1088}) can be solved to give
\begin{eqnarray}
T &=& \frac{\cos\phi}{\sin(\theta+\phi)}\,M\,g,\label{e1077}\\[0.5ex]
R &=&  \frac{\cos\theta}{\sin(\theta+\phi)}\,M\,g.\label{e1075}
\end{eqnarray}
Substituting Eq.~(\ref{e1077}) into Eq.~(\ref{e1066}), we obtain
\begin{equation}
\sin(\theta+\phi) = 2\,\sin\theta\,\cos\phi.
\end{equation}
The physical solution of this equation is $\phi =\theta$ (recall that $\sin 2\,\theta
= 2\,\sin\theta\,\cos\theta$), which determines the direction of the reaction at
the pivot. Finally,  Eqs.~(\ref{e1077}) and (\ref{e1075}) yield
\begin{equation}
T = R = \frac{M\,g}{2\,\sin\theta},
\end{equation}
which determines both the magnitude of the tension in the cable and that of the reaction at the pivot.

One important point to note about the above solution is that if $\phi=\theta$
then the lines of action of the three forces---$R$, $M\,g$, and $T$---intersect at
the same point, as shown in Fig.~\ref{f93}. This is an illustration of a general
rule. Namely, whenever a rigid body is in equilibrium under the action of {\em three}
forces, then these forces are either {\em mutually parallel}, as shown in Fig.~\ref{f92},
or their lines of action pass through the {\em same point}, as shown in Fig.~\ref{f93}.

\subsection{Ladders and Walls}
Suppose that a ladder of length $l$ and negligible mass is leaning against a
vertical wall, making  an angle $\theta$ with the horizontal.
 A workman of mass $M$ climbs a distance $x$ along the ladder,
measured  from
the bottom. See Fig.~\ref{f94}.
 Suppose that the wall is completely frictionless, but  that the ground
possesses a coefficient of static friction $\mu$. How far up the ladder can the
workman climb before it slips along the ground? Is it possible for the workman to climb to
the top of the ladder without any slippage occurring?

\begin{figure}
\epsfysize=2in
\centerline{\epsffile{Chapter10/fig094.eps}}
\caption{\em A ladder leaning against a vertical wall.}\label{f94}  
\end{figure}

There are four forces acting on the ladder: the weight, $M\,g$, of the workman; the
reaction, $S$, at the wall; the reaction, $R$, at the ground; and the frictional force,
$f$, due to the ground. The weight acts at the position of the workman, and is directed
vertically downwards. The reaction, $S$, acts at the top of the ladder, and is directed
horizontally ({\em i.e.}, normal to the surface of the wall). The reaction, $R$, acts at
the bottom of the ladder, and is directed vertically upwards  ({\em i.e.}, normal to the ground).
Finally, the frictional force, $f$, also acts at the bottom of the ladder, and is directed
horizontally.

Resolving horizontally, and setting the net horizontal force acting on the ladder to
zero, we obtain
\begin{equation}
S - f = 0.
\end{equation}
Resolving vertically, and setting the net vertically force acting on the ladder to
zero, we obtain
\begin{equation}
R - M\,g = 0.
\end{equation}
Evaluating the torque acting about the point where the ladder touches the ground, we
note that only the forces $M\,g$ and $S$ contribute. The lever arm associated with the
force $M\,g$ is $x\,\cos\theta$. The lever arm associated with the force $S$ is $l\,\sin\theta$.
Furthermore, the torques associated with these two forces act in opposite directions.
Hence, setting the net torque about the bottom of the ladder to zero, we obtain
\begin{equation}
M\,g\,x\,\cos\theta - S\,l\,\sin\theta = 0.
\end{equation}
The above three equations can be solved to give
\begin{equation}
R = M\,g,
\end{equation}
and 
\begin{equation}
f = S = \frac{x}{l\,\tan\theta} \,M\,g.
\end{equation}

Now, the condition for the ladder not to slip with respect to the ground is
\begin{equation}
f < \mu\,R.
\end{equation}
This condition reduces to
\begin{equation}
x<l\, \mu\,\tan\theta.
\end{equation}
Thus, the furthest distance that the workman can climb along the ladder before it
slips is
\begin{equation}
x_{\rm max} = l\,\mu\,\tan\theta.
\end{equation}
Note that if $\tan\theta > 1/\mu$ then the workman can climb all the way along the
ladder without any slippage occurring. This result suggests that ladders leaning against
walls are less likely to slip when they
are almost vertical ({\em i.e.}, when $\theta\rightarrow 90^\circ$).

\subsection{Jointed Rods}
Suppose that three identical uniform rods of mass $M$ and length $l$ are joined together to form
an equilateral triangle, and are then suspended from a cable, as shown in
Fig.~\ref{f95}. What is the tension in the cable, and what 
are the reactions at the joints?

\begin{figure}
\epsfysize=3in
\centerline{\epsffile{Chapter10/fig095.eps}}
\caption{\em Three identical jointed rods.}\label{f95}  
\end{figure}

Let $X_1$, $X_2$, and $X_3$ be the horizontal reactions at the three joints, and
let $Y_1$, $Y_2$, and $Y_3$ be the corresponding vertical reactions, as shown in Fig.~\ref{f95}.
In drawing this diagram, we have made use of the fact that the rods exert equal and
opposite reactions on one another, in accordance with Newton's third law. Let $T$
be the tension in the cable.

Setting the horizontal and vertical forces acting on rod $AB$ to zero, we obtain
\begin{eqnarray}
X_1 - X_3 &=& 0,\\[0.5ex]
T + Y_1 + Y_3 - M\,g  &=& 0,
\end{eqnarray}
respectively.
Setting the horizontal and vertical forces acting on rod $AC$ to zero, we obtain
\begin{eqnarray}
 X_2 - X_1 &=& 0,\\[0.5ex]
 Y_2 - Y_1 - M\,g  &=& 0,
\end{eqnarray}
respectively.
Finally, setting the horizontal and vertical forces acting on rod $BC$ to zero, we obtain
\begin{eqnarray}
 X_3 - X_2 &=& 0,\\[0.5ex]
 -Y_2 - Y_3 - M\,g  &=& 0,
\end{eqnarray}
respectively.
Incidentally, it is clear, from symmetry, that $X_1 = X_3$ and $Y_1=Y_3$. Thus, the above equations can
be solved to give
\begin{eqnarray}
T &=& 3\,M\,g,\\[0.5ex]
Y_2 &=& 0,\\[0.5ex]
X_1=X_2=X_3 &=& X,\\[0.5ex]
Y_1=Y_3 &=& -M\,g.
\end{eqnarray}
There now remains only one unknown, $X$.

Now, it is clear, from symmetry, that there is zero net torque acting on rod $AB$. Let us evaluate
the torque acting on rod $AC$ about point $A$. (By symmetry, this
is the same as the torque acting on rod $BC$ about point $B$). The two forces which contribute to
this torque are the weight, $M\,g$, and the reaction $X_2=X$. (Recall that the reaction $Y_2$
is zero). The lever arms associated with these two torques (which act in the same direction)
are $(l/2)\,\cos\theta$ and $l\,\sin\theta$, respectively. Thus, setting the net torque
to zero, we obtain
\begin{equation}
M\,g\,(l/2)\,\cos\theta + X\,l\,\sin\theta = 0,
\end{equation}
which yields
\begin{equation}
X = -\frac{M\,g}{2\,\tan\theta} = -\frac{M\,g}{2\,\sqrt{3}},
\end{equation}
since $\theta = 60^\circ$, and $\tan 60^\circ = \sqrt{3}$. We have now fully determined the
tension in the cable, and all the  reactions
at the joints.

\subsection*{\em Worked Example 10.1: Equilibrium  of Two Rods}
\noindent{\em Question:} Suppose that two uniform rods (of negligible thickness)
 are welded together at right-angles, as shown in the
diagram below. Let the first rod be of mass $m_1=5.2\,{\rm kg}$ and length $l_1=1.3\,{\rm m}$.
Let the second rod be of mass $m_2=3.4\,{\rm kg}$ and length $l_2=0.7\,{\rm m}$. Suppose that
the system is suspended from a pivot point located at the free end of the first rod, and then allowed to
reach a stable equilibrium state. What angle $\theta$ does the first rod subtend with the
downward vertical in this state?
\begin{figure*}[h]
\epsfysize=2.5in
\centerline{\epsffile{Chapter10/fig095a.eps}}
\end{figure*}

\noindent{\em Answer:} Let us adopt a coordinate system in which the $x$-axis runs
parallel to the second rod, whereas the $y$-axis runs parallel to the first. Let the origin
of our coordinate system correspond to the pivot point. The centre of mass
of the first rod is situated at its mid-point, whose coordinates are
$$
(x_1, y_1) = (0, l_1/2).
$$
Likewise, the centre of mass of the second rod is situated at its mid-point, whose
coordinates are 
$$
(x_2, y_2) = (l_2/2, l_1).
$$
It follows that the coordinates of the centre of mass of the whole system are
given by
$$
x_{cm} = \frac{m_1\,x_1+ m_2\,x_2}{m_1+m_2} = \frac{1}{2} \frac{m_2\,l_2}{m_1+m_2}=
\frac{3.4\times 0.7}{2\times 8.6} = 0.138\,{\rm m},
$$
and 
$$
y_{cm} = \frac{m_1\,y_1+ m_2\,y_2}{m_1+m_2} = \frac{m_1\,l_1/2+m_2\,l_1}{m_1+m_2}
= \frac{5.2\times 1.3/2+3.4\times 1.3}{8.6} = 0.907\,{\rm m}.
$$
The angle $\theta$  subtended between the line joining the pivot point and the overall centre of mass,
and the first rod is simply
$$
\theta = \tan^{-1}\left(\frac{x_{cm}}{y_{cm}}\right) = \tan^{-1}0.152= 8.65^\circ.
$$
When the system reaches a stable equilibrium state then its centre of mass is aligned
directly below the pivot point. This implies that the first rod
subtends an angle $\theta=8.65^\circ$ with the downward vertical.

\subsection*{\em Worked Example 10.2: Rod Supported by a Cable}
\noindent{\em Question:} A uniform rod of mass $m=\,15\,{\rm kg}$ and
length $l=3\,{\rm m}$ is supported in a horizontal position by a
pin and a cable, as shown in the figure below. Masses $m_1=36\,{\rm kg}$
and $m_2=24\,{\rm kg}$ are suspended from the rod at positions
$l_1=0.5\,{\rm m}$ and $l_2=2.3\,{\rm m}$. The angle $\theta$ is $40^\circ$.
What is the tension $T$ in the cable?

\begin{figure*}[h]
\epsfysize=2.2in
\centerline{\epsffile{Chapter10/fig095b.eps}}
\end{figure*}

\noindent{\em Answer:} Consider the torque acting on the rod about the pin. Note that the
reaction at the pin makes no contribution to this torque (since the
length of the associated lever arm is zero). 
The torque due to the
weight of the rod is $m\,g\,l/2$ ({\em i.e.}, the weight times the length of the lever arm).
Note that the weight of the rod acts at its centre of mass, which is located at the
rod's mid-point.
 The torque due to the weight of the first mass is
$m_1\,g\,l_1$.
 The torque due to the weight of the second mass is $m_2\,g\,l_2$. 
Finally, the torque due to the tension in the cable is $-T\,l\,\sin\theta$ (this
torque is negative since it  twists the rod in the opposite sense to the
other three torques).
Hence, setting the net torque to zero, we obtain
$$
m\,g\,\frac{l}{2} + m_1\,g\,l_1 + m_2\,g\,l_2 -T\,l\,\sin\theta = 0,
$$
or
\begin{eqnarray}
T &=& \frac{ [m/2 + m_1\,(l_1/l) + m_2\,(l_2/l)]\,g }{\sin\theta} \nonumber\\[0.5ex]
&=& \frac{[0.5\times 15 + 36\times(0.5/3)+24\times(2.3/3)]\times 9.81}{\sin 40^\circ}\nonumber\\[0.5ex]
 &=&
 486.84\,{\rm N}.\nonumber
\end{eqnarray}

\subsection*{\em Worked Example 10.3: Leaning Ladder}
\noindent{\em Question:} A uniform ladder of mass $m = 40\,{\rm kg}$ and length
$l=10\,{\rm m}$ is leaned against a smooth vertical wall. A person
of mass $M=80\,{\rm kg}$ stands on the ladder a distance  $x=7\,{\rm m}$
from the bottom, as measured along the ladder. The foot of the ladder
is $d=1.2\,{\rm m}$ from the bottom of the wall. What is the force exerted by the
wall on the ladder? What is the normal force exerted by the floor on the ladder?

\begin{figure*}[h]
\epsfysize=2.5in
\centerline{\epsffile{Chapter10/fig095c.eps}}
\end{figure*}

\noindent{\em Answer:} The angle $\theta$ subtended by the ladder with the ground 
satisfies
$$
\theta = \cos^{-1}(d/l)= \cos^{-1}(1.2/10) = 83.11^\circ.
$$
Let $S$ be the normal reaction at the wall, let $R$ be the normal reaction at the ground,
and let $f$ be the frictional force exerted by the ground on the ladder, as shown
in the diagram. Consider the torque acting on the ladder about the point where
it meets the ground. Only three forces contribute to this torque: the weight, $m\,g$, of
the ladder, which acts half-way along the ladder; the weight, $M\,g$, of the person,
which acts a distance $x$ along the ladder; and the reaction, $S$, at the wall, which acts
at the top of the ladder. The lever arms associated with these three forces are
$(l/2)\,\cos\theta$, $x\,\cos\theta$, and $l\,\sin\theta$, respectively. 
Note that the reaction force acts to twist the ladder in the opposite sense to
the two weights. Hence, setting the net torque to zero, we obtain
$$
m\,g\,\frac{l}{2}\,\cos\theta + M\,g\,x\,\cos\theta - S\,l\,\sin\theta = 0,
$$
which yields
$$
S = \frac{(m\,g/2 + M\,g\,x/l)}{\tan\theta} = \frac{(0.5\times 40\times 9.81+
80\times 9.81\times 7/10)}{\tan 83.11^\circ} = 90.09\,{\rm N}.
$$

The condition that zero net vertical force acts on the ladder yields
$$
R - m\,g - M\,g = 0.
$$
Hence,
$$
R = (m+M)\,g = (40+80)\times 9.81 = 1177.2\,{\rm N}.
$$

\subsection*{\em Worked Example 10.4: Truck Crossing a Bridge}
\noindent{\em Question:} A truck of mass $M=5000\,{\rm kg}$ is crossing
a uniform horizontal bridge of mass $m=1000\,{\rm kg}$ and length $l=100\,{\rm m}$.
The bridge is supported at its two end-points. What are the reactions
at these supports when the truck is one third of the way across the bridge?

\begin{figure*}[h]
\epsfysize=1.5in
\centerline{\epsffile{Chapter10/fig095d.eps}}
\end{figure*}

\noindent{\em Answer:} Let $R$ and $S$ be the reactions at the bridge supports.
Here, $R$ is the reaction at the support closest to the truck. Setting the net vertical
force acting on the bridge to zero, we obtain
$$
R+S-M\,g - m\,g = 0.
$$
Setting the torque acting on the bridge about the left-most support to zero,
we get
$$
M\,g\,l/3 + m\,g\,l/2 - S\,l = 0.
$$
Here, we have made use of the fact that centre of mass of the bridge lies at its
mid-point. It follows from the above two equations that
$$
S = M\,g/3 + m\,g/2 = 5000\times 9.81/3 + 1000\times 9.81 /2 = 2.13\times 10^4\,{\rm N},
$$
and
$$
R = M\,g + m\,g - S = (5000+1000)\times 9.81- 2.13\times 10^4 = 3.76\times 10^4\,{\rm N}.
$$

\subsection*{\em Worked Example 10.5: Rod Supported by a Strut}

\noindent{\em Question:} A uniform horizontal rod of mass $m=15\,{\rm kg}$ is
attached to a vertical wall at one end, and is supported, from below, by a
light rigid strut at the other. The strut is attached to the rod at one end,
and the wall at the other, and subtends an angle of $\theta =30^\circ$ with the
rod. Find the horizontal and vertical reactions at the point where the strut is
attached to the rod, and the points where the rod and the strut are attached to
the wall.

\begin{figure*}[h]
\epsfysize=2.5in
\centerline{\epsffile{Chapter10/fig095e.eps}}
\end{figure*}

\noindent{\em Answer:} Let us call the vertical reactions at the joints $X_1$, $X_2$, and $X_3$.
Let the corresponding horizontal reactions be $Y_1$, $Y_2$, and $Y_3$. See the diagram.
Here, we have made use of the fact that the strut and the rod exert equal and opposite
reactions on one another, in accordance with Newton's third law.
Setting the net vertical force on the rod to zero yields
$$
X_1 + X_3 - m\,g = 0.
$$
Setting the net horizontal force on the rod to zero gives
$$
Y_1+Y_3 = 0.
$$
Setting the net vertical force on the strut to zero yields
$$
X_2-X_3=0.
$$
Finally, setting the net horizontal force on the strut to zero yields
$$
Y_2-Y_3 = 0.
$$
The above equations can be solved to give
$$
-Y_1 = Y_2= Y_3 = Y,
$$
and
$$
X_2 = X_3 = X,
$$
with 
$$
X_1 = m\,g -X.
$$
There now remain only two unknowns, $X$ and $Y$.

Setting the net torque acting on the rod about the point where it
is connected to the wall to zero, we obtain
$$
m\,g\,l/2 - X_3\,l =0,
$$
where $l$ is the length of the rod. Here, we have used the fact that the
centre of gravity of the rod lies at its mid-point. The above equation implies that
$$
X_3 = X = m\,g/2= 15\times 9.81/2 = 73.58\,{\rm N}.
$$
We also have $X_1=m\,g-X = 73.58\,{\rm N}$.
Setting the net torque acting on the strut about the point where it is
connected to the wall to zero, we find
$$
Y_3\,h\,\sin\theta - X_3\,h\,\cos\theta = 0,
$$
where $h$ is the length of the strut. Thus,
$$
Y_3 = Y = \frac{X}{\tan \theta} = \frac{73.58}{\tan 30^\circ}
=127.44\,{\rm N}.
$$

In summary, the vertical reactions are $X_1=X_2=X_3=
73.58\,{\rm N}$, and the horizontal reactions are $-Y_1=Y_2=Y_3=127.44\,
{\rm N}.$
