\section{Oscillatory motion}
\subsection{Introduction}
We have  seen previously (for instance, in Sect.~\ref{s103}) that when systems
are perturbed from a {\em stable} equilibrium state they experience a {\em restoring
force} which acts to return them to that state. In many cases of interest, the magnitude of the 
restoring force is directly  proportional to the displacement from equilibrium.
In this section, we shall investigate the motion of systems subject to such a  force.

\subsection{Simple Harmonic Motion}
Let us reexamine the problem of a mass on a spring (see Sect.~\ref{s56}).
Consider a mass $m$ which slides over a horizontal frictionless surface. Suppose that
the mass is attached
to a light horizontal spring whose other end is anchored to an immovable object. See
Fig.~\ref{f42}. Let $x$ be the extension of the spring: {\em i.e.}, the difference between
the spring's actual length and its unstretched length. Obviously, $x$ can also be used as
a coordinate to determine the horizontal displacement of the mass. 

The equilibrium state of the system corresponds to the situation where
the mass is at rest, and the spring is unextended ({\em i.e.}, $x=0$).
In this state, zero net force acts on the mass, so there is no reason for it to start to move.
If the system is perturbed from this equilibrium state ({\em i.e.}, if the mass is moved, so that the
spring becomes extended) then the mass experiences a {\em restoring force} given by Hooke's law:
\begin{equation}
f = -k\,x.
\end{equation}
Here, $k>0$ is the {\em force constant} of the spring. The negative sign indicates that $f$ is indeed a restoring force. 
Note that the magnitude of the restoring
force is {\em directly proportional} to the displacement of the system from equilibrium 
({\em i.e.}, $f\propto x$). Of course, Hooke's law only holds for {\em small}\/ spring extensions.
Hence, the displacement from equilibrium cannot be made too large.
The motion of this system is representative of the
motion of a wide range of systems when they are {\em slightly} disturbed from a stable equilibrium state.

Newton's second law gives following equation of motion for the system:
\begin{equation}
m\,\ddot{x} = - k\,x.\label{eshm}
\end{equation}
This {\em differential equation} is known as the {\em simple harmonic equation}, and its solution has been known
for centuries. In fact, the solution is
\begin{equation}
x = a\,\cos(\omega\,t-\phi),\label{shm}
\end{equation}
where $a$, $\omega$, and $\phi$ are constants. We can demonstrate that Eq.~(\ref{shm}) is indeed a
solution of Eq.~(\ref{eshm}) by direct substitution. Substituting  Eq.~(\ref{shm}) into
Eq.~(\ref{eshm}), and recalling from calculus that $d(\cos\theta)/d\theta = -\sin\theta$ and
$d (\sin\theta)/d\theta = \cos\theta$, we obtain
\begin{equation}
-m\,\omega^2\,a\,\cos(\omega\,t-\phi) =- k\,a\,\cos(\omega\,t-\phi).
\end{equation}
It follows that Eq.~(\ref{shm}) is the correct solution provided
\begin{equation}\label{eomega}
\omega = \sqrt{\frac{k}{m}}.
\end{equation}

Figure~\ref{f96} shows a graph of $x$ versus $t$ obtained from Eq.~(\ref{shm}). The type of motion shown here is
called {\em simple harmonic motion}.
It can be seen that
the displacement $x$ {\em oscillates} between $x=-a$ and $x=+a$. Here, $a$ is termed the {\em amplitude}
of the oscillation. Moreover, the motion is {\em periodic} in time ({\em i.e.}, it repeats exactly after
a certain time period has elapsed). In fact, the {\em period} is
\begin{equation}
T = \frac{2\,\pi}{\omega}.
\end{equation}
This result is easily obtained from Eq.~(\ref{shm}) by noting that $\cos\theta$ is a periodic function 
of $\theta$ with
period $2\,\pi$. The frequency of the motion ({\em i.e.}, the number of oscillations completed per
second) is
\begin{equation}
f = \frac{1}{T} = \frac{\omega}{2\,\pi}.
\end{equation}
It can be seen that $\omega$ is the motion's {\em angular frequency}  ({\em i.e.}, the frequency
$f$ converted into radians per second).
Finally, the {\em phase angle} $\phi$ determines the times at which the oscillation attains its maximum amplitude,
$x=a$: in fact,
\begin{equation}
t_{\rm max} = T \left(n + \frac{\phi}{2\,\pi}\right).
\end{equation}
Here, $n$ is an arbitrary integer. 

\begin{table}\centering
\begin{tabular}{c|cccc}\hline
$\omega\,t-\phi$ & $0^\circ$ & $90^\circ$ & $180^\circ$ & $270^\circ$\\[0.5ex]\hline
$x$              & $+a$     & 0         & $-a$       & 0 \\[0.5ex]
$\dot{x}$        & 0        & $-\omega\,a$ & 0 & $+\omega\,a$ \\[0.5ex]
$\ddot{x}$       & $-\omega^2\,a$ &0& $+\omega^2\,a$ & 0
\end{tabular}
\caption{\em Simple harmonic motion.}\label{tshm}
\end{table}

Table~\ref{tshm} lists the displacement, velocity, and acceleration of the mass at various phases of the
simple harmonic cycle. The information contained in this table can easily be derived from the
simple harmonic equation, Eq.~(\ref{shm}). Note that all of the non-zero values
shown in this table represent either the maximum or the minimum value taken by   the quantity in question during the
oscillation  cycle.

\begin{figure}
\epsfysize=3in
\centerline{\epsffile{Chapter11/fig096.eps}}
\caption{\em Simple harmonic motion.}\label{f96}  
\end{figure}

We have seen that when a mass on a spring is disturbed from equilibrium it executes {\em simple harmonic
motion} about its equilibrium state. In physical terms, if the initial displacement is positive ($x>0$) then the
restoring force {\em overcompensates}, and sends the system past the equilibrium state ($x=0$) to
negative displacement states ($x<0$). The restoring force again overcompensates, and sends the
system back through $x=0$ to positive displacement states. The motion then repeats itself {\em ad infinitum}. 
The frequency of the oscillation is determined by the spring stiffness, $k$,  and the system
inertia, $m$,  via Eq.~(\ref{eomega}). 
In contrast, the amplitude and phase angle of the oscillation are determined by the {\em initial conditions}. 
Suppose that the instantaneous displacement and velocity of the mass at $t=0$ are $x_0$ and $v_0$,
respectively. It follows from Eq.~(\ref{shm}) that
\begin{eqnarray}
x_0 &=& x(t=0) = a\,\cos\phi,\\[0.5ex]
v_0 &=& \dot{x}(t=0) =a\,\omega\,\sin\phi.
\end{eqnarray}
Here, use has been made of the well-known identities $\cos(-\theta) =\cos\theta$ and $\sin(-\theta)
=-\sin\theta$. Hence, we obtain
\begin{equation}
a = \sqrt{x_0^{\,2} + (v_0/\omega)^2},
\end{equation}
and
\begin{equation}
\phi =\tan^{-1}\!\left(\frac{v_0}{\omega\,x_0}\right),
\end{equation}
since $\sin^2\theta+\cos^2\theta =1$ and $\tan\theta = \sin\theta/\cos\theta$.

The kinetic energy of the system is written
\begin{equation}
K = \frac{1}{2}\,m\,\dot{x}^2 = \frac{m\,a^2\,\omega^2\,\sin^2(\omega\,t-\phi)}{2}.
\end{equation}
Recall, from Sect.~\ref{s56}, that the potential energy takes the form
\begin{equation}
U = \frac{1}{2}\,k\,x^2= \frac{k\,a^2\,\cos^2(\omega\,t-\phi)}{2}.
\end{equation}
Hence, the total energy can be written
\begin{equation}
E = K + U = \frac{a^2\,k}{2},
\end{equation}
since $m\,\omega^2 = k$ and $\sin^2\theta+\cos^2\theta = 1$. Note that the
total energy is a {\em constant of the motion}, as expected for an isolated system. Moreover,
the energy is proportional to the {\em amplitude squared}\/ of the motion.
It is clear, from the above expressions, that simple harmonic motion is characterized
by a constant backward and forward flow of energy between kinetic and potential components.
The kinetic energy attains its maximum value, and the potential energy attains
it minimum value,  when the displacement is zero ({\em i.e.}, when $x=0$). Likewise,
the potential energy attains its maximum value, and the kinetic energy attains
its minimum value, when the displacement is maximal ({\em i.e.}, when $x=\pm a$). 
Note that the minimum value of $K$ is {\em zero}, since the system is instantaneously at rest
when the displacement is maximal.

\subsection{The Torsion Pendulum}
Consider a disk suspended from a torsion wire attached to its centre. See Fig.~\ref{f97}.  This setup is known as a {\em torsion
pendulum}.
A torsion wire is essentially inextensible, but is free to {\em twist} about its axis.
Of course, as the wire twists it also causes the disk attached to it to {\em rotate} in the horizontal
plane. Let $\theta$ be the angle of rotation of the disk, and let $\theta=0$
correspond to the case in which the wire is untwisted.

\begin{figure}
\epsfysize=2in
\centerline{\epsffile{Chapter11/fig097.eps}}
\caption{\em A torsion pendulum.}\label{f97}  
\end{figure}

Any twisting of the wire is inevitably associated with mechanical deformation. The wire resists such deformation by developing
a {\em restoring torque}, $\tau$, which acts to restore the wire to its untwisted state. For relatively small angles
of twist, the magnitude of this torque is directly proportional to the twist angle. Hence, we can write
\begin{equation}\label{thook}
\tau = -k\,\theta,
\end{equation}
where $k>0$ is the {\em torque constant} of the wire. The above equation is essentially a torsional
equivalent to Hooke's law. The rotational equation of motion of the system is
written
\begin{equation}
I\,\ddot{\theta} = \tau,
\end{equation}
where $I$ is the moment of inertia of the disk (about a perpendicular axis through its centre). The moment
of inertia of the wire is assumed to be negligible. Combining the previous two equations, we obtain
\begin{equation}\label{shm1}
I\,\ddot{\theta} = - k\,\theta.
\end{equation} 

Equation~(\ref{shm1}) is clearly a simple harmonic equation [{\em cf.}, Eq.~(\ref{eshm})]. Hence,
we can immediately write the standard solution [{\em cf.}, Eq.~(\ref{shm})]
\begin{equation}
\theta = a\,\cos(\omega\,t -\phi),
\end{equation}
where [{\em cf.}, Eq.~(\ref{eomega})]
\begin{equation}
\omega  =\sqrt{\frac{k}{I}}.
\end{equation}
We conclude that when a torsion pendulum is perturbed from its equilibrium state ({\em i.e.}, $\theta=0$),
it executes torsional oscillations about this state at a fixed frequency, $\omega$, which
depends only on the torque constant of the wire and the moment of inertia of the disk. Note,
in particular, that the frequency is independent of the amplitude of the oscillation [provided $\theta$
remains small enough that Eq.~(\ref{thook}) still applies]. Torsion pendulums are often
used for time-keeping purposes. For instance, the balance wheel in a mechanical wristwatch is a torsion
pendulum in which the restoring torque is provided by a coiled spring.

\subsection{The Simple Pendulum}
Consider a mass $m$ suspended from a light inextensible string of length $l$, such that the
mass is free to swing from side to side in a vertical plane, as shown in Fig.~\ref{f98}.
This setup is known as a {\em simple pendulum}. 
 Let $\theta$ be the angle subtended between the string and
the downward vertical. Obviously, the equilibrium state of the simple pendulum corresponds to
the situation in which the mass is stationary and hanging vertically down ({\em i.e.}, $\theta=0$).
The angular equation of motion of the pendulum is simply
\begin{equation}
I\,\ddot{\theta} = \tau,
\end{equation}
where $I$ is the moment of inertia of the mass, and $\tau$ is the torque acting on the system. For the
case in hand, given that the mass is essentially a point particle, and is situated a distance $l$ from
the axis of rotation ({\em i.e.}, the pivot point), it is easily seen that $I=m\,l^2$. 

\begin{figure}
\epsfysize=3in
\centerline{\epsffile{Chapter11/fig098.eps}}
\caption{\em A simple pendulum.}\label{f98}  
\end{figure}

The two forces acting on the mass are the downward gravitational force, $m \,g$, and the tension, $T$, in the string.
Note, however, that the tension makes no contribution to the torque, since its line of action clearly passes
through the pivot point. From simple trigonometry, 
the line of action of the gravitational force passes a distance $l\,\sin\theta$ from the
pivot point. Hence, the magnitude of the gravitational torque is $m\,g\,l\,\sin\theta$.
Moreover, the gravitational torque is  a {\em restoring torque}: {\em i.e.}, if the mass is
displaced slightly from its equilibrium state ({\em i.e.}, $\theta =0$) then the gravitational force  clearly acts
 to push the mass back toward that state. Thus, we can write
\begin{equation}
\tau = - m\,g\,l\,\sin\theta.
\end{equation}
Combining the previous two equations, we obtain the following  angular equation of motion of the pendulum:
\begin{equation}\label{epend}
l\,\ddot{\theta} = - g\,\sin\theta.
\end{equation}
Unfortunately, this is {\em not} the simple harmonic equation. Indeed, the above equation possesses no
closed solution which can be expressed in terms of simple functions. 

Suppose that
we restrict our attention to relatively {\em  small} deviations from the equilibrium state. In other
words, suppose that the angle $\theta$ is constrained to take fairly small values. We know,
from trigonometry, that for $|\theta|$ less than about $6^\circ$ it is a good approximation to
write
\begin{equation}
\sin\theta \simeq \theta.
\end{equation}
Hence, in the {\em small angle limit}, Eq.~(\ref{epend}) reduces to
\begin{equation}
l\,\ddot{\theta} = - g\,\theta,
\end{equation}
which {\em is} in the familiar form of a simple harmonic equation. Comparing with our original simple harmonic
equation, Eq.~(\ref{eshm}), and its solution, we conclude that the angular frequency of small amplitude
oscillations of a simple pendulum is given by
\begin{equation}
\omega = \sqrt{\frac{g}{l}}.
\end{equation}
In this case, the pendulum frequency is dependent only on the length of the pendulum and the local gravitational acceleration,
and is independent of the mass of the pendulum and the amplitude of the pendulum swings (provided that $\sin\theta\simeq \theta$ remains
a good approximation). Historically, 
the simple pendulum was the basis of virtually all accurate time-keeping devices before the
advent of electronic clocks. Simple pendulums can also be used to measure local variations in $g$.

\subsection{The Compound Pendulum}
Consider an extended body of mass $M$ with a hole drilled though it. Suppose that the body is suspended
from a fixed peg, which passes through the hole, such that it is free to swing from side to side,
as shown in Fig.~\ref{f99}. This setup is known as a {\em compound pendulum}.

\begin{figure}
\epsfysize=3in
\centerline{\epsffile{Chapter11/fig099.eps}}
\caption{\em A compound pendulum.}\label{f99}  
\end{figure}

Let $P$ be the pivot point, and let $C$ be the body's centre of mass, which is located a distance $d$ from the
pivot. Let $\theta$ be the angle subtended between the downward vertical (which passes through point $P$) and the
line $PC$. The equilibrium state of the compound pendulum corresponds to the case in which the centre of
mass lies  vertically below the pivot point: {\em i.e.}, $\theta=0$. See Sect.~\ref{s103}.
The angular equation of motion of the pendulum is simply
\begin{equation}
I\,\ddot{\theta} = \tau,
\end{equation}
where $I$ is the moment of inertia of the body about the pivot point, and $\tau$ is the torque.
Using similar arguments to those employed for the case of the simple pendulum (recalling that all
the weight of the pendulum acts at its centre of mass), we can write
\begin{equation}
\tau = - M\,g\,d\,\sin\theta.
\end{equation}
Note that the reaction, $R$, at the peg does not contribute to the torque, since its line of action passes
through the pivot point.
Combining the previous two equations, we obtain the following  angular equation of motion of the pendulum:
\begin{equation}
I\,\ddot{\theta} = - M\,g\,d\,\sin\theta.
\end{equation}
Finally, adopting the small angle approximation, $\sin\theta \simeq \theta$, we arrive at the
 simple harmonic equation:
\begin{equation}
I\,\ddot{\theta} = - M\,g\,d\,\theta.
\end{equation}
It is clear, by analogy with our previous solutions of such equations, that the angular frequency of small
amplitude oscillations of a compound pendulum is given by
\begin{equation}\label{ecomp}
\omega = \sqrt{\frac{M\,g\,d}{I}}.
\end{equation}

It is helpful to  define the length
\begin{equation}
L = \frac{I}{M\,d}.
\end{equation}
Equation~(\ref{ecomp})  reduces to
\begin{equation}
\omega = \sqrt{\frac{g}{L}},
\end{equation}
which is identical in form to the corresponding expression for a simple pendulum. We conclude that a
compound pendulum behaves like a simple pendulum with {\em effective length} $L$. 

\subsection{Uniform Circular Motion}
Consider an object executing uniform circular motion of radius $a$. Let us set
up a cartesian coordinate system whose origin coincides with the centre of the circle,
and which is such that the motion is confined to the $x$-$y$ plane.
As illustrated in Fig.~\ref{f100}, the instantaneous position of the object can be conveniently
parameterized in terms of an angle $\theta$. 

\begin{figure}
\epsfysize=3in
\centerline{\epsffile{Chapter11/fig100.eps}}
\caption{\em Uniform circular motion.}\label{f100}  
\end{figure}

Since the object is executing {\em uniform} circular motion, we expect the angle $\theta$ to increase
{\em linearly} with time. In other words, we can write
\begin{equation}
\theta = \omega\,t,
\end{equation}
where $\omega$ is the angular rotation frequency ({\em i.e.}, the number of radians through which the
object rotates per second). Here, it is assumed that $\theta=0$ at $t=0$, for the sake of convenience. 

From simple trigonometry, the $x$- and $y$-coordinates of the object can be written
\begin{eqnarray}
x &=& a\,\cos\theta,\\[0.5ex]
y &=& a\,\sin\theta,
\end{eqnarray}
respectively. Hence, combining the previous equations, we obtain
\begin{eqnarray}
x &=& a\,\cos(\omega\,t),\\[0.5ex]
y &=& a\,\cos(\omega\,t - \pi/2).
\end{eqnarray}
Here, use has been made of the trigonometric identity $\sin\theta = \cos(\theta-\pi/2)$. 
A comparison of the above two equations with the standard equation of simple harmonic motion,
Eq.~(\ref{shm}), reveals that our object is executing simple harmonic motion simultaneously along both the
$x$- and the $y$ -axes. Note, however, that these two motions are $90^\circ$ ({\em i.e.}, $\pi/2$ radians)
{\em out of phase}. Moreover, the amplitude of the motion equals the radius of the circle.
 Clearly, there is a close relationship between simple harmonic motion
and circular motion.

\subsection*{\em Worked Example 11.1: Piston in Steam Engine}
\noindent{\em Question:} A piston in a stream engine executes simple harmonic motion.
Given that the maximum displacement of the piston from its centre-line is $\pm 7\,{\rm cm}$,
and that the mass of the piston is $4\,{\rm kg}$, find the maximum velocity of the piston
when the steam engine is running at 4000\,rev./min. What is the maximum acceleration?

\noindent{\em Answer:} We are told that the amplitude of the oscillation is $a=0.07\,{\rm m}$. 
Moreover, when converted to cycles per second ({\em i.e.}, hertz), the frequency of the oscillation becomes
$$
f = \frac{4000}{60} = 66.6666\,{\rm Hz}.
$$
Hence, the angular frequency is
$$
\omega = 2\,\pi\,f = 418.88\,{\rm rad./sec}.
$$
Consulting Tab.~\ref{tshm}, we note that the maximum velocity of an object executing
simple harmonic motion is $v_{\rm max} = a\,\omega$. Hence, the maximum velocity is
$$
v_{\rm max} = a\,\omega = 0.07\times 418.88 = 29.32\,{\rm m/s}.
$$
Likewise, according to Tab.~\ref{tshm}, the maximum acceleration is given by
$$
a_{\rm max} = a\,\omega^2 = 0.07 \times 418.88\times 418.88 = 1.228\times 10^4\,{\rm m/s^2}.
$$

\subsection*{\em Worked Example 11.2: Block and Spring}
\noindent{\em Question:} A block attached to a spring executes simple harmonic motion
in a horizontal plane with an amplitude of $0.25\,{\rm m}$. At a point $0.15\,{\rm m}$
away from the equilibrium position, the velocity of the block is $0.75\,{\rm m/s}$.
What is the period of oscillation of the block?

\noindent{\em Answer:} The equation of simple harmonic motion is
$$
x = a\,\cos(\omega\,t-\phi),
$$
where $x$ is the displacement, and $a$ is the amplitude. We are told that  $a=0.25\,{\rm m}$.
The velocity of the block is obtained by taking the time derivative of the above expression:
$$
\dot{x} =-a\,\omega\,\sin(\omega\,t -\phi).
$$
We are told that at $t=0$ (say), $x = 0.15\,{\rm m}$ and $\dot{x} = 0.75\,{\rm m/s}$.
Hence, 
\begin{eqnarray}
0.15 &=& 0.25\,\cos(\phi),\nonumber\\[0.5ex]
0.75 &=& 0.25\,\omega\,\sin(\phi).\nonumber
\end{eqnarray}
The first equation gives $\phi = \cos^{-1}(0.15/0.25) = 53.13^\circ$. The
second equation yields
$$
\omega = \frac{0.75}{0.25\times\sin(53.13^\circ)} = 3.75\,{\rm rad./s}.
$$
Hence, the period of the motion is
$$
T = \frac{2\,\pi}{\omega} = 1.676\,{\rm s}.
$$

\subsection*{\em Worked Example 11.3: Block and Two Springs}
\noindent{\em Question:} A block of mass $m=3\,{\rm kg}$ is attached to
two springs, as shown below, and slides over a horizontal frictionless
surface. Given that the force constants of the two springs
are $k_1= 1200\,{\rm N/m}$ and $k_2= 400\,{\rm N/m}$, find the period
of oscillation of the system.

\begin{figure*}[h]
\epsfysize=1.5in
\centerline{\epsffile{Chapter11/fig100a.eps}}
\end{figure*}

\noindent{\em Answer:}
Let $x_1$ and $x_2$ represent the extensions of the first and second springs, respectively. 
The net displacement $x$ of the mass from its equilibrium position is then given by
$$
x = x_1 + x_2.
$$
Let $f_1 = k_1\,x_1$ and $f_2=k_2\,x_2$ be the  magnitudes of the forces exerted by the first and second springs,
respectively. Since the springs  (presumably)  possess negligible inertia,
they must exert equal and opposite forces on one another. This implies that $f_1 = f_2$, or
$$
k_1\, x_1 = k_2\,x_2.
$$
Finally, if $f$ is the magnitude of the restoring force acting on the mass, then 
force balance implies that $f = f_1 = f_2$, or
$$
f = k_{\rm eff}\, x = k_1\,x_1.
$$
Here, $k_{\rm eff}$ is the effective force constant of the two springs. 
The above equations can be combined to give
$$
k_{\rm eff} = \frac{k_1\,x_1}{x_1+x_2} = \frac{k_1}{1+ k_1/k_2} = \frac{k_1\,k_2}{k_1+k_2}.
$$
Thus, the problem reduces to that of a block of mass $m=3\,{\rm kg}$ attached to
a spring of effective force constant
$$
k_{\rm eff} = \frac{k_1\,k_2}{k_1+k_2} = \frac{1200\times 400}{1200+400} = 300\,{\rm N/m}.
$$
The angular frequency of oscillation is immediately given by the standard formula
$$
\omega = \sqrt{\frac{k_{\rm eff}}{m}}=\sqrt{\frac{300}{3}} = 10\,{\rm rad./s}.
$$
Hence, the period of oscillation is
$$
T = \frac{2\,\pi}{\omega} = 0.6283\,{\rm s}.
$$

\subsection*{\em Worked Example 11.4: Energy in Simple Harmonic Motion}
\noindent{\em Question:}
A block of mass $m=4\,{\rm kg}$ is attached to a spring, and undergoes simple harmonic
motion with a period of $T=0.35\,{\rm s}$. The total energy of the system is
$E=2.5\,{\rm J}$. What is the force constant of the spring? What is the amplitude
of the motion?

\noindent{\em Answer:}
The angular frequency of the motion is
$$
\omega = \frac{2\,\pi}{T} = \frac{2\,\pi}{0.35} = 17.95\,{\rm rad./s}.
$$
Now, $\omega=\sqrt{k/m}$ for a mass on a spring. Rearrangement of this formula yields
$$
k = m\,\omega^2 = 4 \times 17.95\times 17.95 = 1289.1\,{\rm N/m}.
$$
The total energy of a system executing simple harmonic motion is
$E= a^2\,k/2$. Rearrangement of this formula gives
$$
a = \sqrt{\frac{2\,E}{k}} = \sqrt{\frac{2\times 2.5}{1289.1}} = 0.06228\,{\rm m}.
$$
Thus, the force constant is $ 1289.1\,{\rm N/m}$ and the amplitude is $0.06228\,{\rm m}$. 

\subsection*{\em Worked Example 11.5: Gravity on a New Planet}
\noindent{\em Question:} Having landed on a newly discovered planet, an astronaut sets up 
a simple pendulum of length $0.6\,{\rm m}$, and finds that it makes 51 complete
oscillations in 1 minute. The amplitude of the oscillations is small compared to the
length of the pendulum. What is the surface gravitational acceleration on the planet?

\noindent{\em Answer:} The frequency of the oscillations is
$$
f = \frac{51}{60}  = 0.85 \,{\rm Hz}.
$$
Hence, the angular frequency is
$$
\omega = 2\,\pi\,f = 2\times \pi\times 1.833 = 5.341\,{\rm rad./s}.
$$
Now, $\omega = \sqrt{g/l}$ for small amplitude oscillations of a simple pendulum. Rearrangement off this formula gives 
$$
g = \omega^2\,l = 5.341\times 5.341\times 0.6 = 17.11\,{\rm m/s^2}.
$$
Hence, the  surface gravitational acceleration is $17.11\,{\rm m/s^2}$.

\subsection*{\em Worked Example 11.6: Oscillating Disk}
\noindent{\em Question:} A uniform disk of radius $r = 0.8\,{\rm m}$ and mass $M = 3\,{\rm kg}$
is freely suspended from a horizontal pivot located a radial distance  $d=0.25\,{\rm m}$ from its centre.
Find the angular frequency of small amplitude oscillations of the disk. 

\noindent{\em Answer:} The moment of inertia of the disk about a perpendicular axis passing through its
centre is $I= (1/2)\,M\,r^2$. From the parallel axis theorem, the moment of inertia of the disk
about the pivot point is
$$
I' = I + M\,d^2 = \frac{3\times 0.8\times 0.8}{2} + 3\times 0.25\times 0.25 = 1.1475\,{\rm kg\,m^2}.
$$
The angular frequency of small amplitude oscillations of a compound pendulum is given by
$$
\omega = \sqrt{\frac{M\,g\,d}{I'}} = \sqrt{\frac{3\times 9.81\times 0.25}{1.1475}} = 2.532\,{\rm rad./s}.
$$
Hence, the answer is $2.532\,{\rm rad./s}$.