\section{Orbital Motion}
\subsection{Introduction}
We have spent this course exploring the theory of motion first outlined by
Sir Isaac Newton in his {\em Principia} (1687). It is, therefore, 
interesting to discuss
the particular application of this theory which made Newton
an international celebrity, and which profoundly and permanently changed 
humankind's outlook on the Universe. This application is, of course, the motion of the Solar System. 

\subsection{Historical Background}
Humankind has always been fascinated by the night sky, and, in particular, by the movements of
the Sun, the Moon, and the objects which the ancient Greeks called {\em plantai} (``wanderers''), and which we call
{\em planets}. In ancient times, much of this interest was of a practical nature. The Sun and the
Moon were important for determining the calendar, and also for navigation. Moreover, the
planets were vital to astrology: {\em i.e.}, the belief---almost universally prevalent in the ancient world---that
the positions of the planets in the sky could be used to foretell important events.

Actually, there were only seven ``wandering'' heavenly bodies
 visible to  ancient peoples: the Sun, the Moon, and the five
planets---Mercury, Venus, Mars, Jupiter, and Saturn. The ancients believed that the stars were fixed
to a ``celestial sphere'' which formed the outer boundary of the Universe. However, it was recognized
that the wandering bodies were located {\em within} this sphere: {\em e.g.},  because the Moon clearly
passes in front of, and blocks the light from, stars in its path. 
It was also recognized that some  bodies were closer to the Earth
than others. For instance,  ancient astronomers
 noted that the Moon occasionally passes in front of the Sun and
each of the planets. Moreover, Mercury and Venus can sometimes be seen to transit in front of the Sun.

The first scientific model of the Solar System was outlined by the Greek philosopher
Eudoxas of Cnidus (409--356\,BC). According to this model,
the Sun, the Moon, and the planets all execute uniform circular orbits around the Earth---which is
fixed, and non-rotating.
 The order of the orbits is as follows: Moon, Mercury, Venus, Sun, Mars, Jupiter, Saturn---with 
the Moon closest to the Earth. For obvious reasons, Eudoxas' model became known as the
{\em geocentric} model of the Solar System. Note that  orbits are circular in this
model for philosophical reasons. The ancients believed the heavens to be the realm of perfection.
Since a circle is the most ``perfect'' imaginable shape, it follows that heavenly objects must execute
circular orbits.

A second Greek philosopher, Aristarchus of Samos (310--230\,BC), proposed an 
alternative model in which the Earth and the planets execute uniform
circular orbits around the Sun---which is fixed. Moreover, the Moon orbits around the Earth, and the Earth
rotates daily about a North-South axis. The
order of the planetary orbits is as follows: Mercury, Venus, Earth, Mars, Jupiter, Saturn---with Mercury
closest to the Sun. This model became known as the {\em heliocentric} model of the Solar System. 

The heliocentric model was generally rejected by the ancient philosophers for three main reasons:
\begin{enumerate}
\item If the Earth is rotating about its axis, and
orbiting around the Sun, then the Earth must be in motion. However, we cannot ``feel''
this motion. Nor does this motion give rise to any obvious observational consequences.
Hence, the Earth must be stationary.
\item If the Earth is executing a circular orbit around the Sun then the positions of the stars
should be slightly different when the Earth is on opposite sides of the Sun. This effect
is known as {\em parallax}. Since no stellar parallax is observable (at least, with the naked eye), the Earth
must be stationary. In order to appreciate the force of this argument, it is important to realize
that  ancient astronomers did not suppose the stars to be significantly further away from the Earth than
the planets. The celestial sphere was assumed to lie just beyond the orbit of Saturn.
\item The geocentric model is far more philosophically attractive than the heliocentric model, since in
the former model the Earth occupies a privileged position in the Universe.
\end{enumerate}

The geocentric model was first converted into a proper scientific theory, capable of accurate 
predictions, by the Alexandrian philosopher Claudius Ptolemy (85--165\,AD). The theory that Ptolemy proposed in his famous book, now known as the {\em Almagest}, remained the dominant scientific picture
of the Solar System  for over a millennium. Basically, Ptolemy acquired and extended the extensive 
set of planetary observations
of his predecessor Hipparchus, and then constructed a geocentric model capable of accounting
for them. However, in order to fit the observations, Ptolemy was forced to make
some significant modifications to the original model of Eudoxas. Let us
discuss these modifications.

\begin{figure}
\epsfysize=3in
\centerline{\epsffile{Chapter12/fig101.eps}}
\caption{\em The Ptolemaic system.}\label{f101}  
\end{figure}

First, we need to introduce some terminology.
As shown in Fig.~\ref{f101},
{\em deferants} are large circles centred on the Earth, and {\em epicyles} are small circles whose
centres move around the circumferences of the deferants. In the  Ptolemaic system,
instead of traveling around deferants,
the planets 
move around the circumference of epicycles, which, in turn, move around the circumference of
 deferants. Ptolemy found, however, that this modification was insufficient to completely account
 for
all of his data. Ptolemy's second modification to Eudoxas' model was to displace the Earth slightly from the
common centre of  the deferants. Moreover, Ptolemy assumed that the Sun, Moon, and planets rotate
uniformly about an imaginary point, called the {\em equant}, which is displaced an equal distance in the opposite
direction to the Earth from the centre of the deferants. In other words, Ptolemy assumed that the line
 $EP$, in Fig.~\ref{f101},
rotates uniformly, rather than the line $CP$.

Figure~\ref{f102} shows more details of the Ptolemaic model.\footnote{R.A.~Hatch, University
of Florida, {\tt http://web.clas.ufl.edu/users/rhatch/}}  Note that this diagram is
not drawn to scale, and the displacement of the Earth from the centre of the deferants has been
omitted for the sake of clarity. It can be seen that the Moon and the Sun do not possess
epicyles. 
Moreover, the motions of the {\em inferior planets} ({\em i.e.},
Mercury and Venus) are closely linked to the motion of the Sun. In fact, the centres of the inferior planet
epicycles move on an imaginary line connecting the Earth and the Sun. Furthermore, the
radius vectors connecting the {\em superior planets} ({\em i.e.}, Mars, Jupiter, and Saturn)
to the centres of their epicycles are always parallel to the geometric line connecting the
Earth and the Sun. Note that, in addition to the motion indicated in the diagram, all of the
heavenly bodies (including the stars) rotate clockwise (assuming that we are looking down on the
Earth's North pole in Fig.~\ref{f102}) with a period of 1 day. Finally, there are epicycles
within the epicycles shown in the diagram. In fact, some planets need as many as 28
epicycles to account for all the details of their motion. These subsidiary epicycles
are not shown in the diagram, for the sake of clarity.

As is quite apparent, the Ptolemaic model of the Solar System is {\em extremely complicated}. 
However, it successfully accounted for the relatively
crude
 naked eye observations made by the ancient Greeks. The Sun-linked epicyles of the
inferior planets are needed to explain why these objects always remain close to the
Sun in the sky. The epicycles of the superior planets are needed to account for
their occasional bouts of {\em retrograde motion}: {\em i.e.}, motion in the opposite
direction to their apparent direction of rotation around the Earth. Finally, the
displacement of the Earth from the centre of the deferants, as well as the introduction
of the equant as the centre of uniform rotation, is needed to explain why the planets
speed up slightly when they are close to the Earth (and, hence, appear
brighter in the night sky), and slow down when they are further away.

\begin{figure}
\epsfysize=5.5in
\centerline{\epsffile{Chapter12/fig102.eps}}
\caption{\em The Ptolemaic model of the Solar System.}\label{f102}  
\end{figure}

Ptolemy's model of the Solar System was rescued from the wreck of ancient European civilization
by the Roman Catholic Church, which, unfortunately, converted it into a minor article
of faith, on the basis of a few references in the Bible which seemed to imply that the Earth is
stationary and the Sun is moving ({\em e.g.}, Joshua 10:12-13, Habakkuk 3:11). 
Consequently, this model was not subject to proper scientific criticism for over a
millennium. Having said this, few medieval or renaissance philosophers were entirely satisfied with
Ptolemy's model. Their dissatisfaction focused, not on the many epicycles (which to the
modern eye seem rather absurd), but on the displacement of the Earth from the
centre of the deferants, and the introduction of the equant as the centre of uniform
rotation. Recall, that the only reason  planetary orbits are constructed from circles
in Ptolemy's model is to preserve the assumed ideal symmetry of the heavens. Unfortunately,
this symmetry is severely compromised when the Earth is displaced from the
apparent centre of the Universe. This problem so perplexed the Polish priest-astronomer 
Nicolaus Copernicus (1473--1543) that he eventually decided to reject the geocentric
model, and revive the heliocentric model of Aristarchus. After many years of mathematical
calculations, Copernicus published a book entitled {\em De revolutionibus orbium coelestium}
(On the revolutions of the celestial spheres) in 1543 which outlined his new heliocentric
theory.

\begin{figure}
\epsfysize=5.5in
\centerline{\epsffile{Chapter12/fig103.eps}}
\caption{\em The Copernican model of the Solar System.}\label{f103}  
\end{figure}

Copernicus' model is illustrated in Fig.~\ref{f103}. Again, this diagram is
not to scale. The planets execute uniform
circular orbits about the Sun, and the Moon orbits about the Earth. Finally, the Earth
revolves about its axis daily. Note that there is no displacement of the Sun from the
centres of the planetary orbits, and there is no equant. Moreover, in this model, the inferior planets
remain close to the Sun in the sky without any special synchronization of their orbits. Furthermore,
the occasional retrograde motion of the superior planets has a more natural explanation than
 in Ptolemy's model. Since the Earth orbits more rapidly than
the superior planets, it occasionally ``overtakes'' them, and they appear to move
backward in the night sky, in much the same manner that  slow moving
cars on a freeway appears to move backward to a driver overtaking them. Copernicus
accounted for the lack of stellar parallax, due to
the Earth's motion,  by postulating that the stars
were a lot further away than had previously been supposed, rendering any parallax undetectably
small. Unfortunately, Copernicus insisted on  retaining  uniform circular motion
in his model (after all, he was trying to construct a more symmetric model than
that of Ptolemy). Consequently, Copernicus also had to resort to epicycles to fit the
data. In fact, Copernicus' model ended up with more epicycles than  Ptolemy's!

The real breakthrough in the understanding of planetary motion occurred---as most
breakthroughs in physics occur---when better data became available. 
The data in question was produced by the Dane Tycho Brahe (1546--1601),
who devoted his life to making naked eye astronomical observations  of
unprecedented accuracy and detail. This data was eventually inherited by Brahe's pupil and
assistant, the German scientist Johannes Kepler (1571--1630). 
Kepler fully accepted Copernicus' heliocentric theory of the Solar System.
Moreover, he was just as firm a believer as Copernicus in the perfection of the heavens, and the
consequent need for
circular motion of planetary bodies. The main  difference was that Kepler's
observational data was considerably better than Copernicus'. After years
of fruitless effort, Kepler eventually concluded that no combination of circular 
deferants and epicycles could completely account for his data. At this stage,
he started to think the unthinkable. Maybe, planetary motion was not circular after all?
After more calculations, Kepler was eventually able to formulate three extraordinarily simple laws
which completely accounted for Brahe's observations. These laws are as follows:
\begin{enumerate}
\item The planets move in elliptical orbits with the Sun at one focus. 
\item A line from the Sun to any given planet sweeps out equal areas in equal time intervals. 
\item The square of a planet's period is proportional to the cube of the planet's mean distance from the Sun.
\end{enumerate}
Note that there are no epicyles or equants in Kepler's model of the Solar System.

Figure~\ref{f104} illustrates Kepler's second law. Here, the ellipse represents a planetary orbit, and
$S$ represents the Sun, which is located at one of the focii of the ellipse. Suppose that the
planet moves from point $A$ to point $B$ in the same time it takes to move from point $C$ to point $D$.
According the Kepler's second law, the areas of the elliptic segments $ASB$ and $CSD$ are equal. Note that this law
basically mandates that planets speed up when they move closer to the Sun. 

\begin{figure}
\epsfysize=2in
\centerline{\epsffile{Chapter12/fig104.eps}}
\caption{\em Kepler's second law.}\label{f104}  
\end{figure}

Table~\ref{tkepler} illustrates Kepler's third law. The mean distance, $a$,  and orbital period, $T$,  as well as the ratio $a^3/T^2$,
are listed for each of the first six planets in the Solar System. It can be seen that the ratio
$a^3/T^2$ is indeed  constant from planet to planet.

\begin{table}
\centering
\begin{tabular}{r|ccc}
Planet & $a({\rm AU})$ & $T({\rm yr})$ & $a^3/T^2$\\[0.5ex] \hline
Mercury & 0.387 & 0.241 & 0.998 \\[0.5ex]
Venus & 0.723 & 0.615 &  0.999 \\[0.5ex]
Earth & 1.000 & 1.000 &  1.000\\ [0.5ex]
Mars & 1.524 & 1.881 &  1.000 \\[0.5ex]
Jupiter & 5.203 & 11.862 &1.001\\ [0.5ex]
Saturn & 9.516 & 29.458 & 0.993\\
\end{tabular}
\caption{\em Kepler's third law. Here, $a$ is the mean distance from the Sun, measured in Astronomical Units (1 AU is the
mean Earth-Sun distance), and $T$ is the orbital period, measured in years.}\label{tkepler}
\end{table}

Since we have now definitely adopted a heliocentric model of the Solar System, let us
discuss the ancient Greek objections to such a model, listed earlier. We have already
dealt with the second objection (the absence of stellar parallax) by stating that the
stars are a lot further away from the Earth than the ancient Greeks supposed. The
third objection (that it is philosophically more attractive to have the Earth
at the centre of the Universe) is not a valid scientific criticism. What about the
first objection? If the Earth is rotating about its axis, and also orbiting
the Sun, why do we not ``feel'' this motion? At first sight, this objection appears
to have some force. After all, the rotation velocity of the Earth's surface 
is about $460\,{\rm m/s}$. Moreover, the Earth's orbital velocity is approximately
$30\,{\rm km/s}$. Surely, we would notice if we were moving this rapidly? Of course,
this reasoning is faulty because we know, from Newton's laws of motion, that we only
``feel'' the acceleration associated with motion, not the motion itself. It turns
out that the acceleration at the Earth's surface due to its axial rotation is only about
$0.034\,{\rm m/s^2}$. Moreover, the Earth's acceleration due to its orbital motion
is only $0.0059\,{\rm m/s^2}$. Nowadays, we can detect such small accelerations, but the
ancient Greeks certainly could not.

Kepler correctly formulated the
 three laws of planetary motion in 1619.  Almost seventy years later, in 1687,
Isaac Newton published his {\em Principia}, in which he presented, for the
first time, a universal theory of motion. Newton then went on to
illustrate his
 theory by using it to deriving Kepler's laws from first principles. Let us now discuss Newton's
monumental achievement in more detail.

\subsection{Gravity}
There is one important question which we have avoided discussing until now. Why do objects fall towards the
surface of the Earth? The ancient Greeks had a very simple answer to this question.
According to Aristotle, all objects have a natural tendency to fall towards the centre of
the Universe. Since the centre of the Earth coincides with the centre of the Universe, all objects also
tend to fall towards the Earth's surface. So, an ancient Greek might ask, why do the planets
not fall towards the Earth? Well, according to Aristotle, the planets are embedded in crystal
spheres which rotate with them whilst holding them in place in the firmament. Unfortunately,
Ptolemy  seriously undermined this explanation by shifting the Earth slightly from the 
centre of the Universe. However, the {\em coup de grace} was delivered by Copernicus,
who converted the Earth into just another planet orbiting the Sun.

So, why do objects fall towards the surface of the Earth? The first person, after Aristotle, to seriously
consider this question was Sir Isaac Newton. Since the Earth is not located in a special
place in the Universe, Newton reasoned, objects must be attracted toward the Earth itself.
Moreover, since the Earth is just another planet, objects must be attracted towards other
planets as well. In fact, all objects must exert a force of attraction on all
other objects in the Universe. What intrinsic property of  objects causes them
to exert this attractive force---which Newton termed {\em gravity}---on other objects? Newton  decided that the
crucial property was {\em mass}. After much thought, he was eventually able to
formulate his famous law of universal gravitation:
\begin{quote}
{\sf Every particle in the Universe attracts every other particle with a
force directly proportional to the product of their masses and inversely
proportional to the square of the distance between them. The direction
of the force is along the line joining the particles.}
\end{quote}
Incidentally, Newton adopted an inverse square law because he knew that this was the
only type of force law which was consistent with Kepler's third law of planetary motion.

 Consider two point
objects of masses $m_1$ and $m_2$, separated by a distance $r$.
As illustrated in Fig.~\ref{f105}, the magnitude of the force of attraction
between these objects  is 
\begin{equation}
f = G\,\frac{m_1\,m_2}{r^2}.
\end{equation}
The direction of the force is along the line joining the two objects.

\begin{figure}
\epsfysize=2in
\centerline{\epsffile{Chapter12/fig105.eps}}
\caption{\em Newton's law of gravity.}\label{f105}  
\end{figure}

Let ${\bf r}_1$ and ${\bf r}_2$ be the vector positions of the two objects,
respectively. The vector gravitational force exerted by object 2 on object 1 can
be written
\begin{equation}
{\bf f}_{12} = G\,\frac{{\bf r}_2-{\bf r}_1}{|{\bf r}_2-{\bf r}_1|^3}.
\end{equation}
Likewise, the vector gravitational force exerted by object 1 on object 2 takes
the form
\begin{equation}
{\bf f}_{21} = G\,\frac{{\bf r}_1-{\bf r}_2}{|{\bf r}_1-{\bf r}_2|^3} = - {\bf f}_{21}.
\end{equation}

The constant of proportionality, $G$, appearing in the above formulae is called the
{\em gravitational constant}. Newton could only estimate the value of this quantity, which was first
directly measured by Henry Cavendish in 1798. The modern value of $G$ is
\begin{equation}
G = 6.6726\times 10^{-11}\,{\rm N\,m^2/ kg^2}.
\end{equation}
Note that the gravitational constant is numerically extremely small. This implies that gravity is
an intrinsically weak force. In fact, gravity usually  only becomes  significant  if at least
one of the masses involved is of astronomical dimensions ({\em e.g.}, it is a planet, or
a star).

Let us use Newton's law of gravity to account for the Earth's surface gravity. 
Consider an object of mass $m$ close to the surface of the Earth, whose mass
and radius are $M_\oplus =5.97\times 10^{24}\,{\rm kg}$ and $R_\oplus = 6.378\times 10^6\,{\rm m}$, 
respectively. Newton proved, after considerable effort, that the gravitational force 
exerted by a spherical body (outside that body) is the same as that exerted by an
equivalent  point mass located at the body's centre. Hence, the gravitational
force exerted by the Earth on the object in question is of magnitude
\begin{equation}
f = G\, \frac{m\,M_\oplus}{R_\oplus^{\,2}},
\end{equation}
and is directed towards the centre of the Earth. It follows that the equation of
motion of the object can be written
\begin{equation}
m\,\ddot{\bf r} = - G\, \frac{m\,M_\oplus}{R_\oplus^{\,2}}\,\hat{\bf z},
\end{equation}
where $\hat{\bf z}$ is a unit vector pointing straight upwards ({\em i.e.}, away
from the Earth's centre). Canceling the factor $m$ on either side of the above
equation, we obtain
\begin{equation}
\ddot{\bf r} = - g_\oplus\,\hat{\bf z},
\end{equation}
where
\begin{equation}
g_\oplus = \frac{G\, M_\oplus}{R_\oplus^{\,2}} =  \frac{ (6.673\times 10^{-11})\times
(5.97\times 10^{24})}{(6.378\times 10^6)^2} = 9.79\,{\rm m/s^2}.\label{sgrav}
\end{equation}
Thus, we conclude that all objects on the Earth's surface, irrespective of
their mass,  accelerate straight down ({\em i.e.}, towards the Earth's centre)
with a constant acceleration of $9.79\,{\rm m/s^2}$. This estimate for the
acceleration due to gravity is slightly off the conventional value of $9.81\,{\rm m/s^2}$
because the Earth is actually not quite spherical.

Since Newton's law of gravitation is universal, we immediately conclude that  any
spherical body of mass $M$ and radius $R$ possesses a surface gravity $g$ given
by the following formula:
\begin{equation}
\frac{g}{g_\oplus} = \frac{M/M_\oplus}{(R/R_\oplus)^2}.
\end{equation}
Table~\ref{tgravity} shows the surface gravity of various bodies in the Solar System,
estimated using the above expression. It can be seen that the surface gravity of the
Moon is only about one fifth of that of the Earth. No wonder Apollo astronauts were able
to jump so far on the Moon's surface! Prospective Mars colonists should note that they
will only weigh about a third of their terrestrial weight on Mars. 

\begin{table}
\centering
\begin{tabular}{r|ccc}
Body & $M/M_\oplus$ & $R/R_\oplus$ & $g/g_\oplus$\\[0.5ex] \hline
Sun & $3.33\times 10^5$ &  109.0 & 28.1\\[0.5ex]
Moon & 0.0123& 0.273 & 0.17 \\[0.5ex]
Mercury & 0.0553 & 0.383 & 0.38 \\[0.5ex]
Venus & 0.816 & 0.949 &  0.91 \\[0.5ex]
Earth & 1.000 & 1.000 &  1.000\\ [0.5ex]
Mars & 0.108 & 0.533 &  0.38 \\[0.5ex]
Jupiter & 318.3 & 11.21 &2.5\\ [0.5ex]
Saturn & 95.14 & 9.45 & 1.07\\
\end{tabular}
\caption{\em The mass, $M$, radius, $R$, and surface gravity, $g$, of various
bodies in the Solar System. All quantities are expressed as fractions of the
corresponding terrestrial quantity.}\label{tgravity}
\end{table}

\subsection{Gravitational Potential Energy}
We saw earlier, in Sect.~\ref{spotn}, that gravity is a conservative force, and, therefore,
has an associated potential energy. Let us obtain a general formula for this energy.
Consider a point object of mass $m$, which is a radial distance $r$ from another point
object of mass $M$. The gravitational force acting on the first mass is
of magnitude $f = G\,M/r^2$, and is directed towards the second mass. Imagine that the
first mass moves radially away from the second mass, until it reaches infinity. What is
the change in the potential energy of the first mass associated with this shift?
According to Eq.~(\ref{epot}), 
\begin{equation}
 U(\infty) - U(r)  = -\int_{r}^\infty [-f(r)]\,dr.
\end{equation}
There is a minus sign  in front of $f$  because this force is oppositely directed to the motion.
The above expression can be integrated to give
\begin{equation}
U(r) = -\frac{G\,M\,m}{r}.
\end{equation}
Here, we have adopted the convenient normalization that the potential energy at infinity is
zero. According to the above formula, the gravitational potential energy of a mass $m$
located a distance $r$ from a mass $M$ is simply $-G\,M\,m/r$. 

Consider an object of mass $m$ moving close to the Earth's surface. The potential
energy of such an object can be written
\begin{equation}
U = -\frac{G\,M_\oplus\,m}{R_\oplus +z},
\end{equation}
where $M_\oplus$ and ${R_\oplus}$ are the mass and radius of the Earth, respectively, and
$z$ is the vertical height of the object above the Earth's surface. In the limit that
$z\ll R_\oplus$, the above expression can be expanded using the binomial theorem
to give
\begin{equation}
U \simeq -\frac{G\,M_\oplus\,m}{R_\oplus} + \frac{G\,M_\oplus\,m}{R_\oplus^{\,2}}\,z,
\end{equation}
Since potential energy is undetermined to an arbitrary additive constant, we could just as
well write
\begin{equation}
U \simeq m\,g\,z,
\end{equation}
where $g = G\,M_\oplus/R_\oplus^{\,2}$ is the acceleration due to gravity at
the Earth's surface [see Eq.~(\ref{sgrav})]. Of course, the above formula is equivalent
to the formula (\ref{e53}) derived earlier on  in this course.

For an object of mass $m$ and speed $v$, moving in the gravitational field of a fixed object
of mass $M$, we expect the total energy,
\begin{equation}
E = K + U,
\end{equation}
to be a constant of the motion. Here, the kinetic energy is written $K = (1/2)\,m\,v^2$, whereas
the potential energy takes the form $U= - G\,M\,m/r$. Of course, $r$ is the distance between the
two objects. Suppose that the fixed object is a sphere of radius $R$. Suppose,
further, that the second object is launched from the surface of this sphere
with some velocity $v_{\rm esc}$ which is such that it {\em only just} escapes the sphere's gravitational
influence. After the object has escaped, it is a long way away from the sphere, and hence 
$U=0$. Moreover, if the object only just escaped, then we also expect $K=0$, since the
object will have expended all of its initial kinetic energy escaping from the sphere's
gravitational well. We conclude that our object possesses zero net energy: 
{\em i.e.}, $E = K+U = 0$. Since $E$ is a constant of the motion, it follows
 that at the launch point
\begin{equation}
E = \frac{1}{2}\,m\,v_{\rm esc}^{\,2} - \frac{G\,M\,m}{R} = 0.
\end{equation}
This expression can be rearranged to give
\begin{equation}
v_{\rm esc} = \sqrt{\frac{2\,G\,M}{R}}.
\end{equation}
The quantity $v_{\rm esc}$ is known as the {\em escape velocity}. Objects launched from
the surface of the sphere with velocities exceeding this value will eventually
escape from the sphere's gravitational influence. Otherwise, the objects
will remain in orbit around the sphere, and may eventually strike its surface.
Note that the escape velocity is independent of the object's mass and launch
direction (assuming that it is not straight into the sphere). 

The escape velocity for the Earth is
\begin{equation}
v_{\rm esc} = \sqrt{\frac{2\,G\,M_\oplus}{R_\oplus}} = 
\sqrt{\frac{2\times(6.673\times 10^{-11})\times(5.97\times 10^{24})}{6.378\times 10^6}}=
11.2\,{\rm km/s}.
\end{equation}
Clearly, NASA must launch deep space probes from the surface of the Earth with  velocities which exceed this value if they are 
to have any hope of eventually reaching their targets.

\subsection{Satellite Orbits}
Consider an artificial satellite executing a circular orbit of radius $r$ around the
Earth. Let $\omega$ be the satellite's orbital angular velocity. 
The satellite experiences an acceleration towards the Earth's centre of magnitude
$\omega^2\,r$. Of course, this acceleration is provided by the 
gravitational attraction between the satellite and the Earth, which yields an acceleration of magnitude
$G\,M_\oplus/r^2$. It follows that
\begin{equation}\label{geo}
\omega^2\,r = \frac{G\,M_\oplus}{r^2}.
\end{equation}

Suppose that the satellite's orbit lies in the Earth's equatorial plane. 
Moreover, suppose that the satellite's orbital angular velocity just matches the
Earth's angular velocity of rotation. In this case, the satellite will appear to
hover in the {\em same place} in the sky to a stationary observer on the Earth's surface. A satellite with
this singular property is known as a {\em geostationary satellite}. 

Virtually all of the satellites used to monitor the Earth's weather patterns
are geostationary in nature. Communications satellites  also tend to be geostationary. 
Of course, the satellites which beam satellite-TV to homes across the world {\em must}  be
geostationary---otherwise, you would need to install an expensive tracking antenna on top
of your house in order to pick up  the transmissions. Incidentally, the  person who first
envisaged rapid global telecommunication via a network of geostationary
 satellites was the science fiction writer Arthur C.\ Clarke in 1945.

Let us calculate the orbital radius of a geostationary satellite. The angular
velocity of the Earth's rotation is
\begin{equation}
\omega = \frac{2\,\pi}{24\times 60 \times 60} = 7.27\times 10^{-5}\,{\rm rad./s}.
\end{equation}
It follows from Eq.~(\ref{geo}) that
\begin{eqnarray}
r_{\rm geo} &=& \left(\frac{G\,M_\oplus}{\omega^2}\right)^{1/3} = 
 \left(\frac{(6.673\times 10^{-11})\times(5.97\times 10^{24})}{(7.27\times 10^{-5})^2}\right)^{1/3}\nonumber
\\[0.5ex]
&=&4.22\times 10^7\,{\rm m} = 6.62\,R_\oplus.
\end{eqnarray}
Thus, a geostationary satellite must be placed in a circular orbit whose radius
is {\em exactly} $6.62$ times the Earth's radius.

\subsection{Planetary Orbits}
Let us now see whether we can use Newton's universal laws of motion to derive
Kepler's laws of planetary motion. Consider a planet orbiting around the Sun.
It is convenient to specify the planet's instantaneous position,
with respect to the Sun, in terms of the {\em polar coordinates}\/ $r$ and $\theta$. 
As illustrated in Fig.~\ref{f106}, $r$ is the radial distance between the planet and the
Sun, whereas $\theta$ is the angular bearing of the planet, from the Sun,
measured with respect to some arbitrarily chosen direction.

\begin{figure}
\epsfysize=2.5in
\centerline{\epsffile{Chapter12/fig106.eps}}
\caption{\em A planetary orbit.}\label{f106}  
\end{figure}

Let us define two unit vectors, ${\bf e}_r$ and ${\bf e}_\theta$. (A unit vector is
simply a vector whose length is unity.) As shown in Fig.~\ref{f106}, the {\em radial}
unit vector ${\bf e}_r$ always points from the Sun towards the instantaneous position
of the planet. Moreover, the {\em tangential} unit vector ${\bf e}_\theta$ is always
normal to ${\bf e}_r$, in the direction of increasing $\theta$. 
In Sect.~\ref{s75}, we demonstrated that when acceleration is written in terms of polar
coordinates, it takes the form
\begin{equation}
{\bf a} = a_r\,{\bf e}_r + a_\theta\,{\bf e}_\theta,
\end{equation}
where
\begin{eqnarray}
a_r  &=& \ddot{r} - r\,\dot{\theta}^2,\\[0.5ex]
a_\theta &=& r\,\ddot{\theta} + 2\,\dot{r}\,\dot{\theta}.
\end{eqnarray}
These expressions are more complicated that the corresponding cartesian expressions
because the unit vectors  ${\bf e}_r$ and ${\bf e}_\theta$ {\em change direction}
as the planet changes position.

Now, the planet is subject to a single force: {\em i.e.}, the force of gravitational attraction  exerted
by the Sun. In polar coordinates, this force takes a particularly simple
form (which is why we are using polar coordinates):
\begin{equation}
{\bf f} = - \frac{G\,M_\odot\,m}{r^2}\,{\bf e}_r.
\end{equation}
The minus sign indicates that the force is directed towards, rather than away from, the Sun.

According to Newton's second law, the planet's equation of motion is written
\begin{equation}
m\,{\bf a} = {\bf f}.
\end{equation}
The above four equations yield
\begin{eqnarray}
 \ddot{r} - r\,\dot{\theta}^2 &=& - \frac{G\,M_\odot}{r^2},\label{e1227}\\[0.5ex]
 r\,\ddot{\theta} + 2\,\dot{r}\,\dot{\theta} &=& 0.\label{e1282}
\end{eqnarray}

Equation~(\ref{e1282}) reduces to
\begin{equation}
\frac{d}{dt}\,(r^2\,\dot{\theta}) = 0,
\end{equation}
or
\begin{equation}
r^2\,\dot{\theta} = h,
\end{equation}
where $h$ is a {\em constant of the motion}. What is the physical interpretation of $h$? Recall,
from Sect.~\ref{sang}, that the angular momentum vector of a point particle can be written
\begin{equation}
{\bf l} = m\,{\bf r}\times{\bf v}.
\end{equation}
For the case in hand, ${\bf r} = r\,{\bf e}_r$ and ${\bf v} = \dot{r}\,{\bf e}_r+r\,\dot{\theta}\,
{\bf e}_\theta$ [see Sect.~\ref{s75}]. Hence,
\begin{equation}
l = m\,r\,v_\theta = m\,r^2\,\dot{\theta},
\end{equation}
yielding
\begin{equation}
h = \frac{l}{m}.
\end{equation}
Clearly,  $h$ represents the {\em angular momentum} (per unit mass) of our planet around the Sun.
Angular momentum is conserved ({\em i.e.}, $h$ is constant)
because the force of gravitational attraction between the planet and
the Sun exerts {\em zero torque} on the planet. (Recall, from Sect.~\ref{sangm}, that torque is the 
rate of change of angular momentum.)
The torque is zero because the gravitational force is {\em radial} in nature: {\em i.e.},
its line of action passes through the Sun, and so its associated lever arm is of  length zero.

\begin{figure}
\epsfysize=1.5in
\centerline{\epsffile{Chapter12/fig107.eps}}
\caption{\em The origin of Kepler's second law.}\label{f107}  
\end{figure}

The quantity $h$ has another physical interpretation. Consider Fig.~\ref{f107}. Suppose that
our planet moves from $P$ to $P'$ in the short time interval $\delta t$. Here, $S$ represents
the position of the Sun. The lines $SP$ and $SP'$ are both approximately of length $r$. 
Moreover, using simple trigonometry, the line $PP'$ is of length $r\,\delta\theta$, where
$\delta\theta$ is the small angle through which the line joining the Sun and the planet 
rotates in the time interval $\delta t$. The area of the triangle $PSP'$ is approximately
\begin{equation}
\delta A = \frac{1}{2}\times r\,\delta\theta \times r:
\end{equation}
{\em i.e.}, half its base times its height. Of course, this area represents the area swept out
by the line joining the Sun and the planet in the  time interval $\delta t$. Hence, the
rate at which this area is swept is given by
\begin{equation}\label{sweep}
\lim_{\delta t\rightarrow 0}\frac{\delta A}{\delta t} = \frac{1}{2}\,r^2\,\lim_{\delta t\rightarrow 0}\frac{\delta \theta}{\delta t}
= \frac{r^2\,\dot{\theta}}{2} = \frac{h}{2}.
\end{equation}
Clearly, the fact that $h$ is a constant of the motion implies that the line joining the planet and
the Sun sweeps out area at a {\em  constant rate}: {\em i.e.}, the line sweeps equal areas in equal time intervals.
But, this is just Kepler's second law. We conclude that Kepler's second law of planetary motion is  a direct  manifestation of
{\em angular momentum conservation}.

Let
\begin{equation}
r = \frac{1}{u},
\end{equation}
where $u(t)\equiv u(\theta)$ is a new radial variable. Differentiating with respect to $t$, we obtain
\begin{equation}\label{e1228}
\dot{r} =- \frac{\dot{u}}{u^2} = - \frac{\dot{\theta}}{u^2}\frac{du}{d\theta} =- h\,\frac{du}{d\theta}.
\end{equation}
The last step follows from the fact that $\dot{\theta} = h\,u^2$. Differentiating a second time
with respect to $t$, we obtain
\begin{equation}\label{e1229}
\ddot{r} =- h\,\frac{d}{dt}\!\left(\frac{du}{d\theta}\right)= - h\,\dot{\theta}\,\frac{d^2 u}{d \theta^2}
= - h^2\,u^2\,\frac{d^2 u}{d\theta^2}.
\end{equation}
Equations~(\ref{e1227}) and (\ref{e1229}) can be combined to give
\begin{equation}
\frac{d^2 u}{d\theta^2} + u = \frac{G\,M_\odot}{h^2}.
\end{equation}
This equation possesses the fairly obvious general solution
\begin{equation}
u = A\,\cos(\theta-\theta_0) + \frac{G\,M_\odot}{h^2},
\end{equation}
where $A$ and $\theta_0$ are arbitrary constants.

The above formula can be inverted to give the following simple orbit equation for our planet:
\begin{equation}
r = \frac{1}{A\,\cos(\theta-\theta_0) + G\,M_\odot/h^2}.
\end{equation}
The constant $\theta_0$ merely determines the orientation of the orbit. Since we are
only interested in the orbit's {\em shape}, we can set this quantity to zero without
loss of generality. Hence, our orbit equation reduces to
\begin{equation}\label{ellipse}
r = r_0\,\frac{1+e}{1+ e\,\cos\theta},
\end{equation}
where
\begin{equation}
e = \frac{A\,h^2}{G\,M_\odot},
\end{equation}
and
\begin{equation}\label{r0}
r_0 = \frac{h^2}{G\,M_\odot\,(1+e)}.
\end{equation}

Formula~(\ref{ellipse}) is the standard equation of an {\em ellipse} (assuming  $e<1$), with the
origin at a focus. Hence, we have now proved Kepler's first law of planetary motion.
It is clear that $r_0$ is the radial distance at $\theta=0$. The radial distance at
$\theta=\pi$ is written
\begin{equation}\label{r1}
r_1 = r_0\,\frac{1+e}{1-e}.
\end{equation}
Here, $r_0$ is termed the {\em perihelion} distance ({\em i.e.}, the closest distance to the
Sun) and $r_1$ is termed the {\em aphelion} distance  ({\em i.e.}, the furthest distance from  the
Sun). The quantity
\begin{equation}
e = \frac{r_1-r_0}{r_1+r_0}
\end{equation}
is termed the {\em eccentricity} of the orbit, and is a measure of its departure from
circularity. Thus, $e=0$ corresponds to a purely circular orbit, whereas $e\rightarrow 1$
corresponds to a highly elongated orbit. As specified in Tab.~\ref{tecc}, the orbital
eccentricities of all of the planets (except Mercury) are fairly small.

\begin{table}
\centering
\begin{tabular}{r|c}
Planet & $e$ \\[0.5ex] \hline
Mercury & 0.206  \\[0.5ex]
Venus & 0.007\\[0.5ex]
Earth & 0.017\\ [0.5ex]
Mars & 0.093 \\[0.5ex]
Jupiter & 0.048 \\ [0.5ex]
Saturn & 0.056 \\
\end{tabular}
\caption{\em The orbital eccentricities of various planets in the Solar System.}\label{tecc}
\end{table}

According to Eq.~(\ref{sweep}), a line joining the Sun and an orbiting planet sweeps area
at the constant rate $h/2$. Let $T$ be the planet's orbital period. We expect the line to
sweep out the {\em whole area} of the ellipse enclosed by the planet's orbit in the time
interval $T$. Since the area of an ellipse is $\pi\,a\,b$, where $a$ and $b$ are the
{\em semi-major} and {\em semi-minor} axes, we can write
\begin{equation}\label{tdef1}
T = \frac{\pi\,a\,b}{h/2}.
\end{equation}
Incidentally, Fig.~\ref{f108} illustrates the relationship between the aphelion distance, the
perihelion distance, and the semi-major and semi-minor axes of a planetary orbit. It is
clear, from the figure, that the semi-major axis is just the mean of the  aphelion and
perihelion distances: {\em i.e.},
\begin{equation}\label{adef}
a = \frac{r_0+r_1}{2}.
\end{equation}
Thus, $a$ is essentially the planet's mean distance from the Sun. Finally, the relationship between $a$, $b$,
and the eccentricity, $e$, is given by the well-known formula
\begin{equation}\label{tdef2}
\frac{b}{a} = \sqrt{1-e^2}.
\end{equation}
This formula can easily be obtained from Eq.~(\ref{ellipse}).

\begin{figure}
\epsfysize=1.5in
\centerline{\epsffile{Chapter12/fig108.eps}}
\caption{\em Anatomy of a planetary orbit.}\label{f108}  
\end{figure}

Equations~(\ref{r0}), (\ref{r1}), and (\ref{adef}) can be combined to give
\begin{equation}\label{tdef3}
a = \frac{h^2}{2\,G\,M_\odot}\left(\frac{1}{1+e}+\frac{1}{1-e}\right) = \frac{h^2}{G\,M_\odot\,(1-e^2)}.
\end{equation}
It follows, from Eqs.~(\ref{tdef1}), (\ref{tdef2}), and (\ref{tdef3}), that the orbital period
can be written
\begin{equation}
T = \frac{2\pi}{\sqrt{G\,M_\odot}}\,\,a^{3/2}.
\end{equation}
Thus, the orbital period of a planet is proportional to its mean distance from the Sun to
the power $3/2$---the constant of proportionality being the {\em same} for all planets. Of course,
this is just Kepler's third law of planetary motion.

\subsection*{\em Worked Example 12.1: Gravity on Callisto}
\noindent {\em Question:} Callisto is the eighth of Jupiter's moons:  its  mass and radius are
 $M= 1.08\times 10^{23}\, {\rm kg}$ and $R= 2403  \,{\rm km}$, respectively. What is the gravitational
acceleration on the surface
of this moon?

\noindent{\em Answer:} The surface gravitational acceleration on a spherical body of mass $M$ and radius $R$ is
simply
$$
g = \frac{G\,M}{R^2}.
$$
Hence,
$$
g = \frac{(6.673\times 10^{-11})\times(1.08\times 10^{23})}{(2.403\times 10^6)^2} = 1.25\,{\rm m/s^2}.
$$

\subsection*{\em Worked Example 12.2: Acceleration of a Rocket}
\noindent {\em Question:} A rocket is located   a distance 3.5 times the radius of the Earth above the
Earth's surface. What is the rocket's free-fall acceleration?

\noindent{\em Answer:} Let $R_\oplus$ be the Earth's radius. The distance of the rocket from the
centre of the Earth is $r_1= (3.5+1)\,R_\oplus= 4.5\,R_\oplus$. We know that the free-fall
acceleration of the rocket when its distance from the Earth's centre is $r_0=R_\oplus$ ({\em i.e.}, when it
is at the Earth's surface)
is
$g_0= 9.81\,{\rm m/s^2}$. Moreover, we know that gravity is an inverse-square law ({\em i.e.}, $g\propto 1/r^2$).
Hence, the rocket's acceleration is
$$
g_1 = g_0\left(\frac{r_0}{r_1}\right)^2 = \frac{9.81\times 1}{(4.5)^2} = 0.484\,{\rm m/s^2}.
$$

\subsection*{\em Worked Example 12.3: Circular Earth Orbit}
\noindent {\em Question:} A satellite moves in a circular orbit around the Earth with speed
 $v=6000\,{\rm m/s}$. Determine the satellite's altitude above the Earth's surface.
Determine the period of the satellite's orbit. The Earth's mass and radius are
$M_\oplus =5.97\times 10^{24}\,{\rm kg}$ and $R_\oplus = 6.378\times 10^6\,{\rm m}$, 
respectively.

\noindent{\em Answer:} The acceleration of the satellite towards the centre of the Earth
is $v^2/r$, where $r$ is its orbital radius. This acceleration must be provided by the
 acceleration $G\,M_\oplus/r^2$ due to the Earth's gravitational attraction.
Hence,
$$
\frac{v^2}{r} = \frac{G\,M_\oplus}{r^2}.
$$
The above  expression can be rearranged to give
$$
r = \frac{G\,M_\oplus}{v^2} = \frac{(6.673\times 10^{-11})\times(5.97\times 10^{24})}{(6000)^2} =1.107\times 10^7\,{\rm m}.
$$
Thus, the satellite's altitude above the Earth's surface is
$$
h = r - R_\oplus = 1.107\times 10^7 -  6.378\times 10^6 = 4.69\times 10^6\,{\rm m}.
$$

The satellite's orbital period is simply
$$
T = \frac{2\,\pi\,r}{v} = \frac{2\times\pi\times (1.107\times 10^7)}{6000} = 3.22\,{\rm hours}.
$$

\subsection*{\em Worked Example 12.4: Halley's Comet}
\noindent {\em Question:} The distance of closest approach of Halley's comet to the
Sun is $0.57\,{\rm AU}$. (1 AU is the mean Earth-Sun distance.) The greatest
distance of the comet from the Sun is 35\,AU. The comet's speed at closest approach
is $54\,{\rm km/s}$. What is its speed when it is furthest from the Sun?

\noindent {\em Answer:} At perihelion and aphelion, the comet's velocity is
perpendicular to its position vector from the Sun. Hence, at these two
special points, the comet's angular momentum (around the Sun) takes the particularly
simple form
$$
l = m\,r\,u.
$$
Here, $m$ is the comet's mass, $r$ is its distance from the Sun, and $u$ is its speed.
According to Kepler's second law, the comet orbits the Sun with {\em constant} angular
momentum. Hence, we can write
$$
r_0\,u_0 = r_1\,u_1,
$$
where $r_0$ and $u_0$ are the perihelion distance and speed, respectively, and $r_1$ and $u_1$
are the corresponding quantities at aphelion. We are told that $r_0=0.57\,{\rm AU}$, $r_1=35\,{\rm AU}$,
and $u_0 = 54\,{\rm km/s}$. It follows that
$$
u_1 = \frac{u_0\,r_0}{r_1} = \frac{54\times 0.57}{35} = 0.879\,{\rm km/s}.
$$

\subsection*{\em Worked Example 12.5: Mass of Star}
\noindent {\em Question:} A planet is in circular orbit around a star. The
period and radius of the orbit are $T= 4.3\times 10^7\,{\rm s}$ and  $r=2.34\times 10^{11}\,{\rm m}$,
respectively. Calculate the mass of the star.

\noindent {\em Answer:} Let $\omega$ be the planet's orbital angular velocity. The
planet accelerates towards the star with acceleration $\omega^2\,r$. The acceleration
due to the star's gravitational attraction is $G\,M_\ast/r^2$, where $M_\ast$ is the
mass of the star. Equating these accelerations, we obtain
$$
\omega^2\,r = \frac{G\,M_\ast}{r^2}.
$$
Now,
$$
T = \frac{2\,\pi}{\omega}.
$$
Hence,
combining the previous two expressions, we get
$$
M_\ast = \frac{4\,\pi^2\,r^3}{G\,T^2}.
$$
Thus, the mass of the star is
$$
M_\ast = \frac{4\times\pi^2\times (2.34\times 10^{11})^3}{(6.673\times 10^{-11})\times
(4.3\times 10^7)^2} = 4.01\times 10^{30}\,{\rm kg}.
$$

\subsection*{\em Worked Example 12.6: Launch Energy}
\noindent {\em Question:} What is the minimum energy required to launch a probe
of mass $m=120\,{\rm kg}$ into outer space? The Earth's mass and radius are
$M_\oplus =5.97\times 10^{24}\,{\rm kg}$ and $R_\oplus = 6.378\times 10^6\,{\rm m}$, 
respectively.

\noindent {\em Answer:} The energy which must be given to the probe should
just match the probe's gain in potential energy as it travels  from the Earth's
surface to outer space. By definition, the probe's potential energy in outer
space is zero. The potential energy of the probe at the Earth's surface
is
$$
U = - \frac{G\,M_\oplus\,m}{R_\oplus} = \frac{(6.673\times 10^{-11})\times (5.97\times 10^{24})\times 120}
{( 6.378\times 10^6)} = -7.495\times 10^9\,{\rm J}.
$$
Thus, the gain in potential energy, which is the same as the minimum launch energy, is
$7.495\times 10^9\,{\rm J}$.
