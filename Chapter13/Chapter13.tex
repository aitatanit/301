\section{Wave Motion}
\subsection{Introduction}
Waves are small amplitude perturbations which propagate through continuous
media: {\em e.g.}, gases, liquids, solids, or---in the special case
of electromagnetic waves---a vacuum. Wave motion is a combination of
oscillatory and translational motion. Waves are
important because they are the means through which virtually all
information regarding the outside world  is transmitted to us. For instance,
we hear things via sound waves propagating through the air, and
we see things via light waves. Now, the physical mechanisms which  underlie sound and
light wave propagation are completely different. Nevertheless, sound and
light waves possesses a number of common properties which are intrinsic to
wave motion itself. In this section, we shall concentrate on the {\em common} properties of waves,
rather than those properties which are peculiar to  particular wave types.

\subsection{Waves on a Stretched String}
Probably the simplest type of wave is that which propagates down a stretched string.
Consider a straight string which is stretched such that it is under uniform tension $T$. Let
the string run along the $x$-axis. Suppose that the string is subject to a {\em small
amplitude} displacement, in the $y$-direction, which can {\em vary} along its length.
 Let $y(x,t)$ be the string's displacement
at position $x$ and time $t$. What is the equation of motion for $y(x,t)$?

\begin{figure}[h]
\epsfysize=2.5in
\centerline{\epsffile{Chapter13/fig108a.eps}}
\caption{\em Forces acting on a segment of a stretched string.}\label{f108a}  
\end{figure}

Consider an infinitesimal segment of the string which extends from $x-\delta x/2$ to $x+\delta x/2$. 
As shown in Fig.~\ref{f108a}, this segment is subject to opposing tension forces, $T$, at its two ends,
which act along the local tangent line to the string. Here, we are assuming that the string
displacement remains sufficiently small that the tension does not vary in magnitude along the string.
Suppose that the local tangent line to the string subtends  angles $\delta\theta_1$ and
$\delta\theta_2$ with the $x$-axis at $x-\delta x/2$ and $x+\delta x/2$, respectively---as shown
in Fig.~\ref{f108a}.
Note that these angles are written as infinitesimal quantities because the string
displacement is assumed to be infinitesimally small, which implies that the
string is everywhere almost parallel with the $x$-axis (the string displacement is greatly
exaggerated in Fig.~\ref{f108a}, for the sake of clarity).

Consider the $y$-component of the string segment's equation of motion. The net force
acting on the segment in the $y$-direction takes the form
\begin{equation}\label{e13x2}
f_y(x,t) = T\,\sin\delta\theta_2 - T\,\sin\delta\theta_1 \simeq T\,(\delta\theta_2-\delta\theta_1),
\end{equation}
since $\sin\theta\simeq \theta$ when $\theta$ is small. Now, from calculus,
\begin{eqnarray}\label{e13x3}
\frac{\partial y(x-\delta x/2,t)}{\partial x}&=& \tan\delta\theta_1\simeq \delta\theta_1,\\[0.5ex]
\frac{\partial y(x+\delta x/2,t)}{\partial x}&=& \tan\delta\theta_2\simeq \delta\theta_2,\label{e13x4}
\end{eqnarray}
since the gradient, $dy(x)/dx$, of the curve $y(x)$ is equal to
the tangent of the angle subtended by this curve with the $x$-axis. Note that $\tan\theta\simeq \theta$
when $\theta$ is small. The quantity $\partial y(x,t)/\partial x$ refers to the derivative of $y(x,t)$
with respect to $x$, {\em keeping $t$  constant}---such a derivative is known as a {\em partial
derivative}. Equations~(\ref{e13x2})--(\ref{e13x4}) can be combined to give
\begin{equation}\label{e13x5}
f_y(x,t) = T\left(\frac{\partial y(x+\delta x/2,t)}{\partial x}-\frac{\partial y(x-\delta x/2,t)}{\partial x}
\right)= T\,\delta x\,\frac{\partial^2 y(x,t)}{\partial x^2}.
\end{equation}
Here, $\partial^2 y(x,t)/\partial x^2$ is the second derivative of $y(x,t)$ with respect to $x$,
keeping $t$ constant.

Suppose that the string has a mass {\em per unit length} $\mu$. It follows that the $y$ equation of
motion of our string segment takes the form
\begin{equation}\label{e13x6}
\mu\,\delta x\,\frac{\partial^2 y(x,t)}{\partial t^2} = f_y (x,t),
\end{equation}
Here, $\partial^2 y(x,t)/\partial t^2$---the second derivative of $y(x,t)$ with respect to $t$,
keeping $x$ constant---is the $y$-acceleration of the string segment at position $x$ and time $t$. 
Equations~(\ref{e13x5}) and (\ref{e13x6}) yield the final expression for the string's equation of motion:
\begin{equation}\label{e13x7}
\frac{\partial^2 y}{\partial t^2}=\frac{T}{\mu}\,\frac{\partial^2 y}{\partial x^2}.
\end{equation}

Equation~(\ref{e13x7}) is an example of a {\em wave equation}. In fact,
all small amplitude waves satisfy an equation of motion of this basic form. A particular solution of
this type of equation has been known for centuries: {\em i.e.}, 
\begin{equation}\label{e13x8}
y(x,t) = y_0\,\cos\,(k\,x - \omega \,t),
\end{equation}
where $y_0$, $k$, and $\omega$ are constants. We can demonstrate that (\ref{e13x8}) satisfies
(\ref{e13x7}) by direct substitution. Thus,
\begin{eqnarray}
\frac{\partial y}{\partial t} &=& y_0\,\omega\,\sin\,(k\,x - \omega \,t),\\[0.5ex]
\frac{\partial^2 y}{\partial t^2} &=& -y_0\,\omega^2\,\cos\,(k\,x - \omega \,t),\label{e13x9}
\end{eqnarray}
and
\begin{eqnarray}
\frac{\partial y}{\partial x} &=& -y_0\,k\,\sin\,(k\,x - \omega \,t),\\[0.5ex]
\frac{\partial^2 y}{\partial x^2} &=& -y_0\,k^2\,\cos\,(k\,x - \omega \,t).\label{e13x10}
\end{eqnarray}
Substituting Eqs.~(\ref{e13x9}) and (\ref{e13x10}) into Eq.~(\ref{e13x7}), we find that the
latter equation is satisfied provided 
\begin{equation}\label{e13xx}
\frac{\omega^2}{k^2} = \frac{T}{\mu}.
\end{equation}

Equation~(\ref{e13x8}) describes a pattern of motion which is {\em periodic in both space and time}.
This periodicity follows from the well-known periodicity property of the cosine function:
namely,
$\cos(\theta+2\,\pi)=\cos\theta$. Thus, the wave pattern is periodic in space,
\begin{equation}
y (x+\lambda,t) = y(x,t),
\end{equation}
with periodicity length 
\begin{equation}\label{e13x11}
\lambda = \frac{2\,\pi}{k}.
\end{equation}
Here, $\lambda$ is known as the {\em wavelength}, whereas $k$ is known as the {\em wavenumber}. 
The wavelength is the distance between successive wave peaks.
The wave pattern is periodic in time,
\begin{equation}
y (x,t+T) = y(x,t),
\end{equation}
with period 
\begin{equation}
T = \frac{2\,\pi}{\omega}.
\end{equation}
The wave period is the oscillation period of the wave disturbance at a given point in space.
The wave {\em frequency} ({\em i.e.}, the number of cycles per second the wave pattern executes
at a given point in space) is written
\begin{equation}\label{e13x12}
f = \frac{1}{T} = \frac{\omega}{2\,\pi}.
\end{equation}
The quantity $\omega$ is termed the {\em angular frequency}
 of the wave. Finally, at any given point in space, the displacement $y$ oscillates
between $+y_0$ and $-y_0$ (since the maximal values of $\cos\theta$ are $\pm 1$).
Hence, $y_0$ corresponds to the wave {\em amplitude}.

Equation~(\ref{e13x8}) also describes a sinusoidal pattern which {\em propagates} along the $x$-axis {\em without
changing shape}. We can see this by examining the motion of the
wave peaks, $y=+y_0$, which correspond to
\begin{equation}
k\,x-\omega\,t = n\,2\,\pi,
\end{equation}
where $n$ is an integer. Differentiating the above expression with respect to time,
we obtain
\begin{equation}
\frac{d x}{dt} = \frac{\omega}{k}.
\end{equation}
In other words, the wave peaks all propagate along the $x$-axis with uniform speed
\begin{equation}
v=\frac{\omega}{k}. \label{e13x13}
\end{equation}
It is easily demonstrated that the wave troughs, $y = -y_0$,   propagate
with the same speed. Thus, it is fairly clear that the whole wave pattern
moves with speed $v$---see Fig.~\ref{f109}. Equations~(\ref{e13x11}), (\ref{e13x12}), and
(\ref{e13x13}) yield
\begin{equation}
v = f\,\lambda:\label{e13xy}
\end{equation}
{\em i.e.}, a wave's speed is the product of its frequency and its wavelength.
This is true for all types of (sinusoidal) wave.

\begin{figure}
\epsfysize=3in
\centerline{\epsffile{Chapter13/fig109.eps}}
\caption{\em A sinusoidal wave propagating down the $x$-axis. The solid, dotted, dashed,
and dot-dashed curves show the wave displacement at four successive and equally spaced
times.}\label{f109}  
\end{figure}

Equations (\ref{e13xx}) and (\ref{e13x13}) imply that
\begin{equation}
v = \sqrt{\frac{T}{\mu}}.\label{e13yy}
\end{equation}
In other words, all waves that propagate down a stretched string do so with the
{\em same speed}. This common speed is determined by the properties of the
string: {\em i.e.}, its tension and  mass per unit length.
 Note, from Eq.~(\ref{e13x8}), that the wavelength $\lambda$ 
is arbitrary. However, once the wavelength is specified, the wave frequency $f$ is
fixed via Eqs.~(\ref{e13xy}) and (\ref{e13yy}). It follows that short wavelength
waves possess high frequencies, and {\em vice versa}.

\subsection{General Waves}
By analogy with the previous discussion, a general wave disturbance propagating
along the $x$-axis satisfies
\begin{equation}\label{e13zz}
\frac{\partial^2 y}{\partial t^2}=v^2\,\frac{\partial^2 y}{\partial x^2},
\end{equation}
where $v$ is the common wave speed. In general, $v$ is determined by the properties of
the  medium through which the wave propagates. Thus, for waves propagating along a string, the wave
speed is determined by the string tension and mass per unit length; for
sound waves propagating through a gas, the wave speed is determined by the
gas pressure and density; and for electromagnetic waves propagating through
a vacuum, the wave speed is a constant of nature: {\em i.e.}, $c=3\times 10^8\,{\rm m}/{\rm s}^2$.

One solution of Eq.~(\ref{e13zz}) is
\begin{equation}
y(x,t) = y_0\,\cos\,[k\,(x - v\,t)].
\end{equation}
This is interpreted as a (sinusoidal) wave of amplitude $y_0$ and wavelength $\lambda=2\,\pi/k$ which
propagates in the $+x$ direction with speed $v$. It is easily demonstrated that another
equally good solution of Eq.~(\ref{e13zz}) is
\begin{equation}
y(x,t) = y_0\,\cos\,[k\,(x + v\,t)].
\end{equation}
This is interpreted as a (sinusoidal) wave of amplitude $y_0$ and wavelength $\lambda=2\,\pi/k$ which
propagates in the $-x$ direction with speed $v$.

Equation (\ref{e13zz}) is a {\em linear} partial differential equation (PDE): {\em i.e.}, it is invariant under the transformation
$y\rightarrow a\,y + b$, where $a$ and $b$ are arbitrary constants. One important mathematical
property of linear PDEs is that their solutions are {\em superposable}: {\em i.e.}, they can be added
together and still remain solutions. Thus,  if $y_1(x,t)$ and $y_2(x,t)$ are two distinct solutions
 of Eq.~(\ref{e13zz}) then $a\,y_1(x,t) + b\,y_2(x,t)$ (where $a$ and $b$ are arbitrary constants)
is also a solution---this can be seen from inspection of Eq.~(\ref{e13zz}). To be
more exact, if
\begin{equation}
y_1(x,t) = a_1\,\cos\,[k_1\,(x - v\,t)]
\end{equation}
represents a wave of amplitude $a_1$ and wavenumber $k_1$ which propagates in the
$+x$ direction, and
\begin{equation}
y_2(x,t) = a_2\,\cos\,[k_2\,(x + v\,t)]
\end{equation}
represents a wave of amplitude $a_2$ and wavenumber $k_2$ which propagates in the
$-x$ direction, then
\begin{equation}
y(x,t) = y_1(x,t) + y_2(x,t)
\end{equation}
is a valid solution of the wave equation, and represents the two aforementioned waves 
propagating in the same region {\em without affecting one another}.

\begin{figure}
\epsfysize=2.5in
\centerline{\epsffile{Chapter13/fig110.eps}}
\caption{\em A wave-pulse propagating down the $x$-axis. The solid, dotted, and dashed
curves show the wave displacement at three successive and equally spaced
times.}\label{f110}  
\end{figure}

\subsection{Wave-Pulses}
As is easily demonstrated, the most general solution of the wave equation (\ref{e13zz}) is written
\begin{equation}
F (x - v\,t),
\end{equation}
where $F(p)$ is an {\em arbitrary} function. The above solution is interpreted
as a pulse of {\em arbitrary shape} which propagates in the $+x$ direction with
speed $v$, {\em without changing shape}---see Fig.~\ref{f110}. Likewise,
\begin{equation}
G (x + v\,t)
\end{equation}
represents  another arbitrary pulse  which propagates in the $-x$ direction with
speed $v$, without changing shape. Note that, unlike our previous sinusoidal wave solutions,
a general wave-pulse possesses a definite propagation speed but  {\em does not} possess
a definite wavelength or frequency.

What is the relationship between these new wave-pulse solutions and our previous
sinusoidal wave solutions? It turns out that any wave-pulse can be built up from
a suitable {\em linear superposition} of sinusoidal waves. For instance,
if $F(x-v\,t)$ represents a wave-pulse propagating down the $x$-axis, then
we can write
\begin{equation}
F (x-v\,t) = \int_0^\infty \bar{F}(k)\,\cos\,[k\,(x-v\, t)]\,dk,
\end{equation}
where we have assumed that $F(-p)=F(p)$, for the sake of simplicity. The above
formula is basically a recipe for generating the propagating wave-pulse $F(x-v\,t)$ 
from a suitable admixture of sinusoidal waves of definite wavelength and frequency:
$\bar{F}(k)$ specifies the required amplitude of the wavelength $\lambda =2\,\pi/k$
component. How do we determine $\bar{F}(k)$ for a given wave-pulse? Well, a
mathematical result known as {\em Fourier's theorem} yields
\begin{equation}
\bar{F} (k) = \frac{2}{\pi}\int_0^\infty F(p)\,\cos\,(k\,p)\,dp,
\end{equation}
The above expression essentially tells us the strength of the wavenumber $k$ component
of the wave-pulse  $F(x-v\,t)$. Note that the function $\bar{F}(k)$ is known as the
{\em Fourier spectrum} of the wave-pulse $F(x-v\,t)$. 

\begin{figure}
\epsfysize=5in
\centerline{\epsffile{Chapter13/fig111.eps}}
\caption{\em A propagating wave-pulse, $F(x-v\,t)$, and its
associated Fourier spectrum, $\bar{F}(k)$.}\label{f111}  
\end{figure}

Figures~\ref{f111} and \ref{f112} show two different wave-pulses and their associated
Fourier spectra. Note how, by combining sinusoidal waves of varying wavenumber in different proportions,
it is possible to build up  wave-pulses of completely different shape.

\begin{figure}
\epsfysize=5in
\centerline{\epsffile{Chapter13/fig112.eps}}
\caption{\em A propagating wave-pulse, $F(x-v\,t)$, and its
associated Fourier spectrum, $\bar{F}(k)$.}\label{f112}  
\end{figure}

\subsection{Standing Waves}
Up to now, all of the wave solutions that we have investigated have been propagating
solutions. Is it possible to construct a wave solution which does not propagate?
Suppose we combine a sinusoidal wave of amplitude $y_0$ and wavenumber $k$ which
propagates in the $+x$ direction,
\begin{equation}
y_1(x,t) = y_0\,\cos\,(k\,x-\omega\,t),
\end{equation}
with a second sinusoidal wave of amplitude $y_0$ and wavenumber $k$ which
propagates in the $-x$ direction,
\begin{equation}
y_2(x,t) = y_0\,\cos\,(k\,x+\omega\,t).
\end{equation}
The net result is
\begin{equation}
y(x,t) = y_1(x,t)+y_2(x,t) = y_0\left[\cos\,(k\,x-\omega\,t)+\cos\,(k\,x+\omega\,t)\right].
\end{equation}
Making use of the standard trigonometric identity
\begin{equation}
\cos x + \cos y = 2\, \cos\left(\frac{x+y}{2}\right) \, \cos\left(\frac{x-y}{2}\right),
\end{equation}
we obtain
\begin{equation}
y(x,t) = 2\,y_0\,\cos \,(k\,x)\,\cos\,(\omega\,t).
\end{equation}
The pattern of motion specified by the above expression is illustrated in Fig.~\ref{f113}.
It can be seen that the wave pattern {\em does not} propagate along the $x$-axis. Note,
however, that the amplitude of the wave now varies with position. At certain points,
called {\em nodes}, the amplitude is zero. At other points, called {\em anti-nodes}, the amplitude
is maximal. The nodes are halfway between successive anti-nodes, and both nodes and
anti-nodes are evenly spaced half a wavelength apart. 

\begin{figure}
\epsfysize=3in
\centerline{\epsffile{Chapter13/fig113.eps}}
\caption{\em A standing wave. The various curves show the wave displacement at
 different times.}\label{f113}  
\end{figure}

The standing wave shown in Fig.~\ref{f113} can be thought of as the {\em interference pattern} generated
by combining the two traveling wave solutions $y_1(x,t)$ and $y_2(x,t)$. At the anti-nodes, the waves
reinforce one another, so that the oscillation amplitude becomes double that associated with each
wave individually---this is termed {\em constructive interference}. At the
nodes, the waves completely cancel one another out---this is termed {\em destructive interference}.

Most musical instruments work by exciting standing waves. 
For instance, stringed instruments excite standing
waves on strings, whereas wind instruments excite standing waves in columns of air.
Consider a
guitar string of length $L$. Suppose that the string runs along the $x$-axis, and
extends from $x=0$ to $x=L$. Since the ends of the string are fixed, any wave excited
on the string must satisfy the constraints
\begin{equation}
y(0,t) = y(L, t) = 0.
\end{equation}
It is fairly clear that no propagating wave solution of the form $y_0\,\cos\,[k\,(x\pm v\,t)]$
can satisfy these constraints. However, a standing wave can easily satisfy the 
constraints, provided two of its nodes coincide with the ends of the string. Since
the nodes in a standing wave pattern are spaced half a wavelength apart, it follows that
the wave frequency must be adjusted such that an integer number of half-wavelengths
fit on the string. In other words,
\begin{equation}
L = n\,\frac{\lambda}{2},
\end{equation}
where $n=1,2,3,\ldots$. Now, from Eqs.~(\ref{e13xy}) and (\ref{e13yy}),
\begin{equation}
f\,\lambda = \sqrt{\frac{T}{\mu}},
\end{equation}
where $T$ and $\mu$ are the tension and mass per unit length of the string, respectively.
The above two equations can be combined to give
\begin{equation}
f = \frac{n}{2\,L}\,\sqrt{\frac{T}{\mu}}.
\end{equation}
Thus, the standing waves that can be excited on a guitar string have frequencies
$f_0$, $2\,f_0$, $3\,f_0$, {\em etc}.,  which are integer multiples of
\begin{equation}
f_0 = \frac{1}{2\,L}\,\sqrt{\frac{T}{\mu}}.
\end{equation}
These frequencies are transmitted to our ear, via sound waves which oscillate in
sympathy with the guitar string, and are interpreted as musical notes. To be
more exact, the frequencies correspond to  notes spaced an octave apart.
The frequency $f_0$ is termed the {\em fundamental} frequency, whereas the
frequencies $2\,f_0$, $3\,f_0$, {\em etc.} are termed the {\em overtone harmonic} frequencies. 
When a guitar string is plucked an admixture of standing waves, consisting predominantly
of the fundamental harmonic wave, is excited on the
string. The fundamental harmonic determines the musical note which the guitar string plays.
However, it is the overtone harmonics which give the note its peculiar timbre. Thus,
a trumpet sounds different to a guitar, even when they are both playing the same note, because
a trumpet excites a different mix of overtone harmonics than a guitar.

\subsection{The Doppler Effect}
Consider a sinusoidal wave of wavenumber $k$ and angular frequency $\omega$ propagating
in the $+x$ direction:
\begin{equation}\label{e13ccc}
y(x,t) = y_0\,\cos\,(k\,x-\omega\,t).
\end{equation}
The wavelength and frequency of the wave, as seen by a stationary observer, are
$\lambda = 2\,\pi/k$ and $f=\omega/2\,\pi$, respectively. Consider a second
observer moving with uniform speed $v_o$ in the $+x$ direction. What are
the wavelength and frequency of the wave, as seen by the second observer?
Well, the $x$-coordinate in the moving observer's frame of reference
is $x'=x-v_0\,t$ (see Sect.~\ref{frames}). Of course, both observers measure
the same time. Hence, in the second observer's frame of reference the
wave takes the form
\begin{equation}
y(x',t) = y_0\,\cos\,(k\,x'-\omega'\,t),
\end{equation}
where
\begin{equation}
\omega' = \omega - k\, v_o.
\end{equation}
Here, we have simply replaced $x$ by $x'+v_o\,t$ in Eq.~(\ref{e13ccc}).
Clearly, the moving observer sees a wave possessing the {\em same wavelength} ({\em
i.e.}, the same $k$) but a {\em different frequency} ({\em i.e.}, a different $\omega$)
to that seen by the stationary observer. This phenomenon is called the {\em Doppler effect}. Since
$v=\omega/k$, it follows that the wave speed is also shifted in the moving observer's
frame of reference. In fact,
\begin{equation}
v' = v - v_o,
\end{equation}
where $v'$ is the wave speed seen by the moving observer.
Finally, since $v=f\,\lambda$, and the wavelength is the same in both the moving and
stationary observers' frames of reference, the wave frequency experienced by
the moving observer is
\begin{equation}
f' = \left(1 - \frac{v_o}{v}\right)\,f. 
\end{equation}
Thus, the moving observer sees a lower frequency wave than the stationary observer. This
occurs because the moving observer is traveling in the same
direction as the wave, and  is therefore effectively trying to catch it up. It is easily demonstrated
that an observer moving in the opposite direction to a wave sees a higher frequency
than a stationary observer. Hence, the general Doppler shift formula (for a moving observer and
a stationary wave source) is
\begin{equation}
f' = \left(1 \mp \frac{v_o}{v}\right)\,f,
\end{equation}
where the upper/lower signs correspond to the observer moving in the same/opposite direction
to the wave.

Consider a stationary observer measuring a wave emitted by a source which is moving towards
the observer with speed $v_s$. Let $v$ be the propagation speed of the wave. 
Consider two neighbouring wave crests emitted by the source. Suppose that the first is
emitted at time $t=0$, and the second at time $t=T$, where $T=1/f$ is the wave period in the
frame of reference of the source. At time $t$, the first wave crest has traveled a distance
$d_1=v\,t$ towards the observer, whereas the second wave crest has traveled a distance
$d_2 =v\,(t-T) + v_s\,T$ (measured from the position of the
source at $t=0$). Here, we have taken into account the fact that the source is
a distance $v_s\,T$ closer to the observer when the second wave crest is emitted. 
The effective wavelength, $\lambda'$,  seen by the observer is the distance between neighbouring
wave crests. Hence,
\begin{equation}
\lambda' = d_1-d_2 = (v - v_s)\,T.
\end{equation}
 Since $v=f'\,\lambda'$,
the effective frequency $f'$ seen by the observer is
\begin{equation}
f' = \frac{f}{1-v_s/v},
\end{equation}
where $f$ is the wave frequency in the frame of reference of the source. We conclude that if the
source is moving {\em  towards} the observer then the wave frequency is shifted {\em upwards}. Likewise,
if the source is moving {\em away} from the observer then the frequency is shifted {\em downwards}.
This manifestation of the Doppler effect should be familiar to everyone. When an ambulance
passes us on the street, its siren has a higher pitch ({\em i.e.}, a high
frequency) when it is coming towards us than when it is moving away from us.
Of course, the oscillation frequency of the siren never changes. It is the Doppler
shift induced by the motion of the siren with respect to a stationary listener which
causes the frequency change.

The general formula for the shift in a wave's frequency induced by
relative motion of the observer and the source is
\begin{equation}
f' = \left(\frac{1\mp v_o/v}{1\pm v_s/v}\right) f,
\end{equation}
where $v_o$ is the speed of the observer, and $v_s$ is the speed of the source.
The upper/lower signs correspond to relative motion by which the observer and the source
move apart/together.

Probably the most notorious use of the Doppler effect in everyday life is in police speed
traps. In a speed trap, a policeman fires radar waves ({\em i.e.}, electromagnetic
waves of centimeter wavelength) of fixed frequency at an oncoming car. These waves
reflect off the car, which effectively becomes a moving source. Hence, by measuring the frequency increase
of the reflected waves, the policeman can determine the car's speed.

\subsection*{\em Worked Example 13.1: Piano Range}
\noindent {\em Question:}  A piano emits sound waves whose frequencies range from $f_l =28\, {\rm Hz}$ to
$f_h = 4200 \,{\rm Hz}$. What range of wavelengths is spanned by these waves? The speed of
sound in air is $v=343\,{\rm m/s}$. 

\noindent{\em Answer:} The relationship between a wave's frequency, $f$, wavelength, $\lambda$,
and speed, $v$, is
$$
v = f\,\lambda.
$$
Hence, $\lambda=v/f$. The shortest wavelength (which corresponds to the highest
frequency) is 
$$
\lambda_l = \frac{v}{f_h} = \frac{343}{4200} = 8.1667\times 10^{-2}\,{\rm m}.
$$
The longest wavelength  (which corresponds to the lowest
frequency) is
$$
\lambda_h = \frac{v}{f_l} = \frac{343}{28} = 12.250\,{\rm m}.
$$

\subsection*{\em Worked Example 13.2: Middle C}
\noindent {\em Question:} A steel wire in a piano has a length of $L=0.9\,{\rm m}$
and a mass of $m= 5.4\,{\rm g}$. To what tension $T$ must this wire be
stretched so that its fundamental vibration corresponds to
middle C: {\em i.e.}, the vibration possess a frequency $f=261.6\,{\rm Hz}$. 

\noindent{\em Answer:} The fundamental standing wave on a stretched wire is such that
the length $L$ of the wire corresponds to half the wavelength $\lambda$ of the wave.
Hence,
$$
\lambda = 2\,L = 1.80\,{\rm m}.
$$
The propagation speed of waves on the wire is given by
$$
v = f\,\lambda = 261.6\times 1.80= 470.88\,{\rm m/s}.
$$
Furthermore, the string's mass per unit length is
$$
\mu = \frac{m}{L} = \frac{5.4\times 10^{-3}}{0.9} = 6.00\times 10^{-3} \,{\rm kg/m}.
$$
Now, the relationship between the wave propagation speed, $v$, the mass per unit
length, $\mu$, and the tension, $T$, of a stretched wire is
$$
v = \sqrt{\frac{T}{\mu}}.
$$
Thus,
$$
T = v^2\,\mu  = (470.88)^2\times 6.00\times 10^{-3} = 1.330\times 10^3\,{\rm N}.
$$

\subsection*{\em Worked Example 13.3: Sinusoidal Wave}
\noindent {\em Question:} A wave is described by
$$
y = A\,\sin\,(k\,x-\omega\,t),
$$
where $A=4\,{\rm cm}$, $k=2.65\,{\rm rad./m}$, and  $\omega
=4.78\,{\rm rad./s}$. Moreover, $x$ is in meters and $t$ is
in seconds. What are the wavelength, frequency, and propagation
speed of the wave?

\noindent{\em Answer:} We identify $A$ as the wave amplitude, $k$ as the wavenumber,
and $\omega$ as the angular frequency. Now, $k=2\,\pi/\lambda$, where $\lambda$ is
the wavelength. Hence,
$$
\lambda = \frac{2\,\pi}{k} = \frac{2\times \pi}{2.65} = 2.371\,{\rm m}.
$$
Furthermore, $\omega = 2\,\pi\,f$, where $f$ is the frequency. Hence,
$$
f = \frac{\omega}{2\,\pi} =  \frac{4.78}{2\times\pi}=0.7608\,{\rm Hz}.
$$
Finally, $v=f\,\lambda$, where $v$ is the propagation speed. Thus,
$$
v = 0.7608\times 2.371 = 1.804\,{\rm m/s}.
$$

\subsection*{\em Worked Example 13.4: Truck Passing Stationary Siren}
\noindent {\em Question:} A truck, moving at $v_o=80\,{\rm km/hr}$, passes a stationary
police car whose siren has a frequency of $f=500\,{\rm Hz}$.
What is the frequency change heard by the truck driver
as the truck passes the police car? The speed of sound is
$v=343\,{\rm m/s}$. 

\noindent{\em Answer:} The truck's speed is
$$
v_o = \frac{80\times 1000}{3600} = 22.22\,{\rm m/s}.
$$
When the truck is moving towards the police car, the siren's
apparent frequency is
$$
f_1 = \left(1+\frac{v_o}{v}\right)\,f = \left(1+\frac{22.22}{343}\right)\times
500 = 532.39\,{\rm Hz}.
$$
When the truck is moving away from the police car, the siren's
apparent frequency is
$$
f_2 = \left(1-\frac{v_o}{v}\right)\,f = \left(1-\frac{22.22}{343}\right)\times
500 = 467.61\,{\rm Hz}.
$$
Hence, the frequency shift is
$$
{\mit\Delta f} = f_1-f_2 = 532.39-467.61 = 64.79\,{\rm Hz}.
$$

\subsection*{\em Worked Example 13.5: Ambulance and Car}
\noindent {\em Question:} An ambulance is traveling down a straight road at speed
$v_s=42\,{\rm m/s}$. The ambulance approaches a car which is traveling on the
same road, in the same direction, at speed $v_o=33\,{\rm m/s}$. The
ambulance driver hears his/her siren at a frequency of $f=500\,{\rm Hz}$. 
At what frequency does the driver of the car hear the siren? The speed  of sound is
$v=343\,{\rm m/s}$. 

\noindent{\em Answer:}
The apparent frequency $f'$ of a sound wave is given by
$$
f' = \left(\frac{1-v_o/v}{1-v_s/v}\right)\,f,
$$
where $v_o$ is the speed of the observer ({\em i.e.}, the car driver), $v_s$ is
the speed of the source ({\em i.e.}, the ambulance), $v$ is the speed of sound,
and $f$ is the wave frequency in the frame of reference of the source. We
have chosen a minus sign in the numerator of the above formula because the
observer is moving {\em away from} the source, leading to a {\em downward} Doppler
shift. We
have chosen a minus sign in the denominator of the above formula because the
source is moving {\em towards}  the observer, leading to a {\em upward} Doppler
shift. Hence,
$$
f' = \left(\frac{1-33/343}{1-42/343}\right)\times 500 = 514.95\,{\rm Hz}.
$$
